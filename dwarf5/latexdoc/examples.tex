\chapter{Examples (Informative)}
\label{app:examplesinformative}

The following sections provide examples that illustrate
various aspects of the DWARF debugging information format.



\section{ Compilation Units and Abbreviations Table Example}
\label{app:compilationunitsandabbreviationstableexample}


Figure \refersec{fig:compilationunitsandabbreviationstable}
depicts the relationship of the abbreviations tables contained
in the .debug\_abbrev section to the information contained in
the .debug\_info section. Values are given in symbolic form,
where possible.

The figure corresponds to the following two trivial source files:

File myfile.c
\begin{lstlisting}
typedef char* POINTER;
\end{lstlisting}
File myfile2.c
\begin{lstlisting}
typedef char* strp;
\end{lstlisting}

% Ensures we get the following float out before we go on.
\clearpage
\begin{figure}[here]
%\centering
\begin{minipage}{0.4\textwidth}
\centering
Compilation Unit 1: .debug\_info
\begin{framed}
\scriptsize
\begin{alltt}
\textit{length}
4
\textit{a1 (abbreviations table offset)}
4
\vspace{0.01cm}
\hrule
1
"myfile.c"
"Best Compiler Corp: Version 1.3"
"/home/mydir/src"
DW\-\_LANG\-\_C89
0x0
0x55
\livelink{chap:DWFORMsecoffset}{DW\-\_FORM\-\_sec\-\_offset}
0x0
\vspace{0.01cm}
\hrule
2
“char”
\livelink{chap:DWATEunsignedchar}{DW\-\_ATE\-\_unsigned\-\_char}
1
\vspace{0.01cm}
\hrule
3
e1
\vspace{0.01cm}
\hrule
4
“POINTER”
e2
\vspace{0.01cm}
\hrule
0
\end{alltt}
%
%
\end{framed}
Compilation Unit 2: .debug\_info
\begin{framed}
\scriptsize
\begin{alltt}
\textit{length}
4
\textit{a1 (abbreviations table offset)}
4
\vspace{0.01cm}
\hrule
...
\vspace{0.01cm}
\hrule
4
“strp”
e2
\vspace{0.01cm}
\hrule
...
\end{alltt}
%
%
\end{framed}
\end{minipage}
\hfill
\begin{minipage}{0.4\textwidth}
\centering
Abbreviation Table: .debug\_abbrev
\begin{framed}
\scriptsize
\begin{alltt}
\livelink{chap:DWTAGcompileunit}{DW\-\_TAG\-\_compile\-\_unit}
\livelink{chap:DWCHILDRENyes}{DW\-\_CHILDREN\-\_yes}
\livelink{chap:DWATname}{DW\-\_AT\-\_name}       \livelink{chap:DWFORMstring}{DW\-\_FORM\-\_string}
\livelink{chap:DWATproducer}{DW\-\_AT\-\_producer}   \livelink{chap:DWFORMstring}{DW\-\_FORM\-\_string}
\livelink{chap:DWATcompdir}{DW\-\_AT\-\_comp\-\_dir}   \livelink{chap:DWFORMstring}{DW\-\_FORM\-\_string}
\livelink{chap:DWATlanguage}{DW\-\_AT\-\_language}   \livelink{chap:DWFORMdata1}{DW\-\_FORM\-\_data1}
\livelink{chap:DWATlowpc}{DW\-\_AT\-\_low\-\_pc}     \livelink{chap:DWFORMaddr}{DW\-\_FORM\-\_addr}
\livelink{chap:DWAThighpc}{DW\-\_AT\-\_high\-\_pc}    \livelink{chap:DWFORMdata1}{DW\-\_FORM\-\_data1}
\livelink{chap:DWATstmtlist}{DW\-\_AT\-\_stmt\-\_list}  \livelink{chap:DWFORMindirect}{DW\-\_FORM\-\_indirect}
0                  0
\vspace{0.01cm}
\hrule
2
\livelink{chap:DWTAGbasetype}{DW\-\_TAG\-\_base\-\_type}
\livelink{chap:DWCHILDRENno}{DW\-\_CHILDREN\-\_no}
\livelink{chap:DWATname}{DW\-\_AT\-\_name}       \livelink{chap:DWFORMstring}{DW\-\_FORM\-\_string}
\livelink{chap:DWATencoding}{DW\-\_AT\-\_encoding}   \livelink{chap:DWFORMdata1}{DW\-\_FORM\-\_data1}
\livelink{chap:DWATbytesize}{DW\-\_AT\-\_byte\-\_size}  \livelink{chap:DWFORMdata1}{DW\-\_FORM\-\_data1}
0
\vspace{0.01cm}
\hrule
3
\livelink{chap:DWTAGpointertype}{DW\-\_TAG\-\_pointer\-\_type}
\livelink{chap:DWCHILDRENno}{DW\-\_CHILDREN\-\_no}
\livelink{chap:DWATtype}{DW\-\_AT\-\_type}       \livelink{chap:DWFORMref4}{DW\-\_FORM\-\_ref4}
0
\vspace{0.01cm}
\hrule
4
\livelink{chap:DWTAGtypedef}{DW\-\_TAG\-\_typedef}
\livelink{chap:DWCHILDRENno}{DW\-\_CHILDREN\-\_no}
\livelink{chap:DWATname}{DW\-\_AT\-\_name}      \livelink{chap:DWFORMstring}{DW\-\_FORM\-\_string}
\livelink{chap:DWATtype}{DW\-\_AT\-\_type}      \livelink{chap:DWFORMrefaddr}{DW\-\_FORM\-\_ref\-\_addr}
0
\vspace{0.01cm}
\hrule
0
\end{alltt}
\end{framed}
\end{minipage}
\caption{Compilation units and abbreviations table} \label{fig:compilationunitsandabbreviationstable}
\end{figure}

% Ensures we get the above float out before we go on.
\clearpage

\section{Aggregate Examples}
\label{app:aggregateexamples}

The following examples illustrate how to represent some of
the more complicated forms of array and record aggregates
using DWARF.

\subsection{Fortran 90 Example}
\label{app:fortran90example}
Consider the Fortran 90 source fragment in 
Figure \refersec{fig:fortran90examplesourcefragment}.

\begin{figure}[here]
\begin{lstlisting}
type array_ptr
real :: myvar
real, dimension (:), pointer :: ap
end type array_ptr
type(array_ptr), allocatable, dimension(:) :: arrays
allocate(arrays(20))
do i = 1, 20
allocate(arrays(i)%ap(i+10))
end do
\end{lstlisting}
\caption{Fortran 90 example: source fragment} \label{fig:fortran90examplesourcefragment}
\end{figure}

For allocatable and pointer arrays, it is essentially required
by the Fortran 90 semantics that each array consist of two
parts, which we here call 1) the descriptor and 2) the raw
data. (A descriptor has often been called a dope vector in
other contexts, although it is often a structure of some kind
rather than a simple vector.) Because there are two parts,
and because the lifetime of the descriptor is necessarily
longer than and includes that of the raw data, there must be
an address somewhere in the descriptor that points to the
raw data when, in fact, there is some (that is, when 
the ``variable'' is allocated or associated).

For concreteness, suppose that a descriptor looks something
like the C structure in 
Figure \refersec{fig:fortran90exampledescriptorrepresentation}.
Note, however, that it is
a property of the design that 1) a debugger needs no builtin
knowledge of this structure and 2) there does not need to
be an explicit representation of this structure in the DWARF
input to the debugger.

\begin{figure}[here]
\begin{lstlisting}
struct desc {
    long el_len; // Element length
    void * base; // Address of raw data
    int ptr_assoc : 1; // Pointer is associated flag
    int ptr_alloc : 1; // Pointer is allocated flag
    int num_dims : 6; // Number of dimensions
    struct dims_str { // For each dimension...  
        long low_bound;
        long upper_bound;
        long stride;
    } dims[63];
};
\end{lstlisting}
\caption{Fortran 90 example: descriptor representation} \label{fig:fortran90exampledescriptorrepresentation}
\end{figure}


In practice, of course, a “real” descriptor will have
dimension substructures only for as many dimensions as are
specified in the num\_dims component. Let us use the notation
desc ,textless n \textgreater\   
to indicate a specialization of the desc struct in
which n is the bound for the dims component as well as the
contents of the num\_dims component.

Because the arrays considered here come in two parts, it is
necessary to distinguish the parts carefully. In particular,
the “address of the variable” or equivalently, the “base
address of the object” always refers to the descriptor. For
arrays that do not come in two parts, an implementation can
provide a descriptor anyway, thereby giving it two parts. (This
may be convenient for general runtime support unrelated to
debugging.) In this case the above vocabulary applies as
stated. Alternatively, an implementation can do without a
descriptor, in which case the “address of the variable”,
or equivalently the “base address of the object”, refers
to the “raw data” (the real data, the only thing around
that can be the object).

If an object has a descriptor, then the DWARF type for that
object will have a \livelink{chap:DWATdatalocation}{DW\-\_AT\-\_data\-\_location} attribute. If an object
does not have a descriptor, then usually the DWARF type for the
object will not have a \livelink{chap:DWATdatalocation}{DW\-\_AT\-\_data\-\_location}. (See the following
Ada example for a case where the type for an object without
a descriptor does have a \livelink{chap:DWATdatalocation}{DW\-\_AT\-\_data\-\_location} attribute. In
that case the object doubles as its own descriptor.)

The Fortran 90 derived type array\_ptr can now be redescribed
in C\dash like terms that expose some of the representation as in

\begin{lstlisting}
struct array_ptr {
    float myvar;
    desc<1> ap;
};
\end{lstlisting}

Similarly for variable arrays:
\begin{lstlisting}
desc<1> arrays;
\end{lstlisting}

(Recall that desc \textless 1 \textgreater 
indicates the 1\dash dimensional version of desc.)

Finally, the following notation is useful:

\begin{enumerate}[1.]
\item  sizeof(type): size in bytes of entities of the given type

\item offset(type, comp): offset in bytes of the comp component
within an entity of the given type
\end{enumerate}


The DWARF description is shown in 
Section \refersec{app:fortran90exampledwarfdescription}.

\subsection{Fortran 90 example: DWARF description}
\label{app:fortran90exampledwarfdescription}

\begin{alltt}
! Description for type of 'ap'

1\$: \livelink{chap:DWTAGarraytype}{DW\-\_TAG\-\_array\-\_type}
        ! No name, default (Fortran) ordering, default stride
        \livelink{chap:DWATtype}{DW\-\_AT\-\_type}(reference to REAL)
        \livelink{chap:DWATassociated}{DW\-\_AT\-\_associated}(expression= ! Test 'ptr\_assoc' flag
            \livelink{chap:DWOPpushobjectaddress}{DW\-\_OP\-\_push\-\_object\-\_address}
            \livelink{chap:DWOPlit}{DW\-\_OP\-\_lit}<n> ! where n == offset(ptr\_assoc)
            \livelink{chap:DWOPplus}{DW\-\_OP\-\_plus}
            \livelink{chap:DWOPderef}{DW\-\_OP\-\_deref}
            \livelink{chap:DWOPlit1}{DW\-\_OP\-\_lit1} ! mask for 'ptr\_assoc' flag
            \livelink{chap:DWOPand}{DW\-\_OP\-\_and})
        \livelink{chap:DWATdatalocation}{DW\-\_AT\-\_data\-\_location}(expression= ! Get raw data address
            \livelink{chap:DWOPpushobjectaddress}{DW\-\_OP\-\_push\-\_object\-\_address}
            \livelink{chap:DWOPlit}{DW\-\_OP\-\_lit}<n> ! where n == offset(base)
            \livelink{chap:DWOPplus}{DW\-\_OP\-\_plus}
            \livelink{chap:DWOPderef}{DW\-\_OP\-\_deref}) ! Type of index of array 'ap'
2\$:     \livelink{chap:DWTAGsubrangetype}{DW\-\_TAG\-\_subrange\-\_type}
        ! No name, default stride
        \livelink{chap:DWATtype}{DW\-\_AT\-\_type}(reference to INTEGER)
        \livelink{chap:DWATlowerbound}{DW\-\_AT\-\_lower\-\_bound}(expression=
        \livelink{chap:DWOPpushobjectaddress}{DW\-\_OP\-\_push\-\_object\-\_address}
            \livelink{chap:DWOPlit}{DW\-\_OP\-\_lit}<n> ! where n ==
                         !  offset(desc, dims) +
                         !  offset(dims\_str, lower\_bound)
            \livelink{chap:DWOPplus}{DW\-\_OP\-\_plus}
            \livelink{chap:DWOPderef}{DW\-\_OP\-\_deref})
        \livelink{chap:DWATupperbound}{DW\-\_AT\-\_upper\-\_bound}(expression=
            \livelink{chap:DWOPpushobjectaddress}{DW\-\_OP\-\_push\-\_object\-\_address}
            \livelink{chap:DWOPlit}{DW\-\_OP\-\_lit}<n> ! where n ==
                         !  offset(desc, dims) +
                         !  offset(dims\_str, upper\_bound)
            \livelink{chap:DWOPplus}{DW\-\_OP\-\_plus}
            \livelink{chap:DWOPderef}{DW\-\_OP\-\_deref})
        !  Note: for the m'th dimension, the second operator becomes
        !  \livelink{chap:DWOPlit}{DW\-\_OP\-\_lit}<x> where
        !  x == offset(desc, dims) +
        !  (m-1)*sizeof(dims\_str) +
        !  offset(dims\_str, [lower|upper]\_bound)
        !  That is, the expression does not get longer for each
        !  successive dimension (other than to express the larger
        !  offsets involved).
3\$: \livelink{chap:DWTAGstructuretype}{DW\-\_TAG\-\_structure\-\_type}
        \livelink{chap:DWATname}{DW\-\_AT\-\_name}("array\_ptr")
        \livelink{chap:DWATbytesize}{DW\-\_AT\-\_byte\-\_size}(constant sizeof(REAL) + sizeof(desc<1>))
4\$:     \livelink{chap:DWTAGmember}{DW\-\_TAG\-\_member}
            \livelink{chap:DWATname}{DW\-\_AT\-\_name}("myvar")
            \livelink{chap:DWATtype}{DW\-\_AT\-\_type}(reference to REAL)
            \livelink{chap:DWATdatamemberlocation}{DW\-\_AT\-\_data\-\_member\-\_location}(constant 0)
5\$:     \livelink{chap:DWTAGmember}{DW\-\_TAG\-\_member}
            \livelink{chap:DWATname}{DW\-\_AT\-\_name}("ap");
            \livelink{chap:DWATtype}{DW\-\_AT\-\_type}(reference to 1\$)
            \livelink{chap:DWATdatamemberlocation}{DW\-\_AT\-\_data\-\_member\-\_location}(constant sizeof(REAL))

6\$: \livelink{chap:DWTAGarraytype}{DW\-\_TAG\-\_array\-\_type}
        ! No name, default (Fortran) ordering, default stride
        \livelink{chap:DWATtype}{DW\-\_AT\-\_type}(reference to 3\$)
        \livelink{chap:DWATallocated}{DW\-\_AT\-\_allocated}(expression=
             ! Test 'ptr\_alloc' flag
            \livelink{chap:DWOPpushobjectaddress}{DW\-\_OP\-\_push\-\_object\-\_address}
            \livelink{chap:DWOPlit}{DW\-\_OP\-\_lit}<n> ! where n == offset(ptr\_alloc)
            \livelink{chap:DWOPplus}{DW\-\_OP\-\_plus}
            \livelink{chap:DWOPderef}{DW\-\_OP\-\_deref}
            \livetarg{chap:DWOPlit2}{DW\-\_OP\-\_lit2}
             ! mask for 'ptr\_alloc' flag
            \livelink{chap:DWOPand}{DW\-\_OP\-\_and})
        \livelink{chap:DWATdatalocation}{DW\-\_AT\-\_data\-\_location}(expression= ! Get raw data address
            \livelink{chap:DWOPpushobjectaddress}{DW\-\_OP\-\_push\-\_object\-\_address}
            \livelink{chap:DWOPlit}{DW\-\_OP\-\_lit}<n> ! where n = offset(base)
            \livelink{chap:DWOPplus}{DW\-\_OP\-\_plus}
            \livelink{chap:DWOPderef}{DW\-\_OP\-\_deref})

7\$: \livelink{chap:DWTAGsubrangetype}{DW\-\_TAG\-\_subrange\-\_type}
        ! No name, default stride
        \livelink{chap:DWATtype}{DW\-\_AT\-\_type}(reference to INTEGER)
        \livelink{chap:DWATlowerbound}{DW\-\_AT\-\_lower\-\_bound}(expression=
            \livelink{chap:DWOPpushobjectaddress}{DW\-\_OP\-\_push\-\_object\-\_address}
            \livelink{chap:DWOPlit}{DW\-\_OP\-\_lit}<n> ! where n == ...
            \livelink{chap:DWOPplus}{DW\-\_OP\-\_plus}
            \livelink{chap:DWOPderef}{DW\-\_OP\-\_deref})
        \livelink{chap:DWATupperbound}{DW\-\_AT\-\_upper\-\_bound}(expression=
            \livelink{chap:DWOPpushobjectaddress}{DW\-\_OP\-\_push\-\_object\-\_address}
            \livelink{chap:DWOPlit}{DW\-\_OP\-\_lit}<n> ! where n == ...
            \livelink{chap:DWOPplus}{DW\-\_OP\-\_plus}
            \livelink{chap:DWOPderef}{DW\-\_OP\-\_deref})

8\$: \livelink{chap:DWTAGvariable}{DW\-\_TAG\-\_variable}
        \livelink{chap:DWATname}{DW\-\_AT\-\_name}("arrays")
        \livelink{chap:DWATtype}{DW\-\_AT\-\_type}(reference to 6\$)
        \livelink{chap:DWATlocation}{DW\-\_AT\-\_location}(expression=
            ...as appropriate...) ! Assume static allocation
\end{alltt}

\subsection{Fortran 90 example continued: DWARF description}
\label{app:fortran90examplecontinueddwarfdescription}

Suppose the program is stopped immediately following completion
of the do loop. Suppose further that the user enters the
following debug command:

\begin{lstlisting}
debug> print arrays(5)%ap(2)
\end{lstlisting}

Interpretation of this expression proceeds as follows:

\begin{enumerate}[1.]

\item Lookup name arrays. We find that it is a variable,
whose type is given by the unnamed type at 6\$. Notice that
the type is an array type.


\item Find the 5 element of that array object. To do array
indexing requires several pieces of information:

\begin{enumerate}[a]

\item  the address of the array data

\item the lower bounds of the array \\
% Using plain [] here gives trouble.
\lbrack To check that 5 is within bounds would require the upper
bound too, but we will skip that for this example. \rbrack

\item the stride 

\end{enumerate}

For a), check for a \livelink{chap:DWATdatalocation}{DW\-\_AT\-\_data\-\_location} attribute. Since
there is one, go execute the expression, whose result is
the address needed. The object address used in this case
is the object we are working on, namely the variable named
arrays , whose address was found in step 1. (Had there been
no \livelink{chap:DWATdatalocation}{DW\-\_AT\-\_data\-\_location} attribute, the desired address would
be the same as the address from step 1.)

For b), for each dimension of the array (only one
in this case), go interpret the usual lower bound
attribute. Again this is an expression, which again begins
with \livelink{chap:DWOPpushobjectaddress}{DW\-\_OP\-\_push\-\_object\-\_address}. This object is 
\textbf{still} arrays,
from step 1, because we have not begun to actually perform
any indexing yet.

For c), the default stride applies. Since there is no
\livelink{chap:DWATbytestride}{DW\-\_AT\-\_byte\-\_stride} attribute, use the size of the array element
type, which is the size of type array\_ptr (at 3\$).

Having acquired all the necessary data, perform the indexing
operation in the usual manner -  which has nothing to do with
any of the attributes involved up to now. Those just provide
the actual values used in the indexing step.

The result is an object within the memory that was dynamically
allocated for arrays.

\item  Find the ap component of the object just identified,
whose type is array\_ptr.

This is a conventional record component lookup and
interpretation. It happens that the ap component in this case
begins at offset 4 from the beginning of the containing object.
Component ap has the unnamed array type defined at 1\$ in the
symbol table.

\item  Find the second element of the array object found in step 3. To do array indexing requires
several pieces of information:

\begin{enumerate}[a]
\item  the address of the array storage

\item  the lower bounds of the array \\
% Using plain [] here gives trouble.
\lbrack To check that 2 is within bounds we would require the upper
%bound too, but we’ll skip that for this example \rbrack

\item  the stride

\end{enumerate}
\end{enumerate}

This is just like step 2), so the details are omitted. Recall
that because the DWARF type 1\$ has a \livelink{chap:DWATdatalocation}{DW\-\_AT\-\_data\-\_location},
the address that results from step 4) is that of a
descriptor, and that address is the address pushed by the
\livelink{chap:DWOPpushobjectaddress}{DW\-\_OP\-\_push\-\_object\-\_address} operations in 1\$ and 2\$.

Note: we happen to be accessing a pointer array here instead
of an allocatable array; but because there is a common
underlying representation, the mechanics are the same. There
could be completely different descriptor arrangements and the
mechanics would still be the same — only the stack machines
would be different.



\subsection{Ada Example}
\label{app:adaexample}

Figure \refersec{fig:adaexamplesourcefragment}
illustrates two kinds of Ada parameterized array, one embedded in a record.


\begin{figure}[here]
\begin{lstlisting}
M : INTEGER := <exp>;
VEC1 : array (1..M) of INTEGER;
subtype TEENY is INTEGER range 1..100;
type ARR is array (INTEGER range <>) of INTEGER;
type REC2(N : TEENY := 100) is record
    VEC2 : ARR(1..N);
end record;

OBJ2B : REC2;
\end{lstlisting}
\caption{Ada 90 example: source fragment} \label{fig:adaexamplesourcefragment}
\end{figure}

VEC1 illustrates an (unnamed) array type where the upper bound
of the first and only dimension is determined at runtime. Ada
semantics require that the value of an array bound is fixed at
the time the array type is elaborated (where elaboration refers
to the runtime executable aspects of type processing). For
the purposes of this example, we assume that there are no
other assignments to M so that it safe for the REC1 type
description to refer directly to that variable (rather than
a compiler generated copy).

REC2 illustrates another array type (the unnamed type of
component VEC2) where the upper bound of the first and only
bound is also determined at runtime. In this case, the upper
bound is contained in a discriminant of the containing record
type. (A discriminant is a component of a record whose value
cannot be changed independently of the rest of the record
because that value is potentially used in the specification
of other components of the record.)

The DWARF description is shown in 
Section \refersec{app:adaexampledwarfdescription}


Interesting aspects about this example are:

\begin{enumerate}[1)]
\item The array VEC2 is ``immediately'' contained within structure
REC2 (there is no intermediate descriptor or indirection),
which is reflected in the absence of a \livelink{chap:DWATdatalocation}{DW\-\_AT\-\_data\-\_location}
attribute on the array type at 28\$.

\item One of the bounds of VEC2 is nonetheless dynamic and part of
the same containing record. It is described as a reference to
a member, and the location of the upper bound is determined
as for any member. That is, the location is determined using
an address calculation relative to the base of the containing
object.  A consumer must notice that the referenced bound is a
member of the same containing object and implicitly push the
base address of the containing object just as for accessing
a data member generally.

\item The lack of a subtype concept in DWARF means that DWARF types
serve the role of subtypes and must replicate information from
what should be the parent type. For this reason, DWARF for
the unconstrained array ARR is not needed for the purposes
of this example and therefore not shown.
\end{enumerate}

\subsubsection{Ada example: DWARF description}
\label{app:adaexampledwarfdescription}

\begin{alltt}
11\$: \livelink{chap:DWTAGvariable}{DW\-\_TAG\-\_variable}
        \livelink{chap:DWATname}{DW\-\_AT\-\_name}("M")
        \livelink{chap:DWATtype}{DW\-\_AT\-\_type}(reference to INTEGER)
12\$: \livelink{chap:DWTAGarraytype}{DW\-\_TAG\-\_array\-\_type}
        ! No name, default (Ada) order, default stride
        \livelink{chap:DWATtype}{DW\-\_AT\-\_type}(reference to INTEGER)
13\$:    \livelink{chap:DWTAGsubrangetype}{DW\-\_TAG\-\_subrange\-\_type}
            \livelink{chap:DWATtype}{DW\-\_AT\-\_type}(reference to INTEGER)
            \livelink{chap:DWATlowerbound}{DW\-\_AT\-\_lower\-\_bound}(constant 1)
            \livelink{chap:DWATupperbound}{DW\-\_AT\-\_upper\-\_bound}(reference to variable M at 11\$)
14\$: \livelink{chap:DWTAGvariable}{DW\-\_TAG\-\_variable}
        \livelink{chap:DWATname}{DW\-\_AT\-\_name}("VEC1")
        \livelink{chap:DWATtype}{DW\-\_AT\-\_type}(reference to array type at 12\$)
    . . .
21\$: \livelink{chap:DWTAGsubrangetype}{DW\-\_TAG\-\_subrange\-\_type}
        \livelink{chap:DWATname}{DW\-\_AT\-\_name}("TEENY")
        \livelink{chap:DWATtype}{DW\-\_AT\-\_type}(reference to INTEGER)
        \livelink{chap:DWATlowerbound}{DW\-\_AT\-\_lower\-\_bound}(constant 1)
        \livelink{chap:DWATupperbound}{DW\-\_AT\-\_upper\-\_bound}(constant 100)

      . . .
26\$: \livelink{chap:DWTAGstructuretype}{DW\-\_TAG\-\_structure\-\_type}
        \livelink{chap:DWATname}{DW\-\_AT\-\_name}("REC2")
27\$:   \livelink{chap:DWTAGmember}{DW\-\_TAG\-\_member}
            \livelink{chap:DWATname}{DW\-\_AT\-\_name}("N")
            \livelink{chap:DWATtype}{DW\-\_AT\-\_type}(reference to subtype TEENY at 21\$)
            \livelink{chap:DWATdatamemberlocation}{DW\-\_AT\-\_data\-\_member\-\_location}(constant 0)
28\$:   \livelink{chap:DWTAGarraytype}{DW\-\_TAG\-\_array\-\_type}
            ! No name, default (Ada) order, default stride
            ! Default data location
            \livelink{chap:DWATtype}{DW\-\_AT\-\_type}(reference to INTEGER)
29\$:       \livelink{chap:DWTAGsubrangetype}{DW\-\_TAG\-\_subrange\-\_type}
                \livelink{chap:DWATtype}{DW\-\_AT\-\_type}(reference to subrange TEENY at 21\$)
                \livelink{chap:DWATlowerbound}{DW\-\_AT\-\_lower\-\_bound}(constant 1)
                \livelink{chap:DWATupperbound}{DW\-\_AT\-\_upper\-\_bound}(reference to member N at 27\$)
30\$:   \livelink{chap:DWTAGmember}{DW\-\_TAG\-\_member}
            \livelink{chap:DWATname}{DW\-\_AT\-\_name}("VEC2")
            \livelink{chap:DWATtype}{DW\-\_AT\-\_type}(reference to array “subtype” at 28\$)
            \livelink{chap:DWATdatamemberlocation}{DW\-\_AT\-\_data\-\_member\-\_location}(machine=
                \livelink{chap:DWOPlit}{DW\-\_OP\-\_lit}<n> ! where n == offset(REC2, VEC2)
                \livelink{chap:DWOPplus}{DW\-\_OP\-\_plus})
      . . .
41\$: \livelink{chap:DWTAGvariable}{DW\-\_TAG\-\_variable}
        \livelink{chap:DWATname}{DW\-\_AT\-\_name}("OBJ2B")
        \livelink{chap:DWATtype}{DW\-\_AT\-\_type}(reference to REC2 at 26\$)
        \livelink{chap:DWATlocation}{DW\-\_AT\-\_location}(...as appropriate...)

\end{alltt}

\subsection{Pascal Example}
\label{app:pascalexample}

The Pascal source in 
Figure \refersec{fig:packedrecordexamplesourcefragment}
is used to illustrate the representation of packed unaligned bit
fields.
\begin{figure}[here]
\begin{lstlisting}
TYPE T : PACKED RECORD ! bit size is 2
F5 : BOOLEAN; ! bit offset is 0
F6 : BOOLEAN; ! bit offset is 1
END;
VAR V : PACKED RECORD
F1 : BOOLEAN; ! bit offset is 0
F2 : PACKED RECORD ! bit offset is 1
F3 : INTEGER; ! bit offset is 0 in F2, 1 in V
END;
F4 : PACKED ARRAY [0..1] OF T; ! bit offset is 33
F7 : T; ! bit offset is 37
END;
\end{lstlisting}
\caption{Packed record example: source fragment} \label{fig:packedrecordexamplesourcefragment}
\end{figure}

The DWARF representation in 
Section \refersec{app:packedrecordexampledwarfdescription} 
is
appropriate. 
\livelink{chap:DWTAGpackedtype}{DW\-\_TAG\-\_packed\-\_type} entries could be added to
better represent the source, but these do not otherwise affect
the example and are omitted for clarity. Note that this same
representation applies to both typical big\dash \ and 
little\dash endian
architectures using the conventions described in 
Section \refersec{chap:datamemberentries}.


\subsection{Packed record example: DWARF description}
\label{app:packedrecordexampledwarfdescription}
% DWARF4 had some entries here as DW_AT_member .
% Those are fixed here to DW_TAG_member
\begin{alltt}

21\$: \livelink{chap:DWTAGstructuretype}{DW\-\_TAG\-\_structure\-\_type} ! anonymous type for F2
        \livelink{chap:DWTAGmember}{DW\-\_TAG\-\_member}
            \livelink{chap:DWATname}{DW\-\_AT\-\_name}("F3")
            \livelink{chap:DWATtype}{DW\-\_AT\-\_type}(reference to 11\$)
22\$: \livelink{chap:DWTAGarraytype}{DW\-\_TAG\-\_array\-\_type} ! anonymous type for F4
        \livelink{chap:DWATtype}{DW\-\_AT\-\_type}(reference to 20\$)
        \livelink{chap:DWTAGsubrangetype}{DW\-\_TAG\-\_subrange\-\_type}
            \livelink{chap:DWATtype}{DW\-\_AT\-\_type}(reference to 11\$)
            \livelink{chap:DWATlowerbound}{DW\-\_AT\-\_lower\-\_bound}(0)
            \livelink{chap:DWATupperbound}{DW\-\_AT\-\_upper\-\_bound}(1)
        \livelink{chap:DWATbitstride}{DW\-\_AT\-\_bit\-\_stride}(2)
        \livelink{chap:DWATbitsize}{DW\-\_AT\-\_bit\-\_size}(4)
23\$: \livelink{chap:DWTAGstructuretype}{DW\-\_TAG\-\_structure\-\_type} ! anonymous type for V
        \livelink{chap:DWATbitsize}{DW\-\_AT\-\_bit\-\_size}(39)
        \livelink{chap:DWTAGmember}{DW\-\_TAG\-\_member}
            \livelink{chap:DWATname}{DW\-\_AT\-\_name}("F1")
            \livelink{chap:DWATtype}{DW\-\_AT\-\_type}(reference to 10\$)
            \livelink{chap:DWATdatabitoffset}{DW\-\_AT\-\_data\-\_bit\-\_offset}(0)! may be omitted
            \livelink{chap:DWATbitsize}{DW\-\_AT\-\_bit\-\_size}(1) ! may be omitted
        \livelink{chap:DWTAGmember}{DW\-\_TAG\-\_member}
            \livelink{chap:DWATname}{DW\-\_AT\-\_name}("F2")
            \livelink{chap:DWATtype}{DW\-\_AT\-\_type}(reference to 21\$)
            \livelink{chap:DWATdatabitoffset}{DW\-\_AT\-\_data\-\_bit\-\_offset}(1)
            \livelink{chap:DWATbitsize}{DW\-\_AT\-\_bit\-\_size}(32) ! may be omitted
        \livelink{chap:DWTAGmember}{DW\-\_TAG\-\_member}
            \livelink{chap:DWATname}{DW\-\_AT\-\_name}("F4")
            \livelink{chap:DWATtype}{DW\-\_AT\-\_type}(reference to 22\$)
            \livelink{chap:DWATdatabitoffset}{DW\-\_AT\-\_data\-\_bit\-\_offset}(33)
            \livelink{chap:DWATbitsize}{DW\-\_AT\-\_bit\-\_size}(4) ! may be omitted
        \livelink{chap:DWTAGmember}{DW\-\_TAG\-\_member}
            \livelink{chap:DWATname}{DW\-\_AT\-\_name}("F7")
            \livelink{chap:DWATtype}{DW\-\_AT\-\_type}(reference to 20\$) ! type T
            \livelink{chap:DWATdatabitoffset}{DW\-\_AT\-\_data\-\_bit\-\_offset}(37)
            \livelink{chap:DWATbitsize}{DW\-\_AT\-\_bit\-\_size}(2) ! may be omitted
     \livelink{chap:DWTAGvariable}{DW\-\_TAG\-\_variable}
        \livelink{chap:DWATname}{DW\-\_AT\-\_name}("V")
        \livelink{chap:DWATtype}{DW\-\_AT\-\_type}(reference to 23\$)
        \livelink{chap:DWATlocation}{DW\-\_AT\-\_location}(...)
        ...
\end{alltt}

\section{Namespace Example}
\label{app:namespaceexample}


The C++ example in 
Figure \refersec{fig:namespaceexamplesourcefragment}
is used to illustrate the representation of namespaces.

\begin{figure}[here]
\begin{lstlisting}
namespace {
    int i;
}

namespace A {
    namespace B {
        int j;
        int myfunc (int a);
        float myfunc (float f) { return f – 2.0; }
        int myfunc2(int a) { return a + 2; }
    }
}
namespace Y {
    using A::B::j;         // (1) using declaration
    int foo;
}

using A::B::j;             // (2) using declaration
namespace Foo = A::B;      // (3) namespace alias
using Foo::myfunc;         // (4) using declaration
using namespace Foo;       // (5) using directive
namespace A {
    namespace B {
        using namespace Y; // (6) using directive
        int k;
    }
}
int Foo::myfunc(int a)
{
    i = 3;
    j = 4;
    return myfunc2(3) + j + i + a + 2;
}
\end{lstlisting}
\caption{Namespace example: source fragment} \label{fig:namespaceexamplesourcefragment}
\end{figure}


The DWARF representation in 
Section \refersec{app:namespaceexampledwarfdescription}
is appropriate.

\subsection{Namespace example: DWARF description}
\label{app:namespaceexampledwarfdescription}
\begin{alltt}

1\$:  \livelink{chap:DWTAGbasetype}{DW\-\_TAG\-\_base\-\_type}
        \livelink{chap:DWATname}{DW\-\_AT\-\_name}("int")
        ...
2\$:  \livelink{chap:DWTAGbasetype}{DW\-\_TAG\-\_base\-\_type}
        \livelink{chap:DWATname}{DW\-\_AT\-\_name}("float")
        ...
6\$:  \livelink{chap:DWTAGnamespace}{DW\-\_TAG\-\_namespace}
        ! no \livelink{chap:DWATname}{DW\-\_AT\-\_name} attribute
7\$:
        \livelink{chap:DWTAGvariable}{DW\-\_TAG\-\_variable}
            \livelink{chap:DWATname}{DW\-\_AT\-\_name}("i")
            \livelink{chap:DWATtype}{DW\-\_AT\-\_type}(reference to 1\$)
            \livelink{chap:DWATlocation}{DW\-\_AT\-\_location} ...
            ...
10\$: \livelink{chap:DWTAGnamespace}{DW\-\_TAG\-\_namespace}
        \livelink{chap:DWATname}{DW\-\_AT\-\_name}("A")
20\$:    \livelink{chap:DWTAGnamespace}{DW\-\_TAG\-\_namespace}
            \livelink{chap:DWATname}{DW\-\_AT\-\_name}("B")
30\$:        \livelink{chap:DWTAGvariable}{DW\-\_TAG\-\_variable}
                \livelink{chap:DWATname}{DW\-\_AT\-\_name}("j")
                \livelink{chap:DWATtype}{DW\-\_AT\-\_type}(reference to 1\$)
                \livelink{chap:DWATlocation}{DW\-\_AT\-\_location} ...
                ...
34\$:        \livelink{chap:DWTAGsubprogram}{DW\-\_TAG\-\_subprogram}
                \livelink{chap:DWATname}{DW\-\_AT\-\_name}("myfunc")
                \livelink{chap:DWATtype}{DW\-\_AT\-\_type}(reference to 1\$)
                ...
36\$:        \livelink{chap:DWTAGsubprogram}{DW\-\_TAG\-\_subprogram}
                \livelink{chap:DWATname}{DW\-\_AT\-\_name}("myfunc")
                \livelink{chap:DWATtype}{DW\-\_AT\-\_type}(reference to 2\$)
                ...
38\$:        \livelink{chap:DWTAGsubprogram}{DW\-\_TAG\-\_subprogram}
                \livelink{chap:DWATname}{DW\-\_AT\-\_name}("myfunc2")
                \livelink{chap:DWATlowpc}{DW\-\_AT\-\_low\-\_pc} ...
                \livelink{chap:DWAThighpc}{DW\-\_AT\-\_high\-\_pc} ...
                \livelink{chap:DWATtype}{DW\-\_AT\-\_type}(reference to 1\$)
                ...

40\$: \livelink{chap:DWTAGnamespace}{DW\-\_TAG\-\_namespace}
        \livelink{chap:DWATname}{DW\-\_AT\-\_name}("Y")
        \livelink{chap:DWTAGimporteddeclaration}{DW\-\_TAG\-\_imported\-\_declaration}    ! (1) using-declaration
            \livelink{chap:DWATimport}{DW\-\_AT\-\_import}(reference to 30\$)
        \livelink{chap:DWTAGvariable}{DW\-\_TAG\-\_variable}
            \livelink{chap:DWATname}{DW\-\_AT\-\_name}("foo")
            \livelink{chap:DWATtype}{DW\-\_AT\-\_type}(reference to 1\$)
            \livelink{chap:DWATlocation}{DW\-\_AT\-\_location} ...
            ...
     \livelink{chap:DWTAGimporteddeclaration}{DW\-\_TAG\-\_imported\-\_declaration}       ! (2) using declaration
        \livelink{chap:DWATimport}{DW\-\_AT\-\_import}(reference to 30\$)
        \livelink{chap:DWTAGimporteddeclaration}{DW\-\_TAG\-\_imported\-\_declaration}    ! (3) namespace alias
            \livelink{chap:DWATname}{DW\-\_AT\-\_name}("Foo")
            \livelink{chap:DWATimport}{DW\-\_AT\-\_import}(reference to 20\$)
        \livelink{chap:DWTAGimporteddeclaration}{DW\-\_TAG\-\_imported\-\_declaration}    ! (4) using declaration
            \livelink{chap:DWATimport}{DW\-\_AT\-\_import}(reference to 34\$) ! - part 1
        \livelink{chap:DWTAGimporteddeclaration}{DW\-\_TAG\-\_imported\-\_declaration}    ! (4) using declaration
            \livelink{chap:DWATimport}{DW\-\_AT\-\_import}(reference to 36\$) !  - part 2
        \livelink{chap:DWTAGimportedmodule}{DW\-\_TAG\-\_imported\-\_module}         ! (5) using directive
            \livelink{chap:DWATimport}{DW\-\_AT\-\_import}(reference to 20\$)
        \livelink{chap:DWTAGnamespace}{DW\-\_TAG\-\_namespace}
            \livelink{chap:DWATextension}{DW\-\_AT\-\_extension}(reference to 10\$)
            \livelink{chap:DWTAGnamespace}{DW\-\_TAG\-\_namespace}
                \livelink{chap:DWATextension}{DW\-\_AT\-\_extension}(reference to 20\$)
                \livelink{chap:DWTAGimportedmodule}{DW\-\_TAG\-\_imported\-\_module} ! (6) using directive
                    \livelink{chap:DWATimport}{DW\-\_AT\-\_import}(reference to 40\$)
                \livelink{chap:DWTAGvariable}{DW\-\_TAG\-\_variable}
                    \livelink{chap:DWATname}{DW\-\_AT\-\_name}("k")
                    \livelink{chap:DWATtype}{DW\-\_AT\-\_type}(reference to 1\$)
                    \livelink{chap:DWATlocation}{DW\-\_AT\-\_location} ...
                    ...
60\$: \livelink{chap:DWTAGsubprogram}{DW\-\_TAG\-\_subprogram}
        \livelink{chap:DWATspecification}{DW\-\_AT\-\_specification}(reference to 34\$)
        \livelink{chap:DWATlowpc}{DW\-\_AT\-\_low\-\_pc} ...
        \livelink{chap:DWAThighpc}{DW\-\_AT\-\_high\-\_pc} ...
        ...
\end{alltt}

\section{Member Function Example}
\label{app:memberfunctionexample}

Consider the member function example fragment in 
Figure \refersec{fig:memberfunctionexamplesourcefragment}.

\begin{figure}[here]
\begin{lstlisting}
class A
{
    void func1(int x1);
    void func2() const;
    static void func3(int x3);
};
void A::func1(int x) {}
\end{lstlisting}
\caption{Member function example: source fragment} \label{fig:memberfunctionexamplesourcefragment}
\end{figure}



The DWARF representation in 
Section \refersec{app:memberfunctionexampledwarfdescription}
is appropriate.

\subsection{Member function example: DWARF description}
\label{app:memberfunctionexampledwarfdescription}


\begin{alltt}
1\$: \livelink{chap:DWTAGunspecifiedtype}{DW\-\_TAG\-\_unspecified\-\_type}
        \livelink{chap:DWATname}{DW\-\_AT\-\_name}("void")
                ...
2\$ \livelink{chap:DWTAGbasetype}{DW\-\_TAG\-\_base\-\_type}
        \livelink{chap:DWATname}{DW\-\_AT\-\_name}("int")
        ...
3\$: \livelink{chap:DWTAGclasstype}{DW\-\_TAG\-\_class\-\_type}
        \livelink{chap:DWATname}{DW\-\_AT\-\_name}("A")
        ...
4\$:    \livelink{chap:DWTAGpointertype}{DW\-\_TAG\-\_pointer\-\_type}
            \livelink{chap:DWATtype}{DW\-\_AT\-\_type}(reference to 3\$)
            ...
5\$:    \livelink{chap:DWTAGconsttype}{DW\-\_TAG\-\_const\-\_type}
            \livelink{chap:DWATtype}{DW\-\_AT\-\_type}(reference to 3\$)
            ...
6\$:    \livelink{chap:DWTAGpointertype}{DW\-\_TAG\-\_pointer\-\_type}
            \livelink{chap:DWATtype}{DW\-\_AT\-\_type}(reference to 5\$)
            ...

7\$:    \livelink{chap:DWTAGsubprogram}{DW\-\_TAG\-\_subprogram}
            \livelink{chap:DWATdeclaration}{DW\-\_AT\-\_declaration}
            \livelink{chap:DWATname}{DW\-\_AT\-\_name}("func1")
            \livelink{chap:DWATtype}{DW\-\_AT\-\_type}(reference to 1\$)
            \livelink{chap:DWATobjectpointer}{DW\-\_AT\-\_object\-\_pointer}(reference to 8\$)
                ! References a formal parameter in this 
                ! member function
            ...
8\$:        \livelink{chap:DWTAGformalparameter}{DW\-\_TAG\-\_formal\-\_parameter}
                \livelink{chap:DWATartificial}{DW\-\_AT\-\_artificial}(true)
                \livelink{chap:DWATname}{DW\-\_AT\-\_name}("this")
                \livelink{chap:DWATtype}{DW\-\_AT\-\_type}(reference to 4\$)
                    ! Makes type of 'this' as 'A*' =>
                    ! func1 has not been marked const 
                    ! or volatile
                \livelink{chap:DWATlocation}{DW\-\_AT\-\_location} ...
                ...
9\$:        \livelink{chap:DWTAGformalparameter}{DW\-\_TAG\-\_formal\-\_parameter}
                \livelink{chap:DWATname}{DW\-\_AT\-\_name}(x1)
                \livelink{chap:DWATtype}{DW\-\_AT\-\_type}(reference to 2\$)
                ...
10\$:    \livelink{chap:DWTAGsubprogram}{DW\-\_TAG\-\_subprogram}
             \livelink{chap:DWATdeclaration}{DW\-\_AT\-\_declaration}
             \livelink{chap:DWATname}{DW\-\_AT\-\_name}("func2")
             \livelink{chap:DWATtype}{DW\-\_AT\-\_type}(reference to 1\$)
             \livelink{chap:DWATobjectpointer}{DW\-\_AT\-\_object\-\_pointer}(reference to 11\$)
             ! References a formal parameter in this 
             ! member function
             ...
11\$:        \livelink{chap:DWTAGformalparameter}{DW\-\_TAG\-\_formal\-\_parameter}
                 \livelink{chap:DWATartificial}{DW\-\_AT\-\_artificial}(true)
                 \livelink{chap:DWATname}{DW\-\_AT\-\_name}("this")
                 \livelink{chap:DWATtype}{DW\-\_AT\-\_type}(reference to 6\$)
                     ! Makes type of 'this' as 'A const*' =>
                 !     func2 marked as const
                 \livelink{chap:DWATlocation}{DW\-\_AT\-\_location} ...
                 ...
12\$:    \livelink{chap:DWTAGsubprogram}{DW\-\_TAG\-\_subprogram}
             \livelink{chap:DWATdeclaration}{DW\-\_AT\-\_declaration}
             \livelink{chap:DWATname}{DW\-\_AT\-\_name}("func3")
             \livelink{chap:DWATtype}{DW\-\_AT\-\_type}(reference to 1\$)
             ...
                 ! No object pointer reference formal parameter
                 ! implies func3 is static
13\$:        \livelink{chap:DWTAGformalparameter}{DW\-\_TAG\-\_formal\-\_parameter}
                 \livelink{chap:DWATname}{DW\-\_AT\-\_name}(x3)
                 \livelink{chap:DWATtype}{DW\-\_AT\-\_type}(reference to 2\$)
                 ...
\end{alltt}

\section{Line Number Program Example}
\label{app:linenumberprogramexample}

Consider the simple source file and the resulting machine
code for the Intel 8086 processor in 
Figure \refersec{fig:linenumberprogramexamplemachinecode}.

\begin{figure}[here]
\begin{lstlisting}
1: int
2: main()
    0x239: push pb
    0x23a: mov bp,sp
3: {
4: printf("Omit needless words\n");
    0x23c: mov ax,0xaa
    0x23f: push ax
    0x240: call _printf
    0x243: pop cx
5: exit(0);
    0x244: xor ax,ax
    0x246: push ax
    0x247: call _exit
    0x24a: pop cx
6: }
    0x24b: pop bp
    0x24c: ret
7: 0x24d:
\end{lstlisting}
\caption{Line number program example: machine code} \label{fig:linenumberprogramexamplemachinecode}
\end{figure}

Suppose the line number program header includes the following
(header fields not needed below are not shown):



\begin{alltt}
version                    4
minimum_instruction_length 1
opcode_base               10 ! Opcodes 10-12 not needed
line_base                  1
line_range                15
\end{alltt}


Figure \refersec{tab:linenumberprogramexampleoneencoding}
shows one encoding of the line number program, which occupies
12 bytes (the opcode SPECIAL(m,n) indicates the special opcode
generated for a line increment of m and an address increment
of n).


\begin{centering}
\setlength{\extrarowheight}{0.1cm}
\begin{longtable}{lll}
  \caption{Line number program example: one encoding} \label{tab:linenumberprogramexampleoneencoding} \\
  \hline \\ \bfseries Opcode &\bfseries Operand &\bfseries Byte Stream \\ \hline
\endfirsthead
  \bfseries Opcode &\bfseries Operand &\bfseries Byte Stream\\ \hline
\endhead
  \hline \emph{Continued on next page}
\endfoot
  \hline
\endlastfoot
DW\-\_LNS\-\_advance\-\_pc&LEB128(0x239)&0x2, 0xb9, 0x04 \\
SPECIAL(2, 0)&0xb  & \\
SPECIAL(2, 3)&0x38 & \\
SPECIAL(1, 8)&0x82 & \\
SPECIAL(1, 7)&0x73 & \\
DW\-\_LNS\-\_advance\-\_pc&LEB128(2)&0x2, 0x2 \\
DW\-\_LNE\-\_end\-\_sequence &&0x0, 0x1, 0x1 \\
\end{longtable}
\end{centering}


Table \refersec{tab:linenumberprogramexamplealternateencoding}
shows an alternate 
encoding of the same program using 
standard opcodes to advance
the program counter; 
this encoding occupies 22 bytes.

\begin{centering}
\setlength{\extrarowheight}{0.1cm}
\begin{longtable}{lll}
  \caption{Line number program example: alternate encoding} \label{tab:linenumberprogramexamplealternateencoding} \\
  \hline \\ \bfseries Opcode &\bfseries Operand &\bfseries Byte Stream \\ \hline
\endfirsthead
  \bfseries Opcode &\bfseries Operand &\bfseries Byte Stream\\ \hline
\endhead
  \hline \emph{Continued on next page}
\endfoot
  \hline
\endlastfoot
DW\-\_LNS\-\_fixed\-\_advance\-\_pc&0x239&0x9, 0x39, 0x2        \\
SPECIAL(2, 0)&& 0xb        \\
DW\-\_LNS\-\_fixed\-\_advance\-\_pc&0x3&0x9, 0x3, 0x0        \\
SPECIAL(2, 0)&&0xb        \\
DW\-\_LNS\-\_fixed\-\_advance\-\_pc&0x8&0x9, 0x8, 0x0        \\
SPECIAL(1, 0)&& 0xa        \\
DW\-\_LNS\-\_fixed\-\_advance\-\_pc&0x7&0x9, 0x7, 0x0        \\
SPECIAL(1, 0) && 0xa        \\
DW\-\_LNS\-\_fixed\-\_advance\-\_pc&0x2&0x9, 0x2, 0x0        \\
DW\-\_LNE\-\_end\-\_sequence&&0x0, 0x1, 0x1        \\
\end{longtable}
\end{centering}


\section{Call Frame Information Example}
\label{app:callframeinformationexample}

The following example uses a hypothetical RISC machine in
the style of the Motorola 88000.

\begin{itemize}
\item Memory is byte addressed.

\item Instructions are all 4 bytes each and word aligned.

\item Instruction operands are typically of the form:

\begin{alltt}
<destination.reg>, <source.reg>, <constant>
\end{alltt}

\item The address for the load and store instructions is computed
by adding the contents of the
source register with the constant.

\item There are 8 4\dash byte registers:

\begin{alltt}
R0 always 0
R1 holds return address on call
R2-R3 temp registers (not preserved on call)
R4-R6 preserved on call
R7 stack pointer.
\end{alltt}

\item  The stack grows in the negative direction.

\item The architectural ABI committee specifies that the
stack pointer (R7) is the same as the CFA

\end{itemize}

The following 
(Figure refersec{fig:callframeinformationexamplemachinecodefragments}
are two code fragments from a subroutine called
foo that uses a frame pointer (in addition to the stack
pointer). The first column values are byte addresses. 
% The \space is so we get a space after >
\textless~fs~\textgreater \  denotes the stack frame size in bytes, namely 12.


\begin{figure}[here]
\begin{lstlisting}
       ;; start prologue
foo    sub R7, R7, <fs>        ; Allocate frame
foo+4  store R1, R7, (<fs>-4)  ; Save the return address
foo+8  store R6, R7, (<fs>-8)  ; Save R6
foo+12 add R6, R7, 0           ; R6 is now the Frame ptr
foo+16 store R4, R6, (<fs>-12) ; Save a preserved reg
       ;; This subroutine does not change R5
       ...
       ;; Start epilogue (R7 is returned to entry value)
foo+64 load R4, R6, (<fs>-12)  ; Restore R4
foo+68 load R6, R7, (<fs>-8)   ; Restore R6
foo+72 load R1, R7, (<fs>-4)   ; Restore return address
foo+76 add R7, R7, <fs>        ; Deallocate frame
foo+80 jump R1 ; Return
foo+84
\end{lstlisting}
\caption{Call frame information example: machine code fragments} \label{fig:callframeinformationexamplemachinecodefragments}
\end{figure}


An abstract table 
(see Section \refersec{chap:structureofcallframeinformation}) 
for the foo subroutine
is shown in 
Table \refersec{tab:callframeinformationexampleconceptualmatrix}.
Corresponding fragments from the
.debug\_frame section are shown in 
Table \refersec{tab:callframeinformationexamplecommoninformationentryencoding}.

The following notations apply in 
Table \refersec{tab:callframeinformationexampleconceptualmatrix}:

\begin{alltt}
1. R8 is the return address
2. s = same\_value rule
3. u = undefined rule
4. rN = register(N) rule
5. cN = offset(N) rule
6. a = architectural rule
\end{alltt}

\begin{centering}
\setlength{\extrarowheight}{0.1cm}
\begin{longtable}{lllllllllll}
  \caption{Call frame inforation example: conceptual matrix} \label{tab:callframeinformationexampleconceptualmatrix} \\
  \hline \\ \bfseries Location & \bfseries CFA & \bfseries R0 & \bfseries R1 & \bfseries R2 & \bfseries R3 & \bfseries R4 & \bfseries R5 & \bfseries R6 & \bfseries R7 & \bfseries R8 \\ \hline
\endfirsthead
  \bfseries Location &\bfseries CFA &\bfseries R0 & \bfseries R1 & \bfseries R2 &\bfseries R3 &\bfseries R4 &\bfseries R5 &\bfseries R6 &\bfseries R7 &\bfseries R8\\ \hline
\endhead
  \hline \emph{Continued on next page}
\endfoot
  \hline
\endlastfoot
foo&[R7]+0&s&u&u&u&s&s&s&a&r1 \\
foo+4&[R7]+fs&s&u&u&u&s&s&s&a&r1 \\
foo+8&[R7]+fs&s&u&u&u&s&s&s&a&c-4 \\
foo+12&[R7]+fs&s&u&u&u&s&s&c-8&a&c-4 \\
foo+16&[R6]+fs&s&u&u&u&s&s&c-8&a&c-4 \\
foo+20&[R6]+fs&s&u&u&u&c-12&s&c-8&a&c-4 \\
...&&&&&&&&&& \\
foo+64&[R6]+fs&s&u&u&u&c-12&s&c-8&a&c-4 \\
foo+68&[R6]+fs&s&u&u&u&s&s&c-8&a&c-4  \\
foo+72&[R7]+fs&s&u&u&u&s&s&s&a&c-4  \\
foo+76&[R7]+fs&s&u&u&u&s&s&s&a&r1 \\
foo+80&[R7]+0&s&u&u&u&s&s&s&a&r1 \\
\end{longtable}
\end{centering}


\begin{centering}
\setlength{\extrarowheight}{0.1cm}
\begin{longtable}{lll}
  \caption{Call frame information example: common information entry encoding} \label{tab:callframeinformationexamplecommoninformationentryencoding} \\
  \hline \\ \bfseries Address &\bfseries Value &\bfseries Comment \\ \hline
\endfirsthead
  \bfseries Address &\bfseries Value &\bfseries Comment \\ \hline
\endhead
  \hline \emph{Continued on next page}
\endfoot
  \hline
\endlastfoot
cie&36&length    \\
cie+4&0xffffffff&CIE\_id    \\
cie+8&4&version    \\
cie+9&0&augmentation     \\
cie+10&4&address size    \\
cie+11&0&segment size    \\
cie+12&4&code\_alignment\_factor, \textless caf \textgreater    \\
cie+13&-4&data\_alignment\_factor, \textless daf \textgreater    \\
cie+14&8&R8 is the return addr.    \\
cie+15&\livelink{chap:DWCFAdefcfa}{DW\-\_CFA\-\_def\-\_cfa} (7, 0)&CFA = [R7]+0    \\
cie+18&\livelink{chap:DWCFAsamevalue}{DW\-\_CFA\-\_same\-\_value} (0)&R0 not modified (=0)    \\
cie+20&\livelink{chap:DWCFAundefined}{DW\-\_CFA\-\_undefined} (1)&R1 scratch    \\
cie+22&\livelink{chap:DWCFAundefined}{DW\-\_CFA\-\_undefined} (2)&R2 scratch    \\
cie+24&\livelink{chap:DWCFAundefined}{DW\-\_CFA\-\_undefined} (3)&R3 scratch    \\
cie+26&\livelink{chap:DWCFAsamevalue}{DW\-\_CFA\-\_same\-\_value} (4)&R4 preserve    \\
cie+28&\livelink{chap:DWCFAsamevalue}{DW\-\_CFA\-\_same\-\_value} (5)&R5 preserve    \\
cie+30&\livelink{chap:DWCFAsamevalue}{DW\-\_CFA\-\_same\-\_value} (6)&R6 preserve    \\
cie+32&\livelink{chap:DWCFAsamevalue}{DW\-\_CFA\-\_same\-\_value} (7)&R7 preserve    \\
cie+34&\livelink{chap:DWCFAregister}{DW\-\_CFA\-\_register} (8, 1)&R8 is in R1    \\
cie+37&\livelink{chap:DWCFAnop}{DW\-\_CFA\-\_nop}&padding    \\
cie+38&\livelink{chap:DWCFAnop}{DW\-\_CFA\-\_nop}& padding \\
cie+39& \livelink{chap:DWCFAnop}{DW\-\_CFA\-\_nop}&padding  \\
cie+40 &&  \\
\end{longtable}
\end{centering}


The following notations apply in 
Table \refersec{tab:callframeinformationexampleframedescriptionentryencoding}:

\begin{alltt}
1. <fs> = frame size
2. <caf> = code alignment factor
3. <daf> = data alignment factor
\end{alltt}


\begin{centering}
\setlength{\extrarowheight}{0.1cm}
\begin{longtable}{lll}
  \caption{Call frame information example: frame description entry encoding} \label{tab:callframeinformationexampleframedescriptionentryencoding} \\
  \hline \\ \bfseries Address &\bfseries Value &\bfseries Comment \\ \hline
\endfirsthead
  \bfseries Address &\bfseries Value &\bfseries Comment \\ \hline
\endhead
  \hline \emph{Continued on next page}
\endfoot
  \hline
\endlastfoot
fde&40&length \\
fde+4&cie&CIE\_ptr \\
fde+8&foo&initial\_location \\
fde+12&84&address\_range \\
fde+16&\livelink{chap:DWCFAadvanceloc}{DW\-\_CFA\-\_advance\-\_loc}(1)&instructions \\
fde+17&\livelink{chap:DWCFAdefcfaoffset}{DW\-\_CFA\-\_def\-\_cfa\-\_offset}(12)& \textless fs \textgreater \\
fde+19&\livelink{chap:DWCFAadvanceloc}{DW\-\_CFA\-\_advance\-\_loc}(1)&4/ \textless caf \textgreater \\
fde+20&\livelink{chap:DWCFAoffset}{DW\-\_CFA\-\_offset}(8,1)&-4/ \textless daf \textgreater (second parameter) \\
fde+22&\livelink{chap:DWCFAadvanceloc}{DW\-\_CFA\-\_advance\-\_loc}(1)& \\
fde+23&\livelink{chap:DWCFAoffset}{DW\-\_CFA\-\_offset}(6,2)&-8/ \textless daf> \textgreater (2nd parameter)  \\
fde+25&\livelink{chap:DWCFAadvanceloc}{DW\-\_CFA\-\_advance\-\_loc}(1) & \\
fde+26&\livelink{chap:DWCFAdefcfaregister}{DW\-\_CFA\-\_def\-\_cfa\-\_register}(6) & \\
fde+28&\livelink{chap:DWCFAadvanceloc}{DW\-\_CFA\-\_advance\-\_loc}(1) & \\
fde+29&\livelink{chap:DWCFAoffset}{DW\-\_CFA\-\_offset}(4,3)&-12/ \textless daf \textgreater (2nd parameter) \\
fde+31&\livelink{chap:DWCFAadvanceloc}{DW\-\_CFA\-\_advance\-\_loc}(12)&44/ \textless caf \textgreater \\
fde+32&\livelink{chap:DWCFArestore}{DW\-\_CFA\-\_restore}(4)& \\
fde+33&\livelink{chap:DWCFAadvanceloc}{DW\-\_CFA\-\_advance\-\_loc}(1) & \\
fde+34&\livelink{chap:DWCFArestore}{DW\-\_CFA\-\_restore}(6) & \\
fde+35&\livelink{chap:DWCFAdefcfaregister}{DW\-\_CFA\-\_def\-\_cfa\-\_register}(7)  & \\
fde+37&\livelink{chap:DWCFAadvanceloc}{DW\-\_CFA\-\_advance\-\_loc}(1) & \\
fde+38&\livelink{chap:DWCFArestore}{DW\-\_CFA\-\_restore}(8) &\\
fde+39&\livelink{chap:DWCFAadvanceloc}{DW\-\_CFA\-\_advance\-\_loc}(1) &\\
fde+40&\livelink{chap:DWCFAdefcfaoffset}{DW\-\_CFA\-\_def\-\_cfa\-\_offset}(0)  &\\
fde+42&\livelink{chap:DWCFAnop}{DW\-\_CFA\-\_nop}&padding \\
fde+43&\livelink{chap:DWCFAnop}{DW\-\_CFA\-\_nop}&padding \\
fde+44 && \\
\end{longtable}
\end{centering}

\section{Inlining Examples}
\label{app:inliningexamples}
The pseudo\dash source in 
Figure \refersec{fig:inliningexamplespseudosourcefragment}
is used to illustrate the
use of DWARF to describe inlined subroutine calls. This
example involves a nested subprogram INNER that makes uplevel
references to the formal parameter and local variable of the
containing subprogram OUTER.

\begin{figure}[here]
\begin{lstlisting}
inline procedure OUTER (OUTER_FORMAL : integer) =
    begin
    OUTER_LOCAL : integer;
    procedure INNER (INNER_FORMAL : integer) =
        begin
        INNER_LOCAL : integer;
        print(INNER_FORMAL + OUTER_LOCAL);
        end;
    INNER(OUTER_LOCAL);
    ...
    INNER(31);
    end;
! Call OUTER
!
OUTER(7);
\end{lstlisting}
\caption{Inlining examples: pseudo-source fragmment} \label{fig:inliningexamplespseudosourcefragment}
\end{figure}


There are several approaches that a compiler might take to
inlining for this sort of example. This presentation considers
three such approaches, all of which involve inline expansion
of subprogram OUTER. (If OUTER is not inlined, the inlining
reduces to a simpler single level subset of the two level
approaches considered here.)

The approaches are:

\begin{itemize}[1.]
\item  Inline both OUTER and INNER in all cases

\item Inline OUTER, multiple INNERs \\
Treat INNER as a non\dash inlinable part of OUTER, compile and
call a distinct normal version of INNER defined within each
inlining of OUTER.

\item Inline OUTER, one INNER \\
Compile INNER as a single normal subprogram which is called
from every inlining of OUTER.
\end{itemize}

This discussion does not consider why a compiler might choose
one of these approaches; it considers only how to describe
the result.

In the examples that follow in this section, the debugging
information entries are given mnemonic labels of the following
form

\begin{lstlisting}
<io>.<ac>.<n>.<s>
\end{lstlisting}

where \begin{verbatim}<io>\end{verbatim}
is either INNER or OUTER to indicate to which
subprogram the debugging information entry applies, 
\begin{verbatim}<ac>\end{verbatim}
is either AI or CI to indicate ``abstract instance'' or
``concrete instance'' respectively, 
\begin{verbatim}<n>\end{verbatim}
is the number of the
alternative being considered, and 
\begin{verbatim}<s>\end{verbatim}
is a sequence number that
distinguishes the individual entries. There is no implication
that symbolic labels, nor any particular naming convention,
are required in actual use.

For conciseness, declaration coordinates and call coordinates are omitted.

\subsection{Alternative 1: inline both OUTER and INNER}
\label{app:inlinebothouterandinner}

A suitable abstract instance for an alternative where both
OUTER and INNER are always inlined is shown in 
Figure \refersec{app:inliningexample1abstractinstance}

Notice in 
Section \refersec{app:inliningexample1abstractinstance} 
that the debugging information entry for
INNER (labelled INNER.AI.1.1) is nested in (is a child of)
that for OUTER (labelled OUTER.AI.1.1). Nonetheless, the
abstract instance tree for INNER is considered to be separate
and distinct from that for OUTER.

The call of OUTER shown in 
Figure \refersec{fig:inliningexamplespseudosourcefragment}
might be described as
shown in 
Section \refersec{app:inliningexample1concreteinstance}.


\subsubsection{Inlining example \#1: abstract instance}
\label{app:inliningexample1abstractinstance}
\begin{alltt}
    ! Abstract instance for OUTER
    !
OUTER.AI.1.1:
    \livelink{chap:DWTAGsubprogram}{DW\-\_TAG\-\_subprogram}
        \livelink{chap:DWATname}{DW\-\_AT\-\_name}("OUTER")
        \livelink{chap:DWATinline}{DW\-\_AT\-\_inline}(DW\-\_INL\-\_declared\-\_inlined)
        ! No low/high PCs
OUTER.AI.1.2:
        \livelink{chap:DWTAGformalparameter}{DW\-\_TAG\-\_formal\-\_parameter}
            \livelink{chap:DWATname}{DW\-\_AT\-\_name}("OUTER\_FORMAL")
            \livelink{chap:DWATtype}{DW\-\_AT\-\_type}(reference to integer)
            ! No location
OUTER.AI.1.3:
        \livelink{chap:DWTAGvariable}{DW\-\_TAG\-\_variable}
            \livelink{chap:DWATname}{DW\-\_AT\-\_name}("OUTER\_LOCAL")
            \livelink{chap:DWATtype}{DW\-\_AT\-\_type}(reference to integer)
            ! No location
        !
        ! Abstract instance for INNER
        !
INNER.AI.1.1:
        \livelink{chap:DWTAGsubprogram}{DW\-\_TAG\-\_subprogram}
            \livelink{chap:DWATname}{DW\-\_AT\-\_name}("INNER")
            \livelink{chap:DWATinline}{DW\-\_AT\-\_inline}(DW\-\_INL\-\_declared\-\_inlined)
            ! No low/high PCs
INNER.AI.1.2:
            \livelink{chap:DWTAGformalparameter}{DW\-\_TAG\-\_formal\-\_parameter}
                \livelink{chap:DWATname}{DW\-\_AT\-\_name}("INNER\_FORMAL")
                \livelink{chap:DWATtype}{DW\-\_AT\-\_type}(reference to integer)
                ! No location
INNER.AI.1.3:
            \livelink{chap:DWTAGvariable}{DW\-\_TAG\-\_variable}
                \livelink{chap:DWATname}{DW\-\_AT\-\_name}("INNER\_LOCAL")
                \livelink{chap:DWATtype}{DW\-\_AT\-\_type}(reference to integer)
                ! No location
            ...
            0
        ! No \livelink{chap:DWTAGinlinedsubroutine}{DW\-\_TAG\-\_inlined\-\_subroutine} (concrete instance)
        ! for INNER corresponding to calls of INNER
        ...
        0
\end{alltt}


\subsubsection{Inlining example \#1: concrete instance}
\label{app:inliningexample1concreteinstance}
\begin{alltt}
! Concrete instance for call "OUTER(7)"
!
OUTER.CI.1.1:
    \livelink{chap:DWTAGinlinedsubroutine}{DW\-\_TAG\-\_inlined\-\_subroutine}
        ! No name
        \livelink{chap:DWATabstractorigin}{DW\-\_AT\-\_abstract\-\_origin}(reference to OUTER.AI.1.1)
        \livelink{chap:DWATlowpc}{DW\-\_AT\-\_low\-\_pc}(...)
        \livelink{chap:DWAThighpc}{DW\-\_AT\-\_high\-\_pc}(...)
OUTER.CI.1.2:
        \livelink{chap:DWTAGformalparameter}{DW\-\_TAG\-\_formal\-\_parameter}
            ! No name
            \livelink{chap:DWATabstractorigin}{DW\-\_AT\-\_abstract\-\_origin}(reference to OUTER.AI.1.2)
            \livelink{chap:DWATconstvalue}{DW\-\_AT\-\_const\-\_value}(7)
OUTER.CI.1.3:
        \livelink{chap:DWTAGvariable}{DW\-\_TAG\-\_variable}
            ! No name
            \livelink{chap:DWATabstractorigin}{DW\-\_AT\-\_abstract\-\_origin}(reference to OUTER.AI.1.3)
            \livelink{chap:DWATlocation}{DW\-\_AT\-\_location}(...)
        !
        ! No \livelink{chap:DWTAGsubprogram}{DW\-\_TAG\-\_subprogram} (abstract instance) for INNER
        !
        ! Concrete instance for call INNER(OUTER\_LOCAL)
        !
INNER.CI.1.1:
        \livelink{chap:DWTAGinlinedsubroutine}{DW\-\_TAG\-\_inlined\-\_subroutine}
            ! No name
            \livelink{chap:DWATabstractorigin}{DW\-\_AT\-\_abstract\-\_origin}(reference to INNER.AI.1.1)
            \livelink{chap:DWATlowpc}{DW\-\_AT\-\_low\-\_pc}(...)
            \livelink{chap:DWAThighpc}{DW\-\_AT\-\_high\-\_pc}(...)
            \livelink{chap:DWATstaticlink}{DW\-\_AT\-\_static\-\_link}(...)
INNER.CI.1.2:
            \livelink{chap:DWTAGformalparameter}{DW\-\_TAG\-\_formal\-\_parameter}
                ! No name
                \livelink{chap:DWATabstractorigin}{DW\-\_AT\-\_abstract\-\_origin}(reference to INNER.AI.1.2)
                \livelink{chap:DWATlocation}{DW\-\_AT\-\_location}(...)
INNER.CI.1.3:
             \livelink{chap:DWTAGvariable}{DW\-\_TAG\-\_variable}
                ! No name
                \livelink{chap:DWATabstractorigin}{DW\-\_AT\-\_abstract\-\_origin}(reference to INNER.AI.1.3)
                \livelink{chap:DWATlocation}{DW\-\_AT\-\_location}(...)
            ...
            0
        ! Another concrete instance of INNER within OUTER
        ! for the call "INNER(31)"
        ...
        0
\end{alltt}

\subsection{Alternative 2: Inline OUTER, multiple INNERs}
\label{app:inlineoutermultiipleinners}


In the second alternative we assume that subprogram INNER
is not inlinable for some reason, but subprogram OUTER is
inlinable. Each concrete inlined instance of OUTER has its
own normal instance of INNER. The abstract instance for OUTER,
which includes INNER, is shown in 
Section \refersec{app:inliningexample2abstractinstance}.

Note that the debugging information in this Figure differs from
that in 
Section \refersec{app:inliningexample1abstractinstance}
in that INNER lacks a \livelink{chap:DWATinline}{DW\-\_AT\-\_inline} attribute
and therefore is not a distinct abstract instance. INNER
is merely an out\dash of\dash line routine that is part of OUTER’s
abstract instance. This is reflected in the Figure 70 by
the fact that the labels for INNER use the substring OUTER
instead of INNER.

A resulting concrete inlined instance of OUTER is shown in
Section \refersec{app:inliningexample2concreteinstance}.

Notice in 
Section \refersec{app:inliningexample2concreteinstance}.
that OUTER is expanded as a concrete
inlined instance, and that INNER is nested within it as a
concrete out\dash of\dash line subprogram. Because INNER is cloned
for each inline expansion of OUTER, only the invariant
attributes of INNER 
(for example, \livelink{chap:DWATname}{DW\-\_AT\-\_name}) are specified
in the abstract instance of OUTER, and the low\dash level,
instance\dash specific attributes of INNER (for example,
\livelink{chap:DWATlowpc}{DW\-\_AT\-\_low\-\_pc}) are specified in each concrete instance of OUTER.

The several calls of INNER within OUTER are compiled as normal
calls to the instance of INNER that is specific to the same
instance of OUTER that contains the calls.


\subsubsection{Inlining example 2: abstract instance}
\label{app:inliningexample2abstractinstance}
\begin{alltt}
    ! Abstract instance for OUTER
    !
OUTER.AI.2.1:
    \livelink{chap:DWTAGsubprogram}{DW\-\_TAG\-\_subprogram}
        \livelink{chap:DWATname}{DW\-\_AT\-\_name}("OUTER")
        \livelink{chap:DWATinline}{DW\-\_AT\-\_inline}(DW\-\_INL\-\_declared\-\_inlined)
        ! No low/high PCs
OUTER.AI.2.2:
        \livelink{chap:DWTAGformalparameter}{DW\-\_TAG\-\_formal\-\_parameter}
            \livelink{chap:DWATname}{DW\-\_AT\-\_name}("OUTER\_FORMAL")
            \livelink{chap:DWATtype}{DW\-\_AT\-\_type}(reference to integer)
            ! No location
OUTER.AI.2.3:
        \livelink{chap:DWTAGvariable}{DW\-\_TAG\-\_variable}
            \livelink{chap:DWATname}{DW\-\_AT\-\_name}("OUTER\_LOCAL")
            \livelink{chap:DWATtype}{DW\-\_AT\-\_type}(reference to integer)
            ! No location
        !
        ! Nested out-of-line INNER subprogram
        !
OUTER.AI.2.4:
        \livelink{chap:DWTAGsubprogram}{DW\-\_TAG\-\_subprogram}
            \livelink{chap:DWATname}{DW\-\_AT\-\_name}("INNER")
            ! No \livelink{chap:DWATinline}{DW\-\_AT\-\_inline}
            ! No low/high PCs, frame\_base, etc.
OUTER.AI.2.5:
            \livelink{chap:DWTAGformalparameter}{DW\-\_TAG\-\_formal\-\_parameter}
                \livelink{chap:DWATname}{DW\-\_AT\-\_name}("INNER\_FORMAL")
                \livelink{chap:DWATtype}{DW\-\_AT\-\_type}(reference to integer)
                ! No location
OUTER.AI.2.6:
            \livelink{chap:DWTAGvariable}{DW\-\_TAG\-\_variable}
                \livelink{chap:DWATname}{DW\-\_AT\-\_name}("INNER\_LOCAL")
                \livelink{chap:DWATtype}{DW\-\_AT\-\_type}(reference to integer)
                ! No location
            ...
            0
        ...
        0
\end{alltt}

\subsubsection{Inlining example 2: concrete instance}
\label{app:inliningexample2concreteinstance}
\begin{alltt}

    ! Concrete instance for call "OUTER(7)"
    !
OUTER.CI.2.1:
    \livelink{chap:DWTAGinlinedsubroutine}{DW\-\_TAG\-\_inlined\-\_subroutine}
        ! No name
        \livelink{chap:DWATabstractorigin}{DW\-\_AT\-\_abstract\-\_origin}(reference to OUTER.AI.2.1)
        \livelink{chap:DWATlowpc}{DW\-\_AT\-\_low\-\_pc}(...)
        \livelink{chap:DWAThighpc}{DW\-\_AT\-\_high\-\_pc}(...)
OUTER.CI.2.2:
        \livelink{chap:DWTAGformalparameter}{DW\-\_TAG\-\_formal\-\_parameter}
            ! No name
            \livelink{chap:DWATabstractorigin}{DW\-\_AT\-\_abstract\-\_origin}(reference to OUTER.AI.2.2)
            \livelink{chap:DWATlocation}{DW\-\_AT\-\_location}(...)
OUTER.CI.2.3:
        \livelink{chap:DWTAGvariable}{DW\-\_TAG\-\_variable}
            ! No name
            \livelink{chap:DWATabstractorigin}{DW\-\_AT\-\_abstract\-\_origin}(reference to OUTER.AI.2.3)
            \livelink{chap:DWATlocation}{DW\-\_AT\-\_location}(...)
        !
        ! Nested out-of-line INNER subprogram
        !
OUTER.CI.2.4:
        \livelink{chap:DWTAGsubprogram}{DW\-\_TAG\-\_subprogram}
            ! No name
            \livelink{chap:DWATabstractorigin}{DW\-\_AT\-\_abstract\-\_origin}(reference to OUTER.AI.2.4)
            \livelink{chap:DWATlowpc}{DW\-\_AT\-\_low\-\_pc}(...)
            \livelink{chap:DWAThighpc}{DW\-\_AT\-\_high\-\_pc}(...)
            \livelink{chap:DWATframebase}{DW\-\_AT\-\_frame\-\_base}(...)
            \livelink{chap:DWATstaticlink}{DW\-\_AT\-\_static\-\_link}(...)
OUTER.CI.2.5:
            \livelink{chap:DWTAGformalparameter}{DW\-\_TAG\-\_formal\-\_parameter}
                ! No name
                \livelink{chap:DWATabstractorigin}{DW\-\_AT\-\_abstract\-\_origin}(reference to OUTER.AI.2.5)
                \livelink{chap:DWATlocation}{DW\-\_AT\-\_location}(...)
OUTER.CI.2.6:
            \livelink{chap:DWTAGvariable}{DW\-\_TAG\-\_variable}
                ! No name
                \livelink{chap:DWATabstractorigin}{DW\-\_AT\-\_abstract\-\_origin}(reference to OUTER.AT.2.6)
                \livelink{chap:DWATlocation}{DW\-\_AT\-\_location}(...)
            ...
            0
        ...
        0
\end{alltt}

\subsection{Alternative 3: inline OUTER, one normal INNER}
\label{app:inlineouteronenormalinner}

In the third approach, one normal subprogram for INNER is
compiled which is called from all concrete inlined instances of
OUTER. The abstract instance for OUTER is shown in 
Section \refersec{app:inliningexample3abstractinstance}.

The most distinctive aspect of that Figure is that subprogram
INNER exists only within the abstract instance of OUTER,
and not in OUTER’s concrete instance. In the abstract
instance of OUTER, the description of INNER has the full
complement of attributes that would be expected for a
normal subprogram. While attributes such as \livelink{chap:DWATlowpc}{DW\-\_AT\-\_low\-\_pc},
\livelink{chap:DWAThighpc}{DW\-\_AT\-\_high\-\_pc}, \livelink{chap:DWATlocation}{DW\-\_AT\-\_location}, and so on, typically are omitted
from an abstract instance because they are not invariant across
instances of the containing abstract instance, in this case
those same attributes are included precisely because they are
invariant -- there is only one subprogram INNER to be described
and every description is the same.

A concrete inlined instance of OUTER is illustrated in
\refersec{app:inliningexample3concreteinstance}.

Notice in 
\refersec{app:inliningexample3abstractinstance}
that there is no DWARF representation for
INNER at all; the representation of INNER does not vary across
instances of OUTER and the abstract instance of OUTER includes
the complete description of INNER, so that the description of
INNER may be (and for reasons of space efficiency, should be)
omitted from each concrete instance of OUTER.

There is one aspect of this approach that is problematical from
the DWARF perspective. The single compiled instance of INNER
is assumed to access up\dash level variables of OUTER; however,
those variables may well occur at varying positions within
the frames that contain the concrete inlined instances. A
compiler might implement this in several ways, including the
use of additional compiler generated parameters that provide
reference parameters for the up\dash level variables, or a compiler
generated static link like parameter that points to the group
of up\dash level entities, among other possibilities. In either of
these cases, the DWARF description for the location attribute
of each uplevel variable needs to be different if accessed
from within INNER compared to when accessed from within the
instances of OUTER. An implementation is likely to require
vendor\dash specific DWARF attributes and/or debugging information
entries to describe such cases.

Note that in C++, a member function of a class defined within
a function definition does not require any vendor\dash specific
extensions because the C++ language disallows access to
entities that would give rise to this problem. (Neither extern
variables nor static members require any form of static link
for accessing purposes.)

\subsubsection{Inlining example 3: abstract instance}
\label{app:inliningexample3abstractinstance}
\begin{alltt}
    ! Abstract instance for OUTER
    !
OUTER.AI.3.1:
    \livelink{chap:DWTAGsubprogram}{DW\-\_TAG\-\_subprogram}
        \livelink{chap:DWATname}{DW\-\_AT\-\_name}("OUTER")
        \livelink{chap:DWATinline}{DW\-\_AT\-\_inline}(DW\-\_INL\-\_declared\-\_inlined)
        ! No low/high PCs
OUTER.AI.3.2:
        \livelink{chap:DWTAGformalparameter}{DW\-\_TAG\-\_formal\-\_parameter}
            \livelink{chap:DWATname}{DW\-\_AT\-\_name}("OUTER\_FORMAL")
            \livelink{chap:DWATtype}{DW\-\_AT\-\_type}(reference to integer)
            ! No location
OUTER.AI.3.3:
        \livelink{chap:DWTAGvariable}{DW\-\_TAG\-\_variable}
            \livelink{chap:DWATname}{DW\-\_AT\-\_name}("OUTER\_LOCAL")
            \livelink{chap:DWATtype}{DW\-\_AT\-\_type}(reference to integer)
            ! No location
        !
        ! Normal INNER
        !
OUTER.AI.3.4:
        \livelink{chap:DWTAGsubprogram}{DW\-\_TAG\-\_subprogram}
            \livelink{chap:DWATname}{DW\-\_AT\-\_name}("INNER")
            \livelink{chap:DWATlowpc}{DW\-\_AT\-\_low\-\_pc}(...)
            \livelink{chap:DWAThighpc}{DW\-\_AT\-\_high\-\_pc}(...)
            \livelink{chap:DWATframebase}{DW\-\_AT\-\_frame\-\_base}(...)
            \livelink{chap:DWATstaticlink}{DW\-\_AT\-\_static\-\_link}(...)
OUTER.AI.3.5:
            \livelink{chap:DWTAGformalparameter}{DW\-\_TAG\-\_formal\-\_parameter}
                \livelink{chap:DWATname}{DW\-\_AT\-\_name}("INNER\_FORMAL")
                \livelink{chap:DWATtype}{DW\-\_AT\-\_type}(reference to integer)
                \livelink{chap:DWATlocation}{DW\-\_AT\-\_location}(...)
OUTER.AI.3.6:
            \livelink{chap:DWTAGvariable}{DW\-\_TAG\-\_variable}
                \livelink{chap:DWATname}{DW\-\_AT\-\_name}("INNER\_LOCAL")
                \livelink{chap:DWATtype}{DW\-\_AT\-\_type}(reference to integer)
                \livelink{chap:DWATlocation}{DW\-\_AT\-\_location}(...)
            ...
            0
        ...
        0
\end{alltt}


\subsubsection{Inlining example 3: concrete instance}
\label{app:inliningexample3concreteinstance}
\begin{alltt}
    ! Concrete instance for call "OUTER(7)"
    !
OUTER.CI.3.1:
    \livelink{chap:DWTAGinlinedsubroutine}{DW\-\_TAG\-\_inlined\-\_subroutine}
        ! No name
        \livelink{chap:DWATabstractorigin}{DW\-\_AT\-\_abstract\-\_origin}(reference to OUTER.AI.3.1)
        \livelink{chap:DWATlowpc}{DW\-\_AT\-\_low\-\_pc}(...)
        \livelink{chap:DWAThighpc}{DW\-\_AT\-\_high\-\_pc}(...)
        \livelink{chap:DWATframebase}{DW\-\_AT\-\_frame\-\_base}(...)
OUTER.CI.3.2:
        \livelink{chap:DWTAGformalparameter}{DW\-\_TAG\-\_formal\-\_parameter}
            ! No name
            \livelink{chap:DWATabstractorigin}{DW\-\_AT\-\_abstract\-\_origin}(reference to OUTER.AI.3.2)
            ! No type
            \livelink{chap:DWATlocation}{DW\-\_AT\-\_location}(...)
OUTER.CI.3.3:
        \livelink{chap:DWTAGvariable}{DW\-\_TAG\-\_variable}
            ! No name
            \livelink{chap:DWATabstractorigin}{DW\-\_AT\-\_abstract\-\_origin}(reference to OUTER.AI.3.3)
            ! No type
            \livelink{chap:DWATlocation}{DW\-\_AT\-\_location}(...)
        ! No \livelink{chap:DWTAGsubprogram}{DW\-\_TAG\-\_subprogram} for "INNER"
        ...
        0
\end{alltt}

\section{Constant Expression Example}
\label{app:constantexpressionexample}
C++ generalizes the notion of constant expressions to include
constant expression user-defined literals and functions.

\begin{figure}[here]
\begin{lstlisting}
constexpr double mass = 9.8;
constexpr int square (int x) { return x * x; }
float arr[square(9)]; // square() called and inlined
\end{lstlisting}
\caption{Constant expressions: C++ source} \label{fig:constantexpressionscsource}
\end{figure}

These declarations can be represented as illustrated in 
Section \refersec{app:constantexpressionsdwarfdescription}.

\subsection{Constant expressions: DWARF description}
\label{app:constantexpressionsdwarfdescription}
\begin{alltt}

      ! For variable mass
      !
1\$:  \livelink{chap:DWTAGconsttype}{DW\-\_TAG\-\_const\-\_type}
        \livelink{chap:DWATtype}{DW\-\_AT\-\_type}(reference to "double")
2\$:  \livelink{chap:DWTAGvariable}{DW\-\_TAG\-\_variable}
        \livelink{chap:DWATname}{DW\-\_AT\-\_name}("mass")
        \livelink{chap:DWATtype}{DW\-\_AT\-\_type}(reference to 1\$)
        \livelink{chap:DWATconstexpr}{DW\-\_AT\-\_const\-\_expr}(true)
        \livelink{chap:DWATconstvalue}{DW\-\_AT\-\_const\-\_value}(9.8)
      ! Abstract instance for square
      !
10\$: \livelink{chap:DWTAGsubprogram}{DW\-\_TAG\-\_subprogram}
        \livelink{chap:DWATname}{DW\-\_AT\-\_name}("square")
        \livelink{chap:DWATtype}{DW\-\_AT\-\_type}(reference to "int")
        \livelink{chap:DWATinline}{DW\-\_AT\-\_inline}(DW\-\_INL\-\_inlined)
11\$:   \livelink{chap:DWTAGformalparameter}{DW\-\_TAG\-\_formal\-\_parameter}
            \livelink{chap:DWATname}{DW\-\_AT\-\_name}("x")
            \livelink{chap:DWATtype}{DW\-\_AT\-\_type}(reference to "int")
      ! Concrete instance for square(9)
      !
20\$: \livelink{chap:DWTAGinlinedsubroutine}{DW\-\_TAG\-\_inlined\-\_subroutine}
        \livelink{chap:DWATabstractorigin}{DW\-\_AT\-\_abstract\-\_origin}(reference to 10\$)
        \livelink{chap:DWATconstexpr}{DW\-\_AT\-\_const\-\_expr}(present)
        \livelink{chap:DWATconstvalue}{DW\-\_AT\-\_const\-\_value}(81)
        \livelink{chap:DWTAGformalparameter}{DW\-\_TAG\-\_formal\-\_parameter}
            \livelink{chap:DWATabstractorigin}{DW\-\_AT\-\_abstract\-\_origin}(reference to 11\$)
            \livelink{chap:DWATconstvalue}{DW\-\_AT\-\_const\-\_value}(9)
      ! Anonymous array type for arr
      !
30\$: \livelink{chap:DWTAGarraytype}{DW\-\_TAG\-\_array\-\_type}
        \livelink{chap:DWATtype}{DW\-\_AT\-\_type}(reference to "float")
        \livelink{chap:DWATbytesize}{DW\-\_AT\-\_byte\-\_size}(324) ! 81*4
        \livelink{chap:DWTAGsubrangetype}{DW\-\_TAG\-\_subrange\-\_type}
            \livelink{chap:DWATtype}{DW\-\_AT\-\_type}(reference to "int")
            \livelink{chap:DWATupperbound}{DW\-\_AT\-\_upper\-\_bound}(reference to 20\$)
      ! Variable arr
      !
40\$: \livelink{chap:DWTAGvariable}{DW\-\_TAG\-\_variable}
        \livelink{chap:DWATname}{DW\-\_AT\-\_name}("arr")
        \livelink{chap:DWATtype}{DW\-\_AT\-\_type}(reference to 30\$)
\end{alltt}


\section{Unicode Character Example}
\label{app:unicodecharacterexample}

Unicode character encodings can be described in DWARF as
illustrated in 
Section \refersec{app:unicodecharacterexamplesub}.

\begin{lstlisting}
// C++ source
//
char16_t chr_a = u'h';
char32_t chr_b = U'h';
\end{lstlisting}

\subsection{Unicode Character Example}
\label{app:unicodecharacterexamplesub}
\begin{alltt}

! DWARF description
!
1\$: \livelink{chap:DWTAGbasetype}{DW\-\_TAG\-\_base\-\_type}
        \livelink{chap:DWATname}{DW\-\_AT\-\_name}("char16\_t")
        \livelink{chap:DWATencoding}{DW\-\_AT\-\_encoding}(\livelink{chap:DWATEUTF}{DW\-\_ATE\-\_UTF})
        \livelink{chap:DWATbytesize}{DW\-\_AT\-\_byte\-\_size}(2)
2\$: \livelink{chap:DWTAGbasetype}{DW\-\_TAG\-\_base\-\_type}
        \livelink{chap:DWATname}{DW\-\_AT\-\_name}("char32\_t")
        \livelink{chap:DWATencoding}{DW\-\_AT\-\_encoding}(\livelink{chap:DWATEUTF}{DW\-\_ATE\-\_UTF})
        \livelink{chap:DWATbytesize}{DW\-\_AT\-\_byte\-\_size}(4)
3\$: \livelink{chap:DWTAGvariable}{DW\-\_TAG\-\_variable}
        \livelink{chap:DWATname}{DW\-\_AT\-\_name}("chr\_a")
        \livelink{chap:DWATtype}{DW\-\_AT\-\_type}(reference to 1\$)
4\$: \livelink{chap:DWTAGvariable}{DW\-\_TAG\-\_variable}
        \livelink{chap:DWATname}{DW\-\_AT\-\_name}("chr\_b")
        \livelink{chap:DWATtype}{DW\-\_AT\-\_type}(reference to 2\$)
\end{alltt}



\section{Type-Safe Enumeration Example}
\label{app:typesafeenumerationexample}


C++ type\dash safe enumerations can be described in DWARF as illustrated in 
Section \refersec{app:ctypesafeenumerationexample}.

\begin{lstlisting}
// C++ source
//
enum class E { E1, E2=100 };
E e1;
\end{lstlisting}

\subsection{C++ type-safe enumeration example}
\label{app:ctypesafeenumerationexample}
\begin{alltt}
! DWARF description
!
11\$: \livelink{chap:DWTAGenumerationtype}{DW\-\_TAG\-\_enumeration\-\_type}
        \livelink{chap:DWATname}{DW\-\_AT\-\_name}("E")
        \livelink{chap:DWATtype}{DW\-\_AT\-\_type}(reference to "int")
        \livelink{chap:DWATenumclass}{DW\-\_AT\-\_enum\-\_class}(present)
12\$:   \livelink{chap:DWTAGenumerator}{DW\-\_TAG\-\_enumerator}
            \livelink{chap:DWATname}{DW\-\_AT\-\_name}("E1")
            \livelink{chap:DWATconstvalue}{DW\-\_AT\-\_const\-\_value}(0)
13\$:
         \livelink{chap:DWTAGenumerator}{DW\-\_TAG\-\_enumerator}
            \livelink{chap:DWATname}{DW\-\_AT\-\_name}("E2")
            \livelink{chap:DWATconstvalue}{DW\-\_AT\-\_const\-\_value}(100)
14\$: \livelink{chap:DWTAGvariable}{DW\-\_TAG\-\_variable}
        \livelink{chap:DWATname}{DW\-\_AT\-\_name}("e1")
        \livelink{chap:DWATtype}{DW\-\_AT\-\_type}(reference to 11\$)
\end{alltt}

\section{Template Example}
\label{app:templateexample}

C++ templates can be described in DWARF as illustrated in 
Section \refersec{app:ctemplateexample1}.



\begin{lstlisting}
// C++ source
//
template<class T>
struct wrapper {
    T comp;
};
wrapper<int> obj;
\end{lstlisting}

\subsection{C++ template example 1}
\label{app:ctemplateexample1}
\begin{alltt}
! DWARF description
!
11\$: \livelink{chap:DWTAGstructuretype}{DW\-\_TAG\-\_structure\-\_type}
        \livelink{chap:DWATname}{DW\-\_AT\-\_name}("wrapper")
12\$: \livelink{chap:DWTAGtemplatetypeparameter}{DW\-\_TAG\-\_template\-\_type\-\_parameter}
        \livelink{chap:DWATname}{DW\-\_AT\-\_name}("T")
        \livelink{chap:DWATtype}{DW\-\_AT\-\_type}(reference to "int")
13\$ \livelink{chap:DWTAGmember}{DW\-\_TAG\-\_member}
        \livelink{chap:DWATname}{DW\-\_AT\-\_name}("comp")
        \livelink{chap:DWATtype}{DW\-\_AT\-\_type}(reference to 12\$)
14\$: \livelink{chap:DWTAGvariable}{DW\-\_TAG\-\_variable}
        \livelink{chap:DWATname}{DW\-\_AT\-\_name}("obj")
        \livelink{chap:DWATtype}{DW\-\_AT\-\_type}(reference to 11\$)
\end{alltt}

The actual type of the component comp is int, but in the DWARF
the type references the \livelink{chap:DWTAGtemplatetypeparameter}{DW\-\_TAG\-\_template\-\_type\-\_parameter} for
T, which in turn references int. This implies that in the
original template comp was of type T and that was replaced
with int in the instance.  There exist situations where it is
not possible for the DWARF to imply anything about the nature
of the original template. 

Consider following C++ source and DWARF 
that can describe it in
Section \refersec{app:ctemplateexample2}.


\begin{lstlisting}
// C++ source
//
    template<class T>
    struct wrapper {
        T comp;
    };
    template<class U>
    void consume(wrapper<U> formal)
    {
        ...
    }
    wrapper<int> obj;
    consume(obj);
\end{lstlisting}

\subsection{C++ template example 2}
\label{app:ctemplateexample2}
\begin{alltt}
! DWARF description
!
11\$: \livelink{chap:DWTAGstructuretype}{DW\-\_TAG\-\_structure\-\_type}
        \livelink{chap:DWATname}{DW\-\_AT\-\_name}("wrapper")
12\$:   \livelink{chap:DWTAGtemplatetypeparameter}{DW\-\_TAG\-\_template\-\_type\-\_parameter}
            \livelink{chap:DWATname}{DW\-\_AT\-\_name}("T")
            \livelink{chap:DWATtype}{DW\-\_AT\-\_type}(reference to "int")
13\$    \livelink{chap:DWTAGmember}{DW\-\_TAG\-\_member}
            \livelink{chap:DWATname}{DW\-\_AT\-\_name}("comp")
            \livelink{chap:DWATtype}{DW\-\_AT\-\_type}(reference to 12\$)
14\$: \livelink{chap:DWTAGvariable}{DW\-\_TAG\-\_variable}
        \livelink{chap:DWATname}{DW\-\_AT\-\_name}("obj")
        \livelink{chap:DWATtype}{DW\-\_AT\-\_type}(reference to 11\$)
21\$: \livelink{chap:DWTAGsubprogram}{DW\-\_TAG\-\_subprogram}
        \livelink{chap:DWATname}{DW\-\_AT\-\_name}("consume")
22\$:   \livelink{chap:DWTAGtemplatetypeparameter}{DW\-\_TAG\-\_template\-\_type\-\_parameter}
            \livelink{chap:DWATname}{DW\-\_AT\-\_name}("U")
            \livelink{chap:DWATtype}{DW\-\_AT\-\_type}(reference to "int")
23\$:   \livelink{chap:DWTAGformalparameter}{DW\-\_TAG\-\_formal\-\_parameter}
            \livelink{chap:DWATname}{DW\-\_AT\-\_name}("formal")
            \livelink{chap:DWATtype}{DW\-\_AT\-\_type}(reference to 11\$)
\end{alltt}

In the \livelink{chap:DWTAGsubprogram}{DW\-\_TAG\-\_subprogram} entry for the instance of consume,
U is described as 
int. 
The type of formal is 
\begin{alltt}
wrapper<U>
\end{alltt}
 in
the source. DWARF only represents instantiations of templates;
there is no entry which represents 
\begin{alltt}
wrapper<U>, 
\end{alltt}
which is neither
a template parameter nor a template instantiation. The type
of formal is described as 
\begin{alltt}
wrapper<int>, 
\end{alltt}
the instantiation of
\begin{alltt}
wrapper<U>, 
\end{alltt}
in the \livelink{chap:DWATtype}{DW\-\_AT\-\_type} attribute at 
23\$. 
There is no
description of the relationship between template type parameter
T at 12\$ and U at 
22\$ which was used to instantiate 
\begin{alltt}
wrapper<U>.
\end{alltt}

A consequence of this is that the DWARF information would
not distinguish between the existing example and one where
the formal of consume were declared in the source to be
\begin{alltt}
wrapper<int>.
\end{alltt}

\section{Template Alias Examples}
\label{app:templatealiasexample}

C++ template aliases can be described in DWARF as illustrated in 
Section \refersec{app:templatealiasexample1}
and 
Section \refersec{app:templatealiasexample2}.


\begin{lstlisting}
// C++ source, template alias example 1
//
template<typename T, typename U>
struct Alpha {
    T tango;
    U uniform;
};
template<typename V> using Beta = Alpha<V,V>;
Beta<long> b;
\end{lstlisting}


\subsection{template alias example 1}
\label{app:templatealiasexample1}
\begin{alltt}
! DWARF representation for variable 'b'
!
20\$: \livelink{chap:DWTAGstructuretype}{DW\-\_TAG\-\_structure\-\_type}
        \livelink{chap:DWATname}{DW\-\_AT\-\_name}("Alpha")
21\$:   \livelink{chap:DWTAGtemplatetypeparameter}{DW\-\_TAG\-\_template\-\_type\-\_parameter}
            \livelink{chap:DWATname}{DW\-\_AT\-\_name}("T")
            \livelink{chap:DWATtype}{DW\-\_AT\-\_type}(reference to "long")
22\$:   \livelink{chap:DWTAGtemplatetypeparameter}{DW\-\_TAG\-\_template\-\_type\-\_parameter}
            \livelink{chap:DWATname}{DW\-\_AT\-\_name}("U")
            \livelink{chap:DWATtype}{DW\-\_AT\-\_type}(reference to "long")
23\$:   \livelink{chap:DWTAGmember}{DW\-\_TAG\-\_member}
            \livelink{chap:DWATname}{DW\-\_AT\-\_name}("tango")
            \livelink{chap:DWATtype}{DW\-\_AT\-\_type}(reference to 21\$)
24\$:   \livelink{chap:DWTAGmember}{DW\-\_TAG\-\_member}
            \livelink{chap:DWATname}{DW\-\_AT\-\_name}("uniform")
            \livelink{chap:DWATtype}{DW\-\_AT\-\_type}(reference to 22\$)
25\$: \livelink{chap:DWTAGtemplatealias}{DW\-\_TAG\-\_template\-\_alias}
        \livelink{chap:DWATname}{DW\-\_AT\-\_name}("Beta")
        \livelink{chap:DWATtype}{DW\-\_AT\-\_type}(reference to 20\$)
26\$:   \livelink{chap:DWTAGtemplatetypeparameter}{DW\-\_TAG\-\_template\-\_type\-\_parameter}
            \livelink{chap:DWATname}{DW\-\_AT\-\_name}("V")
            \livelink{chap:DWATtype}{DW\-\_AT\-\_type}(reference to "long")
27\$: \livelink{chap:DWTAGvariable}{DW\-\_TAG\-\_variable}
        \livelink{chap:DWATname}{DW\-\_AT\-\_name}("b")
        \livelink{chap:DWATtype}{DW\-\_AT\-\_type}(reference to 25\$)
\end{alltt}


\begin{lstlisting}
// C++ source, template alias example 2
//
template<class TX> struct X { };
template<class TY> struct Y { };
template<class T> using Z = Y<T>;
X<Y<int>> y;
X<Z<int>> z;
\end{lstlisting}


\subsection{template alias example 2}
\label{app:templatealiasexample2}
\begin{alltt}
! DWARF representation for X<Y<int>>
!
30\$: \livelink{chap:DWTAGstructuretype}{DW\-\_TAG\-\_structure\-\_type}
        \livelink{chap:DWATname}{DW\-\_AT\-\_name}("Y")
31\$:   \livelink{chap:DWTAGtemplatetypeparameter}{DW\-\_TAG\-\_template\-\_type\-\_parameter}
            \livelink{chap:DWATname}{DW\-\_AT\-\_name}("TY")
            \livelink{chap:DWATtype}{DW\-\_AT\-\_type}(reference to "int")
32\$: \livelink{chap:DWTAGstructuretype}{DW\-\_TAG\-\_structure\-\_type}
        \livelink{chap:DWATname}{DW\-\_AT\-\_name}("X")
33\$:   \livelink{chap:DWTAGtemplatetypeparameter}{DW\-\_TAG\-\_template\-\_type\-\_parameter}
            \livelink{chap:DWATname}{DW\-\_AT\-\_name}("TX")
            \livelink{chap:DWATtype}{DW\-\_AT\-\_type}(reference to 30\$)

! DWARF representation for X<Z<int>>
!
40\$: \livelink{chap:DWTAGtemplatealias}{DW\-\_TAG\-\_template\-\_alias}
using Z = Y<int>;
        \livelink{chap:DWATname}{DW\-\_AT\-\_name}("Z")
        \livelink{chap:DWATtype}{DW\-\_AT\-\_type}(reference to 30\$)
41\$:   \livelink{chap:DWTAGtemplatetypeparameter}{DW\-\_TAG\-\_template\-\_type\-\_parameter}
            \livelink{chap:DWATname}{DW\-\_AT\-\_name}("T")
            \livelink{chap:DWATtype}{DW\-\_AT\-\_type}(reference to "int")
42\$: \livelink{chap:DWTAGstructuretype}{DW\-\_TAG\-\_structure\-\_type}
        \livelink{chap:DWATname}{DW\-\_AT\-\_name}("X")
43\$:   \livelink{chap:DWTAGtemplatetypeparameter}{DW\-\_TAG\-\_template\-\_type\-\_parameter}
            \livelink{chap:DWATname}{DW\-\_AT\-\_name}("TX")
            \livelink{chap:DWATtype}{DW\-\_AT\-\_type}(reference to 40\$)
! Note that 32\$ and 42\$ are actually the same type
!
50\$: \livelink{chap:DWTAGvariable}{DW\-\_TAG\-\_variable}
        \livelink{chap:DWATname}{DW\-\_AT\-\_name}("y")
        \livelink{chap:DWATtype}{DW\-\_AT\-\_type}(reference to \$32)
51\$: \livelink{chap:DWTAGvariable}{DW\-\_TAG\-\_variable}
        \livelink{chap:DWATname}{DW\-\_AT\-\_name}("z")
        \livelink{chap:DWATtype}{DW\-\_AT\-\_type}(reference to \$42)
\end{alltt}
