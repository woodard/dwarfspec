\chapter{Program Scope Entries}
\label{chap:programscopeentries} 
This section describes debugging information entries that
relate to different levels of program scope: compilation,
module, subprogram, and so on. Except for separate type
entries (see Section \refersec{chap:separatetypeunitentries}), 
these entries may be thought of
as bounded by ranges of text addresses within the program.

\section{Unit Entries}
An object file may contain one or more compilation units,
of which there are three kinds: normal compilation units,
partial compilation units and type units. A partial compilation
unit is related to one or more other compilation units that
import it. A type unit represents a single complete type in a
separate unit. Either a normal compilation unit or a partial
compilation unit may be logically incorporated into another
compilation unit using an imported unit entry.

\subsection[Normal and Partial CU Entries]{Normal and Partial Compilation Unit Entries}
\label{chap:normalandpartialcompilationunitentries}

A normal compilation unit is represented by a debugging
information entry with the 
tag \livetarg{chap:DWTAGcompileunit}{DW\-\_TAG\-\_compile\-\_unit}. A partial
compilation unit is represented by a debugging information
entry with the 
tag \livetarg{chap:DWTAGpartialunit}{DW\-\_TAG\-\_partial\-\_unit}.

In a simple normal compilation, a single compilation unit with
the tag 
\livelink{chap:DWTAGcompileunit}{DW\-\_TAG\-\_compile\-\_unit} represents a complete object file
and the tag 
\livelink{chap:DWTAGpartialunit}{DW\-\_TAG\-\_partial\-\_unit} is not used. 
In a compilation
employing the DWARF space compression and duplicate elimination
techniques from 
Appendix \refersec{app:usingcompilationunits}, 
multiple compilation units using
the tags 
\livelink{chap:DWTAGcompileunit}{DW\-\_TAG\-\_compile\-\_unit} and/or 
\livelink{chap:DWTAGpartialunit}{DW\-\_TAG\-\_partial\-\_unit} are
used to represent portions of an object file.

\textit{A normal compilation unit typically represents the text and
data contributed to an executable by a single relocatable
object file. It may be derived from several source files,
including pre\dash processed ``include files.'' A partial
compilation unit typically represents a part of the text
and data of a relocatable object file, in a manner that can
potentially be shared with the results of other compilations
to save space. It may be derived from an ``include file'',
template instantiation, or other implementation\dash dependent
portion of a compilation. A normal compilation unit can also
function in a manner similar to a partial compilation unit
in some cases.}

A compilation unit entry owns debugging information
entries that represent all or part of the declarations
made in the corresponding compilation. In the case of a
partial compilation unit, the containing scope of its owned
declarations is indicated by imported unit entries in one
or more other compilation unit entries that refer to that
partial compilation unit (see 
Section \refersec{chap:importedunitentries}).


Compilation unit entries may have the following 
attributes:

\begin{enumerate}[1]
\item Either a \livelink{chap:DWATlowpc}{DW\-\_AT\-\_low\-\_pc} and \livelink{chap:DWAThighpc}{DW\-\_AT\-\_high\-\_pc} pair of
attributes or a \livelink{chap:DWATranges}{DW\-\_AT\-\_ranges} attribute whose values encode the
contiguous or non\dash contiguous address ranges, respectively,
of the machine instructions generated for the compilation
unit (see Section {chap:codeaddressesandranges}).  
A \livelink{chap:DWATlowpc}{DW\-\_AT\-\_low\-\_pc} attribute may also
be specified in combination with \livelink{chap:DWATranges}{DW\-\_AT\-\_ranges} to specify the
default base address for use in location lists (see Section
\refersec{chap:locationlists}) and range lists 
(see Section \refersec{chap:noncontiguousaddressranges}).

\item A \livelink{chap:DWATname}{DW\-\_AT\-\_name} attribute whose value is a null\dash terminated
string 
\hypertarget{chap:DWATnamepathnameofcompilationsource}
containing the full or relative path name of the primary
source file from which the compilation unit was derived.

\item A \livelink{chap:DWATlanguage}{DW\-\_AT\-\_language} attribute whose constant value is an
\hypertarget{chap:DWATlanguageprogramminglanguage}
integer code indicating the source language of the compilation
unit. The set of language names and their meanings are given
in 
Figure \refersec{fig:languagenames}.

\begin{figure}[here]
\centering
\caption{Language names}
\label{fig:languagenames}
\begin{tabular}{ll}
Language name & Meaning\\ \hline
\livetarg{chap:DWLANGAda83}{DW\-\_LANG\-\_Ada83} \dag&ISO Ada:1983 \\
\livetarg{chap:DWLANGAda95}{DW\-\_LANG\-\_Ada95} \dag&ISO Ada:1995 \\
\livetarg{chap:DWLANGC}{DW\-\_LANG\-\_C}&Non-standardized C, such as K\&R \\
\livetarg{chap:DWLANGC89}{DW\-\_LANG\-\_C89}&ISO C:1989 \\
\livetarg{chap:DWLANGC99}{DW\-\_LANG\-\_C99} & ISO C:1999 \\
\livetarg{chap:DWLANGCplusplus}{DW\-\_LANG\-\_C\-\_plus\-\_plus}&ISO C++:1998 \\
\livetarg{chap:DWLANGCobol74}{DW\-\_LANG\-\_Cobol74}& ISO Cobol:1974 \\
\livetarg{chap:DWLANGCobol85}{DW\-\_LANG\-\_Cobol85} & ISO Cobol:1985 \\
\livetarg{chap:DWLANGD}{DW\-\_LANG\-\_D} \dag & D \\
\livetarg{chap:DWLANGFortran77}{DW\-\_LANG\-\_Fortran77} &ISO FORTRAN 77\\
\livetarg{chap:DWLANGFortran90}{DW\-\_LANG\-\_Fortran90} & ISO Fortran 90\\
\livetarg{chap:DWLANGFortran95}{DW\-\_LANG\-\_Fortran95} & ISO Fortran 95\\
\livetarg{chap:DWLANGJava}{DW\-\_LANG\-\_Java} & Java\\
\livetarg{chap:DWLANGModula2}{DW\-\_LANG\-\_Modula2} & ISO Modula\dash 2:1996\\
\livetarg{chap:DWLANGObjC}{DW\-\_LANG\-\_ObjC} & Objective C\\
\livetarg{chap:DWLANGObjCplusplus}{DW\-\_LANG\-\_ObjC\-\_plus\-\_plus} & Objective C++\\
\livetarg{chap:DWLANGPascal83}{DW\-\_LANG\-\_Pascal83} & ISO Pascal:1983\\
\livetarg{chap:DWLANGPLI}{DW\-\_LANG\-\_PLI} \dag & ANSI PL/I:1976\\
\livetarg{chap:DWLANGPython}{DW\-\_LANG\-\_Python} \dag & Python\\
\livetarg{chap:DWLANGUPC}{DW\-\_LANG\-\_UPC} &Unified Parallel C\\ \hline
\dag \ \ Support for these languages is limited.& \\
\end{tabular}
\end{figure}

\item A \livelink{chap:DWATstmtlist}{DW\-\_AT\-\_stmt\-\_list} attribute whose value is a section
\hypertarget{chap:DWATstmtlistlinenumberinformationforunit}
offset to the line number information for this compilation
unit.  This information is placed in a separate object file
section from the debugging information entries themselves. The
value of the statement list attribute is the offset in the
.debug\_line section of the first byte of the line number
information for this compilation unit 
(see Section \refersec{chap:linenumberinformation}).

\item A \livelink{chap:DWATmacroinfo}{DW\-\_AT\-\_macro\-\_info} attribute whose value is a section
\hypertarget{chap:DWATmacroinfomacroinformation}
offset to the macro information for this compilation unit.
This information is placed in a separate object file section
from the debugging information entries themselves. The
value of the macro information attribute is the offset in
the .debug\_macinfo section of the first byte of the macro
information for this compilation unit 
(see Section \refersec{chap:macroinformation}).

\item  A 
\livelink{chap:DWATcompdir}{DW\-\_AT\-\_comp\-\_dir} 
attribute 
\hypertarget{chap:DWATcompdircompilationdirectory}
whose value is a
null\dash terminated string containing the current working directory
of the compilation command that produced this compilation
unit in whatever form makes sense for the host system.

\item  A \livelink{chap:DWATproducer}{DW\-\_AT\-\_producer} attribute whose value is a null\dash
terminated string containing information about the compiler
\hypertarget{chap:DWATproducercompileridentification}
that produced the compilation unit. The actual contents of
the string will be specific to each producer, but should
begin with the name of the compiler vendor or some other
identifying character sequence that should avoid confusion
with other producer values.


\item  A \livelink{chap:DWATidentifiercase}{DW\-\_AT\-\_identifier\-\_case} 
attribute whose integer
\hypertarget{chap:DWATidentifiercaseidentifiercaserule}
constant value is a code describing the treatment
of identifiers within this compilation unit. The
set of identifier case codes is given in Figure
\refersec{fig:identifiercasecodes}.

\begin{figure}[here]
\autorows[0pt]{c}{1}{l}{
\livelink{chap:DWIDcasesensitive}{DW\-\_ID\-\_case\-\_sensitive},
\livelink{chap:DWIDupcase}{DW\-\_ID\-\_up\-\_case},
\livelink{chap:DWIDdowncase}{DW\-\_ID\-\_down\-\_case},
\livelink{chap:DWIDcaseinsensitive}{DW\-\_ID\-\_case\-\_insensitive}
}
\caption{Identifier case codes}\label{fig:identifiercasecodes}
\end{figure}

\livetarg{chap:DWIDcasesensitive}{DW\-\_ID\-\_case\-\_sensitive} is the default for all compilation units
that do not have this attribute.  It indicates that names given
as the values of \livelink{chap:DWATname}{DW\-\_AT\-\_name} attributes in debugging information
entries for the compilation unit reflect the names as they
appear in the source program. The debugger should be sensitive
to the case of identifier names when doing identifier lookups.

\livetarg{chap:DWIDupcase}{DW\-\_ID\-\_up\-\_case} means that the producer of the debugging
information for this compilation unit converted all source
names to upper case. The values of the name attributes may not
reflect the names as they appear in the source program. The
debugger should convert all names to upper case when doing
lookups.

\livetarg{chap:DWIDdowncase}{DW\-\_ID\-\_down\-\_case} means that the producer of the debugging
information for this compilation unit converted all source
names to lower case. The values of the name attributes may not
reflect the names as they appear in the source program. The
debugger should convert all names to lower case when doing
lookups.

\livetarg{chap:DWIDcaseinsensitive}{DW\-\_ID\-\_case\-\_insensitive} means that the values of the name
attributes reflect the names as they appear in the source
program but that a case insensitive lookup should be used to
access those names.

\item A \livelink{chap:DWATbasetypes}{DW\-\_AT\-\_base\-\_types} attribute whose value is a reference.


This 
\hypertarget{chap:DWATbasetypesprimitivedatatypesofcompilationunit}
attribute points to a debugging information entry
representing another compilation unit.  It may be used
to specify the compilation unit containing the base type
entries used by entries in the current compilation unit
(see Section \refersec{chap:basetypeentries}).

This attribute provides a consumer a way to find the definition
of base types for a compilation unit that does not itself
contain such definitions. This allows a consumer, for example,
to interpret a type conversion to a base type correctly.

\item A \livelink{chap:DWATuseUTF8}{DW\-\_AT\-\_use\-\_UTF8} attribute, 
which is a \livelink{chap:flag}{flag} whose
presence indicates that all strings (such as the names of
declared entities in the source program) are represented
using the UTF\dash 8 representation 
(see Section \refersec{datarep:attributeencodings}).


\item A \livelink{chap:DWATmainsubprogram}{DW\-\_AT\-\_main\-\_subprogram} attribute, which is a \livelink{chap:flag}{flag}
whose presence indicates 
\hypertarget{chap:DWATmainsubprogramunitcontainingmainorstartingsubprogram}
that the compilation unit contains a
subprogram that has been identified as the starting function
of the program. If more than one compilation unit contains
this \nolink{flag}, any one of them may contain the starting function.

\textit{Fortran has a PROGRAM statement which is used
to specify and provide a user\dash specified name for the main
subroutine of a program. C uses the name “main” to identify
the main subprogram of a program. Some other languages provide
similar or other means to identify the main subprogram of
a program.}

\end{enumerate}

The  base address of a compilation unit is defined as the
value of the \livelink{chap:DWATlowpc}{DW\-\_AT\-\_low\-\_pc} attribute, if present; otherwise,
it is undefined. If the base address is undefined, then any
DWARF entry or structure defined in terms of the base address
of that compilation unit is not valid.


\subsection{Imported Unit Entries}
\label{chap:importedunitentries}
The 
\hypertarget{chap:DWATimportimportedunit}
place where a normal or partial unit is imported is
represented by a debugging information entry with the 
tag \livetarg{chap:DWTAGimportedunit}{DW\-\_TAG\-\_imported\-\_unit}. 
An imported unit entry contains a
\livelink{chap:DWATimport}{DW\-\_AT\-\_import} attribute whose value is a reference to the
normal or partial compilation unit whose declarations logically
belong at the place of the imported unit entry.

An imported unit entry does not necessarily correspond to
any entity or construct in the source program. It is merely
“glue” used to relate a partial unit, or a compilation
unit used as a partial unit, to a place in some other
compilation unit.

\subsection{Separate Type Unit Entries}
\label{chap:separatetypeunitentries}
An object file may contain any number of separate type
unit entries, each representing a single complete type
definition. Each type unit must be uniquely identified by
a 64\dash bit signature, stored as part of the type unit, which
can be used to reference the type definition from debugging
information entries in other compilation units and type units.

A type unit is represented by a debugging information entry
with the tag \livetarg{chap:DWTAGtypeunit}{DW\-\_TAG\-\_type\-\_unit}. 
A type unit entry owns debugging
information entries that represent the definition of a single
type, plus additional debugging information entries that may
be necessary to include as part of the definition of the type.

A type unit entry may have a \livelink{chap:DWATlanguage}{DW\-\_AT\-\_language} attribute, whose
constant value is an integer code indicating the source
language used to define the type. The set of language names
and their meanings are given in Figure \refersec{fig:languagenames}.

A type unit entry for a given type T owns a debugging
information entry that represents a defining declaration
of type T. If the type is nested within enclosing types or
namespaces, the debugging information entry for T is nested
within debugging information entries describing its containers;
otherwise, T is a direct child of the type unit entry.

A type unit entry may also own additional debugging information
entries that represent declarations of additional types that
are referenced by type T and have not themselves been placed in
separate type units. Like T, if an additional type U is nested
within enclosing types or namespaces, the debugging information
entry for U is nested within entries describing its containers;
otherwise, U is a direct child of the type unit entry.

The containing entries for types T and U are declarations,
and the outermost containing entry for any given type T or
U is a direct child of the type unit entry. The containing
entries may be shared among the additional types and between
T and the additional types.

Types are not required to be placed in type units. In general,
only large types such as structure, class, enumeration, and
union types included from header files should be considered
for separate type units. Base types and other small types
are not usually worth the overhead of placement in separate
type units. Types that are unlikely to be replicated, such
as those defined in the main source file, are also better
left in the main compilation unit.

\section{Module, Namespace and Importing Entries}
\textit{Modules and namespaces provide a means to collect related
entities into a single entity and to manage the names of
those entities.}

\subsection{Module Entries}
\label{chap:moduleentries}
\textit{Several languages have the concept of a ``module.''
A Modula\dash 2 definition module may be represented by a module
entry containing a declaration attribute (\livelink{chap:DWATdeclaration}{DW\-\_AT\-\_declaration}). A
Fortran 90 module may also be represented by a module entry
(but no declaration attribute is warranted because Fortran
has no concept of a corresponding module body).}

A module is represented by a debugging information entry
with the 
tag \livetarg{chap:DWTAGmodule}{DW\-\_TAG\-\_module}.  
Module entries may own other
debugging information entries describing program entities
whose declaration scopes end at the end of the module itself.

If the module has a name, the module entry has a \livelink{chap:DWATname}{DW\-\_AT\-\_name}
attribute whose value is a null\dash terminated string containing
the module name as it appears in the source program.

The module entry may have either a \livelink{chap:DWATlowpc}{DW\-\_AT\-\_low\-\_pc} and
\livelink{chap:DWAThighpc}{DW\-\_AT\-\_high\-\_pc} pair of attributes or a \livelink{chap:DWATranges}{DW\-\_AT\-\_ranges} attribute
whose values encode the contiguous or non\dash contiguous address
ranges, respectively, of the machine instructions generated for
the module initialization code 
(see Section \refersec{chap:codeaddressesandranges}). 
\hypertarget{chap:DWATentrypcentryaddressofmoduleinitialization}
It may also
have a \livelink{chap:DWATentrypc}{DW\-\_AT\-\_entry\-\_pc} attribute whose value is the address of
the first executable instruction of that initialization code
(see Section \refersec{chap:entryaddress}).

If 
\hypertarget{chap:DWATprioritymodulepriority}
the module has been assigned a priority, it may have a
\livelink{chap:DWATpriority}{DW\-\_AT\-\_priority} attribute. The value of this attribute is a
reference to another debugging information entry describing
a variable with a constant value. The value of this variable
is the actual constant value of the module’s priority,
represented as it would be on the target architecture.

\subsection{Namespace Entries}
\label{chap:namespaceentries}
\textit{C++ has the notion of a namespace, which provides a way to
implement name hiding, so that names of unrelated things
do not accidentally clash in the global namespace when an
application is linked together.}

A namespace is represented by a debugging information entry
with the 
tag \livetarg{chap:DWTAGnamespace}{DW\-\_TAG\-\_namespace}. 
A namespace extension is
\hypertarget{chap:DWATextensionpreviousnamespaceextensionororiginalnamespace}
represented by a 
\livelink{chap:DWTAGnamespace}{DW\-\_TAG\-\_namespace} entry 
with a 
\livelink{chap:DWATextension}{DW\-\_AT\-\_extension}
attribute referring to the previous extension, or if there
is no previous extension, to the original 
\livelink{chap:DWTAGnamespace}{DW\-\_TAG\-\_namespace}
entry. A namespace extension entry does not need to duplicate
information in a previous extension entry of the namespace
nor need it duplicate information in the original namespace
entry. (Thus, for a namespace with a name, 
a \livelink{chap:DWATname}{DW\-\_AT\-\_name}
attribute need only be attached directly to the original
\livelink{chap:DWTAGnamespace}{DW\-\_TAG\-\_namespace} entry.)

Namespace and namespace extension entries may own other
debugging information entries describing program entities
whose declarations occur in the namespace.

\textit{For C++, such owned program entities may be declarations,
including certain declarations that are also object or
function definitions.}

If a type, variable, or function declared in a namespace is
defined outside of the body of the namespace declaration,
that type, variable, or function definition entry has a
\livelink{chap:DWATspecification}{DW\-\_AT\-\_specification} attribute whose value is a reference to the
debugging information entry representing the declaration of
the type, variable or function. Type, variable, or function
entries with a \livelink{chap:DWATspecification}{DW\-\_AT\-\_specification} attribute do not need
to duplicate information provided by the declaration entry
referenced by the specification attribute.

\textit{The C++ global namespace (the namespace referred to by
``::f'', for example) is not explicitly represented in
DWARF with a namespace entry (thus mirroring the situation
in C++ source).  Global items may be simply declared with no
reference to a namespace.}

\textit{The C++ compilation unit specific ``unnamed namespace'' may
be represented by a namespace entry with no name attribute in
the original namespace declaration entry (and therefore no name
attribute in any namespace extension entry of this namespace).
}

\textit{A compiler emitting namespace information may choose to
explicitly represent namespace extensions, or to represent the
final namespace declaration of a compilation unit; this is a
quality\dash of\dash implementation issue and no specific requirements
are given here. If only the final namespace is represented,
it is impossible for a debugger to interpret using declaration
references in exactly the manner defined by the C++ language.
}

\textit{Emitting all namespace declaration information in all
compilation units can result in a significant increase in the
size of the debug information and significant duplication of
information across compilation units. The C++ namespace std,
for example, is large and will probably be referenced in
every C++ compilation unit.
}

\textit{For a C++ namespace example, see Appendix \refersec{app:namespaceexample}.
}



\subsection{Imported (or Renamed) Declaration Entries} 
\label{chap:importedorrenameddeclarationentries}
\textit{Some languages support the concept of importing into or making
accessible in a given unit declarations made in a different
module or scope. An imported declaration may sometimes be
given another name.
}

An imported declaration is represented by one or
more debugging information entries with the 
tag \livetarg{chap:DWTAGimporteddeclaration}{DW\-\_TAG\-\_imported\-\_declaration}. 
When 
\hypertarget{chap:DWATimportimporteddeclaration}
an overloaded entity
is imported, there is one imported declaration entry for
each overloading. Each imported declaration entry has a
\livelink{chap:DWATimport}{DW\-\_AT\-\_import} attribute, whose value is a reference to the
debugging information entry representing the declaration that
is being imported.

An imported declaration may also have a 
\livelink{chap:DWATname}{DW\-\_AT\-\_name}
attribute
whose value is a null\dash terminated string containing the
name, as it appears in the source program, by which the
imported entity is to be known in the context of the imported
declaration entry (which may be different than the name of
the entity being imported). If no name is present, then the
name by which the entity is to be known is the same as the
name of the entity being imported.

An imported declaration entry with a name attribute may be
used as a general means to rename or provide an alias for
an entity, regardless of the context in which the importing
declaration or the imported entity occurs.

\textit{A C++ namespace alias may be represented by an imported
\hypertarget{chap:DWATimportnamespacealias}
declaration entry with a name attribute whose value is
a null\dash terminated string containing the alias name as it
appears in the source program and an import attribute whose
value is a reference to the applicable original namespace or
namespace extension entry.
}

\textit{A C++ using declaration may be represented by one or more
\hypertarget{chap:DWATimportnamespaceusingdeclaration}
imported declaration entries.  When the using declaration
refers to an overloaded function, there is one imported
declaration entry corresponding to each overloading. Each
imported declaration entry has no name attribute but it does
have an import attribute that refers to the entry for the
entity being imported. (C++ provides no means to ``rename''
an imported entity, other than a namespace).
}

\textit{A Fortran use statement with an ``only list'' may be
represented by a series of imported declaration entries,
one (or more) for each entity that is imported. An entity
that is renamed in the importing context may be represented
by an imported declaration entry with a name attribute that
specifies the new local name.
}

\subsection{Imported Module Entries}
\label{chap:importedmoduleentries}

\textit{Some languages support the concept of importing into or making
accessible in a given unit all of the declarations contained
within a separate module or namespace.
}

An imported module declaration is represented by a debugging
information entry with the 
tag \livetarg{chap:DWTAGimportedmodule}{DW\-\_TAG\-\_imported\-\_module}.
An
imported module entry contains a \livelink{chap:DWATimport}{DW\-\_AT\-\_import} attribute
whose value is a reference to the module or namespace entry
containing the definition and/or declaration entries for
the entities that are to be imported into the context of the
imported module entry.

An imported module declaration may own a set of imported
declaration entries, each of which refers to an entry in the
module whose corresponding entity is to be known in the context
of the imported module declaration by a name other than its
name in that module. Any entity in the module that is not
renamed in this way is known in the context of the imported
module entry by the same name as it is declared in the module.

\textit{A C++ using directive may be represented by an imported module
\hypertarget{chap:DWATimportnamespaceusingdirective}
entry, with an import attribute referring to the namespace
entry of the appropriate extension of the namespace (which
might be the original namespace entry) and no owned entries.
}

\textit{A Fortran use statement with a “rename list” may be
represented by an imported module entry with an import
attribute referring to the module and owned entries
corresponding to those entities that are renamed as part of
being imported.
}

\textit{A Fortran use statement with neither a “rename list” nor
an “only list” may be represented by an imported module
entry with an import attribute referring to the module and
no owned child entries.
}

\textit{A use statement with an “only list” is represented by a
series of individual imported declaration entries as described
in Section \refersec{chap:importedorrenameddeclarationentries}.
}

\textit{A Fortran use statement for an entity in a module that is
itself imported by a use statement without an explicit mention
may be represented by an imported declaration entry that refers
to the original debugging information entry. For example, given
}
\begin{lstlisting}
module A
integer X, Y, Z
end module

module B
use A
end module

module C
use B, only Q => X
end module
\end{lstlisting}

the imported declaration entry for Q within module C refers
directly to the variable declaration entry for A in module A
because there is no explicit representation for X in module B.

A similar situation arises for a C++ using declaration that
imports an entity in terms of a namespace alias. See 
Appendix  \refersec{app:namespaceexample}
for an example.


\section{Subroutine and Entry Point Entries}
\label{chap:subroutineandentrypointentries}

The following tags exist to describe debugging information entries for subroutines and entry
points:

\begin{tabular}{lp{9.0cm}}
\livetarg{chap:DWTAGsubprogram}{DW\-\_TAG\-\_subprogram} & A subroutine or function. \\
\livelink{chap:DWTAGinlinedsubroutine}{DW\-\_TAG\-\_inlined\-\_subroutine} & A particular inlined 
  instance of a subroutine or function. \\
\livetarg{chap:DWTAGentrypoint}{DW\-\_TAG\-\_entry\-\_point} & An alternate entry point. \\
\end{tabular}

\subsection{General Subroutine and Entry Point Information}
\label{chap:generalsubroutineandentrypointinformation}

It may also have a \livelink{chap:DWATlinkagename}{DW\-\_AT\-\_linkage\-\_name} attribute as
described in Section \refersec{chap:linkagenames}.

If the name of the subroutine described by an entry with the
tag \livelink{chap:DWTAGsubprogram}{DW\-\_TAG\-\_subprogram}
is visible outside of its containing
\hypertarget{chap:DWATexternalexternalsubroutine}
compilation unit, that entry has a 
\livelink{chap:DWATexternal}{DW\-\_AT\-\_external} attribute,
which is a \livelink{chap:flag}{flag}.

\textit{Additional attributes for functions that are members of a
class or structure are described in 
Section \refersec{chap:memberfunctionentries}.
}

A 
\hypertarget{chap:DWATmainsubprogrammainorstartingsubprogram}
subroutine entry may contain a \livelink{chap:DWATmainsubprogram}{DW\-\_AT\-\_main\-\_subprogram}
attribute which is 
a \livelink{chap:flag}{flag} whose presence indicates that the
subroutine has been identified as the starting function of
the program.  If more than one subprogram contains this 
\nolink{flag},
any one of them may be the starting subroutine of the program.

\textit{Fortran has a PROGRAM statement which is used to specify
and provide a user\dash supplied name for the main subroutine of
a program.
}

\textit{A common debugger feature is to allow the debugger user to call
a subroutine within the subject program. In certain cases,
however, the generated code for a subroutine will not obey
the standard calling conventions for the target architecture
and will therefore not be safe to call from within a debugger.
}

A subroutine entry may 
\hypertarget{chap:DWATcallingconventionsubprogramcallingconvention}
contain a 
\livelink{chap:DWATcallingconvention}{DW\-\_AT\-\_calling\-\_convention}
attribute, whose value is an integer constant. The set of
calling convention codes is given in 
Figure \refersec{fig:callingconventioncodes}.

\begin{figure}[here]
\autorows[0pt]{c}{1}{l}{
\addtoindex{DW\-\_CC\-\_normal},
\addtoindex{DW\-\_CC\-\_program},
\addtoindex{DW\-\_CC\-\_nocall},
}
\caption{Calling convention codes}\label{fig:callingconventioncodes}
\end{figure}

If this attribute is not present, or its value is the constant
\livetarg{chap:DWCCnormal}{DW\-\_CC\-\_normal}, then the subroutine may be safely called by
obeying the ``standard'' calling conventions of the target
architecture. If the value of the calling convention attribute
is the constant \livetarg{chap:DWCCnocall}{DW\-\_CC\-\_nocall}, the subroutine does not obey
standard calling conventions, and it may not be safe for the
debugger to call this subroutine.

If the semantics of the language of the compilation unit
containing the subroutine entry distinguishes between ordinary
subroutines and subroutines that can serve as the ``main
program,'' that is, subroutines that cannot be called
directly according to the ordinary calling conventions,
then the debugging information entry for such a subroutine
may have a calling convention attribute whose value is the
constant \livetarg{chap:DWCCprogram}{DW\-\_CC\-\_program}.

\textit{The \livelink{chap:DWCCprogram}{DW\-\_CC\-\_program} value is intended to support Fortran main
programs which in some implementations may not be callable
or which must be invoked in a special way. It is not intended
as a way of finding the entry address for the program.
}

\textit{In C there is a difference between the types of functions
declared using function prototype style declarations and
those declared using non\dash prototype declarations.
}

A subroutine entry declared with a function prototype style
declaration may have a 
\livelink{chap:DWATprototyped}{DW\-\_AT\-\_prototyped} attribute, which is
a \livelink{chap:flag}{flag}.

\textit{The Fortran language allows the keywords elemental, pure
and recursive to be included as part of the declaration of
a subroutine; these attributes reflect that usage. These
attributes are not relevant for languages that do not support
similar keywords or syntax. In particular, the \livelink{chap:DWATrecursive}{DW\-\_AT\-\_recursive}
attribute is neither needed nor appropriate in languages such
as C where functions support recursion by default.
}

A subprogram entry 
\hypertarget{chap:DWATelementalelementalpropertyofasubroutine}
may have a 
\livelink{chap:DWATelemental}{DW\-\_AT\-\_elemental} attribute, which
is a \livelink{chap:flag}{flag}. 
The attribute indicates whether the subroutine
or entry point was declared with the ``elemental'' keyword
or property.

A 
\hypertarget{chap:DWATpurepurepropertyofasubroutine}
subprogram entry may have a 
\livelink{chap:DWATpure}{DW\-\_AT\-\_pure} attribute, which is
a \livelink{chap:flag}{flag}. 
The attribute indicates whether the subroutine was
declared with the ``pure'' keyword or property.

A 
\hypertarget{chap:DWATrecursiverecursivepropertyofasubroutine}
subprogram entry may have a 
\livelink{chap:DWATrecursive}{DW\-\_AT\-\_recursive} attribute, which
is a \livelink{chap:flag}{flag}. 
The attribute indicates whether the subroutine
or entry point was declared with the ``recursive'' keyword
or property.



\subsection{Subroutine and Entry Point Return Types}
\label{chap:subroutineandentrypointreturntypes}

If the subroutine or entry point is a function that returns a
value, then its debugging information entry has a \livelink{chap:DWATtype}{DW\-\_AT\-\_type}
attribute to denote the type returned by that function.

\textit{Debugging information entries for C void functions should
not have an attribute for the return type.  }


\subsection{Subroutine and Entry Point Locations}
\label{chap:subroutineandentrypointlocations}

A subroutine entry may have either a \livelink{chap:DWATlowpc}{DW\-\_AT\-\_low\-\_pc} and
\livelink{chap:DWAThighpc}{DW\-\_AT\-\_high\-\_pc} pair of attributes or a \livelink{chap:DWATranges}{DW\-\_AT\-\_ranges} attribute
whose values encode the contiguous or non\dash contiguous address
ranges, respectively, of the machine instructions generated
for the subroutine (see 
Section \refersec{chap:codeaddressesandranges}).

A 
\hypertarget{chap:DWATentrypcentryaddressofsubprogram}
subroutine entry may also have a 
\livelink{chap:DWATentrypc}{DW\-\_AT\-\_entry\-\_pc} attribute
whose value is the address of the first executable instruction
of the subroutine (see 
Section \refersec{chap:entryaddress}).

An entry point has a \livelink{chap:DWATlowpc}{DW\-\_AT\-\_low\-\_pc} attribute whose value is the
relocated address of the first machine instruction generated
for the entry point.

\textit{While the \livelink{chap:DWATentrypc}{DW\-\_AT\-\_entry\-\_pc} attribute might also seem appropriate
for this purpose, historically the \livelink{chap:DWATlowpc}{DW\-\_AT\-\_low\-\_pc} attribute
was used before the \livelink{chap:DWATentrypc}{DW\-\_AT\-\_entry\-\_pc} was introduced (in DWARF
Version 3). There is insufficient reason to change this.}


Subroutines and entry points 
\hypertarget{chap:DWATaddressclasssubroutineorsubroutinetype}
may also have 
\livelink{chap:DWATsegment}{DW\-\_AT\-\_segment} 
and
\livelink{chap:DWATaddressclass}{DW\-\_AT\-\_address\-\_class} attributes,
as appropriate, to specify
which segments the code for the subroutine resides in and
the addressing mode to be used in calling that subroutine.

A subroutine entry representing a subroutine declaration
that is not also a definition does not have code address or
range attributes.


\subsection{Declarations Owned by Subroutines and Entry Points} 
\label{chap:declarationsownedbysubroutinesandentrypoints}

The declarations enclosed by a subroutine or entry point are
represented by debugging information entries that are owned
by the subroutine or entry point entry. Entries representing
the formal parameters of the subroutine or entry point appear
in the same order as the corresponding declarations in the
source program.

\textit{There is no ordering requirement for entries for declarations
that are children of subroutine or entry point entries but
that do not represent formal parameters. The formal parameter
entries may be interspersed with other entries used by formal
parameter entries, such as type entries.}

The unspecified parameters of a variable parameter list are
represented by a debugging information entry with the tag
\livetarg{chap:DWTAGunspecifiedparameters}{DW\-\_TAG\-\_unspecified\-\_parameters}.

The entry for a subroutine that includes a 
Fortran \livelink{chap:fortrancommonblock}{common} \livelink{chap:commonblockentry}{block}
\index{common block|see{Fortran common block}}
has a child entry with the 
tag \livetarg{chap:DWTAGcommoninclusion}{DW\-\_TAG\-\_common\-\_inclusion}. 
The
\hypertarget{chap:commonreferencecommonblockusage}
common inclusion entry has a 
\livelink{chap:DWATcommonreference}{DW\-\_AT\-\_common\-\_reference} attribute
whose value is a reference to the debugging information entry
for the common \nolink{block} being included 
(see Section \refersec{chap:commonblockentries}).

\subsection{Low-Level Information}
\label{chap:lowlevelinformation}

A 
\hypertarget{chap:DWATreturnaddrsubroutinereturnaddresssavelocation}
subroutine or entry point entry may have a 
\livelink{chap:DWATreturnaddr}{DW\-\_AT\-\_return\-\_addr}
attribute, whose value is a location description. The location
calculated is the place where the return address for the
subroutine or entry point is stored.

A 
\hypertarget{chap:DWATframebasesubroutineframebaseaddress}
subroutine or entry point entry may also have a
\livelink{chap:DWATframebase}{DW\-\_AT\-\_frame\-\_base} attribute, whose value is a location
description that computes the “frame base” for the
subroutine or entry point. If the location description is
a simple register location description, the given register
contains the frame base address. If the location description is
a DWARF expression, the result of evaluating that expression
is the frame base address. Finally, for a location list,
this interpretation applies to each location description
contained in the list of location list entries.

\textit{The use of one of the \livelink{chap:DWOPreg}{DW\-\_OP\-\_reg}~\textless~n~\textgreater 
operations in this
context is equivalent to using 
\livelink{chap:DWOPbreg}{DW\-\_OP\-\_breg}~\textless~n~\textgreater(0) 
but more
compact. However, these are not equivalent in general.}

\textit{The frame base for a procedure is typically an address fixed
relative to the first unit of storage allocated for the
procedure’s stack frame. The \livelink{chap:DWATframebase}{DW\-\_AT\-\_frame\-\_base} attribute
can be used in several ways:}

\begin{enumerate}[1.]
\item \textit{In procedures that need location lists to locate local
variables, the \livelink{chap:DWATframebase}{DW\-\_AT\-\_frame\-\_base} can hold the needed location
list, while all variables’ location descriptions can be
simpler ones involving the frame base.}

\item \textit{It can be used in resolving ``up\dash level'' addressing
within nested routines. 
(See also \livelink{chap:DWATstaticlink}{DW\-\_AT\-\_static\-\_link}, below)}
%The -See also- here is ok, the DW\-\_AT should be
%a hyperref to the def itself, which is earlier in this document.
\end{enumerate}

\textit{Some languages support nested subroutines. In such languages,
it is possible to reference the local variables of an
outer subroutine from within an inner subroutine. The
\livelink{chap:DWATstaticlink}{DW\-\_AT\-\_static\-\_link} and \livelink{chap:DWATframebase}{DW\-\_AT\-\_frame\-\_base} attributes allow
debuggers to support this same kind of referencing.}

If 
\hypertarget{chap:DWATstaticlinklocationofuplevelframe}
a subroutine or entry point is nested, it may have a
\livelink{chap:DWATstaticlink}{DW\-\_AT\-\_static\-\_link}
attribute, whose value is a location
description that computes the frame base of the relevant
instance of the subroutine that immediately encloses the
subroutine or entry point.

In the context of supporting nested subroutines, the
\livelink{chap:DWATframebase}{DW\-\_AT\-\_frame\-\_base} attribute value should obey the following
constraints:

\begin{enumerate}[1.]
\item It should compute a value that does not change during the
life of the procedure, and

\item The computed value should be unique among instances of
the same subroutine. (For typical \livelink{chap:DWATframebase}{DW\-\_AT\-\_frame\-\_base} use, this
means that a recursive subroutine’s stack frame must have
non\dash zero size.)
\end{enumerate}

\textit{If a debugger is attempting to resolve an up\dash level reference
to a variable, it uses the nesting structure of DWARF to
determine which subroutine is the lexical parent and the
\livelink{chap:DWATstaticlink}{DW\-\_AT\-\_static\-\_link} value to identify the appropriate active
frame of the parent. It can then attempt to find the reference
within the context of the parent.}



\subsection{Types Thrown by Exceptions}
\label{chap:typesthrownbyexceptions}

\textit{In C++ a subroutine may declare a set of types which
it may validly throw.}

If a subroutine explicitly declares that it may throw
an exception for one or more types, each such type is
represented by a debugging information entry with the tag
\livetarg{chap:DWTAGthrowntype}{DW\-\_TAG\-\_thrown\-\_type}.  
Each such entry is a child of the entry
representing the subroutine that may throw this type. Each
thrown type entry contains a \livelink{chap:DWATtype}{DW\-\_AT\-\_type} attribute, whose
value is a reference to an entry describing the type of the
exception that may be thrown.

\subsection{Function Template Instantiations}
\label{chap:functiontemplateinstantiations}

\textit{In C++, a function template is a generic definition of
a function that is instantiated differently when called with
values of different types. DWARF does not represent the generic
template definition, but does represent each instantiation.}

A template instantiation is represented by a debugging
information entry with the tag \livelink{chap:DWTAGsubprogram}{DW\-\_TAG\-\_subprogram}. With four
exceptions, such an entry will contain the same attributes and
will have the same types of child entries as would an entry
for a subroutine defined explicitly using the instantiation
types. The exceptions are:

\begin{enumerate}[1.]
\item Each formal parameterized type declaration appearing in the
template definition is represented by a debugging information
entry with the 
tag \livetarg{chap:DWTAGtemplatetypeparameter}{DW\-\_TAG\-\_template\-\_type\-\_parameter}. 
Each
such entry has a \livelink{chap:DWATname}{DW\-\_AT\-\_name} attribute, whose value is a
null\dash terminated string containing the name of the formal
type parameter as it appears in the source program. The
template type parameter entry also has a \livelink{chap:DWATtype}{DW\-\_AT\-\_type} attribute
describing the actual type by which the formal is replaced
for this instantiation.

\item The subprogram entry and each of its child entries reference
a template type parameter entry in any circumstance where
the template definition referenced a formal parameterized type.

\item If the compiler has generated a special compilation unit
to hold the template instantiation and that compilation unit
has a different name from the compilation unit containing
the template definition, the name attribute for the debugging
information entry representing that compilation unit is empty
or omitted.

\item If the subprogram entry representing the template
instantiation or any of its child entries contain declaration
coordinate attributes, those attributes refer to the source
for the template definition, not to any source generated
artificially by the compiler for this instantiation.
\end{enumerate}



\subsection{Inlinable and Inlined Subroutines}
A declaration or a definition of an inlinable subroutine
is represented by a debugging information entry with the
tag 
\livelink{chap:DWTAGsubprogram}{DW\-\_TAG\-\_subprogram}.
The entry for a subroutine that is
\hypertarget{chap:DWATinlineinlinedsubroutine}
explicitly declared to be available for inline expansion or
that was expanded inline implicitly by the compiler has a
\livelink{chap:DWATinline}{DW\-\_AT\-\_inline} attribute whose value is an integer constant. The
set of values for the \livelink{chap:DWATinline}{DW\-\_AT\-\_inline} attribute is given in
Figure \refersec{fig:inlinecodes}.

\begin{figure}[here]
\centering
\caption{Inline codes}
\label{fig:inlinecodes}
\begin{tabular}{lp{9cm}}
Name&Meaning\\ \hline
\livetarg{chap:DWINLnotinlined}{DW\-\_INL\-\_not\-\_inlined} & Not delared inline nor inlined by the
  compiler(equivalent to the absense of the containing
  \livelink{chap:DWATinline}{DW\-\_AT\-\_inline} attribute) \\
\livetarg{chap:DWINLinlined}{DW\-\_INL\-\_inlined} & Not declared inline but inlined by the compiler \\
\livetarg{chap:DWINLdeclarednotinlined}{DW\-\_INL\-\_declared\-\_not\-\_inlined} & Declared inline but 
  not inlined by the compiler \\
\livetarg{chap:DWINLdeclaredinlined}{DW\-\_INL\-\_declared\-\_inlined} & Declared inline and inlined by the compiler \\
\end{tabular}
\end{figure}

\textit{In C++, a function or a constructor declared with
constexpr is implicitly declared inline. The abstract inline
instance (see below) is represented by a debugging information
entry with the tag \livelink{chap:DWTAGsubprogram}{DW\-\_TAG\-\_subprogram}. Such an entry has a
\livelink{chap:DWATinline}{DW\-\_AT\-\_inline} attribute whose value is \livelink{chap:DWINLinlined}{DW\-\_INL\-\_inlined}.}


\paragraph{Abstract Instances}
\label{chap:abstractinstances}
Any debugging information entry that is owned (either
\hypertarget{chap:DWATinlineabstracttinstance}
directly or indirectly) by a debugging information entry
that contains the 
\livelink{chap:DWATinline}{DW\-\_AT\-\_inline} attribute is referred to
as an ``abstract instance entry.'' Any subroutine entry
that contains a \livelink{chap:DWATinline}{DW\-\_AT\-\_inline} attribute whose value is other
than \livelink{chap:DWINLnotinlined}{DW\-\_INL\-\_not\-\_inlined} is known as 
an ``abstract instance root.'' 
Any set of abstract instance entries that are all
children (either directly or indirectly) of some abstract
instance root, together with the root itself, is known as
an ``abstract instance tree.'' However, in the case where
an abstract instance tree is nested within another abstract
instance tree, the entries in the nested abstract instance
tree are not considered to be entries in the outer abstract
instance tree.

Each abstract instance root is either part of a larger
tree (which gives a context for the root) or uses
\livelink{chap:DWATspecification}{DW\-\_AT\-\_specification} to refer to the declaration in context.

\textit{For example, in C++ the context might be a namespace
declaration or a class declaration.}

\textit{Abstract instance trees are defined so that no entry is part
of more than one abstract instance tree. This simplifies the
following descriptions.}

A debugging information entry that is a member of an abstract
instance tree should not contain any attributes which describe
aspects of the subroutine which vary between distinct inlined
expansions or distinct out\dash of\dash line expansions. For example,
the \livelink{chap:DWATlowpc}{DW\-\_AT\-\_low\-\_pc},
\livelink{chap:DWAThighpc}{DW\-\_AT\-\_high\-\_pc}, \livelink{chap:DWATranges}{DW\-\_AT\-\_ranges}, 
\livelink{chap:DWATentrypc}{DW\-\_AT\-\_entry\-\_pc}, \livelink{chap:DWATlocation}{DW\-\_AT\-\_location},
\livelink{chap:DWATreturnaddr}{DW\-\_AT\-\_return\-\_addr}, \livelink{chap:DWATstartscope}{DW\-\_AT\-\_start\-\_scope}, and 
\livelink{chap:DWATsegment}{DW\-\_AT\-\_segment}
attributes typically should be omitted; however, this list
is not exhaustive.

\textit{It would not make sense normally to put these attributes into
abstract instance entries since such entries do not represent
actual (concrete) instances and thus do not actually exist at
run\dash time.  However, 
see Appendix \refersec{app:inlineouteronenormalinner} 
for a contrary example.}

The rules for the relative location of entries belonging to
abstract instance trees are exactly the same as for other
similar types of entries that are not abstract. Specifically,
the rule that requires that an entry representing a declaration
be a direct child of the entry representing the scope of the
declaration applies equally to both abstract and non\dash abstract
entries. Also, the ordering rules for formal parameter entries,
member entries, and so on, all apply regardless of whether
or not a given entry is abstract.

\paragraph{Concrete Inlined Instances}
\label{chap:concreteinlinedinstances}

Each inline expansion of a subroutine is represented
by a debugging information entry with the 
tag \livetarg{chap:DWTAGinlinedsubroutine}{DW\-\_TAG\-\_inlined\-\_subroutine}. 
Each such entry should be a direct
child of the entry that represents the scope within which
the inlining occurs.

Each inlined subroutine entry may have either a 
\livelink{chap:DWATlowpc}{DW\-\_AT\-\_low\-\_pc}
and \livelink{chap:DWAThighpc}{DW\-\_AT\-\_high\-\_pc} pair of attributes or a \livelink{chap:DWATranges}{DW\-\_AT\-\_ranges}
attribute whose values encode the contiguous or non\dash contiguous
address ranges, respectively, of the machine instructions
generated for the inlined subroutine (see 
Section \refersec{chap:codeaddressesandranges}). 
An
\hypertarget{chap:DWATentrypcentryaddressofinlinedsubprogram}
inlined subroutine entry may also contain a 
\livelink{chap:DWATentrypc}{DW\-\_AT\-\_entry\-\_pc}
attribute, representing the first executable instruction of
the inline expansion (see 
Section \refersec{chap:entryaddress}).

% Positions of the 3 targets here is a bit arbitrary.
An inlined 
\hypertarget{chap:DWATcalllinelinenumberofinlinedsubroutinecall}
subroutine 
\hypertarget{chap:DWATcallcolumncolumnpositionofinlinedsubroutinecall}
entry 
\hypertarget{chap:DWATcallfilefilecontaininginlinedsubroutinecall}
may also have \livelink{chap:DWATcallfile}{DW\-\_AT\-\_call\-\_file},
\livelink{chap:DWATcallline}{DW\-\_AT\-\_call\-\_line} and \livelink{chap:DWATcallcolumn}{DW\-\_AT\-\_call\-\_column} attributes, 
each of whose
value is an integer constant. These attributes represent the
source file, source line number, and source column number,
respectively, of the first character of the statement or
expression that caused the inline expansion. The call file,
call line, and call column attributes are interpreted in
the same way as the declaration file, declaration line, and
declaration column attributes, respectively (see 
Section \refersec{chap:declarationcoordinates}).

The call file, call line and call column coordinates do not
describe the coordinates of the subroutine declaration that
was inlined, rather they describe the coordinates of the call.

An inlined subroutine entry 
\hypertarget{chap:DWATconstexprcompiletimeconstantfunction}
may have a 
\livelink{chap:DWATconstexpr}{DW\-\_AT\-\_const\-\_expr}
attribute, which is a \livelink{chap:flag}{flag} 
whose presence indicates that the
subroutine has been evaluated as a compile\dash time constant. Such
an entry may also have a \livelink{chap:DWATconstvalue}{DW\-\_AT\-\_const\-\_value} attribute,
whose value may be of any form that is appropriate for the
representation of the subroutine's return value. The value of
this attribute is the actual return value of the subroutine,
represented as it would be on the target architecture.

\textit{In C++, if a function or a constructor declared with constexpr
is called with constant expressions, then the corresponding
concrete inlined instance has a \livelink{chap:DWATconstexpr}{DW\-\_AT\-\_const\-\_expr} attribute,
as well as a \livelink{chap:DWATconstvalue}{DW\-\_AT\-\_const\-\_value} attribute whose value represents
the actual return value of the concrete inlined instance.}

Any debugging information entry that is owned (either
directly or indirectly) by a debugging information entry
with the tag \livelink{chap:DWTAGinlinedsubroutine}{DW\-\_TAG\-\_inlined\-\_subroutine} is referred to as a
``concrete inlined instance entry.'' Any entry that has
the tag \livelink{chap:DWTAGinlinedsubroutine}{DW\-\_TAG\-\_inlined\-\_subroutine} 
is known as a ``concrete inlined instance root.'' Any set of concrete inlined instance
entries that are all children (either directly or indirectly)
of some concrete inlined instance root, together with the root
itself, is known as a ``concrete inlined instance tree.''
However, in the case where a concrete inlined instance tree
is nested within another concrete instance tree, the entries
in the nested concrete instance tree are not considered to
be entries in the outer concrete instance tree.

\textit{Concrete inlined instance trees are defined so that no entry
is part of more than one concrete inlined instance tree. This
simplifies later descriptions.}

Each concrete inlined instance tree is uniquely associated
with one (and only one) abstract instance tree.

\textit{Note, however, that the reverse is not true. Any given abstract
instance tree may be associated with several different concrete
inlined instance trees, or may even be associated with zero
concrete inlined instance trees.}

Concrete inlined instance entries may omit attributes that
are not specific to the concrete instance (but present in
the abstract instance) and need include only attributes that
are specific to the concrete instance (but omitted in the
abstract instance). In place of these omitted attributes, each
\hypertarget{chap:DWATabstractorigininlineinstance}
concrete inlined instance entry 
has a 
\livelink{chap:DWATabstractorigin}{DW\-\_AT\-\_abstract\-\_origin}
attribute that may be used to obtain the missing information
(indirectly) from the associated abstract instance entry. The
value of the abstract origin attribute is a reference to the
associated abstract instance entry.

If an entry within a concrete inlined instance tree contains
attributes describing the declaration coordinates of that
entry, then those attributes should refer to the file, line
and column of the original declaration of the subroutine,
not to the point at which it was inlined. As a consequence,
they may usually be omitted from any entry that has an abstract
origin attribute.

For each pair of entries that are associated via a
\livelink{chap:DWATabstractorigin}{DW\-\_AT\-\_abstract\-\_origin} attribute, both members of the pair
have the same tag. So, for example, an entry with the tag
\livelink{chap:DWTAGvariable}{DW\-\_TAG\-\_variable} can only be associated with another entry
that also has the tag \livelink{chap:DWTAGvariable}{DW\-\_TAG\-\_variable}. The only exception
to this rule is that the root of a concrete instance tree
(which must always have the tag \livelink{chap:DWTAGinlinedsubroutine}{DW\-\_TAG\-\_inlined\-\_subroutine})
can only be associated with the root of its associated abstract
instance tree (which must have the tag \livelink{chap:DWTAGsubprogram}{DW\-\_TAG\-\_subprogram}).

In general, the structure and content of any given concrete
inlined instance tree will be closely analogous to the
structure and content of its associated abstract instance
tree. There are a few exceptions:

\begin{enumerate}[1.]
\item An entry in the concrete instance tree may be omitted if
it contains only a \livelink{chap:DWATabstractorigin}{DW\-\_AT\-\_abstract\-\_origin} attribute and either
has no children, or its children are omitted. Such entries
would provide no useful information. In C\dash like languages,
such entries frequently include types, including structure,
union, class, and interface types; and members of types. If any
entry within a concrete inlined instance tree needs to refer
to an entity declared within the scope of the relevant inlined
subroutine and for which no concrete instance entry exists,
the reference should refer to the abstract instance entry.

\item Entries in the concrete instance tree which are associated
with entries in the abstract instance tree such that neither
has a \livelink{chap:DWATname}{DW\-\_AT\-\_name} attribute, and neither is referenced by
any other debugging information entry, may be omitted. This
may happen for debugging information entries in the abstract
instance trees that became unnecessary in the concrete instance
tree because of additional information available there. For
example, an anonymous variable might have been created and
described in the abstract instance tree, but because of
the actual parameters for a particular inlined expansion,
it could be described as a constant value without the need
for that separate debugging information entry.

\item A concrete instance tree may contain entries which do
not correspond to entries in the abstract instance tree
to describe new entities that are specific to a particular
inlined expansion. In that case, they will not have associated
entries in the abstract instance tree, should not contain
\livelink{chap:DWATabstractorigin}{DW\-\_AT\-\_abstract\-\_origin} attributes, and must contain all their
own attributes directly. This allows an abstract instance tree
to omit debugging information entries for anonymous entities
that are unlikely to be needed in most inlined expansions. In
any expansion which deviates from that expectation, the
entries can be described in its concrete inlined instance tree.

\end{enumerate}

\paragraph{Out-of-Line Instances of Inlined Subroutines}
\label{chap:outoflineinstancesofinlinedsubroutines}
Under some conditions, compilers may need to generate concrete
executable instances of inlined subroutines other than at
points where those subroutines are actually called. Such
concrete instances of inlined subroutines are referred to as
``concrete out\dash of\dash line instances.''

\textit{In C++, for example, taking the address of a function declared
to be inline can necessitate the generation of a concrete
out\dash of\dash line instance of the given function.}

The DWARF representation of a concrete out\dash of\dash line instance
of an inlined subroutine is essentially the same as for a
concrete inlined instance of that subroutine (as described in
the preceding section). The representation of such a concrete
% It is critical that the hypertarget and livelink be
% separated to avoid problems with latex.
out\dash of\dash line instance 
\hypertarget{chap:DWATabstractoriginoutoflineinstance}
makes use of 
\livelink{chap:DWATabstractorigin}{DW\-\_AT\-\_abstract\-\_origin}
attributes in exactly the same way as they are used for
a concrete inlined instance (that is, as references to
corresponding entries within the associated abstract instance
tree).

The differences between the DWARF representation of a
concrete out\dash of\dash line instance of a given subroutine and the
representation of a concrete inlined instance of that same
subroutine are as follows:

\begin{enumerate}[1.]
\item  The root entry for a concrete out\dash of\dash line instance
of a given inlined subroutine has the same tag as does its
associated (abstract) inlined subroutine entry (that is, tag
\livelink{chap:DWTAGsubprogram}{DW\-\_TAG\-\_subprogram} rather than \livelink{chap:DWTAGinlinedsubroutine}{DW\-\_TAG\-\_inlined\-\_subroutine}).

\item The root entry for a concrete out\dash of\dash line instance tree
is normally owned by the same parent entry that also owns
the root entry of the associated abstract instance. However,
it is not required that the abstract and out\dash of\dash line instance
trees be owned by the same parent entry.

\end{enumerate}

\paragraph{Nested Inlined Subroutines}
\label{nestedinlinedsubroutines}
Some languages and compilers may permit the logical nesting of
a subroutine within another subroutine, and may permit either
the outer or the nested subroutine, or both, to be inlined.

For a non\dash inlined subroutine nested within an inlined
subroutine, the nested subroutine is described normally in
both the abstract and concrete inlined instance trees for
the outer subroutine. All rules pertaining to the abstract
and concrete instance trees for the outer subroutine apply
also to the abstract and concrete instance entries for the
nested subroutine.

For an inlined subroutine nested within another inlined
subroutine, the following rules apply to their abstract and
concrete instance trees:

\begin{enumerate}[1.]
\item The abstract instance tree for the nested subroutine is
described within the abstract instance tree for the outer
subroutine according to the rules in 
Section \refersec{chap:abstractinstances}, and
without regard to the fact that it is within an outer abstract
instance tree.

\item Any abstract instance tree for a nested subroutine is
always omitted within the concrete instance tree for an
outer subroutine.

\item  A concrete instance tree for a nested subroutine is
always omitted within the abstract instance tree for an
outer subroutine.

\item The concrete instance tree for any inlined or out-of-line
expansion of the nested subroutine is described within a
concrete instance tree for the outer subroutine according
to the rules in 
Sections \refersec{chap:concreteinlinedinstances} or 
\refersec{chap:outoflineinstancesofinlinedsubroutines}
, respectively,
and without regard to the fact that it is within an outer
concrete instance tree.
\end{enumerate}

See Appendix \refersec{app:inliningexamples} 
for discussion and examples.

\subsection{Trampolines}
\label{chap:trampolines}

\textit{A trampoline is a compiler\dash generated subroutine that serves as
an intermediary in making a call to another subroutine. It may
adjust parameters and/or the result (if any) as appropriate
to the combined calling and called execution contexts.}

A trampoline is represented by a debugging information entry
with the tag \livelink{chap:DWTAGsubprogram}{DW\-\_TAG\-\_subprogram} or \livelink{chap:DWTAGinlinedsubroutine}{DW\-\_TAG\-\_inlined\-\_subroutine}
that has a \livelink{chap:DWATtrampoline}{DW\-\_AT\-\_trampoline} attribute. The value of that
attribute indicates the target subroutine of the trampoline,
that is, the subroutine to which the trampoline passes
control. (A trampoline entry may but need not also have a
\livelink{chap:DWATartificial}{DW\-\_AT\-\_artificial} attribute.)

The value of the trampoline attribute may be represented
using any of the following forms, which are listed in order
of preference:

\begin{itemize}
\item If the value is of class reference, then the value
specifies the debugging information entry of the target
subprogram.

\item If the value is of class address, then the value is
the relocated address of the target subprogram.

\item If the value is of class string, then the value is the
(possibly mangled) name of the target subprogram.

\item If the value is of class \livelink{chap:flag}{flag}, then the value true
indicates that the containing subroutine is a trampoline but
that the target subroutine is not known.
\end{itemize}


The target subprogram may itself be a trampoline. (A sequence
of trampolines necessarily ends with a non\dash trampoline
subprogram.)

\textit{In C++, trampolines may be used to implement derived virtual
member functions; such trampolines typically adjust the
implicit this pointer parameter in the course of passing
control.  Other languages and environments may use trampolines
in a manner sometimes known as transfer functions or transfer
vectors.}

\textit{Trampolines may sometimes pass control to the target
subprogram using a branch or jump instruction instead of a
call instruction, thereby leaving no trace of their existence
in the subsequent execution context. }

\textit{This attribute helps make it feasible for a debugger to arrange
that stepping into a trampoline or setting a breakpoint in
a trampoline will result in stepping into or setting the
breakpoint in the target subroutine instead. This helps to
hide the compiler generated subprogram from the user. }

\textit{If the target subroutine is not known, a debugger may choose
to repeatedly step until control arrives in a new subroutine
which can be assumed to be the target subroutine. }



\section{Lexical Block Entries}
\label{chap:lexicalblockentries}

\textit{A lexical \livetargi{chap:lexicalblock}{block}{lexical block} is a bracketed sequence of source statements
that may contain any number of declarations. In some languages
(including C and C++), \nolink{blocks} can be nested within other
\nolink{blocks} to any depth.}

% We do not need to link to the preceeding paragraph.
A lexical \nolink{block} is represented by a debugging information
entry with the 
tag \livetarg{chap:DWTAGlexicalblock}{DW\-\_TAG\-\_lexical\-\_block}.

The lexical \livetargi{chap:lexicalblockentry}{block}{lexical block entry} entry
may have 
either a \livelink{chap:DWATlowpc}{DW\-\_AT\-\_low\-\_pc} and
\livelink{chap:DWAThighpc}{DW\-\_AT\-\_high\-\_pc} pair of 
attributes or a \livelink{chap:DWATranges}{DW\-\_AT\-\_ranges} attribute
whose values encode the contiguous or non-contiguous address
ranges, respectively, of the machine instructions generated
for the lexical \livelink{chap:lexicalblock}{block} 
(see Section \refersec{chap:codeaddressesandranges}).

If a name has been given to the 
lexical \livelink{chap:lexicalblock}{block} 
in the source
program, then the corresponding 
lexical \livelink{chap:lexicalblockentry}{block} entry has a
\livelink{chap:DWATname}{DW\-\_AT\-\_name} attribute whose 
value is a null\dash terminated string
containing the name of the lexical \livelink{chap:lexicalblock}{block} 
as it appears in
the source program.

\textit{This is not the same as a C or C++ label (see below).}

The lexical \livelink{chap:lexicalblockentry}{block} entry owns 
debugging information entries that
describe the declarations within that lexical \livelink{chap:lexicalblock}{block}. 
There is
one such debugging information entry for each local declaration
of an identifier or inner lexical \livelink{chap:lexicalblock}{block}.

\section{Label Entries}
\label{chap:labelentries}

A label is a way of identifying a source statement. A labeled
statement is usually the target of one or more ``go to''
statements.

A label is represented by a debugging information entry with
the 
tag \livetarg{chap:DWTAGlabel}{DW\-\_TAG\-\_label}. 
The entry for a label should be owned by
the debugging information entry representing the scope within
which the name of the label could be legally referenced within
the source program.

The label entry has a \livelink{chap:DWATlowpc}{DW\-\_AT\-\_low\-\_pc} attribute whose value
is the relocated address of the first machine instruction
generated for the statement identified by the label in
the source program.  The label entry also has a \livelink{chap:DWATname}{DW\-\_AT\-\_name}
attribute whose value is a null-terminated string containing
the name of the label as it appears in the source program.


\section{With Statement Entries}
\label{chap:withstatemententries}

\textit{Both Pascal and Modula\dash 2 support the concept of a ``with''
statement. The with statement specifies a sequence of
executable statements within which the fields of a record
variable may be referenced, unqualified by the name of the
record variable.}

A with statement is represented by a debugging information
entry with the tag \livetarg{chap:DWTAGwithstmt}{DW\-\_TAG\-\_with\-\_stmt}.

A with statement entry may have either a \livelink{chap:DWATlowpc}{DW\-\_AT\-\_low\-\_pc} and
\livelink{chap:DWAThighpc}{DW\-\_AT\-\_high\-\_pc} pair of attributes or a \livelink{chap:DWATranges}{DW\-\_AT\-\_ranges} attribute
whose values encode the contiguous or non\dash contiguous address
ranges, respectively, of the machine instructions generated
for the with statement 
(see Section \refersec{chap:codeaddressesandranges}).

The with statement entry has a \livelink{chap:DWATtype}{DW\-\_AT\-\_type} attribute, denoting
the type of record whose fields may be referenced without full
qualification within the body of the statement. It also has
a \livelink{chap:DWATlocation}{DW\-\_AT\-\_location} attribute, describing how to find the base
address of the record object referenced within the body of
the with statement.

\section{Try and Catch Block Entries}
\label{chap:tryandcatchblockentries}

\textit{In C++ a lexical \livelink{chap:lexicalblock}{block} may be 
designated as a ``catch \nolink{block}.'' 
A catch \livetargi{chap:catchblock}{block}{catch block} is an 
exception handler that handles
exceptions thrown by an immediately 
preceding ``try \livelink{chap:tryblock}{block}.''
A catch \livelink{chap:catchblock}{block} 
designates the type of the exception that it
can handle.}

A try \livetargi{chap:tryblock}{block}{try block} is represented 
by a debugging information entry
with the tag \livetarg{chap:DWTAGtryblock}{DW\-\_TAG\-\_try\-\_block}.  
A catch \livelink{chap:catchblock}{block} is represented by
a debugging information entry with 
the tag \livetarg{chap:DWTAGcatchblock}{DW\-\_TAG\-\_catch\-\_block}.

% nolink as we have links just above and do not have a combo link for both
Both try and catch \nolink{block} entries may have either a
\livelink{chap:DWATlowpc}{DW\-\_AT\-\_low\-\_pc} and \livelink{chap:DWAThighpc}{DW\-\_AT\-\_high\-\_pc} pair of attributes or a
\livelink{chap:DWATranges}{DW\-\_AT\-\_ranges} attribute whose values encode the contiguous
or non\dash contiguous address ranges, respectively, of the
machine instructions generated for the \livelink{chap:lexicalblock}{block}
(see Section
\refersec{chap:codeaddressesandranges}).

Catch \livelink{chap:catchblock}{block} entries have at 
least one child entry, an
entry representing the type of exception accepted by
that catch \livelink{chap:catchblock}{block}. 
This child entry has one of the tags
\livelink{chap:DWTAGformalparameter}{DW\-\_TAG\-\_formal\-\_parameter} or \livelink{chap:DWTAGunspecifiedparameters}{DW\-\_TAG\-\_unspecified\-\_parameters},
and will have the same form as other parameter entries.

The siblings immediately following 
a try \livelink{chap:tryblock}{block} entry are its
corresponding catch \livelink{chap:catchblock}{block} entries.







