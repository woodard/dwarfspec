\chapter{Other Debugging Information}
\label{chap:otherdebugginginformation}
% references to chapter 7 look like  {datarep:...}
This section describes debugging information that is not
represented in the form of debugging information entries and
is not contained within a \addtoindex{.debug\_info} or 
\addtoindex{.debug\_types} section.

In the descriptions that follow, these terms are used to
specify the representation of DWARF sections:

Initial length, section offset and section length, which are
defined in 
Sections \refersec{datarep:locationdescriptions} and 
\refersec{datarep:32bitand64bitdwarfformats}.

Sbyte, ubyte, uhalf, and uword, which are defined in 
Section \refersec{datarep:integerrepresentationnames}.

\section{Accelerated Access}
\label{chap:acceleratedaccess}

\textit{A debugger frequently needs to find the debugging information
\addtoindexx{accelerated access}
for a program entity defined outside of the compilation unit
where the debugged program is currently stopped. Sometimes the
debugger will know only the name of the entity; sometimes only
the address. To find the debugging information associated with
a global entity by name, using the DWARF debugging information
entries alone, a debugger would need to run through all
entries at the highest scope within each compilation unit.}

\textit{Similarly, in languages in which the name of a type is
required to always refer to the same concrete type (such as
C++), a compiler may choose to elide type definitions in
all compilation units except one. In this case a debugger
needs a rapid way of locating the concrete type definition
by name. As with the definition of global data objects, this
would require a search of all the top level type definitions
of all compilation units in a program.}

\textit{To find the debugging information associated with a subroutine,
given an address, a debugger can use the low and high pc
attributes of the compilation unit entries to quickly narrow
down the search, but these attributes only cover the range
of addresses for the text associated with a compilation unit
entry. To find the debugging information associated with a
data object, given an address, an exhaustive search would be
needed. Furthermore, any search through debugging information
entries for different compilation units within a large program
would potentially require the access of many memory pages,
probably hurting debugger performance.}

To make lookups of program entities (data objects, functions
and types) by name or by address faster, a producer of DWARF
information may provide three different types of tables
containing information about the debugging information
entries owned by a particular compilation unit entry in a
more condensed format.

\subsection{Lookup by Name}

For lookup by name, 
\addtoindexx{lookup!by name}
two tables are maintained in separate
\addtoindex{accelerated access!by name}
object file sections named 
\addtoindex{.debug\_pubnames} for objects and
functions, and 
\addtoindex{.debug\_pubtypes}
for types. Each table consists
of sets of variable length entries. Each set describes the
names of global objects and functions, or global types,
respectively, whose definitions are represented by debugging
information entries owned by a single compilation unit.

\textit{C++ member functions with a definition in the class declaration
are definitions in every compilation unit containing the
class declaration, but if there is no concrete out\dash of\dash line
instance there is no need to have a 
\addtoindex{.debug\_pubnames} entry
for the member function.}

Each set begins with a header containing four values:
\begin{enumerate}[1.]

\item unit\_length (initial length) \\
The total length of the all of the entries for that set,
not including the length field itself 
(see Section \refersec{datarep:locationdescriptions}).

\item  version (uhalf) \\
A version number\addtoindexx{version number!name lookup table}\addtoindexx{version number!type lookup table} 
(see Section \refersec{datarep:namelookuptables}). 
This number is specific
to the name lookup table and is independent of the DWARF
version number.

\item debug\_info\_offset (section offset) \\
The offset from the beginning of the 
\addtoindex{.debug\_info} section of
the compilation unit header referenced by the set.

\item debug\_info\_length (section length) \\
The size in bytes of the contents of the 
\addtoindex{.debug\_info} section
generated to represent that compilation unit.
\end{enumerate}

This header is followed by a variable number of offset/name
pairs. Each pair consists of the section offset from the
beginning of the compilation unit corresponding to the current
set to the debugging information entry for the given object,
followed by a null\dash terminated character string representing
the name of the object as given by the \livelink{chap:DWATname}{DW\-\_AT\-\_name} attribute
of the referenced debugging information entry. Each set of
names is terminated by an offset field containing zero (and
no following string).


In the case of the name of a function member or static data
member of a C++ structure, class or union, the name presented
in the 
\addtoindex{.debug\_pubnames} 
section is not the simple name given
by the \livelink{chap:DWATname}{DW\-\_AT\-\_name} attribute of the referenced debugging
information entry, but rather the fully qualified name of
the data or function member.

\subsection{Lookup by Address}

For 
\addtoindexx{lookup!by address}
lookup by address, a table is maintained in a separate
\addtoindex{accelerated access!by address}
object file section called 
\addtoindex{.debug\_aranges}. The table consists
of sets of variable length entries, each set describing the
portion of the program’s address space that is covered by
a single compilation unit.

Each set begins with a header containing five values:

\begin{enumerate}[1.]

\item unit\_length (initial length) \\
The total length of all of the
entries for that set, not including the length field itself
(see Section \refersec{datarep:initiallengthvalues}).

\item version (uhalf) \\
A version number\addtoindexx{version number!address lookup table} 
(see Appendix \refersec{app:dwarfsectionversionnumbersinformative}). 
This number is specific to the address lookup table and is
independent of the DWARF version number.

\item debug\_info\_offset (section offset) \\
The offset from the
beginning of the \addtoindex{.debug\_info} or 
\addtoindex{.debug\_types} section of the
compilation unit header referenced by the set.

\item address\_size (ubyte) \\
The size of an address in bytes on
\addtoindexx{address\_size}
the target architecture. For 
\addtoindexx{address space!segmented}
segmented addressing, this is
the size of the offset portion of the address.

\item segment\_size (ubyte) \\
The size of a segment selector in
bytes on the target architecture. If the target system uses
a flat address space, this value is 0.

\end{enumerate}


This header is followed by a variable number of address range
descriptors. Each descriptor is a triple consisting of a
segment selector, the beginning address within that segment
of a range of text or data covered by some entry owned by
the corresponding compilation unit, followed by the non\dash zero
length of that range. A particular set is terminated by an
entry consisting of three zeroes. When the segment\_size value
is zero in the header, the segment selector is omitted so that
each descriptor is just a pair, including the terminating
entry. By scanning the table, a debugger can quickly decide
which compilation unit to look in to find the debugging
information for an object that has a given address.

\textit{If the range of addresses covered by the text and/or data
of a compilation unit is not contiguous, then there may be
multiple address range descriptors for that compilation unit.}




\section{Line Number Information}
\label{chap:linenumberinformation}
\textit{A source\dash level debugger will need to know how to
%FIXME: the see here is not 'see also'. Fix?
\addtoindexx{line number information|see{statement list attribute}}
associate locations in the source files with the corresponding
machine instruction addresses in the executable object or
the shared objects used by that executable object. Such an
association would make it possible for the debugger user
to specify machine instruction addresses in terms of source
locations. This would be done by specifying the line number
and the source file containing the statement. The debugger
can also use this information to display locations in terms
of the source files and to single step from line to line,
or statement to statement.}

Line number information generated for a compilation unit is
represented in the 
\addtoindex{.debug\_line} section of an object file and
is referenced by a corresponding compilation unit debugging
information entry 
(see Section \refersec{chap:generalsubroutineandentrypointinformation}) 
in the \addtoindex{.debug\_info}
section.

\textit{Some computer architectures employ more than one instruction
set (for example, the ARM 
\addtoindexx{ARM instruction set architecture}
and 
MIPS architectures support
\addtoindexx{MIPS instruction set architecture}
a 32\dash bit as well as a 16\dash bit instruction set). Because the
instruction set is a function of the program counter, it is
convenient to encode the applicable instruction set in the
\addtoindex{.debug\_line} section as well.}

\textit{If space were not a consideration, the information provided
in the \addtoindex{.debug\_line} 
section could be represented as a large
matrix, with one row for each instruction in the emitted
object code. The matrix would have columns for:}

\begin{itemize}
\item \textit{the source file name}
\item \textit{the source line number}
\item \textit{the source column number}
\item \textit{whether this insruction is the beginning of a \addtoindex{basic block}}
\item \textit{and so on}
\end{itemize}

\textit{Such a matrix, however, would be impractically large. We
shrink it with two techniques. First, we delete from
the matrix each row whose file, line, source column and
\addtoindex{discriminator} information 
is identical with that of its
predecessors. Any deleted row would never be the beginning of
a source statement. Second, we design a byte\dash coded language
for a state machine and store a stream of bytes in the object
file instead of the matrix. This language can be much more
compact than the matrix. When a consumer of the line number
information executes, it must ``run'' the state machine
to generate the matrix for each compilation unit it is
interested in.  The concept of an encoded matrix also leaves
room for expansion. In the future, columns can be added to the
matrix to encode other things that are related to individual
instruction addresses.}

\textit{When the set of addresses of a compilation unit cannot be
described as a single contiguous range, there will be a
separate matrix for each contiguous subrange.}

\subsection{Definitions}

The following terms are used in the description of the line
number information format:


\begin{tabular} {lp{9cm}}
state machine &
The hypothetical machine used by a consumer of the line number
information to expand the byte\dash coded 
instruction stream into a matrix of
line number information. \\

line number program &
A series of byte\dash coded 
line number information instructions representing
one compilation unit. \\

\addtoindex{basic block} &
 A sequence of instructions where only the first instruction may be a
branch target and only the last instruction may transfer control. A
procedure invocation is defined to be an exit from a 
\addtoindex{basic block}.

\textit{A \addtoindex{basic block} does not 
necessarily correspond to a specific source code
construct.} \\

sequence &
A series of contiguous target machine instructions. One compilation unit
may emit multiple sequences (that is, not all instructions within a
compilation unit are assumed to be contiguous). \\
\end{tabular}

\subsection{State Machine Registers}
\label{chap:statemachineregisters}

The line number information state machine has the following 
registers:
\begin{longtable}{l|p{9cm}}
  \caption{State Machine Registers } \\
  \hline \\ \bfseries Register name&\bfseries Meaning\\ \hline
\endfirsthead
  \bfseries Register name&\bfseries Meaning\\ \hline
\endhead
  \hline \emph{Continued on next page}
\endfoot
  \hline
\endlastfoot
\addtoindexi{address}{address register!in line number machine}&
The program\dash counter value corresponding to a machine instruction
generated by the compiler. \\

\addtoindex{op\_index} &
An unsigned integer representing the index of an operation within a VLIW
instruction. The index of the first operation is 0. For non\dash VLIW
architectures, this register will always be 0.

The address and op\_index registers, taken together, form an operation
pointer that can reference any individual operation with the instruction
stream. \\


\addtoindex{file} &
An unsigned integer indicating the identity of the source file
corresponding to a machine instruction. \\

\addtoindex{line} &
An unsigned integer indicating a source line number. Lines are numbered
beginning at 1. The compiler may emit the value 0 in cases where an
instruction cannot be attributed to any source line. \\

\addtoindex{column} &
An unsigned integer indicating a column number within a source line.
Columns are numbered beginning at 1. The value 0 is reserved to indicate
that a statement begins at the ``left edge'' of the line. \\

\addtoindex{is\_stmt} &
A boolean indicating that the current instruction is a recommended
breakpoint location. A recommended breakpoint location 
is intended to ``represent'' a line, a 
statement and/or a semantically distinct subpart of a
statement. \\

\addtoindex{basic\_block}  &
A boolean indicating that the current instruction is the beginning of a
\addtoindex{basic block}. \\

\addtoindex{end\_sequence} &
A boolean indicating that the current address is that of the first byte after
the end of a sequence of target machine instructions. 
\addtoindex{end\_sequence}
terminates a sequence of lines; therefore other information in the same
row is not meaningful. \\

\addtoindex{prologue\_end} &
A boolean indicating that the current address is one (of possibly many)
where execution should be suspended for an entry breakpoint of a
function. \\

\addtoindex{epilogue\_begin} &
A boolean indicating that the current address is one (of possibly many)
where execution should be suspended for an exit breakpoint of a function. \\

\addtoindex{isa} &
An unsigned integer whose value encodes the applicable
instruction set architecture for the current instruction.
The encoding of instruction sets should be shared by all
users of a given architecture. It is recommended that this
encoding be defined by the ABI authoring committee for each
architecture. \\

\addtoindex{discriminator} &
An unsigned integer identifying the block to which the
current instruction belongs. Discriminator values are assigned
arbitrarily by the DWARF producer and serve to distinguish
among multiple blocks that may all be associated with the
same source file, line, and column. Where only one block
exists for a given source position, the discriminator value
should be zero. \\
\end{longtable}

At the beginning  of each sequence within a line number
program, the state of the registers is:

\begin{tabular}{lp{8cm}}
address & 0 \\
op\_index & 0 \\
file & 1 \\
line & 1 \\
column & 0 \\
\addtoindex{is\_stmt} & determined by \addtoindex{default\_is\_stmt} in the line number program header \\
\addtoindex{basic\_block} & ``false'' \addtoindexx{basic block} \\
\addtoindex{end\_sequence} & ``false'' \\
\addtoindex{prologue\_end} & ``false'' \\
\addtoindex{epilogue\_begin} & ``false'' \\
\addtoindex{isa} & 0 \\
discriminator & 0 \\
\end{tabular}

\textit{The 
\addtoindex{isa} value 0 specifies that the instruction set is the
architecturally determined default instruction set. This may
be fixed by the ABI, or it may be specified by other means,
for example, by the object file description.}

\subsection{Line Number Program Instructions}

The state machine instructions in a line number program belong to one of three categories:

\begin{tabular}{lp{10cm}}
special opcodes &
These have a ubyte opcode field and no operands.

\textit{Most of the instructions in a 
line number program are special opcodes.} \\

standard opcodes &
These have a ubyte opcode field which may be followed by zero or more
LEB128 operands (except for 
\livelink{chap:DWLNSfixedadvancepc}{DW\-\_LNS\-\_fixed\-\_advance\-\_pc}, see below).
The opcode implies the number of operands and their meanings, but the
line number program header also specifies the number of operands for
each standard opcode. \\

extended opcodes &
These have a multiple byte format. The first byte is zero; the next bytes
are an unsigned LEB128 integer giving the number of bytes in the
instruction itself (does not include the first zero byte or the size). The
remaining bytes are the instruction itself (which begins with a ubyte
extended opcode). \\
\end{tabular}


\subsection{The Line Number Program Header}

The optimal encoding of line number information depends to a
certain degree upon the architecture of the target machine. The
line number program header provides information used by
consumers in decoding the line number program instructions for
a particular compilation unit and also provides information
used throughout the rest of the line number program.

The line number program for each compilation unit begins with
a header containing the following fields in order:

\begin{enumerate}[1.]
\item unit\_length (initial length)  \\
The size in bytes of the line number information for this
compilation unit, not including the unit\_length field itself
(see Section \refersec{datarep:initiallengthvalues}). 

\item version (uhalf) 
A version number\addtoindexx{version number!line number information} 
(see Appendix \refersec{app:dwarfsectionversionnumbersinformative}). 
This number is specific to
the line number information and is independent of the DWARF
version number. 

\item header\_length  \\
The number of bytes following the \addtoindex{header\_length} field to the
beginning of the first byte of the line number program itself.
In the 32\dash bit DWARF format, this is a 4\dash byte unsigned
length; in the 64\dash bit DWARF format, this field is an
8\dash byte unsigned length 
(see Section \refersec{datarep:32bitand64bitdwarfformats}). 

\item minimum\_instruction\_length (ubyte)  \\
\addtoindexx{minimum\_instruction\_length}
The size in bytes of the smallest target machine
instruction. Line number program opcodes that alter
the address and op\_index registers use this and
\addtoindexx{maximum\_operations\_per\_instruction}
maximum\-\_operations\-\_per\-\_instruction in their calculations. 

\item maximum\_operations\_per\_instruction (ubyte) \\
The 
\addtoindexx{maximum\_operations\_per\_instruction}
maximum number of individual operations that may be
encoded in an instruction. Line number program opcodes
that alter the address and 
\addtoindex{op\_index} registers use this and
\addtoindex{minimum\_instruction\_length}
in their calculations.
For non-VLIW
architectures, this field is 1, the op\_index register is always
0, and the operation pointer is simply the address register.

\item default\_is\_stmt (ubyte) \\
\addtoindexx{default\_is\_stmt}
The initial value of the \addtoindex{is\_stmt} register.  

\textit{A simple approach
to building line number information when machine instructions
are emitted in an order corresponding to the source program
is to set \addtoindex{default\_is\_stmt} 
to ``true'' and to not change the
value of the \addtoindex{is\_stmt} register 
within the line number program.
One matrix entry is produced for each line that has code
generated for it. The effect is that every entry in the
matrix recommends the beginning of each represented line as
a breakpoint location. This is the traditional practice for
unoptimized code.}

\textit{A more sophisticated approach might involve multiple entries in
the matrix for a line number; in this case, at least one entry
(often but not necessarily only one) specifies a recommended
breakpoint location for the line number. \livelink{chap:DWLNSnegatestmt}{DW\-\_LNS\-\_negate\-\_stmt}
opcodes in the line number program control which matrix entries
constitute such a recommendation and 
\addtoindex{default\_is\_stmt} might
be either ``true'' or ``false''. This approach might be
used as part of support for debugging optimized code.}

\item line\_base (sbyte) \\
\addtoindexx{line\_base}
This parameter affects the meaning of the special opcodes. See below.

\item line\_range (ubyte) \\
\addtoindexx{line\_range}
This parameter affects the meaning of the special opcodes. See below.

\item opcode\_base (ubyte) \\
The 
\addtoindex{opcode\_base}
number assigned to the first special opcode.

\textit{Opcode base is typically one greater than the highest-numbered
\addtoindex{opcode\_base}
standard opcode defined for the specified version of the line
number information (12 in 
\addtoindex{DWARF Version 3} and 
\addtoindexx{DWARF Version 4}
Version 4, 9 in
\addtoindexx{DWARF Version 2}
Version 2).  
If opcode\_base is less than the typical value,
\addtoindex{opcode\_base}
then standard opcode numbers greater than or equal to the
opcode base are not used in the line number table of this unit
(and the codes are treated as special opcodes). If opcode\_base
is greater than the typical value, then the numbers between
that of the highest standard opcode and the first special
opcode (not inclusive) are used for vendor specific extensions.}

\item standard\_opcode\_lengths (array of ubyte) \\
\addtoindexx{standard\_opcode\_lengths}
This array specifies the number of LEB128 operands for each
of the standard opcodes. The first element of the array
corresponds to the opcode whose value is 1, and the last
element corresponds to the opcode whose value 
is opcode\_base - 1.

By increasing opcode\_base, and adding elements to this array,
\addtoindex{opcode\_base}
new standard opcodes can be added, while allowing consumers who
do not know about these new opcodes to be able to skip them.

Codes for vendor specific extensions, if any, are described
just like standard opcodes.

\item include\_directories (sequence of path names) \\
Entries 
\addtoindexx{include\_directories}
in this sequence describe each path that was searched
for included source files in this compilation. (The paths
include those directories specified explicitly by the user for
the compiler to search and those the compiler searches without
explicit direction.) Each path entry is either a full path name
or is relative to the current directory of the compilation.

The last entry is followed by a single null byte.

The line number program assigns numbers to each of the file
entries in order, beginning with 1. The current directory of
the compilation is understood to be the zeroth entry and is
not explicitly represented.

\item  file\_names (sequence of file entries) \\
Entries 
\addtoindexx{file names}
in 
\addtoindexx{file\_names}
this sequence describe source files that contribute
to the line number information for this compilation unit or is
used in other contexts, such as in a declaration coordinate or
a macro file inclusion. Each entry consists of the following
values:


\begin{itemize}
\item A null\dash terminated string containing the full or relative
path name of a source file. If the entry contains a file
name or relative path name, the file is located relative
to either the compilation directory (as specified by the
\livelink{chap:DWATcompdir}{DW\-\_AT\-\_comp\-\_dir} attribute given in the compilation unit) or one
of the directories listed in the 
\addtoindex{include\_directories} section.

\item An unsigned LEB128 number representing the directory
index of a directory in the 
\addtoindex{include\_directories} section.


\item An unsigned LEB128 number representing the
(implementation\dash defined) time of last modification for
the file, or 0 if not available.

\item An unsigned LEB128 number representing the length in
bytes of the file, or 0 if not available.  

\end{itemize}

The last entry is followed by a single null byte.

The directory index represents an entry in the
\addtoindex{include\_directories} section. 
The index is 0 if the file was
found in the current directory of the compilation, 1 if it
was found in the first directory in the 
\addtoindex{include\_directories}
section, and so on. The directory index is ignored for file
names that represent full path names.

The primary source file is described by an entry whose path
name exactly matches that given in the \livelink{chap:DWATname}{DW\-\_AT\-\_name} attribute
in the compilation unit, and whose directory is understood
to be given by the implicit entry with index 0.

The line number program assigns numbers to each of the file
entries in order, beginning with 1, and uses those numbers
instead of file names in the file register.

\textit{A compiler may generate a single null byte for the file
names field and define file names using the extended opcode
\livelink{chap:DWLNEdefinefile}{DW\-\_LNE\-\_define\-\_file}.}


\end{enumerate}

\subsection{The Line Number Program}

As stated before, the goal of a line number program is to build
a matrix representing one compilation unit, which may have
produced multiple sequences of target machine instructions.
Within a sequence, addresses (operation pointers) may only
increase. (Line numbers may decrease in cases of pipeline
scheduling or other optimization.)

\subsubsection{Special Opcodes} 
\label{chap:specialopcodes}
Each ubyte special opcode has the following effect on the state machine:

\begin{enumerate}[1.]

\item  Add a signed integer to the line register.

\item  Modify the operation pointer by incrementing the
address and \addtoindex{op\_index} registers as described below.

\item  Append a row to the matrix using the current values
of the state machine registers.

\item  Set the \addtoindex{basic\_block} register to ``false.'' \addtoindexx{basic block}
\item  Set the \addtoindex{prologue\_end} register to ``false.''
\item  Set the \addtoindex{epilogue\_begin} register to ``false.''
\item  Set the \addtoindex{discriminator} register to 0.

\end{enumerate}

All of the special opcodes do those same seven things; they
differ from one another only in what values they add to the
line, address and op\_index registers.


\textit{Instead of assigning a fixed meaning to each special opcode,
the line number program uses several parameters in the header
to configure the instruction set. There are two reasons
for this.  First, although the opcode space available for
special opcodes now ranges from 13 through 255, the lower
bound may increase if one adds new standard opcodes. Thus, the
opcode\_base field of the line number program header gives the
value of the first special opcode. Second, the best choice of
special\dash opcode meanings depends on the target architecture. For
example, for a RISC machine where the compiler\dash generated code
interleaves instructions from different lines to schedule
the pipeline, it is important to be able to add a negative
value to the line register to express the fact that a later
instruction may have been emitted for an earlier source
line. For a machine where pipeline scheduling never occurs,
it is advantageous to trade away the ability to decrease
the line register (a standard opcode provides an alternate
way to decrease the line number) in return for the ability
to add larger positive values to the address register. To
permit this variety of strategies, the line number program
header defines a 
\addtoindexx{line\_base}
field that specifies the minimum
value which a special opcode can add to the line register
and a line\_range field that defines the range of values it
can add to the line register.}


A special opcode value is chosen based on the amount that needs
to be added to the line, address and op\_index registers. The
maximum line increment for a special opcode is the value
of the 
\addtoindexx{line\_base}
field in the header, plus the value of
the line\_range field, minus 1 (line base + 
line range - 1). 
If the desired line increment is greater than the maximum
line increment, a standard opcode must be used instead of a
special opcode. The operation advance represents the number
of operations to skip when advancing the operation pointer.

The special opcode is then calculated using the following formula:

  opcode = ( \textit{desired line increment} - \addtoindex{line\_base}) +
(\addtoindex{line\_range} * \textit{operation advance} ) + \addtoindex{opcode\_base}

If the resulting opcode is greater than 255, a standard opcode
must be used instead.

When \addtoindex{maximum\_operations\_per\_instruction} is 1, the operation
advance is simply the address increment divided by the
\addtoindex{minimum\_instruction\_length}.

To decode a special opcode, subtract the opcode\_base from
the opcode itself to give the \textit{adjusted opcode}. 
The \textit{operation advance} 
is the result of the adjusted opcode divided by the
line\_range. The new address and op\_index values are given by
\begin{myindentpara}{1cm}

\textit{adjusted opcode} = opcode – opcode\_base

\textit{operation advance} = \textit{adjusted opcode} / line\_range

\begin{myindentpara}{1cm}
new address =

address +

\addtoindex{minimum\_instruction\_length} *
((\addtoindex{op\_index} + operation advance) / 
\addtoindex{maximum\_operations\_per\_instruction})
\end{myindentpara}
new op\_index =

\begin{myindentpara}{1cm}
(op\_index + operation advance) \% \addtoindex{maximum\_operations\_per\_instruction}
\end{myindentpara}

\end{myindentpara}

\textit{When the \addtoindex{maximum\_operations\_per\_instruction} field is 1,
op\_index is always 0 and these calculations simplify to those
given for addresses in 
\addtoindex{DWARF Version 3}.}

The amount to increment the line register is the 
\addtoindex{line\_base} plus
the result of the 
\addtoindex{adjusted opcode} modulo the 
\addtoindex{line\_range}. That
is,

\begin{myindentpara}{1cm}
line increment = \addtoindex{line\_base} + (adjusted opcode \% \addtoindex{line\_range})
\end{myindentpara}

\textit{As an example, suppose that the opcode\_base is 13, 
\addtoindex{line\_base}
is -3, \addtoindex{line\_range} is 12, 
\addtoindex{minimum\_instruction\_length} is 1
and 
\addtoindex{maximum\_operations\_per\_instruction} is 1. 
This means that
we can use a special opcode whenever two successive rows in
the matrix have source line numbers differing by any value
within the range [-3, 8] and (because of the limited number
of opcodes available) when the difference between addresses
is within the range [0, 20], but not all line advances are
available for the maximum operation advance (see below).}

\textit{The opcode mapping would be:}
% FIXME: This should be a tabular or the like, not a verbatim
\begin{verbatim}
            \       Line advance
   Operation \
     Advance  \ -3  -2  -1   0   1   2   3   4   5   6   7   8
   ---------   -----------------------------------------------
           0    13  14  15  16  17  18  19  20  21  22  23  24
           1    25  26  27  28  29  30  31  32  33  34  35  36
           2    37  38  39  40  41  42  43  44  45  46  47  48
           3    49  50  51  52  53  54  55  56  57  58  59  60
           4    61  62  63  64  65  66  67  68  69  70  71  72
           5    73  74  75  76  77  78  79  80  81  82  83  84
           6    85  86  87  88  89  90  91  92  93  94  95  96
           7    97  98  99 100 101 102 103 104 105 106 107 108
           8   109 110 111 112 113 114 115 116 117 118 119 120
           9   121 122 123 124 125 126 127 128 129 130 131 132
          10   133 134 135 136 137 138 139 140 141 142 143 144
          11   145 146 147 148 149 150 151 152 153 154 155 156
          12   157 158 159 160 161 162 163 164 165 166 167 168
          13   169 170 171 172 173 174 175 176 177 178 179 180
          14   181 182 183 184 185 186 187 188 189 190 191 192
          15   193 194 195 196 197 198 199 200 201 202 203 204
          16   205 206 207 208 209 210 211 212 213 214 215 216
          17   217 218 219 220 221 222 223 224 225 226 227 228 
          18   229 230 231 232 233 234 235 236 237 238 239 240 
          19   241 242 243 244 245 246 247 248 249 250 251 252
          20   253 254 255
\end{verbatim}


\textit{There is no requirement that the expression 
255 - \addtoindex{line\_base} + 1 be an integral multiple of
\addtoindex{line\_range}. }

\subsubsection{Standard Opcodes}
\label{chap:standardopcodes}


The standard opcodes, their applicable operands and the
actions performed by these opcodes are as follows:

\begin{enumerate}[1.]

\item \textbf{DW\-\_LNS\-\_copy} \\
The \livetarg{chap:DWLNScopy}{DW\-\_LNS\-\_copy} opcode takes no operands. It appends a row
to the matrix using the current values of the state machine
registers. Then it sets the \addtoindex{discriminator} register to 0,
and sets the \addtoindex{basic\_block}, 
\addtoindex{prologue\_end} and 
\addtoindex{epilogue\_begin}
registers to ``false.''

\item \textbf{DW\-\_LNS\-\_advance\-\_pc} \\
The \livetarg{chap:DWLNSadvancepc}{DW\-\_LNS\-\_advance\-\_pc} opcode takes a single unsigned LEB128
operand as the operation advance and modifies the address
and \addtoindex{op\_index} registers as specified in 
Section \refersec{chap:specialopcodes}.

\item \textbf{DW\-\_LNS\-\_advance\-\_line} \\
The \livetarg{chap:DWLNSadvanceline}{DW\-\_LNS\-\_advance\-\_line} opcode takes a single signed LEB128
operand and adds that value to the line register of the
state machine.

\item \textbf{DW\-\_LNS\-\_set\-\_file} \\ 
The \livetarg{chap:DWLNSsetfile}{DW\-\_LNS\-\_set\-\_file} opcode takes a single
unsigned LEB128 operand and stores it in the file register
of the state machine.

\item \textbf{DW\-\_LNS\-\_set\-\_column} \\ 
The \livetarg{chap:DWLNSsetcolumn}{DW\-\_LNS\-\_set\-\_column} opcode takes a
single unsigned LEB128 operand and stores it in the column
register of the state machine.

\item \textbf{DW\-\_LNS\-\_negate\-\_stmt} \\
The \livetarg{chap:DWLNSnegatestmt}{DW\-\_LNS\-\_negate\-\_stmt} opcode takes no
operands. It sets the \addtoindex{is\_stmt} register of the state machine
to the logical negation of its current value.

\item \textbf{DW\-\_LNS\-\_set\-\_basic\-\_block} \\
The \livetarg{chap:DWLNSsetbasicblock}{DW\-\_LNS\-\_set\-\_basic\-\_block}
opcode
\addtoindexx{basic block}
takes no operands. 
It sets the basic\_block register of the
state machine to ``true.''



\item \textbf{DW\-\_LNS\-\_const\-\_add\-\_pc} \\
The \livetarg{chap:DWLNSconstaddpc}{DW\-\_LNS\-\_const\-\_add\-\_pc} opcode takes
no operands. It advances the address and op\_index registers
by the increments corresponding to special opcode 255.

\textit{When the line number program needs to advance the address
by a small amount, it can use a single special opcode,
which occupies a single byte. When it needs to advance the
address by up to twice the range of the last special opcode,
it can use \livelink{chap:DWLNSconstaddpc}{DW\-\_LNS\-\_const\-\_add\-\_pc} followed by a special opcode,
for a total of two bytes. Only if it needs to advance the
address by more than twice that range will it need to use
both \livelink{chap:DWLNSadvancepc}{DW\-\_LNS\-\_advance\-\_pc} and a special opcode, requiring three
or more bytes.}

\item \textbf{DW\-\_LNS\-\_fixed\-\_advance\-\_pc} \\ 
The \livetarg{chap:DWLNSfixedadvancepc}{DW\-\_LNS\-\_fixed\-\_advance\-\_pc} opcode
takes a single uhalf (unencoded) operand and adds it to the
address register of the state machine and sets the op\_index
register to 0. This is the only standard opcode whose operand
is \textbf{not} a variable length number. It also does 
\textbf{not} multiply the
operand by the \addtoindex{minimum\_instruction\_length} field of the header.

\textit{Existing assemblers cannot emit \livelink{chap:DWLNSadvancepc}{DW\-\_LNS\-\_advance\-\_pc} or special
opcodes because they cannot encode LEB128 numbers or judge when
the computation of a special opcode overflows and requires
the use of \livelink{chap:DWLNSadvancepc}{DW\-\_LNS\-\_advance\-\_pc}. Such assemblers, however, can
use \livelink{chap:DWLNSfixedadvancepc}{DW\-\_LNS\-\_fixed\-\_advance\-\_pc} instead, sacrificing compression.}

\item \textbf{DW\-\_LNS\-\_set\-\_prologue\-\_end} \\
The \livetarg{chap:DWLNSsetprologueend}{DW\-\_LNS\-\_set\-\_prologue\-\_end}
opcode takes no operands. It sets the 
\addtoindex{prologue\_end} register
to ``true''.

\textit{When a breakpoint is set on entry to a function, it is
generally desirable for execution to be suspended, not on the
very first instruction of the function, but rather at a point
after the function's frame has been set up, after any language
defined local declaration processing has been completed,
and before execution of the first statement of the function
begins. Debuggers generally cannot properly determine where
this point is. This command allows a compiler to communicate
the location(s) to use.}

\textit{In the case of optimized code, there may be more than one such
location; for example, the code might test for a special case
and make a fast exit prior to setting up the frame.}

\textit{Note that the function to which the 
\addtoindex{prologue end} applies cannot
be directly determined from the line number information alone;
it must be determined in combination with the subroutine
information entries of the compilation (including inlined
subroutines).}


\item \textbf{DW\-\_LNS\-\_set\-\_epilogue\-\_begin} \\
The \livetarg{chap:DWLNSsetepiloguebegin}{DW\-\_LNS\-\_set\-\_epilogue\-\_begin} opcode takes no operands. It
sets the \addtoindex{epilogue\_begin} register to ``true''.

\textit{When a breakpoint is set on the exit of a function or execution
steps over the last executable statement of a function, it is
generally desirable to suspend execution after completion of
the last statement but prior to tearing down the frame (so that
local variables can still be examined). Debuggers generally
cannot properly determine where this point is. This command
allows a compiler to communicate the location(s) to use.}

\textit{Note that the function to which the 
\addtoindex{epilogue end} applies cannot
be directly determined from the line number information alone;
it must be determined in combination with the subroutine
information entries of the compilation (including inlined
subroutines).}

\textit{In the case of a trivial function, both 
\addtoindex{prologue end} and
\addtoindex{epilogue begin} may occur at the same address.}

\item \textbf{DW\-\_LNS\-\_set\-\_isa} \\
The \livetarg{chap:DWLNSsetisa}{DW\-\_LNS\-\_set\-\_isa} opcode takes a single
unsigned LEB128 operand and stores that value in the 
\addtoindex{isa}
register of the state machine.
\end{enumerate}

\subsubsection{ExtendedOpcodes}
\label{chap:extendedopcodes}

The extended opcodes are as follows:

\begin{enumerate}[1.]

\item \textbf{DW\-\_LNE\-\_end\-\_sequence} \\
The \livetarg{chap:DWLNEendsequence}{DW\-\_LNE\-\_end\-\_sequence} opcode takes no operands. It sets the
\addtoindex{end\_sequence}
register of the state machine to “true” and
appends a row to the matrix using the current values of the
state-machine registers. Then it resets the registers to the
initial values specified above 
(see Section \refersec{chap:statemachineregisters}). 
Every line
number program sequence must end with a \livelink{chap:DWLNEendsequence}{DW\-\_LNE\-\_end\-\_sequence}
instruction which creates a row whose address is that of the
byte after the last target machine instruction of the sequence.

\item \textbf{DW\-\_LNE\-\_set\-\_address} \\
The \livetarg{chap:DWLNEsetaddress}{DW\-\_LNE\-\_set\-\_address} opcode takes a single relocatable
address as an operand. The size of the operand is the size
of an address on the target machine. It sets the address
register to the value given by the relocatable address and
sets the op\_index register to 0.

\textit{All of the other line number program opcodes that
affect the address register add a delta to it. This instruction
stores a relocatable value into it instead.}


\item \textbf{DW\-\_LNE\-\_define\-\_file} \\

The \livetarg{chap:DWLNEdefinefile}{DW\-\_LNE\-\_define\-\_file} opcode takes four operands:

\begin{enumerate}[1.]

\item A null\dash terminated string containing the full or relative
path name of a source file. If the entry contains a file
name or a relative path name, the file is located relative
to either the compilation directory (as specified by the
\livelink{chap:DWATcompdir}{DW\-\_AT\-\_comp\-\_dir} attribute given in the compilation unit)
or one of the directories in the 
\addtoindex{include\_directories} section.

\item An unsigned LEB128 number representing the directory index
of the directory in which the file was found.  

\item An unsigned
LEB128 number representing the time of last modification
of the file, or 0 if not available.  

\item An unsigned LEB128
number representing the length in bytes of the file, or 0 if
not available.
\end{enumerate}

The directory index represents an entry in the
\addtoindex{include\_directories} section of the line number program
header. The index is 0 if the file was found in the current
directory of the compilation, 1 if it was found in the first
directory in the \addtoindex{include\_directories} section,
and so on. The
directory index is ignored for file names that represent full
path names.

The primary source file is described by an entry whose path
name exactly matches that given in the \livelink{chap:DWATname}{DW\-\_AT\-\_name} attribute
in the compilation unit, and whose directory index is 0. The
files are numbered, starting at 1, in the order in which they
appear; the names in the header come before names defined by
the \livelink{chap:DWLNEdefinefile}{DW\-\_LNE\-\_define\-\_file} instruction. These numbers are used
in the file register of the state machine.

\item \textbf{DW\-\_LNE\-\_set\-\_discriminator} \\
The \livetarg{chap:DWLNEsetdiscriminator}{DW\-\_LNE\-\_set\-\_discriminator}
opcode takes a single
parameter, an unsigned LEB128 integer. It sets the
\addtoindex{discriminator} register to the new value.



\end{enumerate}

\textit{Appendix \refersec{app:linenumberprogramexample} 
gives some sample line number programs.}

\section{Macro Information}
\label{chap:macroinformation}
\textit{Some languages, such as 
\addtoindex{C} and 
addtoindex{C++}, provide a way to replace
\addtoindex{macro information}
text in the source program with macros defined either in the
source file itself, or in another file included by the source
file.  Because these macros are not themselves defined in the
target language, it is difficult to represent their definitions
using the standard language constructs of DWARF. The debugging
information therefore reflects the state of the source after
the macro definition has been expanded, rather than as the
programmer wrote it. The macro information table provides a way
of preserving the original source in the debugging information.}

As described in 
Section \refersec{chap:normalandpartialcompilationunitentries},
the macro information for a
given compilation unit is represented in the 
\addtoindex{.debug\_macinfo}
section of an object file. The macro information for each
compilation unit is represented as a series of “macinfo”
entries. Each macinfo entry consists of a “type code” and
up to two additional operands. The series of entries for a
given compilation unit ends with an entry containing a type
code of 0.

\subsection{Macinfo Types}
\label{chap:macinfotypes}

The valid \addtoindex{macinfo types} are as follows:

\begin{tabular}{ll}
\livelink{chap:DWMACINFOdefine}{DW\-\_MACINFO\-\_define} 
&A macro definition.\\
\livelink{chap:DWMACINFOundef}{DW\-\_MACINFO\-\_undef}
&A macro undefinition.\\
\livelink{chap:DWMACINFOstartfile}{DW\-\_MACINFO\-\_start\-\_file}
&The start of a new source file inclusion.\\
\livelink{chap:DWMACINFOendfile}{DW\-\_MACINFO\-\_end\-\_file}
&The end of the current source file inclusion.\\
\livelink{chap:DWMACINFOvendorext}{DW\-\_MACINFO\-\_vendor\-\_ext}
& Vendor specific macro information directives.\\
\end{tabular}

\subsubsection{Define and Undefine Entries}
\label{chap:defineandundefineentries}

All 
\livetarg{chap:DWMACINFOdefine}{DW\-\_MACINFO\-\_define} and 
\livetarg{chap:DWMACINFOundef}{DW\-\_MACINFO\-\_undef} entries have two
operands. The first operand encodes the line number of the
source line on which the relevant defining or undefining
macro directives appeared.

The second operand consists of a null-terminated character
string. In the case of a 
\livelink{chap:DWMACINFOundef}{DW\-\_MACINFO\-\_undef} entry, the value
of this string will be simply the name of the pre- processor
symbol that was undefined at the indicated source line.

In the case of a \livelink{chap:DWMACINFOdefine}{DW\-\_MACINFO\-\_define} entry, the value of this
string will be the name of the macro symbol that was defined
at the indicated source line, followed immediately by the 
\addtoindex{macro formal parameter list}
including the surrounding parentheses (in
the case of a function-like macro) followed by the definition
string for the macro. If there is no formal parameter list,
then the name of the defined macro is followed directly by
its definition string.

In the case of a function-like macro definition, no whitespace
characters should appear between the name of the defined
macro and the following left parenthesis. Also, no whitespace
characters should appear between successive formal parameters
in the formal parameter list. (Successive formal parameters
are, however, separated by commas.) Also, exactly one space
character should separate the right parenthesis that terminates
the formal parameter list and the following definition string.

In the case of a ``normal'' (i.e. non-function-like) macro
definition, exactly one space character should separate the
name of the defined macro from the following definition text.



\subsubsection{Start File Entries}
\label{chap:startfileentries}
Each \livetarg{chap:DWMACINFOstartfile}{DW\-\_MACINFO\-\_start\-\_file} entry also has two operands. The
first operand encodes the line number of the source line on
which the inclusion macro directive occurred.

The second operand encodes a source file name index. This index
corresponds to a file number in the line number information
table for the relevant compilation unit. This index indicates
(indirectly) the name of the file that is being included by
the inclusion directive on the indicated source line.

\subsubsection{End File Entries}
\label{chap:endfileentries}
A \livetarg{chap:DWMACINFOendfile}{DW\-\_MACINFO\-\_end\-\_file} entry has no operands. The presence of
the entry marks the end of the current source file inclusion.

\subsubsection{Vendor Extension Entries}
\label{chap:vendorextensionentries}
A \livetarg{chap:DWMACINFOvendorext}{DW\-\_MACINFO\-\_vendor\-\_ext} entry has two operands. The first
is a constant. The second is a null-terminated character
string. The meaning and/or significance of these operands is
intentionally left undefined by this specification.

\textit{A consumer must be able to totally ignore all
\livelink{chap:DWMACINFOvendorext}{DW\-\_MACINFO\-\_vendor\-\_ext} entries that it does not understand
(see Section \refersec{datarep:vendorextensibility}).}


\subsection{Base Source Entries} 
\label{chap:basesourceentries}

A producer shall generate \livelink{chap:DWMACINFOstartfile}{DW\-\_MACINFO\-\_start\-\_file} and
\livelink{chap:DWMACINFOendfile}{DW\-\_MACINFO\-\_end\-\_file} entries for the source file submitted to
the compiler for compilation. This \livelink{chap:DWMACINFOstartfile}{DW\-\_MACINFO\-\_start\-\_file} entry
has the value 0 in its line number operand and references
the file entry in the line number information table for the
primary source file.


\subsection{Macinfo Entries For Command Line Options}
\label{chap:macinfoentriesforcommandlineoptions}

In addition to producing \livelink{chap:DWMACINFOdefine}{DW\-\_MACINFO\-\_define} and \livelink{chap:DWMACINFOundef}{DW\-\_MACINFO\-\_undef}
entries for each of the define and undefine directives
processed during compilation, the DWARF producer should
generate a \livelink{chap:DWMACINFOdefine}{DW\-\_MACINFO\-\_define} or \livelink{chap:DWMACINFOundef}{DW\-\_MACINFO\-\_undef} entry for
each pre-processor symbol which is defined or undefined by
some means other than via a define or undefine directive
within the compiled source text. In particular, pre-processor
symbol definitions and un- definitions which occur as a
result of command line options (when invoking the compiler)
should be represented by their own \livelink{chap:DWMACINFOdefine}{DW\-\_MACINFO\-\_define} and
\livelink{chap:DWMACINFOundef}{DW\-\_MACINFO\-\_undef} entries.

All such \livelink{chap:DWMACINFOdefine}{DW\-\_MACINFO\-\_define} and \livelink{chap:DWMACINFOundef}{DW\-\_MACINFO\-\_undef} entries
representing compilation options should appear before the
first \livelink{chap:DWMACINFOstartfile}{DW\-\_MACINFO\-\_start\-\_file} entry for that compilation unit
and should encode the value 0 in their line number operands.


\subsection{General rules and restrictions}
\label{chap:generalrulesandrestrictions}

All macinfo entries within a \addtoindex{.debug\_macinfo}
section for a
given compilation unit appear in the same order in which the
directives were processed by the compiler.

All macinfo entries representing command line options appear
in the same order as the relevant command line options
were given to the compiler. In the case where the compiler
itself implicitly supplies one or more macro definitions or
un-definitions in addition to those which may be specified on
the command line, macinfo entries are also produced for these
implicit definitions and un-definitions, and these entries
also appear in the proper order relative to each other and
to any definitions or undefinitions given explicitly by the
user on the command line.



\section{Call Frame Information}
\label{chap:callframeinformation}




\textit{Debuggers often need to be able to view and modify the state of any subroutine activation that is
\addtoindexx{activation!call frame}
on the call stack. An activation consists of:}

\begin{itemize}
\item \textit{A code location that is within the
subroutine. This location is either the place where the program
stopped when the debugger got control (e.g. a breakpoint), or
is a place where a subroutine made a call or was interrupted
by an asynchronous event (e.g. a signal).}

\item \textit{An area of memory that is allocated on a stack called a
``call frame.'' The call frame is identified by an address
on the stack. We refer to this address as the Canonical
Frame Address or CFA. Typically, the CFA is defined to be the
value of the stack pointer at the call site in the previous
frame (which may be different from its value on entry to the
current frame).}

\item \textit{A set of registers that are in use by the subroutine
at the code location.}

\end{itemize}

\textit{Typically, a set of registers are designated to be preserved
across a call. If a callee wishes to use such a register, it
saves the value that the register had at entry time in its call
frame and restores it on exit. The code that allocates space
on the call frame stack and performs the save operation is
called the subroutine’s \addtoindex{prologue}, and the code that performs
the restore operation and deallocates the frame is called its
\addtoindex{epilogue}. Typically, the 
\addtoindex{prologue} code is physically at the
beginning of a subroutine and the 
\addtoindex{epilogue} code is at the end.}

\textit{To be able to view or modify an activation that is not
on the top of the call frame stack, the debugger must
``virtually unwind'' the stack of activations until
it finds the activation of interest.  A debugger unwinds
a stack in steps. Starting with the current activation it
virtually restores any registers that were preserved by the
current activation and computes the predecessor’s CFA and
code location. This has the logical effect of returning from
the current subroutine to its predecessor. We say that the
debugger virtually unwinds the stack because the actual state
of the target process is unchanged.}

\textit{The unwinding operation needs to know where registers are
saved and how to compute the predecessor’s CFA and code
location. When considering an architecture-independent way
of encoding this information one has to consider a number of
special things.}


\begin{itemize} % bullet list

\item \textit{Prologue 
\addtoindexx{prologue}
and 
\addtoindex{epilogue} code is not always in 
distinct block
at the beginning and end of a subroutine. It is common
to duplicate the \addtoindex{epilogue} code 
at the site of each return
from the code. Sometimes a compiler breaks up the register
save/unsave operations and moves them into the body of the
subroutine to just where they are needed.}


\item \textit{Compilers use different ways to manage the call
frame. Sometimes they use a frame pointer register, sometimes
not.}

\item \textit{The algorithm to compute CFA changes as you progress through
the \addtoindex{prologue} 
and \addtoindex{epilogue code}. 
(By definition, the CFA value
does not change.)}

\item \textit{Some subroutines have no call frame.}

\item \textit{Sometimes a register is saved in another register that by
convention does not need to be saved.}

\item \textit{Some architectures have special instructions that perform
some or all of the register management in one instruction,
leaving special information on the stack that indicates how
registers are saved.}

\item \textit{Some architectures treat return address values specially. For
example, in one architecture, the call instruction guarantees
that the low order two bits will be zero and the return
instruction ignores those bits. This leaves two bits of
storage that are available to other uses that must be treated
specially.}


\end{itemize}


\subsection{Structure of Call Frame Information}
\label{chap:structureofcallframeinformation}

DWARF supports virtual unwinding by defining an architecture
independent basis for recording how procedures save and restore
registers during their lifetimes. This basis must be augmented
on some machines with specific information that is defined by
an architecture specific ABI authoring committee, a hardware
vendor, or a compiler producer. The body defining a specific
augmentation is referred to below as the ``augmenter.''

Abstractly, this mechanism describes a very large table that
has the following structure:

\begin{verbatim}
        LOC CFA R0 R1 ... RN
        L0
        L1
        ...
        LN
\end{verbatim}


The first column indicates an address for every location
that contains code in a program. (In shared objects, this
is an object-relative offset.) The remaining columns contain
virtual unwinding rules that are associated with the indicated
location.

The CFA column defines the rule which computes the Canonical
Frame Address value; it may be either a register and a signed
offset that are added together, or a DWARF expression that
is evaluated.

The remaining columns are labeled by register number. This
includes some registers that have special designation on
some architectures such as the PC and the stack pointer
register. (The actual mapping of registers for a particular
architecture is defined by the augmenter.) The register columns
contain rules that describe whether a given register has been
saved and the rule to find the value for the register in the
previous frame.

The register rules are:


\begin{tabular}{lp{8cm}}
undefined 
&A register that has this rule has no recoverable value in the previous frame.
(By convention, it is not preserved by a callee.) \\

same value
&This register has not been modified from the previous frame. (By convention,
it is preserved by the callee, but the callee has not modified it.) \\

offset(N)
&The previous value of this register is saved at the address CFA+N where CFA
is the current CFA value and N is a signed offset.\\

val\_offset(N)
&The previous value of this register is the value CFA+N where CFA is the
current CFA value and N is a signed offset.\\

register(R)
&The previous value of this register is stored 
in another register numbered R.\\

expression(E)
&The previous value of this register is located at the address produced by
executing the DWARF expression E.\\

val\_expression(E) 
&The previous value of this register is the value produced by executing the
DWARF expression E.\\

architectural
&The rule is defined externally to this specification by the augmenter.\\

\end{tabular}

\textit{This table would be extremely large if actually constructed
as described. Most of the entries at any point in the table
are identical to the ones above them. The whole table can be
represented quite compactly by recording just the differences
starting at the beginning address of each subroutine in
the program.}

The virtual unwind information is encoded in a self-contained
section called 
\addtoindex{.debug\_frame}.  Entries in a 
\addtoindex{.debug\_frame} section
are aligned on a multiple of the address size relative to
the start of the section and come in two forms: a Common
\addtoindexx{common information entry}
Information Entry (CIE) and a 
\addtoindexx{frame description entry}
Frame Description Entry (FDE).

\textit{If the range of code addresses for a function is not
contiguous, there may be multiple CIEs and FDEs corresponding
to the parts of that function.}


A Common Information Entry holds information that is shared
among many Frame Description Entries. There is at least one
CIE in every non-empty \addtoindex{.debug\_frame} section. A CIE contains
the following fields, in order:

\begin{enumerate}[1.]
\item length (initial length)  \\
A constant that gives the number of bytes of the CIE structure,
not including the length field itself 
(see Section \refersec{datarep:initiallengthvalues}). 
The
size of the length field plus the value of length must be an
integral multiple of the address size.

\item  CIE\_id (4 or 8 bytes, see Section \refersec{datarep:32bitand64bitdwarfformats}) \\
A constant that is used to distinguish CIEs from FDEs.

\item  version (ubyte) \\
A version number\addtoindexx{version number!call frame information} 
(see Section \refersec{datarep:callframeinformation}). 
This number is specific to the call frame information
and is independent of the DWARF version number.


\item  augmentation (UTF\dash 8 string) \\
A null\dash terminated UTF\dash 8 string that identifies the augmentation
to this CIE or to the FDEs that use it. If a reader encounters
an augmentation string that is unexpected, then only the
following fields can be read:


\begin{itemize}

\item CIE: length, CIE\_id, version, augmentation

\item FDE: length, CIE\_pointer, initial\_location, address\_range

\end{itemize}
If there is no augmentation, this value is a zero byte.

\textit{The augmentation string allows users to indicate that there
is additional target\dash specific information in the CIE or FDE
which is needed to unwind a stack frame. For example, this
might be information about dynamically allocated data which
needs to be freed on exit from the routine.}

\textit{Because the \addtoindex{.debug\_frame} section is useful independently of
any \addtoindex{.debug\_info} section, the augmentation string always uses
UTF\dash 8 encoding.}

\item  address\_size (ubyte) \\
The size of a target address
\addtoindexx{address\_size}
in this CIE and any FDEs that
use it, in bytes. If a compilation unit exists for this frame,
its address size must match the address size here.

\item  segment\_size (ubyte) \\
The size of a segment selector in this CIE and any FDEs that
use it, in bytes.

\item  \addtoindex{code\_alignment\_factor} (unsigned LEB128) \\
\addtoindex{code alignment factor}
A 
\addtoindexx{\textless caf\textgreater|see{code alignment factor}}
constant that is factored out of all advance location
instructions (see 
Section \refersec{chap:rowcreationinstructions}).


\item  \addtoindex{data\_alignment\_factor} (signed LEB128) \\
\addtoindexx{data alignment factor}
A 
\addtoindexx{\textless daf\textgreater|see{data alignment factor}}
constant that is factored out of certain offset instructions
(see below). The resulting value is  \textit{(operand *
data\_alignment\_factor)}.

\item  return\_address\_register (unsigned LEB128) \\
An unsigned LEB128 constant that indicates which column in the
rule table represents the return address of the function. Note
that this column might not correspond to an actual machine
register.

\item initial\_instructions (array of ubyte) \\
A sequence of rules that are interpreted to create the initial
setting of each column in the table.  The default rule for
all columns before interpretation of the initial instructions
is the undefined rule. However, an ABI authoring body or a
compilation system authoring body may specify an alternate
default value for any or all columns.

\item padding (array of ubyte) \\
Enough \livelink{chap:DWCFAnop}{DW\-\_CFA\-\_nop} instructions to make the size of this entry
match the length value above.
\end{enumerate}

An FDE contains the following fields, in order:

\begin{enumerate}[1.]
\item length (initial length)  \\

A constant that gives the number of bytes of the header and
instruction stream for this function, not including the length
field itself 
(see Section  \refersec{datarep:initiallengthvalues}). 
The size of the length field
plus the value of length must be an integral multiple of the
address size.

\item   CIE\_pointer (4 or 8 bytes, see Section \refersec{datarep:32bitand64bitdwarfformats}) \\
A constant offset into the \addtoindex{.debug\_frame}
section that denotes
the CIE that is associated with this FDE.

\item  initial\_location (segment selector and target address) \\
The address of the first location associated with this table
entry. If the segment\_size field of this FDE's CIE is non-zero,
the initial location is preceded by a segment selector of
the given length.

\item  address\_range (target address) \\
The number of bytes of program instructions described by this entry.

\item instructions (array of ubyte) \\
A sequence of table defining instructions that are described below.

\item 6. padding (array of ubyte) \\
Enough \livelink{chap:DWCFAnop}{DW\-\_CFA\-\_nop} instructions to make the size of this
entry match the length value above.
\end{enumerate}

\subsection{Call Frame Instructions}
\label{chap:callframeinstructions}

Each call frame instruction is defined to take 0 or more
operands. Some of the operands may be encoded as part of the
opcode 
(see Section \refersec{datarep:callframeinformation}). 
The instructions are defined in
the following sections.

Some call frame instructions have operands that are encoded
as DWARF expressions 
(see Section \refersec{chap:generaloperations}). 
The following DWARF
operators cannot be used in such operands:


\begin{itemize}
\item \livelink{chap:DWOPcall2}{DW\-\_OP\-\_call2}, \livelink{chap:DWOPcall4}{DW\-\_OP\-\_call4} 
and \livelink{chap:DWOPcallref}{DW\-\_OP\-\_call\-\_ref} operators
are not meaningful in an operand of these instructions
because there is no mapping from call frame information to
any corresponding debugging compilation unit information,
thus there is no way to interpret the call offset.

\item \livelink{chap:DWOPpushobjectaddress}{DW\-\_OP\-\_push\-\_object\-\_address} is not meaningful in an operand
of these instructions because there is no object context to
provide a value to push.

\item \livelink{chap:DWOPcallframecfa}{DW\-\_OP\-\_call\-\_frame\-\_cfa} is not meaningful in an operand of
these instructions because its use would be circular.
\end{itemize}

\textit{Call frame instructions to which these restrictions apply
include \livelink{chap:DWCFAdefcfaexpression}{DW\-\_CFA\-\_def\-\_cfa\-\_expression}, \livelink{chap:DWCFAexpression}{DW\-\_CFA\-\_expression}
and \livelink{chap:DWCFAvalexpression}{DW\-\_CFA\-\_val\-\_expression}.}

\subsubsection{Row Creation Instructions}
\label{chap:rowcreationinstructions}

\begin{enumerate}[1.]

\item \textbf{DW\-\_CFA\-\_set\-\_loc} \\
The \livetarg{chap:DWCFAsetloc}{DW\-\_CFA\-\_set\-\_loc} instruction takes a single operand that
represents a target address. The required action is to create a
new table row using the specified address as the location. All
other values in the new row are initially identical to the
current row. The new location value is always greater than
the current one. If the segment\_size field of this FDE's CIE
is non- zero, the initial location is preceded by a segment
selector of the given length.


\item \textbf{DW\-\_CFA\-\_advance\-\_loc} \\
The \livetarg{chap:DWCFAadvanceloc}{DW\-\_CFA\-\_advanceloc} instruction takes a single operand (encoded
with the opcode) that represents a constant delta. The required
action is to create a new table row with a location value that
is computed by taking the current entry’s location value
and adding the value of 
\textit{delta * \addtoindex{code\_alignment\_factor}}. 
All
other values in the new row are initially identical to the
current row

\item \textbf{DW\-\_CFA\-\_advance\-\_loc1} \\
The \livetarg{chap:DWCFAadvanceloc1}{DW\-\_CFA\-\_advance\-\_loc1} instruction takes a single ubyte
operand that represents a constant delta. This instruction
is identical to \livelink{chap:DWCFAadvanceloc}{DW\-\_CFA\-\_advance\-\_loc} except for the encoding
and size of the delta operand.

\item \textbf{DW\-\_CFA\-\_advance\-\_loc2} \\
The \livetarg{chap:DWCFAadvanceloc2}{DW\-\_CFA\-\_advance\-\_loc2} instruction takes a single uhalf
operand that represents a constant delta. This instruction
is identical to \livelink{chap:DWCFAadvanceloc}{DW\-\_CFA\-\_advance\-\_loc} except for the encoding
and size of the delta operand.

\item \textbf{DW\-\_CFA\-\_advance\-\_loc4} \\
The \livetarg{chap:DWCFAadvanceloc4}{DW\-\_CFA\-\_advance\-\_loc4} instruction takes a single uword
operand that represents a constant delta. This instruction
is identical to \livelink{chap:DWCFAadvanceloc}{DW\-\_CFA\-\_advance\-\_loc} except for the encoding
and size of the delta operand.

\end{enumerate}

\subsubsection{CFA Definition Instructions}
\label{chap:cfadefinitioninstructions}

\begin{enumerate}[1.]
\item \textbf{DW\-\_CFA\-\_def\-\_cfa} \\
The \livetarg{chap:DWCFAdefcfa}{DW\-\_CFA\-\_def\-\_cfa} instruction takes two unsigned LEB128
operands representing a register number and a (non\dash factored)
offset. The required action is to define the current CFA rule
to use the provided register and offset.

\item \textbf{ DW\-\_CFA\-\_def\-\_cfa\-\_sf} \\
The \livetarg{chap:DWCFAdefcfasf}{DW\-\_CFA\-\_def\-\_cfa\-\_sf} instruction takes two operands:
an unsigned LEB128 value representing a register number and a
signed LEB128 factored offset. This instruction is identical
to \livelink{chap:DWCFAdefcfa}{DW\-\_CFA\-\_def\-\_cfa} except that the second operand is signed
and factored. The resulting offset is factored\_offset *
\addtoindex{data\_alignment\_factor}.


\item \textbf{DW\-\_CFA\-\_def\-\_cfa\-\_register} \\
The \livetarg{chap:DWCFAdefcfaregister}{DW\-\_CFA\-\_def\-\_cfa\-\_register} instruction takes a single
unsigned LEB128 operand representing a register number. The
required action is to define the current CFA rule to use
the provided register (but to keep the old offset). This
operation is valid only if the current CFA rule is defined
to use a register and offset.



\item \textbf{DW\-\_CFA\-\_def\-\_cfa\-\_offset} \\
The \livetarg{chap:DWCFAdefcfaoffset}{DW\-\_CFA\-\_def\-\_cfa\-\_offset} instruction takes a single
unsigned LEB128 operand representing a (non-factored)
offset. The required action is to define the current CFA rule
to use the provided offset (but to keep the old register). This
operation is valid only if the current CFA rule is defined
to use a register and offset.


\item \textbf{DW\-\_CFA\-\_def\-\_cfa\-\_offset\-\_sf} \\
The \livetarg{chap:DWCFAdefcfaoffsetsf}{DW\-\_CFA\-\_def\-\_cfa\-\_offset\-\_sf} instruction takes a signed
LEB128 operand representing a factored offset. This instruction
is identical to \livelink{chap:DWCFAdefcfaoffset}{DW\-\_CFA\-\_def\-\_cfa\-\_offset} except that the
operand is signed and factored. The resulting offset is
factored\_offset * \addtoindex{data\_alignment\_factor}.
This operation
is valid only if the current CFA rule is defined to use a
register and offset.

\item \textbf{DW\-\_CFA\-\_def\-\_cfa\-\_expression} \\
The \livetarg{chap:DWCFAdefcfaexpression}{DW\-\_CFA\-\_def\-\_cfa\-\_expression} instruction takes a 
\addtoindexx{exprloc class}
single
operand encoded as a \livelink{chap:DWFORMexprloc}{DW\-\_FORM\-\_exprloc} value representing a
DWARF expression. The required action is to establish that
expression as the means by which the current CFA is computed.
See 
Section \refersec{chap:callframeinstructions} 
regarding restrictions on the DWARF
expression operators that can be used.

\end{enumerate}

\subsubsection{Register Rule Instructions}
\label{chap:registerruleinstructions}

\begin{enumerate}[1.]
\item \textbf{DW\-\_CFA\-\_undefined} \\
The \livetarg{chap:DWCFAundefined}{DW\-\_CFA\-\_undefined} instruction takes a single unsigned
LEB128 operand that represents a register number. The required
action is to set the rule for the specified register to
``undefined.''

\item \textbf{DW\-\_CFA\-\_same\-\_value} \\
The \livetarg{chap:DWCFAsamevalue}{DW\-\_CFA\-\_same\-\_value} instruction takes a single unsigned
LEB128 operand that represents a register number. The required
action is to set the rule for the specified register to
``same value.''

\item \textbf{DW\-\_CFA\-\_offset} \\
The \livetarg{chap:DWCFAoffset}{DW\-\_CFA\-\_offset} instruction takes two operands: a register
number (encoded with the opcode) and an unsigned LEB128
constant representing a factored offset. The required action
is to change the rule for the register indicated by the
register number to be an offset(N) rule where the value of
N is 
\textit{factored offset * \addtoindex{data\_alignment\_factor}}.

\item \textbf{DW\-\_CFA\-\_offset\-\_extended} \\
The \livetarg{chap:DWCFAoffsetextended}{DW\-\_CFA\-\_offset\-\_extended} instruction takes two unsigned
LEB128 operands representing a register number and a factored
offset. This instruction is identical to \livelink{chap:DWCFAoffset}{DW\-\_CFA\-\_offset} except
for the encoding and size of the register operand.

\item \textbf{ DW\-\_CFA\-\_offset\-\_extended\-\_sf} \\
The \livetarg{chap:DWCFAoffsetextendedsf}{DW\-\_CFA\-\_offset\-\_extended\-\_sf} instruction takes two operands:
an unsigned LEB128 value representing a register number and a
signed LEB128 factored offset. This instruction is identical
to \livelink{chap:DWCFAoffsetextended}{DW\-\_CFA\-\_offset\-\_extended} except that the second operand is
signed and factored. The resulting offset is 
\textit{factored\_offset * \addtoindex{data\_alignment\_factor}}.

\item \textbf{DW\-\_CFA\-\_val\-\_offset} \\
The \livetarg{chap:DWCFAvaloffset}{DW\-\_CFA\-\_val\-\_offset} instruction takes two unsigned
LEB128 operands representing a register number and a
factored offset. The required action is to change the rule
for the register indicated by the register number to be a
val\_offset(N) rule where the value of N is 
\textit{factored\_offset * \addtoindex{data\_alignment\_factor}}.

\item \textbf{DW\-\_CFA\-\_val\-\_offset\-\_sf} \\
The \livetarg{chap:DWCFAvaloffsetsf}{DW\-\_CFA\-\_val\-\_offset\-\_sf} instruction takes two operands: an
unsigned LEB128 value representing a register number and a
signed LEB128 factored offset. This instruction is identical
to \livelink{chap:DWCFAvaloffset}{DW\-\_CFA\-\_val\-\_offset} except that the second operand is signed
and factored. The resulting offset is 
\textit{factored\_offset * \addtoindex{data\_alignment\_factor}}.

\item \textbf{DW\-\_CFA\-\_register} \\
The \livetarg{chap:DWCFAregister}{DW\-\_CFA\-\_register} instruction takes two unsigned LEB128
operands representing register numbers. The required action
is to set the rule for the first register to be register(R)
where R is the second register.

\item \textbf{DW\-\_CFA\-\_expression} \\
The \livetarg{chap:DWCFAexpression}{DW\-\_CFA\-\_expression} 
instruction takes two operands: an
unsigned LEB128 value representing a register number, and
a \livelink{chap:DWFORMblock}{DW\-\_FORM\-\_block} 
value representing a DWARF expression. 
The
required action is to change the rule for the register
indicated by the register number to be an expression(E)
rule where E is the DWARF expression. That is, the DWARF
expression computes the address. The value of the CFA is
pushed on the DWARF evaluation stack prior to execution of
the DWARF expression.

See Section \refersec{chap:callframeinstructions} 
regarding restrictions on the DWARF
expression operators that can be used.

\item \textbf{DW\-\_CFA\-\_val\-\_expression} \\
The \livetarg{chap:DWCFAvalexpression}{DW\-\_CFA\-\_val\-\_expression} instruction takes two operands:
an unsigned LEB128 value representing a register number, and
a \livelink{chap:DWFORMblock}{DW\-\_FORM\-\_block} 
value representing a DWARF expression. The
required action is to change the rule for the register
indicated by the register number to be a val\_expression(E)
rule where E is the DWARF expression. That is, the DWARF
expression computes the value of the given register. The value
of the CFA is pushed on the DWARF evaluation stack prior to
execution of the DWARF expression.

See Section \refersec{chap:callframeinstructions} 
regarding restrictions on the DWARF
expression operators that can be used.

\item \textbf{ DW\-\_CFA\-\_restore} \\
The \livetarg{chap:DWCFArestore}{DW\-\_CFA\-\_restore} instruction takes a single operand (encoded
with the opcode) that represents a register number. The
required action is to change the rule for the indicated
register to the rule assigned it by the initial\_instructions
in the CIE.

\item \textbf{DW\-\_CFA\-\_restore\-\_extended} \\
The \livetarg{chap:DWCFArestoreextended}{DW\-\_CFA\-\_restore\-\_extended} instruction takes a single unsigned
LEB128 operand that represents a register number. This
instruction is identical to \livelink{chap:DWCFArestore}{DW\-\_CFA\-\_restore} except for the
encoding and size of the register operand.

\end{enumerate}

\subsubsection{Row State Instructions}
\label{chap:rowstateinstructions}

\textit{The next two instructions provide the ability to stack and
retrieve complete register states. They may be useful, for
example, for a compiler that moves \addtoindex{epilogue} code 
into the
body of a function.}


\begin{enumerate}[1.]

\item \textbf{DW\-\_CFA\-\_remember\-\_state} \\
The \livetarg{chap:DWCFArememberstate}{DW\-\_CFA\-\_remember\-\_state} instruction takes no operands. The
required action is to push the set of rules for every register
onto an implicit stack.


\item \textbf{DW\-\_CFA\-\_restore\-\_state} \\
The \livetarg{chap:DWCFArestorestate}{DW\-\_CFA\-\_restore\-\_state} instruction takes no operands. The
required action is to pop the set of rules off the implicit
stack and place them in the current row.

\end{enumerate}

\subsubsection{Padding Instruction}
\label{chap:paddinginstruction}
\begin{enumerate}[1.]
\item \textbf{DW\-\_CFA\-\_nop} \\
The \livetarg{chap:DWCFAnop}{DW\-\_CFA\-\_nop} instruction has no operands and no required
actions. It is used as padding to make a CIE or FDE an
appropriate size

\end{enumerate}

\subsection{Call Frame Instruction Usage} 
\label{chap:callframeinstructionusage}

\textit{To determine the virtual unwind rule set for a given location
(L1), one searches through the FDE headers looking at the
initial\_location and address\_range values to see if L1 is
contained in the FDE. If so, then:}

\begin{enumerate}[1.]

\item \textit{Initialize a register set by reading the
initial\_instructions field of the associated CIE.}

\item \textit{Read and process the FDE’s instruction
sequence until a \livelink{chap:DWCFAadvanceloc}{DW\-\_CFA\-\_advance\-\_loc}, 
\livelink{chap:DWCFAsetloc}{DW\-\_CFA\-\_set\-\_loc}, or the
end of the instruction stream is encountered.}

\item \textit{ If a \livelink{chap:DWCFAadvanceloc}{DW\-\_CFA\-\_advance\-\_loc} or \livelink{chap:DWCFAsetloc}{DW\-\_CFA\-\_set\-\_loc}
instruction is encountered, then compute a new location value
(L2). If L1 >= L2 then process the instruction and go back
to step 2.}

\item \textit{ The end of the instruction stream can be thought
of as a \livelink{chap:DWCFAsetloc}{DW\-\_CFA\-\_set\-\_loc} (initial\_location + address\_range)
instruction. Note that the FDE is ill-formed if L2 is less
than L1.}

\end{enumerate}

\textit{The rules in the register set now apply to location L1.}

\textit{For an example, see 
Appendix \refersec{app:callframeinformationexample}.}



\subsection{Call Frame Calling Address}
\label{chap:callframecallingaddress}

\textit{When unwinding frames, consumers frequently wish to obtain the
address of the instruction which called a subroutine. This
information is not always provided. Typically, however,
one of the registers in the virtual unwind table is the
Return Address.}

If a Return Address register is defined in the virtual
unwind table, and its rule is undefined (for example, by
\livelink{chap:DWCFAundefined}{DW\-\_CFA\-\_undefined}), then there is no return address and no
call address, and the virtual unwind of stack activations
is complete.

\textit{In most cases the return address is in the same context as the
calling address, but that need not be the case, especially if
the producer knows in some way the call never will return. The
context of the 'return address' might be on a different line,
in a different lexical \livelink{chap:lexicalblock}{block}, 
or past the end of the calling
subroutine. If a consumer were to assume that it was in the
same context as the calling address, the unwind might fail.}

\textit{For architectures with constant-length instructions where
the return address immediately follows the call instruction,
a simple solution is to subtract the length of an instruction
from the return address to obtain the calling instruction. For
architectures with variable-length instructions (e.g.  x86),
this is not possible. However, subtracting 1 from the return
address, although not guaranteed to provide the exact calling
address, generally will produce an address within the same
context as the calling address, and that usually is sufficient.}



