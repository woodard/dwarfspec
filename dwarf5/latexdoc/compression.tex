\chapter[Compression (Informative)]{DWARF Compression and Duplicate Elimination (Informative)}
\label{dwarfcompressionandduplicateeliminationinformative}

% It seemed difficult to get close to the same layout and 
% captioning as DWARF4 here with figures as they moved (floated)
% making it hard to follow.  Hence this uses fewer figures.

DWARF 
\addtoindexx{DWARF compression}
can 
\addtoindexx{DWARF duplicate elimination}
use a lot of disk space.

This is especially true for C++, where the depth and complexity
of headers can mean that many, many (possibly thousands of)
declarations are repeated in every compilation unit. C++
templates can also mean that some functions and their DWARF
descriptions get duplicated.

This Appendix describes techniques for using the DWARF
representation in combination with features and characteristics
of some common object file representations to reduce redundancy
without losing information. It is worth emphasizing that none
of these techniques are necessary to provide a complete and
accurate DWARF description; they are solely concerned with
reducing the size of DWARF information.

The techniques described here depend more directly and more
obviously on object file concepts and linker mechanisms than
most other parts of DWARF. While the presentation tends to
use the vocabulary of specific systems, this is primarily to
aid in describing the techniques by appealing to well\dash known
terminology. These techniques can be employed on any system
that supports certain general functional capabilities
(described below).


\section{Using Compilation Units}
\label{app:usingcompilationunits}

\subsection{Overview}
The general approach is to break up the debug information of
a compilation into separate normal and partial compilation
units, each consisting of one or more sections. By arranging
that a sufficiently similar partitioning occurs in other
compilations, a suitable system linker can delete redundant
groups of sections when combining object files.

\textit{The following uses some traditional section naming here
but aside from the DWARF sections, the names are just meant
to suggest traditional contents as a way of explaining the
approach, not to be limiting.}

A traditional relocatable object output 
from a single compilation might contain sections 
named:
\begin{alltt}
    \addtoindex{.data}
    \addtoindex{.text}
    \addtoindex{.debug\_info}
    \addtoindex{.debug\_abbrev}
    \addtoindex{.debug\_line}
    \addtoindex{.debug\_aranges}
\end{alltt}
A relocatable object from a compilation system 
attempting duplicate DWARF elimination might
contain sections as in:

\begin{alltt}
    \addtoindex{.data}
    \addtoindex{.text}
    \addtoindex{.debug\_info}
    \addtoindex{.debug\_abbrev}
    \addtoindex{.debug\_line}
    \addtoindex{.debug\_aranges}
\end{alltt}

followed (or preceded, the order is not significant) 
by a series of 
\addtoindexx{section group}
section groups:
\begin{alltt}
==== Section group 1
    \addtoindex{.debug\_info}
    \addtoindex{.debug\_abbrev}
    \addtoindex{.debug\_line}
==== ...
==== Section group N
    \addtoindex{.debug\_info}
    \addtoindex{.debug\_abbrev}
    \addtoindex{.debug\_line}
\end{alltt}

where each \addtoindex{section group} might or might not contain executable
code (\textit{.text} sections) or data (\textit{.data} sections).

\needlines{6}
A \textit{\addtoindex{section group}} is a named set 
of section contributions
within an object file with the property that the entire set
of section contributions must be retained or discarded as a
whole; no partial elimination is allowed. Section groups can
generally be handled by a linker in two ways:
\begin{enumerate}[1. ]

\item Given multiple identical (duplicate) section groups,
\addtoindexx{section group}
one of them is chosen to be kept and used, while the rest
are discarded.

\item Given a \addtoindex{section group} 
that is not referenced from any
section outside of the \addtoindex{section group}, 
the section group
is discarded.

\end{enumerate}


Which handling applies may be indicated by the 
\addtoindex{section group}
itself and/or selection of certain linker options.

For example, if a linker determines that 
\addtoindex{section group} 1
from A.o and 
\addtoindex{section group} 3 from B.o are identical, it could
discard one group and arrange that all references in A.o and
B.o apply to the remaining one of the two identical section
groups. This saves space.

An important part of making it possible to \doublequote{redirect}
references to the surviving 
\addtoindex{section group} is the use of
consistently chosen linker global symbols for referring to
locations within each 
\addtoindex{section group}.
It follows that references
are simply to external names and the linker already knows
how to match up references and definitions.

What is minimally needed from the object file format and system
linker (outside of DWARF itself, and normal object/linker
facilities such as simple relocations) are:
\begin{enumerate}[1. ]

\item A means of referencing from inside one \addtoindex{.debug\_info}
compilation unit to another 
\addtoindex{.debug\_info} compilation unit
(\livelink{chap:DWFORMrefaddr}{DW\-\_FORM\-\_ref\-\_addr} provides this).

\item A means of having multiple contributions to specific sections
(for example, \addtoindex{.debug\_info}, and so on) in a single object file.

\item  A means of identifying a \addtoindex{section group} (giving it a name).

\item A means of identifying which sections go together to make
up a \addtoindex{section group}, 
so that the group can be treated as a unit
(kept or discarded).

\item  A means of indicating how each \addtoindex{section group} should be
processed by the linker.

\end{enumerate}

\textit{The notion of section and section contribution used here
corresponds closely to the similarly named concepts in the
ELF object file representation. 
The notion of \addtoindex{section group} is
an abstraction of common extensions of the ELF representation
widely known as ``COMDATs'' or ``COMDAT sections''. (Other
object file representations provide COMDAT\dash style mechanisms as
well.) There are several variations in the COMDAT schemes in
common use, any of which should be sufficient for the purposes
of the 
\addtoindexx{duplication elimination|see{DWARF duplicate elimination}}
DWARF duplicate elimination techniques described here.}

\subsection{Naming and Usage Considerations}
\label{app:namingandusageconsiderations}

A precise description of the means of deriving names usable
by the linker to access DWARF entities is not part of this
specification. Nonetheless, an outline of a usable approach
is given here to make this more understandable and to guide
implementors.

Implementations should clearly document their naming conventions.

In the following, it will be helpful to refer to the examples
in 
Figure \ref{fig:duplicateeliminationexample1csource}
through 
Figure \ref{fig:duplicateeliminationexample2companiondwarf}
of 
Section \refersec{app:examples}.

\textbf{Section Group Names}

Section groups must have a \addtoindex{section group} name.
\addtoindexx{section group!name}
For the subsequent 
\addtoindex{C++} example, a name like
\begin{alltt}
    <producer-prefix>.<file-designator>.<gid-number>
\end{alltt}
will suffice, where

\begin{description}

\item  [\textless producer\dash prefix\textgreater] 
is some string specific to the
producer, which has a language\dash designation embedded in the
name when appropriate. (Alternatively, the language name
could be embedded in the 
\textless gid\dash number\textgreater).


\item  [\textless file\dash designator\textgreater]
names the file, such as wa.h in
the example.


\item  [\textless gid\dash number\textgreater]
is a string generated to identify the
specific wa.h header file in such a way that

\begin{itemize}

\item  a 'matching' output from another compile generates
the same 
\textless gid\dash number\textgreater,
and

\item  a non\dash matching output (say because of \#defines)
generates a different 
\textless gid\dash number\textgreater.
\end{itemize}

\end{description}

\textit{It may be useful to think of a 
\textless gid\dash number\textgreater
as a kind
of ``digital signature'' that allows a fast test for the
equality of two 
\addtoindexx{section group}
section groups.}

So, for example, the \addtoindex{section group} 
corresponding to file wa.h
above is given the name \textit{my.compiler.company.cpp.wa.h.123456}.



\textbf{Debugging Information Entry Names}

Global labels for 
\addtoindexx{debugging information entry!ownership relation}
debugging information entries (need explained
below) within a \addtoindex{section group}
can be given names of the form

\begin{alltt}
    <prefix>.<file-designator>.<gid-number>.<die-number>
\end{alltt}

such as

\begin{alltt}
    my.compiler.company.wa.h.123456.987
\end{alltt}

where
\begin{description}
\item [\textless prefix\textgreater]  
distinguishes this as a DWARF debug info name, and should identify the producer
and, when appropriate, the language.
\item [\textless file\dash designator\textgreater]  
and 
\texttt{\textless gid\dash number\textgreater} 
are as above.

\item  [\textless die\dash number\textgreater]
could be a number sequentially assigned 
to entities (tokens, perhaps) found
during compilation.

\end{description}

In general, every point in the 
\addtoindexx{section group}
section group 
\addtoindex{.debug\_info} that
could be referenced from outside by \emph{any} compilation unit must
normally have an external name generated for it in the linker
symbol table, whether the current compilation references all
those points or not.

\textit{The completeness of the set of names generated is a
quality\dash of\dash implementation issue.}

It is up to the producer to ensure that if 
\textless die\dash numbers\textgreater\ 
in separate compilations would not match properly then a
distinct 
\textless gid\dash number\textgreater\ 
is generated.

Note that only 
\addtoindexx{section group}
section groups that are designated as
duplicate\dash removal\dash applies actually require the
\begin{alltt}
    <prefix>.<file-designator>.<gid-number>.<die-number>
\end{alltt}
external labels for debugging information entries as all other
\addtoindex{section group} sections can use 'local' labels 
(section\dash relative
relocations).

(This is a consequence of separate compilation, not a rule
imposed by this document.)

\textit{Local labels use references with form \livelink{chap:DWFORMref4}{DW\-\_FORM\-\_ref4}
or 
\livelink{chap:DWFORMref8}{DW\-\_FORM\-\_ref8}. 
(These are affected by relocations
so 
\livelink{chap:DWFORMrefudata}{DW\-\_FORM\-\_ref\-\_udata}, 
\livelink{chap:DWFORMref1}{DW\-\_FORM\-\_ref1} and 
\livelink{chap:DWFORMref2}{DW\-\_FORM\-\_ref2} are
normally not usable and 
\livelink{chap:DWFORMrefaddr}{DW\-\_FORM\-\_ref\-\_addr} is not necessary
for a local label.)}


\subsubsection{Use of DW\-\_TAG\-\_compile\-\_unit versus 
DW\-\_TAG\-\_partial\-\_unit}

A \addtoindex{section group} compilation unit that uses 
\livelink{chap:DWTAGcompileunit}{DW\-\_TAG\-\_compile\-\_unit}
is like any other compilation unit, in that its contents
are evaluated by consumers as though it were an ordinary
compilation unit.

An \#include directive appearing outside any other
declarations is a good candidate to be represented using
\livelink{chap:DWTAGcompileunit}{DW\-\_TAG\-\_compile\-\_unit}. 
However, an \#include appearing inside
a \addtoindex{C++} namespace declaration or a function, for example, is
not a good candidate because the entities included are not
necessarily file level entities.

This also applies to \addtoindex{Fortran} INCLUDE lines when declarations
are included into a procedure or module context.

Consequently a compiler must use \livelink{chap:DWTAGpartialunit}{DW\-\_TAG\-\_partial\-\_unit} (instead
of \livelink{chap:DWTAGcompileunit}{DW\-\_TAG\-\_compile\-\_unit}) 
in a \addtoindex{section group} 
whenever the section group 
contents are not necessarily globally visible. 
This
directs consumers to ignore that compilation unit when scanning
top level declarations and definitions.

The \livelink{chap:DWTAGpartialunit}{DW\-\_TAG\-\_partial\-\_unit} compilation unit will be referenced
from elsewhere and the referencing locations give the
appropriate context for interpreting the partial compilation
unit.

A \livelink{chap:DWTAGpartialunit}{DW\-\_TAG\-\_partial\-\_unit} entry may have, as appropriate, any of
the attributes assigned to a \livelink{chap:DWTAGcompileunit}{DW\-\_TAG\-\_compile\-\_unit}.


\subsubsection{Use of DW\-\_TAG\-\_imported\-\_unit}

A \livelink{chap:DWTAGimportedunit}{DW\-\_TAG\-\_imported\-\_unit} debugging information entry has an
\livelink{chap:DWATimport}{DW\-\_AT\-\_import} attribute referencing a \livelink{chap:DWTAGcompileunit}{DW\-\_TAG\-\_compile\-\_unit} or
\livelink{chap:DWTAGpartialunit}{DW\-\_TAG\-\_partial\-\_unit} debugging information entry.

A \livelink{chap:DWTAGimportedunit}{DW\-\_TAG\-\_imported\-\_unit} debugging information entry refers
to a 
\livelink{chap:DWTAGcompileunit}{DW\-\_TAG\-\_compile\-\_unit} or 
\livelink{chap:DWTAGpartialunit}{DW\-\_TAG\-\_partial\-\_unit} debugging
information entry to specify that the 
\livelink{chap:DWTAGcompileunit}{DW\-\_TAG\-\_compile\-\_unit} or
\livelink{chap:DWTAGpartialunit}{DW\-\_TAG\-\_partial\-\_unit} contents logically appear at the point
of the 
\livelink{chap:DWTAGimportedunit}{DW\-\_TAG\-\_imported\-\_unit} entry.


\subsubsection{Use of DW\-\_FORM\-\_ref\-\_addr}

Use 
\livelink{chap:DWFORMrefaddr}{DW\-\_FORM\-\_ref\-\_addr} to reference from one compilation
unit's debugging information entries to those of another
compilation unit.

When referencing into a removable \addtoindex{section group}
\addtoindex{.debug\_info}
from another \addtoindex{.debug\_info} (from anywhere), the
\texttt{<prefix>.<file-designator>.<gid-number>.<die-number>}
name should be used for an external symbol and a relocation
generated based on that name.

\textit{When referencing into a 
\addtoindexx{section group}
non-section group 
\addtoindex{.debug\_info},
from another \addtoindex{.debug\_info} (from anywhere) 
\livelink{chap:DWFORMrefaddr}{DW\-\_FORM\-\_ref\-\_addr} is
still the form to be used, but a section\dash relative relocation
generated by use of a non-exported name (often called an
``internal name'') may be used for references within the
same object file.}

\subsection{Examples}
\label{app:examples}

This section provides several 
\addtoindexx{DWARF duplicate elimination!examples}
examples in order to have a
concrete basis for discussion.

In these examples, the focus is on the arrangement of DWARF
information into sections (specifically the 
\addtoindex{.debug\_info}
section) and the naming conventions used to achieve references
into 
\addtoindexx{section group}
section groups. 
In practice, all of the examples that
follow involve DWARF sections other than just 
\addtoindex{.debug\_info}
(for example, \addtoindex{.debug\_line}, 
\addtoindex{.debug\_aranges}, or others);
however, only the \addtoindex{.debug\_info}
section is shown to keep the
examples compact and easier to read.

The grouping of sections into a named set is shown, but the means for achieving this in terms of
the underlying object language is not (and varies from system to system).

\subsubsection{C++ Example}

The \addtoindex{C++} source 
\addtoindexx{DWARF duplicate elimination!C++ example}
in 
Figure \refersec{fig:duplicateeliminationexample1csource}
is used to illustrate the DWARF
representation intended to allow duplicate elimination.

\begin{figure}[ht]
\textit{File wa.h}
\begin{lstlisting}[numbers=none]
struct A {
   int i;
};
\end{lstlisting}
\textit{File wa.c}
\begin{lstlisting}[numbers=none]
#include "wa.h";
int
f(A &a)
{
    return a.i + 2;
}
\end{lstlisting}
\caption{Duplicate elimination example \#1: C++ Source}
\label{fig:duplicateeliminationexample1csource}
\end{figure}

Figure \refersec{fig:duplicateeliminationexample1dwarfsectiongroup}
shows the \addtoindex{section group} corresponding to the included file 
wa.h.

\begin{figure}
\begin{dwflisting}
% FIXME: the DWFORMrefn could use rethinking
\begin{alltt}
==== Section group name:
    my.compiler.company.cpp.wa.h.123456
== section \addtoindex{.debug\_info}
DW.cpp.wa.h.123456.1:     ! linker global symbol
    \livelink{chap:DWTAGcompileunit}{DW\-\_TAG\-\_compile\-\_unit}
        \livelink{chap:DWATlanguage}{DW\-\_AT\-\_language}(\livelink{chap:DWLANGCplusplus}{DW\-\_LANG\-\_C\-\_plus\-\_plus})
        ...  ! other unit attributes
DW.cpp.wa.h.123456.2:     ! linker global symbol
    \livelink{chap:DWTAGbasetype}{DW\-\_TAG\-\_base\-\_type}
        \livelink{chap:DWATname}{DW\-\_AT\-\_name}("int")
DW.cpp.wa.h.123456.3:     ! linker global symbol
    \livelink{chap:DWTAGstructuretype}{DW\-\_TAG\-\_structure\-\_type}
        \livelink{chap:DWATname}{DW\-\_AT\-\_name}("A")
DW.cpp.wa.h.123456.4:     ! linker global symbol
        \livelink{chap:DWTAGmember}{DW\-\_TAG\-\_member}
        \livelink{chap:DWATname}{DW\-\_AT\-\_name}("i")
        \livelink{chap:DWATtype}{DW\-\_AT\-\_type}(\livelink{chap:DWFORMrefn}{DW\-\_FORM\-\_refn} to DW.cpp.wa.h.123456.2)
            ! (This is a local reference, so the more
            ! compact form \livetarg{chap:DWFORMrefn}{DW\-\_FORM\-\_refn} 
            ! for n 1,2,4, or 8 can be used)
\end{alltt}
\end{dwflisting}
\vspace{2mm}
\caption{Duplicate elimination example \#1: DWARF section group} 
\label{fig:duplicateeliminationexample1dwarfsectiongroup}
\end{figure}

Figure \refersec{fig:duplicateeliminationexample1primarycompilationunit}
shows the ``normal'' DWARF sections, which are not part of
any \addtoindex{section group}, 
and how they make use of the information
in the \addtoindex{section group} shown above.

\begin{figure}
\begin{dwflisting}
\begin{alltt}
== section .text
    [generated code for function f]
== section \addtoindex{.debug\_info}
    \livelink{chap:DWTAGcompileunit}{DW\-\_TAG\-\_compile\-\_unit}
.L1:                           ! local (non-linker) symbol
        \livelink{chap:DWTAGreferencetype}{DW\-\_TAG\-\_reference\-\_type}
            \livelink{chap:DWATtype}{DW\-\_AT\-\_type}(reference to DW.cpp.wa.h.123456.3)
        \livelink{chap:DWTAGsubprogram}{DW\-\_TAG\-\_subprogram}
            \livelink{chap:DWATname}{DW\-\_AT\-\_name}("f")
            \livelink{chap:DWATtype}{DW\-\_AT\-\_type}(reference to DW.cpp.wa.h.123456.2)
            \livelink{chap:DWTAGvariable}{DW\-\_TAG\-\_variable}
                \livelink{chap:DWATname}{DW\-\_AT\-\_name}("a")
                \livelink{chap:DWATtype}{DW\-\_AT\-\_type}(reference to .L1)
        ...
\end{alltt}
\end{dwflisting}
\caption{Duplicate elimination example \#1: primary compilation unit} 
\label{fig:duplicateeliminationexample1primarycompilationunit}
\end{figure}

This example uses \livelink{chap:DWTAGcompileunit}{DW\-\_TAG\-\_compile\-\_unit} for the \addtoindex{section group},
implying that the contents of the compilation unit are
globally visible (in accordance with 
\addtoindex{C++} language rules).
\livelink{chap:DWTAGpartialunit}{DW\-\_TAG\-\_partial\-\_unit} is not needed for the same reason.

\subsubsection{Fortran Example}


For a \addtoindex{Fortran}
\addtoindexx{DWARF duplicate elimination!Fortran example}
example, consider 
Figure \refersec{fig:duplicateeliminationexample2fortransource}.

\begin{figure}
\textit{File CommonStuff.fh}
\addtoindexx{Fortran}
\begin{lstlisting}[numbers=none]
IMPLICIT INTEGER(A-Z)
COMMON /Common1/ C(100)
PARAMETER(SEVEN = 7)
\end{lstlisting}

\textit{File Func.f}
\begin{lstlisting}[numbers=none]
FUNCTION FOO (N)
INCLUDE 'CommonStuff.fh'
FOO = C(N + SEVEN)
RETURN
END
\end{lstlisting}
\caption{Duplicate elimination example \#2: Fortran source} 
\label{fig:duplicateeliminationexample2fortransource}
\end{figure}


Figure \refersec{fig:duplicateeliminationexample2dwarfsectiongroup}
shows the \addtoindex{section group}
corresponding to the included file 
\addtoindexx{Fortran example}
CommonStuff.fh.

\begin{figure}
\begin{dwflisting}
\begin{alltt}
==== Section group name:

    my.f90.company.f90.CommonStuff.fh.654321

== section \addtoindex{.debug\_info}

DW.myf90.CommonStuff.fh.654321.1:    ! linker global symbol
    \livelink{chap:DWTAGpartialunit}{DW\-\_TAG\-\_partial\-\_unit}
        ! ...compilation unit attributes, including...
        \livelink{chap:DWATlanguage}{DW\-\_AT\-\_language}(\livelink{chap:DWLANGFortran90}{DW\-\_LANG\-\_Fortran90})
        \livelink{chap:DWATidentifiercase}{DW\-\_AT\-\_identifier\-\_case}(\livelink{chap:DWIDcaseinsensitive}{DW\-\_ID\-\_case\-\_insensitive})

DW.myf90.CommonStuff.fh.654321.2:    ! linker global symbol
3\$: \livelink{chap:DWTAGarraytype}{DW\-\_TAG\-\_array\-\_type}
        ! unnamed
        \livelink{chap:DWATtype}{DW\-\_AT\-\_type}(reference to DW.f90.F90\$main.f.2)
            ! base type INTEGER
        \livelink{chap:DWTAGsubrangetype}{DW\-\_TAG\-\_subrange\-\_type}
            \livelink{chap:DWATtype}{DW\-\_AT\-\_type}(reference to DW.f90.F90\$main.f.2)
                ! base type INTEGER)
            \livelink{chap:DWATlowerbound}{DW\-\_AT\-\_lower\-\_bound}(constant 1)
            \livelink{chap:DWATupperbound}{DW\-\_AT\-\_upper\-\_bound}(constant 100)

DW.myf90.CommonStuff.fh.654321.3:    ! linker global symbol
    \livelink{chap:DWTAGcommonblock}{DW\-\_TAG\-\_common\-\_block}
        \livelink{chap:DWATname}{DW\-\_AT\-\_name}("Common1")
        \livelink{chap:DWATlocation}{DW\-\_AT\-\_location}(Address of common \nolink{block} Common1)
        \livelink{chap:DWTAGvariable}{DW\-\_TAG\-\_variable}
            \livelink{chap:DWATname}{DW\-\_AT\-\_name}("C")
            \livelink{chap:DWATtype}{DW\-\_AT\-\_type}(reference to 3\$)
            \livelink{chap:DWATlocation}{DW\-\_AT\-\_location}(address of C)

DW.myf90.CommonStuff.fh.654321.4:    ! linker global symbol
    \livelink{chap:DWTAGconstant}{DW\-\_TAG\-\_constant}
        \livelink{chap:DWATname}{DW\-\_AT\-\_name}("SEVEN")
        \livelink{chap:DWATtype}{DW\-\_AT\-\_type}(reference to DW.f90.F90\$main.f.2)
            ! base type INTEGER
        \livelink{chap:DWATconstvalue}{DW\-\_AT\-\_const\-\_value}(constant 7)
\end{alltt}
\end{dwflisting}
\caption{Duplicate elimination example \#2: DWARF section group}
\label{fig:duplicateeliminationexample2dwarfsectiongroup}
\end{figure}

Figure \refersec{fig:duplicateeliminationexample2primaryunit}
shows the sections for the primary compilation unit.

\begin{figure}
\begin{dwflisting}
\begin{alltt}
== section .text
    [code for function Foo]

== section \addtoindex{.debug\_info}
    \livelink{chap:DWTAGcompileunit}{DW\-\_TAG\-\_compile\-\_unit}
        \livelink{chap:DWTAGsubprogram}{DW\-\_TAG\-\_subprogram}
            \livelink{chap:DWATname}{DW\-\_AT\-\_name}("Foo")
            \livelink{chap:DWATtype}{DW\-\_AT\-\_type}(reference to DW.f90.F90\$main.f.2)
                ! base type INTEGER
            \livelink{chap:DWTAGimportedunit}{DW\-\_TAG\-\_imported\-\_unit}
                \livelink{chap:DWATimport}{DW\-\_AT\-\_import}(reference to
                    DW.myf90.CommonStuff.fh.654321.1)
            \livelink{chap:DWTAGcommoninclusion}{DW\-\_TAG\-\_common\-\_inclusion} ! For Common1
                \livelink{chap:DWATcommonreference}{DW\-\_AT\-\_common\-\_reference}(reference to
                    DW.myf90.CommonStuff.fh.654321.3)
            \livelink{chap:DWTAGvariable}{DW\-\_TAG\-\_variable} ! For function result
                \livelink{chap:DWATname}{DW\-\_AT\-\_name}("Foo")
                    \livelink{chap:DWATtype}{DW\-\_AT\-\_type}(reference to DW.f90.F90\$main.f.2)
                        ! base type INTEGER
\end{alltt}
\end{dwflisting}
\caption{Duplicate elimination example \#2: primary unit}
\label{fig:duplicateeliminationexample2primaryunit}
\end{figure}

A companion main program is shown in 
Figure \refersec{fig:duplicateeliminationexample2companionsource}

\begin{figure}
\textit{File Main.f} 
\begin{lstlisting}[numbers=none]
INCLUDE 'CommonStuff.fh'
C(50) = 8
PRINT *, 'Result = ', FOO(50 - SEVEN)
END
\end{lstlisting}
\caption{Duplicate elimination example \#2: companion source }
\label{fig:duplicateeliminationexample2companionsource}
\end{figure}

\needlines{3}
That main program results in an object file that
contained a duplicate of the \addtoindex{section group} named
my.f90.company.f90.CommonStuff.fh.654321 
corresponding to the
included file as well as the remainder of the main subprogram
as shown in 
Figure \refersec{fig:duplicateeliminationexample2companiondwarf}.

\begin{figure}
\begin{dwflisting}
\begin{alltt}
== section \addtoindex{.debug\_info}
    \livelink{chap:DWTAGcompileunit}{DW\-\_TAG\-\_compile\-\_unit}
        \livelink{chap:DWATname}{DW\-\_AT\-\_name}(F90\$main)
        \livelink{chap:DWTAGbasetype}{DW\-\_TAG\-\_base\-\_type}
            \livelink{chap:DWATname}{DW\-\_AT\-\_name}("INTEGER")
            \livelink{chap:DWATencoding}{DW\-\_AT\-\_encoding}(\livelink{chap:DWATEsigned}{DW\-\_ATE\-\_signed})
            \livelink{chap:DWATbytesize}{DW\-\_AT\-\_byte\-\_size}(...)

        \livelink{chap:DWTAGbasetype}{DW\-\_TAG\-\_base\-\_type}
            ...
        ...  ! other base types
        \livelink{chap:DWTAGsubprogram}{DW\-\_TAG\-\_subprogram}
            \livelink{chap:DWATname}{DW\-\_AT\-\_name}("F90\$main")
            \livelink{chap:DWTAGimportedunit}{DW\-\_TAG\-\_imported\-\_unit}
                \livelink{chap:DWATimport}{DW\-\_AT\-\_import}(reference to
                    DW.myf90.CommonStuff.fh.654321.1)
            \livelink{chap:DWTAGcommoninclusion}{DW\-\_TAG\-\_common\-\_inclusion} ! for Common1
                \livelink{chap:DWATcommonreference}{DW\-\_AT\-\_common\-\_reference}(reference to
                    DW.myf90.CommonStuff.fh.654321.3)
            ...
\end{alltt}
\end{dwflisting}
\caption{Duplicate elimination example \#2: companion DWARF }
\label{fig:duplicateeliminationexample2companiondwarf}
\end{figure}

This example uses \livelink{chap:DWTAGpartialunit}{DW\-\_TAG\-\_partial\-\_unit} for the \addtoindex{section group}
because the included declarations are not independently
visible as global entities.


\needlines{6}
\subsubsection{C Example}

The \addtoindex{C} example 
\addtoindexx{DWARF duplicate elimination!C example}
in this Section might appear to be equally
valid as a \addtoindex{C} example. However, it is prudent to include
a \livelink{chap:DWTAGimportedunit}{DW\-\_TAG\-\_imported\-\_unit}
in the primary unit 
(see Figure \refersec{fig:duplicateeliminationexample1primarycompilationunit})
with an \livelink{chap:DWATimport}{DW\-\_AT\-\_import} attribute that refers to the proper unit
in the \addtoindex{section group}.

\textit{The \addtoindex{C} rules for consistency of global (file scope) symbols
across compilations are less strict than for C++; inclusion
of the import unit attribute assures that the declarations of
the proper \addtoindex{section group} are considered before declarations
from other compilations.}


\section{Using Type Units}
\label{app:usingtypeunits}

A large portion of debug information is type information, and
in a typical compilation environment, many types are duplicated
many times. One method of controlling the amount of duplication
is separating each type into a separate 
\addtoindex{.debug\_types} section
and arranging for the linker to recognize and eliminate
duplicates at the individual type level.

Using this technique, each substantial type definition is
placed in its own individual section, while the remainder
of the DWARF information (non-type information, incomplete
type declarations, and definitions of trivial types) is
placed in the usual debug information section. In a typical
implementation, the relocatable object file may contain one
of each of these debug sections:

\begin{alltt}
\addtoindex{.debug\_abbrev}
\addtoindex{.debug\_info}
\addtoindex{.debug\_line}
\end{alltt}

and any number of these additional sections:

\begin{alltt}
\addtoindex{.debug\_types}
\end{alltt}

As discussed in the previous section 
(Section \refersec{app:usingcompilationunits}), 
many
linkers today support the concept of a COMDAT group or
linkonce section. The general idea is that a ``key'' can be
attached to a section or a group of sections, and the linker
will include only one copy of a \addtoindex{section group}
(or individual section) for any given key. 
For 
\addtoindex{.debug\_types} sections, the
key is the \addtoindex{type signature}
formed from the algorithm given in
Section \refersec{datarep:typesignaturecomputation}.

\subsection{Signature Computation Example}
\label{app:signaturecomputationexample}

As an example, 
\addtoindexx{type signature!example computation}
consider a \addtoindex{C++} header file 
containing the type definitions shown
in Figure \refersec{fig:typesignatureexamplescsource}.

\begin{figure}[h]
\begin{lstlisting}
namespace N {

    struct B;

    struct C {
        int x;
        int y;
    };

    class A {
    public:
        A(int v);
        int v();
    private:
        int v_;
        struct A *next;
        struct B *bp;
        struct C c;
    };
}
\end{lstlisting}
\caption{Type signature examples: C++ source}
\label{fig:typesignatureexamplescsource}
\end{figure}

Next, consider one possible representation of the DWARF
information that describes the type \doublequote{struct C} as shown
in 
\refersec{fig:typesignaturecomputation1dwarfrepresentation}:

\begin{figure}
\begin{dwflisting}
% We keep the : (colon) away from the attribute so tokenizing in the python tools
% does not result in adding : into the attribute name.
\begin{alltt}
  \livelink{chap:DWTAGtypeunit}{DW\-\_TAG\-\_type\-\_unit}
      \livelink{chap:DWATlanguage}{DW\-\_AT\-\_language} : \livelink{chap:DWLANGCplusplus}{DW\-\_LANG\-\_C\-\_plus\-\_plus} (4)
    \livelink{chap:DWTAGnamespace}{DW\-\_TAG\-\_namespace}
        \livelink{chap:DWATname}{DW\-\_AT\-\_name} : "N"
L1:
      \livelink{chap:DWTAGstructuretype}{DW\-\_TAG\-\_structure\-\_type}
          \livelink{chap:DWATname}{DW\-\_AT\-\_name} : "C"
          \livelink{chap:DWATbytesize}{DW\-\_AT\-\_byte\-\_size} : 8
          \livelink{chap:DWATdeclfile}{DW\-\_AT\-\_decl\-\_file} : 1
          \livelink{chap:DWATdeclline}{DW\-\_AT\-\_decl\-\_line} : 5
        \livelink{chap:DWTAGmember}{DW\-\_TAG\-\_member}
            \livelink{chap:DWATname}{DW\-\_AT\-\_name} : "x"
            \livelink{chap:DWATdeclfile}{DW\-\_AT\-\_decl\-\_file} : 1
            \livelink{chap:DWATdeclline}{DW\-\_AT\-\_decl\-\_line} : 6
            \livelink{chap:DWATtype}{DW\-\_AT\-\_type} : reference to L2
            \livelink{chap:DWATdatamemberlocation}{DW\-\_AT\-\_data\-\_member\-\_location} : 0
        \livelink{chap:DWTAGmember}{DW\-\_TAG\-\_member}
            \livelink{chap:DWATname}{DW\-\_AT\-\_name} : "y"
            \livelink{chap:DWATdeclfile}{DW\-\_AT\-\_decl\-\_file} : 1
            \livelink{chap:DWATdeclline}{DW\-\_AT\-\_decl\-\_line} : 7
            \livelink{chap:DWATtype}{DW\-\_AT\-\_type} : reference to L2
            \livelink{chap:DWATdatamemberlocation}{DW\-\_AT\-\_data\-\_member\-\_location} : 4
L2:
     \livelink{chap:DWTAGbasetype}{DW\-\_TAG\-\_base\-\_type}
         \livelink{chap:DWATbytesize}{DW\-\_AT\-\_byte\-\_size} : 4
         \livelink{chap:DWATencoding}{DW\-\_AT\-\_encoding} : \livelink{chap:DWATEsigned}{DW\-\_ATE\-\_signed}
         \livelink{chap:DWATname}{DW\-\_AT\-\_name} : "int"
\end{alltt}
\end{dwflisting}
\caption{Type signature computation \#1: DWARF representation}
\label{fig:typesignaturecomputation1dwarfrepresentation}
\end{figure}

\needlines{3}
In computing a signature for the type N::C, flatten the type
\addtoindexx{type signature}
description into a byte stream according to the procedure
outlined in 
Section \refersec{datarep:typesignaturecomputation}.
The result is shown in 
Figure \refersec{fig:typesignaturecomputation1flattenedbytestream}.

\begin{figure}
\begin{dwflisting}
\begin{alltt}
// Step 2: 'C' \livelink{chap:DWTAGnamespace}{DW\-\_TAG\-\_namespace} "N"
0x43 0x39 0x4e 0x00
// Step 3: 'D' \livelink{chap:DWTAGstructuretype}{DW\-\_TAG\-\_structure\-\_type}
0x44 0x13
// Step 4: 'A' \livelink{chap:DWATname}{DW\-\_AT\-\_name} \livelink{chap:DWFORMstring}{DW\-\_FORM\-\_string} "C"
0x41 0x03 0x08 0x43 0x00
// Step 4: 'A' \livelink{chap:DWATbytesize}{DW\-\_AT\-\_byte\-\_size} \livelink{chap:DWFORMsdata}{DW\-\_FORM\-\_sdata} 8
0x41 0x0b 0x0d 0x08
// Step 7: First child ("x")
    // Step 3: 'D' \livelink{chap:DWTAGmember}{DW\-\_TAG\-\_member}
    0x44 0x0d
    // Step 4: 'A' \livelink{chap:DWATname}{DW\-\_AT\-\_name} \livelink{chap:DWFORMstring}{DW\-\_FORM\-\_string} "x"
    0x41 0x03 0x08 0x78 0x00
    // Step 4: 'A' \livelink{chap:DWATdatamemberlocation}{DW\-\_AT\-\_data\-\_member\-\_location} \livelink{chap:DWFORMsdata}{DW\-\_FORM\-\_sdata} 0
    0x41 0x38 0x0d 0x00
    // Step 6: 'T' \livelink{chap:DWATtype}{DW\-\_AT\-\_type} (type \#2)
    0x54 0x49
        // Step 3: 'D' \livelink{chap:DWTAGbasetype}{DW\-\_TAG\-\_base\-\_type}
        0x44 0x24
        // Step 4: 'A' \livelink{chap:DWATname}{DW\-\_AT\-\_name} \livelink{chap:DWFORMstring}{DW\-\_FORM\-\_string} "int"
        0x41 0x03 0x08 0x69 0x6e 0x74 0x00
        // Step 4: 'A' \livelink{chap:DWATbytesize}{DW\-\_AT\-\_byte\-\_size} \livelink{chap:DWFORMsdata}{DW\-\_FORM\-\_sdata} 4
        0x41 0x0b 0x0d 0x04
        // Step 4: 'A' \livelink{chap:DWATencoding}{DW\-\_AT\-\_encoding} \livelink{chap:DWFORMsdata}{DW\-\_FORM\-\_sdata} \livelink{chap:DWATEsigned}{DW\-\_ATE\-\_signed}
        0x41 0x3e 0x0d 0x05
        // Step 7: End of \livelink{chap:DWTAGbasetype}{DW\-\_TAG\-\_base\-\_type} "int"
        0x00
    // Step 7: End of \livelink{chap:DWTAGmember}{DW\-\_TAG\-\_member} "x"
    0x00
// Step 7: Second child ("y")
    // Step 3: 'D' \livelink{chap:DWTAGmember}{DW\-\_TAG\-\_member}
    0x44 0x0d
    // Step 4: 'A' \livelink{chap:DWATname}{DW\-\_AT\-\_name} \livelink{chap:DWFORMstring}{DW\-\_FORM\-\_string} "y"
    0x41 0x03 0x08 0x78 0x00
    // Step 4: 'A' \livelink{chap:DWATdatamemberlocation}{DW\-\_AT\-\_data\-\_member\-\_location} \livelink{chap:DWFORMsdata}{DW\-\_FORM\-\_sdata} 4
    0x41 0x38 0x0d 0x04
    // Step 6: 'R' \livelink{chap:DWATtype}{DW\-\_AT\-\_type} (type \#2)
    0x52 0x49 0x02
    // Step 7: End of \livelink{chap:DWTAGmember}{DW\-\_TAG\-\_member} "y"
    0x00
// Step 7: End of \livelink{chap:DWTAGstructuretype}{DW\-\_TAG\-\_structure\-\_type} "C"
0x00
\end{alltt}
\end{dwflisting}
\caption{Type signature computation \#1: flattened byte stream}
\label{fig:typesignaturecomputation1flattenedbytestream}
\end{figure}

\needlines{4}
Running an \addtoindex{MD5 hash} over this byte stream, and taking the
low\dash order 64 bits, yields the final signature: 
0xd28081e8 dcf5070a.

Next, consider a representation of the DWARF information that
describes the type ``class A'' as shown in 
Figure \refersec{fig:typesignaturecomputation2dwarfrepresentation}.

\begin{figure}
\figurepart{1}{2}
\begin{dwflisting}
\begin{alltt}
  \livelink{chap:DWTAGtypeunit}{DW\-\_TAG\-\_type\-\_unit}
      \livelink{chap:DWATlanguage}{DW\-\_AT\-\_language} : \livelink{chap:DWLANGCplusplus}{DW\-\_LANG\-\_C\-\_plus\-\_plus} (4)
    \livelink{chap:DWTAGnamespace}{DW\-\_TAG\-\_namespace}
        \livelink{chap:DWATname}{DW\-\_AT\-\_name} : "N"
L1:
        \livelink{chap:DWTAGclasstype}{DW\-\_TAG\-\_class\-\_type}
             \livelink{chap:DWATname}{DW\-\_AT\-\_name} : "A"
             \livelink{chap:DWATbytesize}{DW\-\_AT\-\_byte\-\_size} : 20
             \livelink{chap:DWATdeclfile}{DW\-\_AT\-\_decl\-\_file} : 1
             \livelink{chap:DWATdeclline}{DW\-\_AT\-\_decl\-\_line} : 10
           \livelink{chap:DWTAGmember}{DW\-\_TAG\-\_member}
                \livelink{chap:DWATname}{DW\-\_AT\-\_name} : "v\_"
                \livelink{chap:DWATdeclfile}{DW\-\_AT\-\_decl\-\_file} : 1
                \livelink{chap:DWATdeclline}{DW\-\_AT\-\_decl\-\_line} : 15
                \livelink{chap:DWATtype}{DW\-\_AT\-\_type} : reference to L2
                \livelink{chap:DWATdatamemberlocation}{DW\-\_AT\-\_data\-\_member\-\_location} : 0
                \livelink{chap:DWATaccessibility}{DW\-\_AT\-\_accessibility} : \livelink{chap:DWACCESSprivate}{DW\-\_ACCESS\-\_private}
          \livelink{chap:DWTAGmember}{DW\-\_TAG\-\_member}
               \livelink{chap:DWATname}{DW\-\_AT\-\_name} : "next"
               \livelink{chap:DWATdeclfile}{DW\-\_AT\-\_decl\-\_file} : 1
               \livelink{chap:DWATdeclline}{DW\-\_AT\-\_decl\-\_line} : 16
               \livelink{chap:DWATtype}{DW\-\_AT\-\_type} : reference to L3
               \livelink{chap:DWATdatamemberlocation}{DW\-\_AT\-\_data\-\_member\-\_location} : 4
               \livelink{chap:DWATaccessibility}{DW\-\_AT\-\_accessibility} : \livelink{chap:DWACCESSprivate}{DW\-\_ACCESS\-\_private}
          \livelink{chap:DWTAGmember}{DW\-\_TAG\-\_member}
               \livelink{chap:DWATname}{DW\-\_AT\-\_name} : "bp"
               \livelink{chap:DWATdeclfile}{DW\-\_AT\-\_decl\-\_file} : 1
               \livelink{chap:DWATdeclline}{DW\-\_AT\-\_decl\-\_line} : 17
               \livelink{chap:DWATtype}{DW\-\_AT\-\_type} : reference to L4
               \livelink{chap:DWATdatamemberlocation}{DW\-\_AT\-\_data\-\_member\-\_location} : 8
               \livelink{chap:DWATaccessibility}{DW\-\_AT\-\_accessibility} : \livelink{chap:DWACCESSprivate}{DW\-\_ACCESS\-\_private}
          \livelink{chap:DWTAGmember}{DW\-\_TAG\-\_member}
               \livelink{chap:DWATname}{DW\-\_AT\-\_name} : "c"
               \livelink{chap:DWATdeclfile}{DW\-\_AT\-\_decl\-\_file} : 1
               \livelink{chap:DWATdeclline}{DW\-\_AT\-\_decl\-\_line} : 18
               \livelink{chap:DWATtype}{DW\-\_AT\-\_type} : 0xd28081e8 dcf5070a (signature for struct C)
               \livelink{chap:DWATdatamemberlocation}{DW\-\_AT\-\_data\-\_member\-\_location} : 12
               \livelink{chap:DWATaccessibility}{DW\-\_AT\-\_accessibility} : \livelink{chap:DWACCESSprivate}{DW\-\_ACCESS\-\_private}
\end{alltt}
\end{dwflisting}
\caption{Type signature computation \#2: DWARF representation}
\label{fig:typesignaturecomputation2dwarfrepresentation}
\end{figure}

\begin{figure}
\figurepart{2}{2}
\begin{dwflisting}
\begin{alltt}
      \livelink{chap:DWTAGsubprogram}{DW\-\_TAG\-\_subprogram}
           \livelink{chap:DWATexternal}{DW\-\_AT\-\_external} : 1
           \livelink{chap:DWATname}{DW\-\_AT\-\_name} : "A"
           \livelink{chap:DWATdeclfile}{DW\-\_AT\-\_decl\-\_file} : 1
           \livelink{chap:DWATdeclline}{DW\-\_AT\-\_decl\-\_line} : 12
           \livelink{chap:DWATdeclaration}{DW\-\_AT\-\_declaration} : 1
        \livelink{chap:DWTAGformalparameter}{DW\-\_TAG\-\_formal\-\_parameter}
           \livelink{chap:DWATtype}{DW\-\_AT\-\_type} : reference to L3
           \livelink{chap:DWATartificial}{DW\-\_AT\-\_artificial} : 1
        \livelink{chap:DWTAGformalparameter}{DW\-\_TAG\-\_formal\-\_parameter}
           \livelink{chap:DWATtype}{DW\-\_AT\-\_type} : reference to L2
       \livelink{chap:DWTAGsubprogram}{DW\-\_TAG\-\_subprogram}
           \livelink{chap:DWATexternal}{DW\-\_AT\-\_external} : 1
           \livelink{chap:DWATname}{DW\-\_AT\-\_name} : "v"
           \livelink{chap:DWATdeclfile}{DW\-\_AT\-\_decl\-\_file} : 1
           \livelink{chap:DWATdeclline}{DW\-\_AT\-\_decl\-\_line} : 13
           \livelink{chap:DWATtype}{DW\-\_AT\-\_type} : reference to L2
           \livelink{chap:DWATdeclaration}{DW\-\_AT\-\_declaration} : 1
         \livelink{chap:DWTAGformalparameter}{DW\-\_TAG\-\_formal\-\_parameter}
           \livelink{chap:DWATtype}{DW\-\_AT\-\_type} : reference to L3
           \livelink{chap:DWATartificial}{DW\-\_AT\-\_artificial} : 1
L2:
    \livelink{chap:DWTAGbasetype}{DW\-\_TAG\-\_base\-\_type}
         \livelink{chap:DWATbytesize}{DW\-\_AT\-\_byte\-\_size} : 4
         \livelink{chap:DWATencoding}{DW\-\_AT\-\_encoding} : \livelink{chap:DWATEsigned}{DW\-\_ATE\-\_signed}
         \livelink{chap:DWATname}{DW\-\_AT\-\_name} : "int"
L3:
    \livelink{chap:DWTAGpointertype}{DW\-\_TAG\-\_pointer\-\_type}
         \livelink{chap:DWATtype}{DW\-\_AT\-\_type} : reference to L1
L4:
    \livelink{chap:DWTAGpointertype}{DW\-\_TAG\-\_pointer\-\_type}
         \livelink{chap:DWATtype}{DW\-\_AT\-\_type} : reference to L5
    \livelink{chap:DWTAGnamespace}{DW\-\_TAG\-\_namespace}
         \livelink{chap:DWATname}{DW\-\_AT\-\_name} : "N"
L5:
       \livelink{chap:DWTAGstructuretype}{DW\-\_TAG\-\_structure\-\_type}
           \livelink{chap:DWATname}{DW\-\_AT\-\_name} : "B"
           \livelink{chap:DWATdeclaration}{DW\-\_AT\-\_declaration} : 1
\end{alltt}
\end{dwflisting}
\begin{center}
Figure~\ref{fig:typesignaturecomputation2dwarfrepresentation} Type signature computation \#2: DWARF representation \textit{(concluded)}
\end{center}
\end{figure}

In this example, the structure types N::A and N::C have each
been placed in separate 
\addtoindexx{type unit}
type units.  For N::A, the actual
definition of the type begins at label L1. The definition
involves references to the int base type and to two pointer
types. The information for each of these referenced types is
also included in this \addtoindex{type unit}, 
since base types and pointer
types are trivial types that are not worth the overhead of a
separate \addtoindex{type unit}. 
The last pointer type contains a reference
to an incomplete type N::B, which is also included here as
a declaration, since the complete type is unknown and its
signature is therefore unavailable. There is also a reference
to N::C, using 
\livelink{chap:DWFORMrefsig8}{DW\-\_FORM\-\_ref\-\_sig8} to 
refer to the type signature
\addtoindexx{type signature}
for that type.

In computing a signature for the type N::A, flatten the type
description into a byte stream according to the procedure
outlined in 
Section \refersec{datarep:typesignaturecomputation}.
The result is shown in 
Figure \refersec{fig:typesignatureexample2flattenedbytestream}.

\begin{figure}
\figurepart{1}{3}
\begin{dwflisting}
% DWARF4 had a \livelink{chap:DWATnamespace}{DW_AT_namespace} below, 
% but this error is fixed here to be \livelink{chap:DWTAGnamespace}{DW_TAG_namespace}.
\begin{alltt}
// Step 2: 'C' \livelink{chap:DWTAGnamespace}{DW\-\_TAG\-\_namespace} "N"
0x43 0x39 0x4e 0x00
// Step 3: 'D' \livelink{chap:DWTAGclasstype}{DW\-\_TAG\-\_class\-\_type}
0x44 0x02
// Step 4: 'A' \livelink{chap:DWATname}{DW\-\_AT\-\_name} \livelink{chap:DWFORMstring}{DW\-\_FORM\-\_string} "A"
0x41 0x03 0x08 0x41 0x00
// Step 4: 'A' \livelink{chap:DWATbytesize}{DW\-\_AT\-\_byte\-\_size} \livelink{chap:DWFORMsdata}{DW\-\_FORM\-\_sdata} 20
0x41 0x0b 0x0d 0x14
// Step 7: First child ("v\_")
    // Step 3: 'D' \livelink{chap:DWTAGmember}{DW\-\_TAG\-\_member}
    0x44 0x0d
    // Step 4: 'A' \livelink{chap:DWATname}{DW\-\_AT\-\_name} \livelink{chap:DWFORMstring}{DW\-\_FORM\-\_string} "v\_"
    0x41 0x03 0x08 0x76 0x5f 0x00
    // Step 4: 'A' \livelink{chap:DWATaccessibility}{DW\-\_AT\-\_accessibility} \livelink{chap:DWFORMsdata}{DW\-\_FORM\-\_sdata} \livelink{chap:DWACCESSprivate}{DW\-\_ACCESS\-\_private}
    0x41 0x32 0x0d 0x03
    // Step 4: 'A' \livelink{chap:DWATdatamemberlocation}{DW\-\_AT\-\_data\-\_member\-\_location} \livelink{chap:DWFORMsdata}{DW\-\_FORM\-\_sdata} 0
    0x41 0x38 0x0d 0x00
    // Step 6: 'T' \livelink{chap:DWATtype}{DW\-\_AT\-\_type} (type \#2)
    0x54 0x49
        // Step 3: 'D' \livelink{chap:DWTAGbasetype}{DW\-\_TAG\-\_base\-\_type}
        0x44 0x24
        // Step 4: 'A' \livelink{chap:DWATname}{DW\-\_AT\-\_name} \livelink{chap:DWFORMstring}{DW\-\_FORM\-\_string} "int"
        0x41 0x03 0x08 0x69 0x6e 0x74 0x00
        // Step 4: 'A' \livelink{chap:DWATbytesize}{DW\-\_AT\-\_byte\-\_size} \livelink{chap:DWFORMsdata}{DW\-\_FORM\-\_sdata} 4
        0x41 0x0b 0x0d 0x04
        // Step 4: 'A' \livelink{chap:DWATencoding}{DW\-\_AT\-\_encoding} \livelink{chap:DWFORMsdata}{DW\-\_FORM\-\_sdata} \livelink{chap:DWATEsigned}{DW\-\_ATE\-\_signed}
        0x41 0x3e 0x0d 0x05
        // Step 7: End of \livelink{chap:DWTAGbasetype}{DW\-\_TAG\-\_base\-\_type} "int"
        0x00
    // Step 7: End of \livelink{chap:DWTAGmember}{DW\-\_TAG\-\_member} "v\_"
    0x00
// Step 7: Second child ("next")
    // Step 3: 'D' \livelink{chap:DWTAGmember}{DW\-\_TAG\-\_member}
    0x44 0x0d
    // Step 4: 'A' \livelink{chap:DWATname}{DW\-\_AT\-\_name} \livelink{chap:DWFORMstring}{DW\-\_FORM\-\_string} "next"
    0x41 0x03 0x08 0x6e 0x65 0x78 0x74 0x00
    // Step 4: 'A' \livelink{chap:DWATaccessibility}{DW\-\_AT\-\_accessibility} \livelink{chap:DWFORMsdata}{DW\-\_FORM\-\_sdata} \livelink{chap:DWACCESSprivate}{DW\-\_ACCESS\-\_private}
    0x41 0x32 0x0d 0x03
    // Step 4: 'A' \livelink{chap:DWATdatamemberlocation}{DW\-\_AT\-\_data\-\_member\-\_location} \livelink{chap:DWFORMsdata}{DW\-\_FORM\-\_sdata} 4
    0x41 0x38 0x0d 0x04
\end{alltt}
\end{dwflisting}
\caption{Type signature example \#2: flattened byte stream}
\label{fig:typesignatureexample2flattenedbytestream}
\end{figure}

\begin{figure}
\figurepart{2}{3}
\begin{dwflisting}
\begin{alltt}    
    // Step 6: 'T' \livelink{chap:DWATtype}{DW\-\_AT\-\_type} (type \#3)
    0x54 0x49
        // Step 3: 'D' \livelink{chap:DWTAGpointertype}{DW\-\_TAG\-\_pointer\-\_type}
        0x44 0x0f
        // Step 5: 'N' \livelink{chap:DWATtype}{DW\-\_AT\-\_type}
        0x4e 0x49
        // Step 5: 'C' \livelink{chap:DWTAGnamespace}{DW\-\_TAG\-\_namespace} "N" 'E'
        0x43 0x39 0x4e 0x00 0x45
        // Step 5: "A"
        0x41 0x00
        // Step 7: End of \livelink{chap:DWTAGpointertype}{DW\-\_TAG\-\_pointer\-\_type}
        0x00
    // Step 7: End of \livelink{chap:DWTAGmember}{DW\-\_TAG\-\_member} "next"
    0x00
// Step 7: Third child ("bp")
    // Step 3: 'D' \livelink{chap:DWTAGmember}{DW\-\_TAG\-\_member}
    0x44 0x0d
    // Step 4: 'A' \livelink{chap:DWATname}{DW\-\_AT\-\_name} \livelink{chap:DWFORMstring}{DW\-\_FORM\-\_string} "bp"
    0x41 0x03 0x08 0x62 0x70 0x00
    // Step 4: 'A' \livelink{chap:DWATaccessibility}{DW\-\_AT\-\_accessibility} \livelink{chap:DWFORMsdata}{DW\-\_FORM\-\_sdata} \livelink{chap:DWACCESSprivate}{DW\-\_ACCESS\-\_private}
    0x41 0x32 0x0d 0x03
    // Step 4: 'A' \livelink{chap:DWATdatamemberlocation}{DW\-\_AT\-\_data\-\_member\-\_location} \livelink{chap:DWFORMsdata}{DW\-\_FORM\-\_sdata} 8
    0x41 0x38 0x0d 0x08
    // Step 6: 'T' \livelink{chap:DWATtype}{DW\-\_AT\-\_type} (type \#4)
    0x54 0x49
        // Step 3: 'D' \livelink{chap:DWTAGpointertype}{DW\-\_TAG\-\_pointer\-\_type}
0x44 0x0f
        // Step 5: 'N' \livelink{chap:DWATtype}{DW\-\_AT\-\_type}
        0x4e 0x49
        // Step 5: 'C' \livelink{chap:DWTAGnamespace}{DW\-\_TAG\-\_namespace} "N" 'E'
        0x43 0x39 0x4e 0x00 0x45
        // Step 5: "B"
        0x42 0x00
        // Step 7: End of \livelink{chap:DWTAGpointertype}{DW\-\_TAG\-\_pointer\-\_type}
        0x00
    // Step 7: End of \livelink{chap:DWTAGmember}{DW\-\_TAG\-\_member} "next"
    0x00
// Step 7: Fourth child ("c")
    // Step 3: 'D' \livelink{chap:DWTAGmember}{DW\-\_TAG\-\_member}
    0x44 0x0d
    // Step 4: 'A' \livelink{chap:DWATname}{DW\-\_AT\-\_name} \livelink{chap:DWFORMstring}{DW\-\_FORM\-\_string} "c"
    0x41 0x03 0x08 0x63 0x00
    // Step 4: 'A' \livelink{chap:DWATaccessibility}{DW\-\_AT\-\_accessibility} \livelink{chap:DWFORMsdata}{DW\-\_FORM\-\_sdata} \livelink{chap:DWACCESSprivate}{DW\-\_ACCESS\-\_private}
    0x41 0x32 0x0d 0x03
\end{alltt}
\end{dwflisting}
\begin{center}
Figure~\ref{fig:typesignatureexample2flattenedbytestream} Type signature example \#2: flattened byte stream \textit{(continued)}
\end{center}
\end{figure}
    
\begin{figure}
\figurepart{3}{3}
\begin{dwflisting}
\begin{alltt}
    // Step 4: 'A' \livelink{chap:DWATdatamemberlocation}{DW\-\_AT\-\_data\-\_member\-\_location} \livelink{chap:DWFORMsdata}{DW\-\_FORM\-\_sdata} 12
    0x41 0x38 0x0d 0x0c
    // Step 6: 'T' \livelink{chap:DWATtype}{DW\-\_AT\-\_type} (type \#5)
    0x54 0x49
        // Step 2: 'C' \livelink{chap:DWTAGnamespace}{DW\-\_TAG\-\_namespace} "N"
        0x43 0x39 0x4e 0x00
        // Step 3: 'D' \livelink{chap:DWTAGstructuretype}{DW\-\_TAG\-\_structure\-\_type}
        0x44 0x13
        // Step 4: 'A' \livelink{chap:DWATname}{DW\-\_AT\-\_name} \livelink{chap:DWFORMstring}{DW\-\_FORM\-\_string} "C"
        0x41 0x03 0x08 0x43 0x00
        // Step 4: 'A' \livelink{chap:DWATbytesize}{DW\-\_AT\-\_byte\-\_size} \livelink{chap:DWFORMsdata}{DW\-\_FORM\-\_sdata} 8
        0x41 0x0b 0x0d 0x08
        // Step 7: First child ("x")
            // Step 3: 'D' \livelink{chap:DWTAGmember}{DW\-\_TAG\-\_member}
            0x44 0x0d
            // Step 4: 'A' \livelink{chap:DWATname}{DW\-\_AT\-\_name} \livelink{chap:DWFORMstring}{DW\-\_FORM\-\_string} "x"
            0x41 0x03 0x08 0x78 0x00
            // Step 4: 'A' \livelink{chap:DWATdatamemberlocation}{DW\-\_AT\-\_data\-\_member\-\_location} \livelink{chap:DWFORMsdata}{DW\-\_FORM\-\_sdata} 0
            0x41 0x38 0x0d 0x00
            // Step 6: 'R' \livelink{chap:DWATtype}{DW\-\_AT\-\_type} (type \#2)
            0x52 0x49 0x02
            // Step 7: End of \livelink{chap:DWTAGmember}{DW\-\_TAG\-\_member} "x"
            0x00
        // Step 7: Second child ("y")
            // Step 3: 'D' \livelink{chap:DWTAGmember}{DW\-\_TAG\-\_member}
            0x44 0x0d
            // Step 4: 'A' \livelink{chap:DWATname}{DW\-\_AT\-\_name} \livelink{chap:DWFORMstring}{DW\-\_FORM\-\_string} "y"
            0x41 0x03 0x08 0x79 0x00
            // Step 4: 'A' \livelink{chap:DWATdatamemberlocation}{DW\-\_AT\-\_data\-\_member\-\_location} \livelink{chap:DWFORMsdata}{DW\-\_FORM\-\_sdata} 4
            0x41 0x38 0x0d 0x04
            // Step 6: 'R' \livelink{chap:DWATtype}{DW\-\_AT\-\_type} (type \#2)
            0x52 0x49 0x02
            // Step 7: End of \livelink{chap:DWTAGmember}{DW\-\_TAG\-\_member} "y"
            0x00
        // Step 7: End of \livelink{chap:DWTAGstructuretype}{DW\-\_TAG\-\_structure\-\_type} "C"
        0x00
    // Step 7: End of \livelink{chap:DWTAGmember}{DW\-\_TAG\-\_member} "c"
    0x00
// Step 7: Fifth child ("A")
    // Step 3: 'S' \livelink{chap:DWTAGsubprogram}{DW\-\_TAG\-\_subprogram} "A"
    0x53 0x2e 0x41 0x00
// Step 7: Sixth child ("v")
    // Step 3: 'S' \livelink{chap:DWTAGsubprogram}{DW\-\_TAG\-\_subprogram} "v"
    0x53 0x2e 0x76 0x00
// Step 7: End of \livelink{chap:DWTAGstructuretype}{DW\-\_TAG\-\_structure\-\_type} "A"
0x00
\end{alltt}
\end{dwflisting}
\begin{center}
Figure~\ref{fig:typesignatureexample2flattenedbytestream} Type signature example \#2: flattened byte stream \textit{(concluded)}
\end{center}
\end{figure}

Running an \addtoindex{MD5 hash} over this byte stream, and taking the
low-order 64 bits, yields the final signature: 0xd6d160f5
5589f6e9.


A source file that includes this header file may declare a
variable of type N::A, and its DWARF information may look
like that shown in 
Figure \refersec{fig:typesignatureexampleusage}.

\begin{figure}
\begin{dwflisting}
\begin{alltt}
  \livelink{chap:DWTAGcompileunit}{DW\-\_TAG\-\_compile\-\_unit}
  ...
  \livelink{chap:DWTAGsubprogram}{DW\-\_TAG\-\_subprogram}
    ...
    \livelink{chap:DWTAGvariable}{DW\-\_TAG\-\_variable}
      \livelink{chap:DWATname}{DW\-\_AT\-\_name} : "a"
      \livelink{chap:DWATtype}{DW\-\_AT\-\_type} : (signature) 0xd6d160f5 5589f6e9
      \livelink{chap:DWATlocation}{DW\-\_AT\-\_location} : ...
    ...
\end{alltt}
\end{dwflisting}
\caption{Type signature example usage}
\label{fig:typesignatureexampleusage}
\end{figure}

\subsection{Type Signature Computation Grammar}
\label{app:typesignaturecomputationgrammar}

Figure \refersec{fig:typesignaturecomputationgrammar}
\addtoindexx{type signature!computation grammar}
presents a semi-formal grammar that may aid in understanding
how the bytes of the flattened type description are formed
during the type signature computation algorithm of
Section \refersec{datarep:typesignaturecomputation}. 

\begin{figure}[h]
\begin{dwflisting}
%FIXME: The index entries here with \addtoindexx are ineffective.
\begin{alltt}
signature
    : opt-context debug-entry attributes children
opt-context               // Step 2
    : 'C' tag-code string opt-context
    : empty
debug-entry               // Step 3
    : 'D' tag-code
attributes                // Steps 4, 5, 6
    : attribute attributes
    : empty
attribute
    : 'A' at-code form-encoded-value     // Normal attributes
    : 'N' at-code opt-context 'E' string // Reference to type by name
    : 'R' at-code back-ref               // Back-reference to visited type
    : 'T' at-code signature              // Recursive type
children                 //  Step 7
    : child children
    : '0'
child
    : 'S' tag-code string
    : signature
tag-code
    : <ULEB128>
at-code
    : <ULEB128>
form-encoded-value
    : \livelink{chap:DWFORMsdata}{DW\-\_FORM\-\_sdata} value \addtoindexx{constant class}
    : \livelink{chap:DWFORMflag}{DW\-\_FORM\-\_flag} value \addtoindexx{flag class}
    : \livelink{chap:DWFORMstring}{DW\-\_FORM\-\_string} string \addtoindexx{string class}
    : \livelink{chap:DWFORMblock}{DW\-\_FORM\-\_block} \nolink{block} \addtoindexx{block class}
\livelink{chap:DWFORMstring}{DW\-\_FORM\-\_string} \addtoindexx{string class}
    : 'x08'
\livelink{chap:DWFORMblock}{DW\-\_FORM\-\_block}  \addtoindexx{block class}
    : 'x09'
\livelink{chap:DWFORMflag}{DW\-\_FORM\-\_flag} \addtoindexx{flag class}
    : 'x0c'
\livelink{chap:DWFORMsdata}{DW\-\_FORM\-\_sdata} \addtoindexx{constant class}
    : 'x0d'
value
    : <SLEB128>
\nolink{block}
    : <ULEB128> <fixed-length-block> // The ULEB128 gives the length of the \nolink{block}
back-ref
    : <ULEB128>
string
    : <null-terminated-string>
empty
    :
\end{alltt}
\end{dwflisting}
\caption{Type signature computation grammar}
\label{fig:typesignaturecomputationgrammar}
\end{figure}

\section{Summary of Compression Techniques}
\label{app:summaryofcompressiontechniques}
\subsection{\#include compression}
\label{app:includecompression}

\addtoindex{C++} has a much greater 
problem than 
\addtoindex{C} with the number and
size of the headers included and the amount of data in each,
but even with \addtoindex{C} 
there is substantial header file information
duplication.

A reasonable approach is to put each header file in its own
\addtoindex{section group}, using the naming rules mentioned above. The
section groups are marked to ensure duplicate removal.

All data instances and code instances (even if they came
from the header files above) are put 
\addtoindexx{section group}
into non-section group
sections such as the base object file 
\addtoindex{.debug\_info} section.

\subsection{Eliminating function duplication}
\label{app:eliminatingfunctionduplication}


Function templates (C++) result in code for the same template
instantiation being compiled into multiple archives or
relocatable objects. The linker wants to keep only one of a
given entity. The DWARF description, and everything else for
this function, should be reduced to just a single copy.

For each such code group (function template in this example)
the compiler assigns a name for the group which will match
all other instantiations of this function but match nothing
else. 
The 
\addtoindexx{section group}
section groups are marked to ensure duplicate
removal, so that the second and subsequent definitions seen
by the static linker are simply discarded.


References to other 
\addtoindex{.debug\_info} sections follow the approach
suggested above, but the naming rule is slightly
different in that the \texttt{\textless file-designator\textgreater} 
should be interpreted as a \texttt{\textless file-designator\textgreater}.



\subsection{Single-function-per-DWARF-compilation-unit}
\label{app:singlefunctionperdwarfcompilationunit}

Section groups can help make it easy for a linker to completely
remove unused functions.

Such 
\addtoindexx{section group}
section groups are not marked for duplicate removal,
since the functions are not duplicates of anything.

Each function is given a compilation unit and a section
group. Each such compilation unit is complete, with its own
text, data, and DWARF sections.

There will also be a compilation unit that has the file\dash level
declarations and definitions. Other per\dash function compilation
unit DWARF information (
\addtoindex{.debug\_info}) points to this common
file\dash level compilation unit using \livelink{chap:DWTAGimportedunit}{DW\-\_TAG\-\_imported\-\_unit}.

Section groups can use \livelink{chap:DWFORMrefaddr}{DW\-\_FORM\-\_ref\-\_addr} and internal labels
(section\dash relative relocations) to refer to the main object
file sections, as the 
\addtoindexx{section group}
section groups here are either deleted
as unused or kept. There is no possibility (aside from error)
of a group from some other compilation being used in place
of one of these groups.


\subsection{Inlining and out-of-line-instances}
\label{app:inliningandoutoflineinstances}

Abstract instances
\addtoindexx{abstract instance}
\addtoindexx{concrete out-of-line instance}
and concrete-out-of-line instances may be
put in distinct compilation units using 
\addtoindexx{section group}
section groups. 
This
makes possible some useful duplicate DWARF elimination.

\textit{No special provision for eliminating class duplication
resulting from template instantiation is made here, though
nothing prevents eliminating such duplicates using section
groups.}


\subsection{Separate Type Units}
\label{app:separatetypeunits}

Each complete declaration of a globally-visible type can be
\addtoindexx{type unit}
placed in its own separate type section, with a group key
derived from the type signature. The linker can then remove
all duplicate type declarations based on the key.

