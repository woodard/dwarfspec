\chapter{Debug Section Relationships (Informative)}
\label{app:debugsectionrelationshipsinformative}
DWARF information is organized into multiple program sections, 
each of which holds a particular kind of information. In some 
cases, information in one section refers to information in one 
or more of the others. These relationships are illustrated by 
the diagrams and associated notes on the following pages.

In the figures, a section is shown as a shaded oval with the
name of the section inside. References from one section to
another are shown by an arrow. In the first figure, the arrow
is annotated with an unshaded box which contains an indication
of the construct (such as an attribute or form) that encodes
the reference. In the second figure, this box is left out
for reasons of space in favor of a label annotation that is
explained in the subsequent notes.

\section{Normal DWARF Section Relationships}
Figure \referfol{fig:debugsectionrelationships} illustrates
the DWARF section relations without split DWARF object files
involved. Similarly, it does not show the 
relationships between the main debugging sections of an executable
or sharable file and a related \addtoindex{supplementary object file}.

\section{Split DWARF Section Relationships}
Figure \ref{fig:splitdwarfsectionrelationships} illustrates
the DWARF section relationships for \splitDWARFobjectfile{s}.
However, it does not show the 
relationships between the main debugging sections of an executable
or shareable file and a related \addtoindex{supplementary object file}.

\begin{landscape}
\begin{figure}
\scriptsize
\begin{tikzpicture}
    [sect/.style={rectangle, rounded corners=10pt, draw, fill=blue!15, 
         inner sep=.2cm, minimum width=4.0cm},
     link/.style={rectangle,                       draw,
         inner sep=.2cm, minimum width=4.5cm},
     circ/.style={circle,                          draw, fill=yellow!25,
         minimum size=0.5cm}]
     
% The first (leftmost) column, first sections, then links, from top to bottom
%
\node(zsectara) at ( 0, 15.0) [sect] {\dotdebugaranges};
\node(zlinka)   at ( 0, 13.5) [link] {To compilation unit~~(a)};
\node(zsectinf) at ( 0,  7.5) [sect] {\begin{tabular}{c}
                                        ~\\
				      \dotdebuginfo\\
                                        ~
				      \end{tabular}};
\node(zcircs)   at (-1.8,4.0) [circ] {\textit{A}};
\node(zlinkb)   at ( 0,  1.5) [link] {To compilation unit~~(b)};
\node(zsectpub) at ( 0,-0.25) [sect] {\begin{tabular}{c} 
				      \dotdebugnames  
				      \end{tabular}};
                                      
\draw[thin,triangle 45-]            (zcircs) -- (zsectinf);
\draw[thin,-to reversed]            (zlinka) -- (zsectara);
\draw[thin,triangle 45-]            (zsectinf) -- (zlinka);
\draw[thin,-triangle 45]            (zlinkb) -- (zsectinf);
\draw[thin,to reversed-]	    (zsectpub) -- (zlinkb);

% The second column, similarly
%
\node(zsectfra) at (5, 15.0)  [sect] {\dotdebugframe};
\node(zlinkc)   at (5, 13.5)  [link] {To abbreviations~~(c)};
\node(zlinkd)   at (5, 12.1)  [link] {\DWFORMstrp{}~~(d)};
\node(zlinke)	at (5, 10.4)  [link] {\begin{tabular}{c}
				      \DWATstroffsetsbase \\
				      \DWFORMstrx{}~~~~~~~~~(e) \\
				      \end{tabular}};
\node(zlinkf)   at (5,  8.4)  [link] {\begin{tabular}{c}
				      \DWOPcallref{}~~~~~(f) \\
				      \DWFORMrefaddr
				      \end{tabular}};
\node(zlinki)   at (5,  6.7)  [link] {\DWATmacros{}~~(g)};
\node(zlinkj)   at (5,  5.4)  [link] {\DWATstmtlist{}~~(h)};
\node(zlinkh)   at (5,  3.9)  [link] {\begin{tabular}{c}
                                      \DWATranges{}~~~~(i) \\
                                      \DWATrnglistsbase
                                      \end{tabular}};
\node(zlinkg)   at (5,  2.4)  [link] {\DWATlocation{}, etc.~~(j)};
\node(zlinkk)	at (5,  0.5)  [link] {\begin{tabular}(c)
				      \DWATaddrbase    \\
				      \DWFORMaddrx \\
				      \DWOPaddrx \\
				      \DWOPconstx
				      \end{tabular} (k)};

% Links between first and second columns
%
\draw[thin,to reversed-]	     (zsectinf) -- (zlinkc.west);
\draw[thin,to reversed-]	     (zsectinf) -- (zlinkd.west);
\draw[thin,to reversed-]	     (zsectinf) -- (zlinke.west);
\draw[thin, triangle 45-triangle 45] (zsectinf) -- (zlinkf.west);
\draw[thin,to reversed-]	     (zsectinf) -- (zlinkg.west);
\draw[thin,to reversed-]	     (zsectinf) -- (zlinkh.west);
\draw[thin,to reversed-]	     (zsectinf) -- (zlinki.west);
\draw[thin,to reversed-]	     (zsectinf) -- (zlinkj.west);
\draw[thin,to reversed-]	     (zsectinf) -- (zlinkk.north west);

% The third column
%
\node(zsectabb)	at (10, 15.00) [sect] {\dotdebugabbrev};
\node(zsectstr)	at (10, 13.75) [sect] {\dotdebugstr};
\node(zlinkl)   at (10, 12.50) [link] {To strings~~(l)};
\node(zsectstx) at (10, 11.25) [sect] {\dotdebugstroffsets};
\node(zlinkm)   at (10,  9.50) [link] {\begin{tabular}{c}
                                      \DWMACROdefinestrx \\
                                      \DWMACROundefstrx \\
                                      (m)
                                      \end{tabular}};
\node(zsectmac)	at (10,  7.80) [sect] {\dotdebugmacro};
\node(zlinkn)   at (10,  6.40) [link] {\begin{tabular}{c}
                                      macroinfo header~~~~~~(n)\\
                                      \DWMACROstartfile
                                      \end{tabular}};
\node(zsectlin)	at (10,  5.00) [sect] {\dotdebugline};
\node(zsectran)	at (10,  3.85) [sect] {\dotdebugrnglists};
\node(zsectloc)	at (10,  2.70) [sect] {\dotdebugloclists{}};
\node(zlinko)   at (10,  1.20) [link] {\begin{tabular}{c}
                                       \DWOPaddrx \\
                                       \DWOPconstx
                                       \end{tabular} (o)};
\node(zsectadx) at (10,  -0.25) [sect] {\dotdebugaddr{}};

\draw[thin,to reversed-]       (zsectstx) -- (zlinkl);
\draw[thin,-triangle 45]       (zlinkl) -- (zsectstr);
\draw[thin,to reversed-]       (zsectmac) -- (zlinkm);
\draw[thin,-triangle 45]       (zlinkm) -- (zsectstx);
\draw[thin,to reversed-]       (zsectmac) -- (zlinkn);
\draw[thin,-triangle 45]       (zlinkn) -- (zsectlin);
\draw[thin,to reversed-]       (zsectloc) -- (zlinko);
\draw[thin,-triangle 45]       (zlinko) -- (zsectadx);

% Links between second and third colums
%
\draw[thin,-triangle 45]	(zlinkc.east) -- (zsectabb.west);
\draw[thin,-triangle 45]	(zlinkd.east) -- (zsectstr.west);
\draw[thin,-triangle 45]	(zlinke.east) -- (zsectstx.west);
\draw[thin,-triangle 45]	(zlinkg.east) -- (zsectloc.west);
\draw[thin,-triangle 45]	(zlinkh.east) -- (zsectran.west);
\draw[thin,-triangle 45]	(zlinki.east) -- (zsectmac.west);
\draw[thin,-triangle 45]	(zlinkj.east) -- (zsectlin.west);
\draw[thin,-triangle 45]	(zlinkk.east) -- (zsectadx.west);

% The fourth column
%
\node(zlinky)   at (15.6, 10.5) [link] {\begin{tabular}{c}
                                        \DWMACROdefinestrp \\
                                        \DWMACROundefstrp \\
                                        (p)
                                        \end{tabular}};
\node(zlinkz)   at (15.6,  6.4) [link] {\begin{tabular}{c}
                                        \DWMACROimport \\
                                        (q)
                                        \end{tabular}};
\node(zcircsp)  at (17.4,  4.0) [circ]  {\textit{A}};
\node(zlinkx)   at (15.6,  2.0) [link]  {\DWFORMlinestrp~(r, s)};
\node(zsectlns) at (15.6,-0.25) [sect]  {\dotdebuglinestr};
             
             
\draw[thin,to reversed-]       (zsectmac.east) -- (zlinky);
\draw[thin,-triangle 45]       (zlinky) -- (zsectstr.east);
\draw[triangle 45-triangle 45] (zsectmac.east) -- (zlinkz);
\draw[thin,to reversed-]       (zcircsp) -- (zlinkx);
\draw[thin,to reversed-]       (zsectlin.east) -- (zlinkx);
\draw[thin,-triangle 45]       (zlinkx) -- (zsectlns);

\end{tikzpicture}
\vspace{5mm}
\caption{Debug section relationships}
\label{fig:debugsectionrelationships}
\end{figure}
\end{landscape}

\clearpage
\begin{center}
   \textbf{Notes for Figure \ref{fig:debugsectionrelationships}}
\end{center}
\begin{description} 
\itembfnl{(a) \dotdebugaranges{} to \dotdebuginfo}
The \texttt{debug\_info\_offset} value in
the header is
the offset in the \dotdebuginfo{} section of the
corresponding compilation unit header (not the compilation
unit entry).

%b
\itembfnl{(b) \dotdebugnames{} to \dotdebuginfo}
The list of compilation units following the header contains the offsets in the
\dotdebuginfo{} section of the 
corresponding compilation unit headers (not
the compilation unit entries). 

%c
\itembfnl{(c) \dotdebuginfo{} to \dotdebugabbrev}
The \HFNdebugabbrevoffset{} value in the header is the offset in the
\dotdebugabbrev{} 
section of the abbreviations for that compilation unit.

%d
\itembfnl{(d) \dotdebuginfo{} to \dotdebugstr}
Attribute values of class string may have form 
\DWFORMstrp, whose
value is the offset in the \dotdebugstr{}
section of the corresponding string.

%e
\itembfnl{(e) \dotdebuginfo{} to \dotdebugstroffsets}
The value of the \DWATstroffsetsbase{} attribute in a
\DWTAGcompileunit{}, \DWTAGtypeunit{} or \DWTAGpartialunit{} 
DIE is the offset in the
\dotdebugstroffsets{} section of the 
\addtoindex{string offsets table}
for that unit.
In addition, attribute values of class string may have form 
\DWFORMstrx, whose value is an index into the
string offsets table.

%f
\itembfnl{(f) \dotdebuginfo{} to \dotdebuginfo}
The operand of the \DWOPcallref{} 
DWARF expression operator is the
offset of a debugging information entry in the 
\dotdebuginfo{} section of another compilation.
Similarly for attribute operands that use
\DWFORMrefaddr.

%g
\itembfnl{(g) \dotdebuginfo{} to \dotdebugmacro}
An attribute value of class 
\livelink{chap:classmacptr}{macptr} (specifically form
\DWFORMsecoffset) is an 
offset within the 
\dotdebugmacro{} section
of the beginning of the macro information for the referencing unit.

%h
\itembfnl{(h) \dotdebuginfo{} to \dotdebugline}
An attribute value of class 
\livelink{chap:classlineptr}{lineptr} (specifically form
\DWFORMsecoffset) 
is an offset in the 
\dotdebugline{} section of the
beginning of the line number information for the referencing unit.

%i
\needlines{5}
\itembfnl{(i) \dotdebuginfo{} to \dotdebugrnglists}
An attribute value of class \CLASSrnglist{} 
(specifically form \DWFORMrnglistx{} or \DWFORMsecoffset) 
is an index or offset within the \dotdebugrnglists{} 
section of a \addtoindex{range list}.

%j
\itembfnl{(j) \dotdebuginfo{} to \dotdebugloclists}
An attribute value of class \CLASSloclist{} 
(specifically form \DWFORMloclistx or \DWFORMsecoffset) 
is an index or offset within the \dotdebugloclists{}
section of a \addtoindex{location list}.

%k
\itembfnl{(k) \dotdebuginfo{} to \dotdebugaddr}
The value of the \DWATaddrbase{} attribute in the
\DWTAGcompileunit{} or \DWTAGpartialunit{} DIE is the
offset in the \dotdebugaddr{} section of the machine
addresses for that unit.
\DWFORMaddrx, \DWOPaddrx{} and \DWOPconstx{} contain
indices relative to that offset.

%l
\itembfnl{(l) \dotdebugstroffsets{} to \dotdebugstr}
Entries in the string offsets table
are offsets to the corresponding string text in the 
\dotdebugstr{} section.

%m
\itembfnl{(m) \dotdebugmacro{} to \dotdebugstroffsets}
The second operand of a 
\DWMACROdefinestrx{} or \DWMACROundefstrx{} 
macro information entry is an index
into the string offset table in the 
\dotdebugstroffsets{} section.

%n
\itembfnl{(n) \dotdebugmacro{} to \dotdebugline}
The second operand of 
\DWMACROstartfile{} refers to a file entry in the 
\dotdebugline{} section relative to the start 
of that section given in the macro information header.

%o
\itembfnl{(o) \dotdebugloclists{} to \dotdebugaddr}
\DWOPaddrx{} and \DWOPconstx{} operators that occur in the 
\dotdebugloclists{} section refer indirectly to the 
\dotdebugaddr{} section by way of the 
\DWATaddrbase{} attribute in the associated \dotdebuginfo{} 
section.

%p
\itembfnl{(p) \dotdebugmacro{} to \dotdebugstr}
The second operand of a 
\DWMACROdefinestrp{} or \DWMACROundefstrp{} macro information
entry is an index into the string table in the 
\dotdebugstr{} section.

%q
\needlines{4}
\itembfnl{(q) \dotdebugmacro{} to \dotdebugmacro}
The operand of a 
\DWMACROimport{} macro information
entry is an offset into another part of the 
\dotdebugmacro{} section to the header for the 
sequence to be replicated.

%r
\needlines{4}
\itembfnl{(r) \dotdebugline{} to \dotdebuglinestr}
The value of a \DWFORMlinestrp{} form refers to a
string section specific to the line number table.
This form can be used in a \dotdebugline{} section
(as well as in a \dotdebuginfo{} section).

%s
\itembfnl{(s) \dotdebuginfo{} to \dotdebuglinestr}
The value of a \DWFORMlinestrp{} form refers to a
string section specific to the line number table.
This form can be used in a \dotdebuginfo{} section
(as well as in a \dotdebugline{} section).\footnote{ 
The circled (s) connects to the circled
(s)' via hyperspace (a wormhole).}
 
\end{description}



\begin{landscape}
\begin{figure}
%\scriptsize
\begin{tikzpicture}
    [sect/.style={rectangle, rounded corners=10pt, draw, fill=blue!15, 
        inner sep=.2cm, minimum width=4.0cm},
     link/.style={rectangle,                       draw,
        inner sep=.2cm, minimum width=4.5cm}]

\fill[yellow!25] (7.5,-1) -- (7.5,14.5) -- (19,14.5) -- (19,-1) -- cycle;

% First column
%
\node(ysectabb)    at ( 5, 13.5) [sect] {\dotdebugabbrev};
\node(ysectadd)    at ( 2, 12.0) [sect] {\dotdebugaddr};
\node(ysectara)    at ( 0, 10.5) [sect] {\dotdebugaranges};
\node(ysectfra)    at ( 0,  9.0) [sect] {\dotdebugframe};
\node(ysectlin)    at ( 0,  7.5) [sect] {\dotdebugline};
\node(ysectlis)    at ( 0,  5.7) [sect] {\dotdebuglinestr};
\node(ysectnam)    at ( 0,  4.0) [sect] {\dotdebugnames};
\node(ysectstr)    at ( 2,  2.0) [sect] {\dotdebugstr};
\node(ysectsto)    at ( 5,  0.0) [sect] {\dotdebugstroffsets};

\node(ysectinf)    at ( 5,  7)   [sect] {\begin{tabular}{c}
                                         ~\\
                                         \dotdebuginfo \\
                                         (skeleton CU)\\
                                         ~
                                         \end{tabular}};

\node(ysectinfdwo) at (10.5,7)   [sect] {\begin{tabular}{c}
                                         \dotdebuginfodwo \\
                                         (one CU, possibly \\
                                         multiple COMDAT \\
                                         type units)
                                         \end{tabular}};

\node(ysectabbdwo) at (10.5, 13.5) [sect] {\dotdebugabbrevdwo};
\node(ysectlocdwo) at (14.0, 11.5) [sect] {\dotdebugloclistsdwo};
\node(ysectrandwo) at (15.8,  9.5) [sect] {\dotdebugrnglistsdwo};
\node(ysectlindwo) at (16.0,  7.0) [sect] {\dotdebuglinedwo};
\node(ysectmacdwo) at (16.0,  4.5) [sect] {\dotdebugmacrodwo};
\node(ysectstrdwo) at (13.5,  2.0) [sect] {\dotdebugstrdwo};
\node(ysectstodwo) at (10.5,  0.0) [sect] {\dotdebugstroffsetsdwo};

\draw[thin,-triangle 45]	(ysectinf) -- (ysectabb) node[midway, right] {(c)};
\draw[thin,-triangle 45]	(ysectinf) -- (ysectadd) node[midway, right] {(k)};
\draw[thin,-triangle 45]	(ysectara.east) -- (ysectinf) node[midway, left] {(a)};
\draw[thin,-triangle 45]	(ysectinf) -- (ysectlin.east) node[midway, above] {(h)};
\draw[thin,-triangle 45]	(ysectlin) -- (ysectlis) node[midway, right] {(l)};
\draw[thin,-triangle 45]	(ysectnam.east) -- (ysectinf) node[midway, left] {(b)};
\draw[thin,-triangle 45]	(ysectinf) -- (ysectstr) node[midway, right] {(d)};
\draw[thin,-triangle 45]	(ysectinf) -- (ysectsto) node[midway, right] {(e)};
\draw[thin,-triangle 45]	(ysectsto) -- (ysectstr) node[midway, right] {(l)};

\draw[dashed, thick,-triangle 45]  (ysectinf) .. controls (7.5, 11) ..(ysectinfdwo) 
                                                       node[midway, above] {(did)};

\draw[thin,-triangle 45]  (ysectinfdwo) -- (ysectabbdwo) node[midway, left] {(co)};
\draw[thin,-triangle 45]  (ysectinfdwo) -- (ysectlindwo.west) node[midway, above] {(ho)};
\draw[thin,-triangle 45]  (ysectinfdwo) -- (ysectrandwo.west) node[midway, above] {(io)};
\draw[thin,-triangle 45]  (ysectinfdwo) -- (ysectlocdwo.west) node[midway, right] {(jo)};
\draw[thin,-triangle 45]  (ysectinfdwo) -- (ysectmacdwo.west) node[midway, above] {(go)};
\draw[thin,-triangle 45]  (ysectinfdwo) -- (ysectstrdwo) node[midway, right] {(do)};
\draw[thin,-triangle 45]  (ysectinfdwo) -- (ysectstodwo) node[midway, left] {(eo)};

\draw[thin,-triangle 45]  (ysectstodwo) -- (ysectstrdwo) node[near end, below] {(lo)};
\draw[thin,-triangle 45]  (ysectmacdwo) -- (ysectstrdwo) node[midway, left] {(po)};
\draw[thin,-triangle 45]  (ysectmacdwo) .. controls (16.2, 1) .. (ysectstodwo.east)
                                                       node[very near start, right] {(mo)};
\draw[thin,-triangle 45]  (ysectlindwo.east) .. controls (19,4) and (18, 0) .. (ysectstodwo.east)
                                                       node[very near start, left] {(lmo)};

\draw (0,  14) node {\begin{tabular}{l} Skeleton DWARF \\ in executable \end{tabular}};
\draw (17, 14) node {\begin{tabular}{r} Split DWARF \\ in separate object \end{tabular}};                                                
\end{tikzpicture}
\vspace{3mm}
\caption{Split DWARF section relationships}
\label{fig:splitdwarfsectionrelationships}
\end{figure}
\end{landscape}

\clearpage
\begin{center}
   \textbf{Notes for Figure \ref{fig:splitdwarfsectionrelationships}}
\end{center}
\begin{description}
\itembfnl{(a)  \dotdebugaranges{} to \dotdebuginfo}
The \texttt{debug\_info\_offset} field in the header is the 
offset in the \dotdebuginfo{} section of the corresponding 
compilation unit header of the skeleton \dotdebuginfo{} section 
(not the compilation unit entry).  The \DWATdwoname{} attribute 
in the \dotdebuginfo{} skeleton connects 
the ranges to the full compilation unit in \dotdebuginfodwo.

\itembfnl{(b) \dotdebugnames{} to \dotdebuginfo}
The \dotdebugnames{} section  offsets lists provide an offset
for the skeleton compilation unit and eight 
byte signatures for the type units that appear only in the 
\dotdebuginfodwo. The DIE offsets for these 
compilation units and type units refer to the DIEs in the 
\dotdebuginfodwo{} section for the respective 
compilation unit and type units.

\itembfnl{(c) \dotdebuginfo{} skeleton to \dotdebugabbrev}
The \HFNdebugabbrevoffset{} value in the header is 
the offset in the \dotdebugabbrev{} section of the 
abbreviations for that compilation unit skeleton.

\itembfnl{(co) \dotdebuginfodwo{} to \dotdebugabbrevdwo}
The \HFNdebugabbrevoffset{} value in the header 
is the offset in the \dotdebugabbrevdwo{} section of the 
abbreviations for that compilation unit.

\itembfnl{(d) \dotdebuginfo{} to \dotdebugstr}
Attribute values of class string may have form \DWFORMstrp, 
whose value is an offset in the 
\dotdebugstr{} section of the corresponding string.

\itembfnl{(did) \dotdebuginfo{} to \dotdebuginfodwo}
The \DWATdwoname{}
attribute in a skeleton unit identifies the file containing 
the corresponding \texttt{.dwo} (split) data. 

\itembfnl{(do) \dotdebuginfodwo{} to \dotdebugstrdwo}
Attribute values of class string may have form 
\DWFORMstrp, whose value is an offset in the 
\dotdebugstrdwo{} section of the corresponding string.

\itembfnl{(e) \dotdebuginfo{} to \dotdebugstroffsets}
Attribute values of class string may have form 
\DWFORMstrx, whose value is an index into  the 
\dotdebugstroffsets{} section for the corresponding string.

\needlines{4}
\itembfnl{(eo)\dotdebuginfodwo{} to \dotdebugstroffsetsdwo}
Attribute values of class string may have form 
\DWFORMstrx, whose value is an index into  the 
\dotdebugstroffsetsdwo{} section for the corresponding string.

\itembfnl{(fo) \dotdebuginfodwo{} to \dotdebuginfodwo}
The operand of the \DWOPcallref{} DWARF expression 
operator is the offset of a debugging 
information entry in the \dotdebuginfodwo{} section of 
another compilation unit.  Similarly for attribute 
operands that use \DWFORMrefaddr. 
See Section \refersec{chap:controlflowoperations}.

\itembfnl{(go) \dotdebuginfodwo{} to \dotdebugmacrodwo}
An attribute of class \CLASSmacptr{} (specifically \DWATmacros{} 
with form \DWFORMsecoffset{}) is an offset within the 
\dotdebugmacrodwo{} section of the beginning of the macro 
information for the referencing unit.

\itembfnl{(h) \dotdebuginfo{} (skeleton) to \dotdebugline}
An attribute value of class \CLASSlineptr{} (specifically 
\DWATstmtlist{} with form \DWFORMsecoffset) 
is an offset within the \dotdebugline{} section of the 
beginning of the line number information for the 
referencing unit.

\itembfnl{(ho) \dotdebuginfodwo{}  to \dotdebuglinedwo{} (skeleton)}
An attribute value of class \CLASSlineptr{} (specifically  
\DWATstmtlist{}  with form \DWFORMsecoffset) 
is an offset within the \dotdebuglinedwo{} section of the 
beginning of the line number header information 
for the referencing unit (the line table details are not in 
\dotdebuglinedwo{} but the line header with its list 
of file names is present).

\itembfnl{(io) \dotdebuginfodwo{} to \dotdebugrnglistsdwo}
An attribute value of class \CLASSrnglist{} (specifically 
\DWATranges{} with form \DWFORMrnglistx{} or \DWFORMsecoffset) 
is an index or offset within the \dotdebugrnglistsdwo{}
section of a \addtoindex{range list}.
The format of \dotdebugrnglistsdwo{} location list entries 
is restricted to a subset of those in \dotdebugrnglists.
See Section \refersec{chap:noncontiguousaddressranges} for details.

\itembfnl{(jo) \dotdebuginfodwo{} to \dotdebugloclistsdwo}
An attribute value of class 
\CLASSloclist{} (specifically with form \DWFORMloclistx
or \DWFORMsecoffset) 
is an index or offset within the \dotdebugloclistsdwo{} 
section of a \addtoindex{location list.}
The format of \dotdebugloclistsdwo{} location list entries 
is restricted to a subset of those in \dotdebugloclists.
See Section \refersec{chap:locationlists} for details.

\needlines{4}
\itembfnl{(k) \dotdebuginfo{} to \dotdebugaddr}
The value of the \DWATaddrbase{} attribute in the 
\DWTAGcompileunit, \DWTAGpartialunit{} or \DWTAGtypeunit{} DIE 
is the offset in the \dotdebugaddr{} section of the machine 
addresses for that unit.
\DWFORMaddrx, \DWOPaddrx{} and \DWOPconstx{} contain indices 
relative to that offset.

\end{description}