\chapter[Selected Glossary (Informative)]{Selected Glossary$^0$ (Informative)}
\footnotetext[0]{\textcolor{green!30!black}{Note to Reviewers--This new Appendix is
an experiment. Comments regarding whether it should be
retained and expanded are solicited. --Ed.}}
\label{app:selectedglossary}

This appendix contains descriptions for selected terms
and vocabulary used in this document. The focus is on groups
of related terms that may sound similar and/or have similar 
meanings but do need to be distinguished. These descriptions
are informal and cannot be used to contradict or change the
meaning given in the main text. 
    
\section{Classification of Object Files}

\begin{description}
\itembf{DWARF object file}: an object file that includes 
DWARF information (includes relocatable, hybrid, split, executable,
shareable and package object files).

\itembf{relocatable DWARF object file (.o\footnote{These 
file extensions are typical of Linux and other Unix-based 
operating systems, but other extensions are used in other 
environments.} file)}: an object file that includes 
relocatable program text (code) and data as well as either
one or more DWARF conventional units, or a skeleton unit.

\itembf{split DWARF object file (.dwo$^1$ file)}: an object 
file that contains no program text (code) or data, but does 
contain the split units that are associated with a specific 
skeleton unit of a different file.
 
\itembf{hybrid DWARF object file (.o$^1$ file)}: an object 
file that contains relocatable program text (code) and data,
the skeleton unit for that code and data, as well as the 
split units that are associated with that skeleton unit.

\itembf{split/hybrid object file}: an object file that is 
either a split DWARF object file (.dwo) or a hybrid DWARF 
object file (.o), that is, an object file that contains 
one or more split units.

\itembf{supplementary object file}: an object file (typically
a shareable image object file) that contains macro units and
related string information that can be shared with other
object files.
\end{description}

\section{Kinds of Units}

\begin{table}[ht]
\caption{Kinds of Units and Their Characteristics}
\label{tab:kindsofunitsandtheir characteristics}
\footnotesize
\begin{tabular}{P{3.6cm}llP{3.3cm}}
\hline
\bfseries
Type of unit &\bfseries Unit Type Code &\bfseries Root DIE TAG &\bfseries Other Key \\
(short name) &                &              &\bfseries Characteristics \\
\hline
conventional full compilation unit \newline(full unit) &\DWUTcompile     &\DWTAGcompileunit&\dotdebuginfo \\
\hline
conventional type unit \newline(type unit)             &\DWUTtype        &\DWTAGtypeunit   &\dotdebuginfo \\
\hline
conventional partial unit (partial unit)       &\DWUTpartial     &\DWTAGpartialunit&\dotdebuginfo \\
\hline
skeleton unit \newline (skeleton unit)                 &\DWUTskeleton    &\DWTAGcompileunit&\dotdebuginfo{}
                                                                                            \DWATdwoname{}
                                                                                            \DWATdwoid \\
\hline
split full compilation unit (split full unit)           &\DWUTsplitcompile&\DWTAGcompileunit&\dotdebuginfodwo{}                                                                                            \DWATdwoid \\
\hline
split type unit                                        &\DWUTsplittype   &\DWTAGtypeunit   &\dotdebuginfodwo \\
\hline
\end{tabular} 
\end{table}

Macro units defined in Section \refersec{chap:macroinformation}
are distinctly different from the above units.

