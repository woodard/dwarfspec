\chapter[Split DWARF Objects (Informative)]{Split DWARF Objects (Informative)}
\label{app:splitdwarfobjectsinformative}

With the traditional DWARF format, debug information is designed
with the expectation that it will be processed by the linker to
produce an output binary with complete debug information, and
with fully-resolved references to locations within the
application. For very large applications, however, this approach
can result in excessively large link times and excessively large
output files. 

Several vendors have independently developed
proprietary approaches that allow the debug information to remain
in the relocatable object files, so that the linker does not have
to process the debug information or copy it to the output file.
These approaches have all required that additional information be
made available to the debug information consumer, and that the
consumer perform some minimal amount of relocation in order to
interpret the debug info correctly. The additional information
required, in the form of load maps or symbol tables, and the
details of the relocation are not covered by the DWARF
specification, and vary with each vendor's implementation.

These limitations are removed by the design described here.

\section{Overview}
\label{app:splitoverview}
\DWARFVersionV{} introduces an optional set of debugging sections
that allow the compiler to partition the debugging information
into a set of (small) sections that require link-time relocation
and a set of (large) sections that do not. The sections that
require relocation are written to the relocatable object file as
usual, and are linked into the final executable. The sections
that do not require relocation, however, can be written to the
relocatable object (.o) file but ignored by the linker, or they
can be written to a separate DWARF object (dwo{}) file.

\needlines{4}
The optional set of debugging sections includes the following:
\begin{itemize}
\item
\dotdebuginfodwo{} - Contains the \DWTAGcompileunit{} and
\DWTAGtypeunit{} DIEs and
their descendants. This is the bulk of the debugging
information for the compilation unit that is normally found
in the \dotdebuginfo{} section.
\item
\dotdebugabbrevdwo{} - Contains the abbreviations tables used by
the \dotdebuginfodwo{} sections.
\item
\dotdebuglocdwo{} - Contains the location lists referenced by
the debugging information entries in the \dotdebuginfodwo{}
section. This contains the location lists normally found in 
the \dotdebugloc{} section,
with a slightly modified format to eliminate the need for
relocations.
\item
\dotdebugstrdwo{} - Contains the string table for all indirect
strings referenced by the debugging information in the
\dotdebuginfodwo{} sections.
\item
\dotdebugstroffsetsdwo{} - Contains the string offsets table
for the strings in the \dotdebugstrdwo{}{} section.
\item
\dotdebugmacrodwo{} - Contains macro definition information,
normally found in the \dotdebugmacro{} section.
\item
\dotdebuglinedwo{} - Contains skeleton line tables for the type
units in the \dotdebuginfodwo{} section. These line tables
contain only the directory and files lists needed to
interpret \DWATdeclfile{} attributes in the debugging
information entries. Actual line number tables remain in the
\dotdebugline{} section, and remain in the relocatable object
(.o) files.
\end{itemize}

In order for the consumer to locate and process the debug
information, the compiler must produce a small amount of debug
information that passes through the linker into the output
binary. A skeleton \dotdebuginfo{} section for each compilation unit
contains a reference to the corresponding ".o" or ".dwo"
file, and the \dotdebugline{} section (which is typically small
compared to the \dotdebuginfo{} sections) is
linked into the output binary, as is the new \dotdebugaddr{}
section.

\needlines{6}
The debug sections that continue to be linked into the
output binary include the following:
\begin{itemize}
\item
\dotdebugabbrev{} - Contains the abbreviation codes used by the
skeleton \dotdebuginfo{} section.
\item
\dotdebuginfo{} - Contains a skeleton \DWTAGcompileunit{} DIE,
but no children.
\item
\dotdebugstr{} - Contains any strings referenced by the skeleton
\dotdebuginfo{} sections (via \DWFORMstrp{} or \DWFORMstrx{}).
\item
\dotdebugstroffsets{} - Contains the string offsets table for
the strings in the \dotdebugstr{} section.
\item
\dotdebugaddr{} - Contains references to loadable sections,
indexed by attributes of form \DWFORMaddrx{} or location
expression \DWOPaddrx{} opcodes.
\item
\dotdebugline{} - Contains the line number tables, unaffected by
this design. (These could be moved to the .dwo file, but in
order to do so, each \DWLNEsetaddress{} opcode would need to
be replaced by a new opcode that referenced an entry in the
\dotdebugaddr{} section. Furthermore, leaving this section in the
.o file allows many debug info consumers to remain unaware of
.dwo files.)
\item
\dotdebugframe{} - Contains the frame tables, unaffected by this
design.
\item
\dotdebugranges{} - Contains the range lists, unaffected by this
design.
\item
\dotdebugpubnames{} - Contains the public names for use in
building an index section. This section has the same
format and use as always. The section header refers to a
compilation unit offset, which is the offset of the
skeleton compilation unit in the \dotdebuginfo{} section.
\item
\dotdebugpubtypes{} - Contains the public types for use in
building an index section. This section has the same
format and use as always. The section header refers to a
compilation unit offset, which is the offset of the
skeleton compilation unit in the \dotdebuginfo{} section.
\item
\dotdebugaranges{} - Contains the accelerated range lookup table
for the compilation unit, unaffected by this design.
\end{itemize}

\needlines{6}
The skeleton \DWTAGcompileunit{} DIE has the following attributes:
\autocols[0pt]{c}{3}{l}{
\DWATaddrbase{},
\DWATcompdir{},
\DWATdwoname{},
\DWATdwoid{},
\DWAThighpc{} \dag,
\DWATlowpc{} \dag,
\DWATranges{} \dag,
\DWATrangesbase{},
\DWATstmtlist{},
\DWATstroffsetsbase{}
}
\dag{} If \DWATranges{} is present, \DWATlowpc{} and \DWAThighpc{} are
not used, and vice versa.

All other attributes of the compilation unit DIE are moved to
the full DIE in the \dotdebuginfodwo{} section.

\needlines{4}
Because of other improvements in \DWARFVersionV, most of the
relocations that would normally be found in the \dotdebuginfodwo{}
sections are moved to the \dotdebugaddr{} and
\dotdebugstroffsetsdwo{} sections. Those in the
\dotdebugstroffsetsdwo{} sections are simply omitted because the
DWARF information in those sections is not combined at link
time, so no relocation is necessary. Similarly,
many of the remaining relocations referring to range lists are
eliminated. 

The relocations that remain fall into the following categories:
\begin{itemize}
\item
References from compilation unit and type unit headers to the
\dotdebugabbrevdwo{} section. Because the new sections are not
combined at link time, these references need no relocations.
\item
References from \DWTAGcompileunit{} DIEs to the
\dotdebuglinedwo{} section, via \DWATstmtlist{}. This attribute
remains in the skeleton \dotdebuginfo{} section, so no
relocation in the \dotdebuginfodwo{} section is necessary.
\item
References from \DWTAGtypeunit{} DIEs to the skeleton
\dotdebuglinedwo{} section, via \DWATstmtlist{}. Because the new
sections are not combined at link time, these references need
no relocations.
\item
References from \DWTAGcompileunit{} and \DWTAGtypeunit{} DIEs
to the \dotdebugstroffsetsdwo{} section, via
\DWATstroffsetsbase{}. Because the new sections are not
combined at link time, the \DWATstroffsetsbase{} attribute
is not required in a \dotdebuginfodwo{}
section.
\item
References from \DWTAGcompileunit{} DIEs to the \dotdebugaddr{}
section, via \DWATaddrbase{}. This attribute remains in
the skeleton \dotdebuginfo{} section, so no relocation in the
\dotdebuginfodwo{} section is necessary.
\needlines{4}
\item
References from \DWTAGcompileunit{} DIEs to the \dotdebugranges{}
section, via \DWATrangesbase{}. This attribute remains in
the skeleton \dotdebuginfo{} section, so no relocation in the
\dotdebuginfodwo{} section is necessary.
\item
References from the \dotdebuglocdwo{} section to machine addresses
via a location list entry or a base address selection entry.
With a minor change to the location list entry format,
described below, these relocations are also eliminated.
\end{itemize}

\needlines{4}
Each location list entry contains beginning and ending address
offsets, which normally may be relocated addresses. In the
\dotdebuglocdwo{} section, these offsets are replaced by indices
into the \dotdebugaddr{} section. Each location list entry begins
with a single byte identifying the entry type:
\begin{itemize}
\item
\DWLLEendoflistentry{} (0) indicates an end-of-list entry,
\item
\DWLLEbaseaddressselectionentry{} (1) indicates a base address
selection entry, 
\item
\DWLLEstartendentry{} (2) indicates a normal
location list entry providing start and end addresses,
\item
\DWLLEstartlengthentry{} (3) indicates a normal location list
entry providing a start address and a length, and
\item
\DWLLEoffsetpairentry{} (4) indicates a normal location list
entry providing start and end offsets relative to the base
address. 
\end{itemize}
An end-of-list entry has no further data. A base address
selection entry contains a single unsigned LEB128
\addtoindexx{LEB128!unsigned} number
following the entry type byte, which is an index into the
\dotdebugaddr{} section that selects the new base address for
subsequent location list entries. A start/end entry contains two
unsigned LEB128\addtoindexx{LEB128!unsigned} numbers 
following the entry type byte, which are
indices into the \dotdebugaddr{} section that select the beginning
and ending addresses. A start/length entry contains one unsigned
LEB128 number and a 4-byte unsigned value (as would be
represented by the form code \DWFORMdatafour). The first number
is an index into the \dotdebugaddr{} section that selects the
beginning offset, and the second number is the length of the
range. Addresses fetched from the \dotdebugaddr{} section are not
relative to the base address. An offset pair entry contains two
4-byte unsigned values (as would be represented by the form code
\DWFORMdatafour){}, treated as the beginning and ending offsets,
respectively, relative to the base address. As in the \dotdebugloc{}
section, the base address is obtained either from the nearest
preceding base address selection entry, or, if there is no such
entry, from the compilation unit base address (as defined in
Section 3.1.1). For the latter three types (start/end,
start/length, and offset pair), the two operand values are
followed by a location description as in a normal location list
entry in the \dotdebugloc{} section.

This design depends on having an index of debugging information
available to the consumer. For name lookups, the consumer can use
the \dotdebugpubnames{} and \dotdebugpubtypes{} sections (or an index
built at link time based on the information in those sections),
which will lead to a skeleton compilation unit. The
\DWATcompdir{} and \DWATdwoname{} attributes in the skeleton
compilation unit can then be used to locate the corresponding
DWARF object file for the compilation unit. Similarly, for an
address lookup, the consumer can use the \dotdebugaranges{} table,
which will also lead to a skeleton compilation unit. For a file
and line number lookup, the skeleton compilation units can be
used to locate the line number tables.

\section{Split DWARF Object Examples}
\label{app:splitdwarfobjectexamples}

[TBD]

\section{DWARF Package File Example}
\label{app:dwarfpackagefileexample}

[TBD]
