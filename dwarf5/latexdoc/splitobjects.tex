\chapter[Split DWARF Object Files (Informative)]{Split DWARF Object Files (Informative)}
\label{app:splitdwarfobjectsinformative}
\addtoindexx{DWARF compression}
\addtoindexx{DWARF duplicate elimination|see{\textit{also} DWARF compression}}
\addtoindexx{DWARF duplicate elimination|see{\textit{also} split DWARF object file}}
With the traditional DWARF format, debug information is designed
with the expectation that it will be processed by the linker to
produce an output binary with complete debug information, and
with fully-resolved references to locations within the
application. For very large applications, however, this approach
can result in excessively large link times and excessively large
output files. 

Several vendors have independently developed
proprietary approaches that allow the debug information to remain
in the relocatable object files, so that the linker does not have
to process the debug information or copy it to the output file.
These approaches have all required that additional information be
made available to the debug information consumer, and that the
consumer perform some minimal amount of relocation in order to
interpret the debug info correctly. The additional information
required, in the form of load maps or symbol tables, and the
details of the relocation are not covered by the DWARF
specification, and vary with each vendor's implementation.

Section \refersec{datarep:splitdwarfobjectfiles} describes a new
platform-independent mechanism that allows a producer to
split the debugging information into relocatable and
non-relocatable partitions. This Appendix describes the use
of \splitDWARFobjectfile{s} and provides some illustrative
examples.

\section{Overview}
\label{app:splitoverview}
\DWARFVersionV{} introduces an optional set of debugging sections
that allow the compiler to partition the debugging information
into a set of (small) sections that require link-time relocation
and a set of (large) sections that do not. The sections that
require relocation are written to the relocatable object file as
usual, and are linked into the final executable. The sections
that do not require relocation, however, can be written to the
relocatable object (.o) file but ignored by the linker, or they
can be written to a separate DWARF object (.dwo{}) 
\addtoindexx{\texttt{.dwo} file extension} file.

\needlines{4}
The optional set of debugging sections includes the following:
\begin{itemize}
\item
\dotdebuginfodwo{} - Contains the \DWTAGcompileunit{} and
\DWTAGtypeunit{} DIEs and
their descendants. This is the bulk of the debugging
information for the compilation unit that is normally found
in the \dotdebuginfo{} section.
\item
\dotdebugabbrevdwo{} - Contains the abbreviations tables used by
the \dotdebuginfodwo{} sections.
\item
\dotdebuglocdwo{} - Contains the location lists referenced by
the debugging information entries in the \dotdebuginfodwo{}
section. This contains the location lists normally found in 
the \dotdebugloc{} section,
with a slightly modified format to eliminate the need for
relocations.
\item
\dotdebugstrdwo{} - Contains the string table for all indirect
strings referenced by the debugging information in the
\dotdebuginfodwo{} sections.
\item
\dotdebugstroffsetsdwo{} - Contains the string offsets table
for the strings in the \dotdebugstrdwo{}{} section.
\item
\dotdebugmacrodwo{} - Contains macro definition information,
normally found in the \dotdebugmacro{} section.
\item
\dotdebuglinedwo{} - Contains \addtoindex{specialized line number table}s 
for the type units in the \dotdebuginfodwo{} section. These tables
contain only the directory and filename lists needed to
interpret \DWATdeclfile{} attributes in the debugging
information entries. Actual line number tables remain in the
\dotdebugline{} section, and remain in the relocatable object
(.o) files.

\end{itemize}

In a .dwo file there is no benefit to having a separate string
section for directories and file names because the primary
string table will never be stripped. Accordingly, no
\texttt{.debug\_line\_str.dwo} is defined. Content descriptions corresponding
to \DWFORMlinestrp{} in an executable file (for example, in the
skeleton compilation unit) instead use \DWFORMstrx. This allows
directory and file name strings to be merged with general
strings and across compilations in package files (which are not
subject to potential stripping).



In order for the consumer to locate and process the debug
information, the compiler must produce a small amount of debug
information that passes through the linker into the output
binary. A skeleton \dotdebuginfo{} section for each compilation unit
contains a reference to the corresponding \texttt{.o} or \texttt{.dwo}
file, and the \dotdebugline{} section (which is typically small
compared to the \dotdebuginfo{} sections) is
linked into the output binary, as is the new \dotdebugaddr{}
section.

\needlines{6}
The debug sections that continue to be linked into the
output binary include the following:
\begin{itemize}
\item
\dotdebugabbrev{} - Contains the abbreviation codes used by the
skeleton \dotdebuginfo{} section.
\item
\dotdebuginfo{} - Contains a skeleton \DWTAGcompileunit{} DIE,
but no children.
\item
\dotdebugstr{} - Contains any strings referenced by the skeleton
\dotdebuginfo{} sections (via \DWFORMstrp{} or \DWFORMstrx{}).
\item
\dotdebugstroffsets{} - Contains the string offsets table for
the strings in the \dotdebugstr{} section.
\item
\dotdebugaddr{} - Contains references to loadable sections,
indexed by attributes of form \DWFORMaddrx{} or location
expression \DWOPaddrx{} opcodes.
\item
\dotdebugline{} - Contains the line number tables, unaffected by
this design. (These could be moved to the .dwo file, but in
order to do so, each \DWLNEsetaddress{} opcode would need to
be replaced by a new opcode that referenced an entry in the
\dotdebugaddr{} section. Furthermore, leaving this section in the
.o file allows many debug info consumers to remain unaware of
.dwo files.)
\item
\dotdebugframe{} - Contains the frame tables, unaffected by this
design.
\item
\dotdebugranges{} - Contains the range lists, unaffected by this
design.
\item
\dotdebugnames{} - Contains the names for use in
building an index section. This section has the same
format and use as always. The section header refers to a
compilation unit offset, which is the offset of the
skeleton compilation unit in the \dotdebuginfo{} section.
\item
\dotdebugaranges{} - Contains the accelerated range lookup table
for the compilation unit, unaffected by this design.
\end{itemize}

\needlines{6}
The skeleton \DWTAGcompileunit{} DIE has the following attributes:
\autocols[0pt]{c}{3}{l}{
\DWATaddrbase{},
\DWATcompdir{},
\DWATdwoname{},
\DWATdwoid{},
\DWAThighpc{} \dag,
\DWATlowpc{} \dag,
\DWATranges{} \dag,
\DWATrangesbase{},
\DWATstmtlist{},
\DWATstroffsetsbase{}
}
\dag{} If \DWATranges{} is present, the \DWATlowpc{}/\DWAThighpc{}
pair is not used, although \DWATlowpc{} may still be present
to provide a default base address for range list entries.
Conversely, if the \DWATlowpc/\linebreak[0]\DWAThighpc{} pair is
present, then \DWATranges{} is not used.

All other attributes of the compilation unit DIE are moved to
the full DIE in the \dotdebuginfodwo{} section.
The \DWATdwoid{} attribute is present
in both the skeleton DIE and the full DIE, so that a consumer
can verify a match.

\needlines{4}
Because of other improvements in \DWARFVersionV, most of the
relocations that would normally be found in the \dotdebuginfodwo{}
sections are moved to the \dotdebugaddr{} and
\dotdebugstroffsetsdwo{} sections. Those in the
\dotdebugstroffsetsdwo{} sections are simply omitted because the
DWARF information in those sections is not combined at link
time, so no relocation is necessary. Similarly,
many of the remaining relocations referring to range lists are
eliminated. 

The relocations that remain fall into the following categories:
\begin{itemize}
\item
References from compilation unit and type unit headers to the
\dotdebugabbrevdwo{} section. Because the new sections are not
combined at link time, these references need no relocations.
\item
References from \DWTAGcompileunit{} DIEs to the
\dotdebuglinedwo{} section, via \DWATstmtlist{}. This attribute
remains in the skeleton \dotdebuginfo{} section, so no
relocation in the \dotdebuginfodwo{} section is necessary.
\item
References from \DWTAGtypeunit{} DIEs to the skeleton
\dotdebuglinedwo{} section, via \DWATstmtlist{}. Because the new
sections are not combined at link time, these references need
no relocations.
\item
References from \DWTAGcompileunit{} and \DWTAGtypeunit{} DIEs
to the \dotdebugstroffsetsdwo{} section, via
\DWATstroffsetsbase{}. Because the new sections are not
combined at link time, the \DWATstroffsetsbase{} attribute
is not required in a \dotdebuginfodwo{}
section.
\item
References from \DWTAGcompileunit{} DIEs to the \dotdebugaddr{}
section, via \DWATaddrbase{}. This attribute remains in
the skeleton \dotdebuginfo{} section, so no relocation in the
\dotdebuginfodwo{} section is necessary.
\needlines{4}
\item
References from \DWTAGcompileunit{} DIEs to the \dotdebugranges{}
section, via \DWATrangesbase{}. This attribute remains in
the skeleton \dotdebuginfo{} section, so no relocation in the
\dotdebuginfodwo{} section is necessary.
\item
References from the \dotdebuglocdwo{} section to machine addresses
via a location list entry or a base address selection entry.
With a minor change to the location list entry format,
described below, these relocations are also eliminated.
\end{itemize}

\needlines{4}
Each location list entry contains beginning and ending address
offsets, which normally may be relocated addresses. In the
\dotdebuglocdwo{} section, these offsets are replaced by indices
into the \dotdebugaddr{} section. Each location list entry begins
with a single byte identifying the entry type:
\begin{itemize}
\item
\DWLLEendoflistentry{} (0) indicates an end-of-list entry,
\item
\DWLLEbaseaddressselectionentry{} (1) indicates a base address
selection entry, 
\item
\DWLLEstartendentry{} (2) indicates a normal
location list entry providing start and end addresses,
\item
\DWLLEstartlengthentry{} (3) indicates a normal location list
entry providing a start address and a length, and
\item
\DWLLEoffsetpairentry{} (4) indicates a normal location list
entry providing start and end offsets relative to the base
address. 
\end{itemize}
An end-of-list entry has no further data. A base address
selection entry contains a single unsigned LEB128
\addtoindexx{LEB128!unsigned} number
following the entry type byte, which is an index into the
\dotdebugaddr{} section that selects the new base address for
subsequent location list entries. A start/end entry contains two
unsigned LEB128\addtoindexx{LEB128!unsigned} numbers 
following the entry type byte, which are
indices into the \dotdebugaddr{} section that select the beginning
and ending addresses. A start/length entry contains one unsigned
LEB128 number and a 4-byte unsigned value (as would be
represented by the form code \DWFORMdatafour). The first number
is an index into the \dotdebugaddr{} section that selects the
beginning offset, and the second number is the length of the
range. Addresses fetched from the \dotdebugaddr{} section are not
relative to the base address. An offset pair entry contains two
4-byte unsigned values (as would be represented by the form code
\DWFORMdatafour){}, treated as the beginning and ending offsets,
respectively, relative to the base address. As in the \dotdebugloc{}
section, the base address is obtained either from the nearest
preceding base address selection entry, or, if there is no such
entry, from the compilation unit base address (as defined in
Section \refersec{chap:normalandpartialcompilationunitentries}). 
For the latter three types (start/end,
start/length, and offset pair), the two operand values are
followed by a location description as in a normal location list
entry in the \dotdebugloc{} section.

\needlines{8}
This design depends on having an index of debugging information
available to the consumer. For name lookups, the consumer can 
use the \dotdebugnames{} index section (see 
Section \refersec{chap:acceleratedaccess}) to 
locate a skeleton compilation unit. The
\DWATcompdir{} and \DWATdwoname{} attributes in the skeleton
compilation unit can then be used to locate the corresponding
DWARF object file for the compilation unit. Similarly, for an
address lookup, the consumer can use the \dotdebugaranges{} table,
which will also lead to a skeleton compilation unit. For a file
and line number lookup, the skeleton compilation units can be
used to locate the line number tables.

\section{Split DWARF Object File Example}
\label{app:splitdwarfobjectfileexample}
\addtoindexx{split DWARF object file!example}
Consider the example source code in 
Figure \refersec{fig:splitobjectexamplesourcefragment1}, 
Figure \refersec{fig:splitobjectexamplesourcefragment2} and
Figure \refersec{fig:splitobjectexamplesourcefragment3}.
When compiled with split DWARF, we will have two object files,
\texttt{demo1.o} and \texttt{demo2.o}, and two \splitDWARFobjectfile{s}, 
\texttt{demo1.dwo} and \texttt{demo2.dwo}.

\begin{figure}[b]
\textit{File demo1.cc}
\begin{lstlisting}
#include "demo.h"

bool Box::contains(const Point& p) const
{
    return (p.x() >= ll_.x() && p.x() <= ur_.x() &&
            p.y() >= ll_.y() && p.y() <= ur_.y());
}
\end{lstlisting}
\caption{Split object example: source fragment \#1}
\label{fig:splitobjectexamplesourcefragment1}
\end{figure}

\begin{figure}[h]
\textit{File demo2.cc}
\begin{lstlisting}
#include "demo.h"

bool Line::clip(const Box& b)
{
  float slope = (end_.y() - start_.y()) / (end_.x() - start_.x());
  while (1) {
    // Trivial acceptance.
    if (b.contains(start_) && b.contains(end_)) return true;

    // Trivial rejection.
    if (start_.x() < b.l() && end_.x() < b.l()) return false;
    if (start_.x() > b.r() && end_.x() > b.r()) return false;
    if (start_.y() < b.b() && end_.y() < b.b()) return false;
    if (start_.y() > b.t() && end_.y() > b.t()) return false;

    if (b.contains(start_)) {
      // Swap points so that start_ is outside the clipping 
      // rectangle.
      Point temp = start_;
      start_ = end_;
      end_ = temp;
    }

    if (start_.x() < b.l())
      start_ = Point(b.l(), 
                     start_.y() + (b.l() - start_.x()) * slope);
    else if (start_.x() > b.r())
      start_ = Point(b.r(), 
                     start_.y() + (b.r() - start_.x()) * slope);
    else if (start_.y() < b.b())
      start_ = Point(start_.x() + (b.b() - start_.y()) / slope, 
                     b.b());
    else if (start_.y() > b.t())
      start_ = Point(start_.x() + (b.t() - start_.y()) / slope, 
                     b.t());
  }
}
\end{lstlisting}
\caption{Split object example: source fragment \#2}
\label{fig:splitobjectexamplesourcefragment2}
\end{figure}

\begin{figure}[h]
\textit{File demo.h}
\begin{lstlisting}
class A {
  public:
    Point(float x, float y) : x_(x), y_(y){}
    float x() const { return x_; }
    float y() const { return y_; }
  private:
    float x_;
    float y_;
};

class Line {
  public:
    Line(Point start, Point end) : start_(start), end_(end){}
    bool clip(const Box& b);
    Point start() const { return start_; }
    Point end() const { return end_; }
  private:
    Point start_;
    Point end_;
};

class Box {
  public:
    Box(float l, float r, float b, float t) : ll_(l, b), ur_(r, t){}
    Box(Point ll, Point ur) : ll_(ll), ur_(ur){}
    bool contains(const Point& p) const;
    float l() const { return ll_.x(); }
    float r() const { return ur_.x(); }
    float b() const { return ll_.y(); }
    float t() const { return ur_.y(); }
  private:
    Point ll_;
    Point ur_;
};

\end{lstlisting}
\caption{Split object example: source fragment \#3}
\label{fig:splitobjectexamplesourcefragment3}
\end{figure}

\clearpage
\subsection{Contents of the Object File}
The object files each contain the following sections of debug
information:
\begin{alltt}
  \dotdebugabbrev
  \dotdebuginfo
  \dotdebugranges
  \dotdebugline
  \dotdebugstr
  \dotdebugaddr
  \dotdebugnames
  \dotdebugaranges
\end{alltt}

The \dotdebugabbrev{} section contains just a single entry describing
the skeleton compilation unit DIE.

The DWARF description in the \dotdebuginfo{} section 
contains just a single DIE, the skeleton compilation unit, 
which may look like 
Figure \referfol{fig:splitdwafexampleskeletondwarfdescription}.

\begin{figure}[h]
\begin{dwflisting}
\begin{alltt}

    \DWTAGcompileunit
      \DWATcompdir: (reference to directory name in .debug_str)
      \DWATdwoname: (reference to "demo1.dwo" in .debug_str)
      \DWATdwoid: 0x44e413b8a2d1b8f
      \DWATaddrbase: (reference to .debug_addr section)
      \DWATrangesbase: (reference to range list in .debug_ranges section)
      \DWATranges: (offset of range list in .debug_ranges section)
      \DWATstmtlist: (reference to .debug_line section)
      \DWATlowpc: 0
      
\end{alltt}
\end{dwflisting}
\caption{Split object example: Skeleton DWARF description}
\label{fig:splitdwafexampleskeletondwarfdescription}
\end{figure}

The \DWATcompdir{} and \DWATdwoname{} attributes provide the
location of the corresponding \splitDWARFobjectfile{} that
contains the full debug information; that file is normally
expected to be in the same directory as the object file itself.

The \DWATdwoid{} attribute provides a hash of the debug
information contained in the \splitDWARFobjectfile. This hash serves
two purposes: it can be used to verify that the debug information
in the \splitDWARFobjectfile{} matches the information in the object
file, and it can be used to find the debug information in a DWARF
package file.

\needlines{4}
The \DWATaddrbase{} attribute contains the relocatable offset of
this object file's contribution to the \dotdebugaddr{} section, and
the \DWATrangesbase{} attribute contains the relocatable offset
of this object file's contribution to the \dotdebugranges{} section.
The \DWATranges{} attribute refers to a specific range list within
that contribution, and its value is a (non-relocatable) offset
relative to the base. In a compilation unit with a single
contiguous range of code, the \DWATranges{} attribute might be
omitted, and instead replaced by the pair \DWATlowpc{} and
\DWAThighpc.

The \DWATstmtlist{} attribute contains the relocatable offset of
this file's contribution to the \dotdebugline{} table.

If there is a \DWATranges{} attribute, the \DWATlowpc{} attribute
provides a default base address for the range list entries in the
\dotdebugranges{} section. It may be omitted if each range list entry
provides an explicit base address selection entry; it may provide
a relocatable base address, in which case the offsets in each
range list entry are relative to it; or it may have the value 0,
in which case the offsets in each range list entry are themselves
relocatable addresses.

The \dotdebugranges{} section contains the range list referenced by
the \DWATranges{} attribute in the skeleton compilation unit DIE,
plus any range lists referenced by \DWATranges{} attributes in the
split DWARF object. In our example, \texttt{demo1.o} would contain range
list entries for the function \texttt{Box::contains}, as well as for
out-of-line copies of the inline functions \texttt{Point::x} and 
\texttt{Point::y}.

The \dotdebugline{} section contains the full line number table for
the compiled code in the object file. In the example in
Figure \refersec{fig:splitobjectexamplesourcefragment1}, the line
number program header would list the two files, \texttt{demo.h} and
\texttt{demo1.cc}, and would contain line number programs for
\texttt{Box::contains}, \texttt{Point::x}, and \texttt{Point::y}.

The \dotdebugstr{} section contains the strings referenced indirectly
by the compilation unit DIE and by the line number program.

The \dotdebugaddr{} section contains relocatable addresses of
locations in the loadable text and data that are referenced by
debugging information entries in the split DWARF object. In the
example in \refersec{fig:splitobjectexamplesourcefragment3}, 
\texttt{demo1.o} may have three entries:
\begin{center}
%\footnotesize
\begin{tabular}{cl}
Slot & Location referenced \\
\hline
   0   &  low PC value for \texttt{Box::contains}  \\
   1   &  low PC value for \texttt{Point::x}       \\
   2   &  low PC value for \texttt{Point::y}       \\
\end{tabular}
\end{center}

\needlines{4}
The \dotdebugnames{}
section contains the names defined by the debugging
information in the \splitDWARFobjectfile{} 
(see Section \refersec{chap:contentsofthenameindex}, 
and references the skeleton compilation unit. 
When linked together into a final executable,
they can be used by a DWARF consumer to lookup a name to find one
or more skeleton compilation units that provide information about
that name. From the skeleton compilation unit, the consumer can
find the \splitDWARFobjectfile{} that it can then read to get the full
DWARF information.

The \dotdebugaranges{} section contains the PC ranges defined in this
compilation unit, and allow a DWARF consumer to map a PC value to
a skeleton compilation unit, and then to a \splitDWARFobjectfile.


\subsection{Contents of the Split DWARF Object Files}
The \splitDWARFobjectfile{s} each contain the following sections:
\begin{alltt}
  \dotdebugabbrevdwo
  \dotdebuginfodwo{} (for the compilation unit)
  \dotdebuginfodwo{} (one COMDAT section for each type unit)
  \dotdebuglocdwo
  \dotdebuglinedwo
  \dotdebugmacrodwo
  \dotdebugstroffsetsdwo
  \dotdebugstrdwo
\end{alltt}
The \dotdebugabbrevdwo{} section contains the abbreviation
declarations for the debugging information entries in the
\dotdebuginfodwo{} section. In general, it looks just like a normal
\dotdebugabbrev{} section in a non-split object file.

The \dotdebuginfodwo{} section containing the compilation unit
contains the full debugging information for the compile unit, and
looks much like a normal \dotdebuginfo{} section in a non-split
object file, with the following exceptions:
\begin{itemize}
\item The \DWTAGcompileunit{} DIE does not need to repeat the
\DWATranges, \DWATlowpc, \DWAThighpc, and
\DWATstmtlist{} attributes that are provided in the skeleton
compilation unit.

\item References to strings in the string table use the new form
code \DWFORMstrx, referring to slots in the
\dotdebugstroffsetsdwo{} section.

\textit{Use of \DWFORMstrp{} is not appropriate in a \splitDWARFobjectfile.}

\needlines{4}
\item References to range lists in the \dotdebugranges{} section are
all relative to the base offset given by \DWATrangesbase{}
in the skeleton compilation unit.

\item References to relocatable addresses in the object file use
the new form code \DWFORMaddrx, referring to slots in the
\dotdebugaddr{} table, relative to the base offset given by
\DWATaddrbase{} in the skeleton compilation unit.
\end{itemize}

Figure \refersec{fig:splitobjectexampledemo1dwodwarfexcerpts} presents
some excerpts from the \dotdebuginfodwo{} section for \texttt{demo1.dwo}.

\begin{figure}[h]
\figurepart{1}{2}
\begin{dwflisting}
\begin{alltt}

    \DWTAGcompileunit
        \DWATproducer [\DWFORMstrx]: (slot 15) (producer string)
        \DWATlanguage: \DWLANGCplusplus
        \DWATname [\DWFORMstrx]: (slot 7) "demo1.cc"
        \DWATcompdir [\DWFORMstrx]: (slot 4) (directory name)
        \DWATdwoid [\DWFORMdataeight]: 0x44e413b8a2d1b8f
1$:   \DWTAGclasstype
          \DWATname [\DWFORMstrx]: (slot 12) "Point"
          \DWATsignature [\DWFORMrefsigeight]: 0x2f33248f03ff18ab
          \DWATdeclaration: true
2$:     \DWTAGsubprogram
          \DWATexternal: true
          \DWATname [\DWFORMstrx]: (slot 12) "Point"
          \DWATdeclfile: 1
          \DWATdeclline: 5
          \DWATlinkagename [\DWFORMstrx]: (slot 16): "_ZN5PointC4Eff"
          \DWATaccessibility: \DWACCESSpublic
          \DWATdeclaration: true
        ...
3$:   \DWTAGclasstype
          \DWATname [\DWFORMstring]: "Box"
          \DWATsignature [\DWFORMrefsigeight]: 0xe97a3917c5a6529b
          \DWATdeclaration: true
        ...
4$:     \DWTAGsubprogram
            \DWATexternal: true
            \DWATname [\DWFORMstrx]: (slot 0) "contains"
            \DWATdeclfile: 1
            \DWATdeclline: 28
            \DWATlinkagename [\DWFORMstrx]: (slot 8) "_ZNK3Box8containsERK5Point"
            \DWATtype: (reference to 7$)
            \DWATaccessibility: \DWACCESSpublic
            \DWATdeclaration: true
        ...
\end{alltt}
\end{dwflisting}
\caption{Split object example: \texttt{demo1.dwo} excerpts}
\label{fig:splitobjectexampledemo1dwodwarfexcerpts}
\end{figure}
        
\begin{figure}
\figurepart{2}{2}
\begin{dwflisting}
\begin{alltt}

5$:   \DWTAGsubprogram
          \DWATspecification: (reference to 4$)
          \DWATdeclfile: 2
          \DWATdeclline: 3
          \DWATlowpc [\DWFORMaddrx]: (slot 0)
          \DWAThighpc [\DWFORMdataeight]: 0xbb
          \DWATframebase: \DWOPcallframecfa
          \DWATobjectpointer: (reference to 6$)
6$:     \DWTAGformalparameter
            \DWATname [\DWFORMstrx]: (slot 13): "this"
            \DWATtype: (reference to 8$)
            \DWATartificial: true
            \DWATlocation: \DWOPfbreg(-24)
        \DWTAGformalparameter
            \DWATname [\DWFORMstring]: "p"
            \DWATdeclfile: 2
            \DWATdeclline: 3
            \DWATtype: (reference to 11$)
            \DWATlocation: \DWOPfbreg(-32)
      ...
7$:   \DWTAGbasetype
          \DWATbytesize: 1
          \DWATencoding: \DWATEboolean
          \DWATname [\DWFORMstrx]: (slot 5) "bool"
      ...
8$:   \DWTAGconsttype
          \DWATtype: (reference to 9$)
9$:   \DWTAGpointertype
          \DWATbytesize: 8
          \DWATtype: (reference to 10$)
10$:  \DWTAGconsttype
          \DWATtype: (reference to 3$)
      ...
11$:  \DWTAGconsttype
          \DWATtype: (reference to 12$)
12$:  \DWTAGreferencetype
          \DWATbytesize: 8
          \DWATtype: (reference to 13$)
13$:  \DWTAGconsttype
          \DWATtype: (reference to 1$)
      ...
\end{alltt}
\end{dwflisting}
\begin{center}
\vspace{3mm}
Figure~\ref{fig:splitobjectexampledemo1dwodwarfexcerpts}: Split object example: \texttt{demo1.dwo} DWARF excerpts \textit{(concluded)}
\end{center}
\end{figure}

In the defining declaration for \texttt{Box::contains} at 5\$, the
\DWATlowpc{} attribute is represented with \DWFORMaddrx,
referring to slot 0 in the \dotdebugaddr{} table from \texttt{demo1.o}.
That slot contains the relocated address of the beginning of the
function.

Each type unit is contained in its own COMDAT \dotdebuginfodwo{}
section, and looks like a normal type unit in a non-split object,
except that the \DWTAGtypeunit{} DIE contains a \DWATstmtlist{}
attribute that refers to a specialized \dotdebuglinedwo{}
\addtoindexx{type unit!specialized \texttt{.debug\_line.dwo} section in}
\addtoindexx{specialized \texttt{.debug\_line.dwo} section}
section. This
section contains a normal line number
program header with a list of include directories and filenames,
but no line number program. This section is used only as a
reference for filenames needed for \DWATdeclfile{} attributes
within the type unit.

The \dotdebugstroffsetsdwo{} section contains an entry for each
unique string in the string table. In the \texttt{demo1.dwo} example,
these string table slots have been assigned as shown in
Figure \refersec{fig:splitobjectexamplestringtableslots}.

\begin{figure}[H]
\begin{center}
\footnotesize
\begin{tabular}{cl|cl}
    Slot & String & Slot & String \\
    \hline
    0  &   contains                         &    10 &   \_ZNK3Box1rEv \\
    1  &   \_ZNK5Point1xEv                  &    11 &   \_ZN3BoxC4E5PointS0\_ \\
    2  &   \_ZNK3Box1lEv                    &    12 &   Point\\
    3  &   \_ZNK3Box1bEv                    &    13 &   this\\
    4  &   \textit{(compilation directory)} &    14 &   float \\
    5  &   bool                             &    15 &   \textit{(producer string)} \\
    6  &   \_ZN3BoxC4Effff                  &    16 &   \_ZN5PointC4Eff \\
    7  &   demo1.cc                         &    17 &   \_ZNK3Box1tEv \\
    8  &   \_ZNK3Box8containsERK5Point      & \\
    9  &   \_ZNK5Point1yEv                  & \\
\end{tabular}
\end{center}
\caption{Split object example: String table slots}
\label{fig:splitobjectexamplestringtableslots}
\end{figure}

Each entry in the table is the offset of the string, which is
contained in the \dotdebugstrdwo{} section. In a split DWARF 
object file, these offsets have no relocations, since they 
are not part of the relocatable object file. When combined into a 
DWARF package file, however, each slot must be adjusted to 
refer to the appropriate offset within the merged string table.
The tool that builds the DWARF package file must understand 
the structure of the section in order to apply the necessary 
adjustments. (See Section \refersec{app:dwarfpackagefileexample} 
for an example of a DWARF package file.)

\needlines{4}
The \dotdebuglocdwo{} section contains the location lists referenced
by \DWATlocation{} attributes in the \dotdebuginfodwo{} section. This
section has a similar format to the \dotdebugloc{} section in a
non-split object, but it has some small differences as explained
in Section \refersec{datarep:locationlistentriesinsplitobjects}. 
In \texttt{demo2.dwo} as shown in 
Figure \refersec{fig:splitobjectexampledemo2dwodwarfdebuginfodwoexcerpts}, 
the debugging information for \texttt{Line::clip} describes a local 
variable \texttt{slope} whose location varies based on the PC.
Figure \refersec{fig:splitobjectexampledemo2dwodwarfdebuglocdwoexcerpts} 
presents some excerpts from the \dotdebuginfodwo{} section for 
\texttt{demo2.dwo}.

\begin{figure}[b]
\figurepart{1}{2}
\begin{dwflisting}
\begin{alltt}

1$: \DWTAGclasstype
        \DWATname [\DWFORMstrx]: (slot 20) "Line"
        \DWATsignature [\DWFORMrefsigeight]: 0x79c7ef0eae7375d1
        \DWATdeclaration: true
    ...
2$:   \DWTAGsubprogram
          \DWATexternal: true
          \DWATname [\DWFORMstrx]: (slot 19) "clip"
          \DWATdeclfile: 2
          \DWATdeclline: 16
          \DWATlinkagename [\DWFORMstrx]: (slot 2) "_ZN4Line4clipERK3Box"
          \DWATtype: (reference to DIE for bool)
          \DWATaccessibility: \DWACCESSpublic
          \DWATdeclaration: true
      ...
\end{alltt}
\end{dwflisting}
\caption{Split object example: \texttt{demo2.dwo} DWARF \dotdebuginfodwo{} excerpts}
\label{fig:splitobjectexampledemo2dwodwarfdebuginfodwoexcerpts}
\end{figure}

\begin{figure}
\figurepart{2}{2}
\begin{dwflisting}
\begin{alltt}

3$:   \DWTAGsubprogram
          \DWATspecification: (reference to 2$)
          \DWATdeclfile: 1
          \DWATdeclline: 3
          \DWATlowpc [\DWFORMaddrx]: (slot 32)
          \DWAThighpc [\DWFORMdataeight]: 0x1ec
          \DWATframebase: \DWOPcallframecfa
          \DWATobjectpointer: (reference to 4$)
4$:     \DWTAGformalparameter
            \DWATname: (indexed string: 0x11): this
            \DWATtype: (reference to DIE for type const Point* const)
            \DWATartificial: 1
            \DWATlocation: 0x0 (location list)
5$:     \DWTAGformalparameter
            \DWATname: b
            \DWATdeclfile: 1
            \DWATdeclline: 3
            \DWATtype: (reference to DIE for type const Box& const)
            \DWATlocation [\DWFORMsecoffset]: 0x2a
6$:     \DWTAGlexicalblock
            \DWATlowpc [\DWFORMaddrx]: (slot 17)
            \DWAThighpc: 0x1d5
7$:       \DWTAGvariable
              \DWATname [\DWFORMstrx]: (slot 28): "slope"
              \DWATdeclfile: 1
              \DWATdeclline: 5
              \DWATtype: (reference to DIE for type float)
              \DWATlocation [\DWFORMsecoffset]: 0x49

\end{alltt}
\end{dwflisting}
\begin{center}
\vspace{3mm}
Figure~\ref{fig:splitobjectexampledemo2dwodwarfdebuginfodwoexcerpts}: Split object example: \texttt{demo2.dwo} DWARF \dotdebuginfodwo{} excerpts \textit{(concluded)}
\end{center}
\end{figure}

In Figure \refersec{fig:splitobjectexampledemo2dwodwarfdebuginfodwoexcerpts},
The \DWTAGformalparameter{} entries at 4\$ and 5\$ refer to the
location lists at offset \texttt{0x0} and \texttt{0x2a}, respectively, and the
\DWTAGvariable{} entry for \texttt{slope} at 7\$ refers to the location
list at offset \texttt{0x49}. 
Figure \refersec{fig:splitobjectexampledemo2dwodwarfdebuglocdwoexcerpts}
shows a representation of the
location lists at those offsets in the \dotdebuglocdwo{} section.

\begin{figure}
\begin{dwflisting}
\begin{tabular}{rlrrl}
\texttt{offset} & entry type & start & length & expression \\
\hline \\
0x00 & \DWLLEstartlengthentry &  [9] & 0x002f & \DWOPregfive~(rdi) \\
0x09 & \DWLLEstartlengthentry & [11] & 0x01b9 & \DWOPregthree~(rbx) \\
0x12 & \DWLLEstartlengthentry & [29] & 0x0003 & \DWOPbregtwelve~(r12):\\
&&&& -8; \DWOPstackvalue \\
0x1d & \DWLLEstartlengthentry & [31] & 0x0001 & \DWOPentryvalue: \\
&&&& (\DWOPregfive~(rdi)); \\
&&&& \DWOPstackvalue \\
0x29 & \DWLLEendoflistentry &&& \\
\\   & \hhline{-} &&& \\
0x2a & \DWLLEstartlengthentry &  [9] & 0x002f & \DWOPregfour~(rsi)) \\
0x33 & \DWLLEstartlengthentry & [11] & 0x01ba & \DWOPregsix~(rbp)) \\
0x3c & \DWLLEstartlengthentry & [30] & 0x0003 & \DWOPentryvalue: \\
&&&& (\DWOPregfour~(rsi)); \\
&&&& \DWOPstackvalue \\
0x48 & \DWLLEendoflistentry &&& \\
\\   & \hhline{-} &&& \\
0x49 & \DWLLEstartlengthentry & [10] & 0x0004 & \DWOPregeighteen~(xmm1) \\
0x52 & \DWLLEstartlengthentry & [11] & 0x01bd & \DWOPfbreg: -36 \\
0x5c & \DWLLEendoflistentry &&& \\
&&&& \\
\end{tabular}
\end{dwflisting}
\caption{Split object example: \texttt{demo2.dwo} DWARF \dotdebuglocdwo{} excerpts}
\label{fig:splitobjectexampledemo2dwodwarfdebuglocdwoexcerpts}
\end{figure}

In each \DWLLEstartlengthentry{}, the start field is the index
of a slot in the \dotdebugaddr{} section, relative to the base
offset defined by the compilations unit's \DWATaddrbase{}
attribute. The \dotdebugaddr{} slots referenced by these entries give
the relocated address of a label within the function where the
address range begins. The length field gives the length of the
address range.

\clearpage
\section{DWARF Package File Example}
\label{app:dwarfpackagefileexample}
\addtoindexx{DWARF duplicate elimination!examples}
[TBD]
