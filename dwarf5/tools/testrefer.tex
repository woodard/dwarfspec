\% Definitions for each of the DWARF names
% These eliminate the need to use escapes for the underscores or
% add entries for indexing
%

\newcommand{\addtoindex}[1]{#1\index{#1}}
\newcommand{\addttindex}[1]{\texttt{#1}\index{#1@\texttt{#1}}}
\newcommand{\refersec}[1]{\vref{#1}}  % beware possible rerun loop
\newcommand{\referfol}[1]{\ref{#1} following}

% Generate a live link in the document
% use like \livelink{chap:DWOPdup}{DW\_OP\_dup}
\newcommand{\livelink}[2]{\hyperlink{#1}{#2}\index{#2}}
% use when the index is different from the text and target.
\newcommand{\livelinki}[3]{\hyperlink{#1}{#2}\index{#3}}
% livetarg is the declaration this is the target of livelinks.
% FIXME: we might want livetarg and livetargi  #2 to be \textbf{#2}
\newcommand{\livetarg}[2]{\hypertarget{#1}{#2}\index{#2}}
% When we want the index entry to look different from the name.
\newcommand{\livetargi}[3]{\hypertarget{#1}{#2}\index{#3}}

\newcommand{\thirtytwobitdwarfformat}[1][]{\livelink{datarep:xxbitdwffmt}{32-bit DWARF format}}
\newcommand{\sixtyfourbitdwarfformat}[1][]{\livelink{datarep:xxbitdwffmt}{64-bit DWARF format}}

% For index entries. The tt-variant of each pair is designed to
% allow a word to appear in tt font in the main test and the index
% but to collate in the index in its non-tt order. (LaTex normally
% sorts all tt words before all non-tt words.)
\newcommand{\addtoindex}[1]{#1\index{#1}}
\newcommand{\addttindex}[1]{\texttt{#1}\index{#1@\texttt{#1}}}
\newcommand{\addtoindexi}[2]{#1\index{#2}}
\newcommand{\addttindexi}[2]{\texttt{#1}\index{#2@\texttt{#2}}}
\newcommand{\addtoindexx}[1]{\index{#1}}
\newcommand{\addttindexx}[1]{\index{#1@\texttt{#1}}}





% A command to define multiple helpful DWARF name commands
%
\newcommand{\newdwfnamecommands}[2]{
	\expandafter\def\csname #1LINK\endcsname{\index{#2}\hyperlink{chap:#1}{#2}}
	\expandafter\def\csname #1TARG\endcsname{\index{#2}\hypertarget{chap:#1}{#2}}
	\expandafter\def\csname #1INDX\endcsname{\index{#2}#2}
	\expandafter\def\csname #1MARK\endcsname{\hypertarget{chap:#1}{}\index{#2}}	
	\expandafter\def\csname #1NAME\endcsname{#2}
	% The normal, most common use in running text...
	\expandafter\def\csname #1\endcsname{\csname #1LINK\endcsname}
	}


% DW_ACCESS
%
\newdwfnamecommands{DWACCESSprivate}{DW\_ACCESS\_private}
\newdwfnamecommands{DWACCESSprotected}{DW\_ACCESS\_protected}
\newdwfnamecommands{DWACCESSpublic}{DW\_ACCESS\_public}
%
% DW_ADDR
%
\newdwfnamecommands{DWADDRnone}{DW\_ADDR\_none}
%
% DW_AT
%
\newdwfnamecommands{DWATabstractorigin}{DW\_AT\_abstract\_origin}
\newdwfnamecommands{DWATaccessibility}{DW\_AT\_accessibility}
\newdwfnamecommands{DWATaddrbase}{DW\_AT\_addr\_base}
\newdwfnamecommands{DWATaddressclass}{DW\_AT\_address\_class}
\newdwfnamecommands{DWATallocated}{DW\_AT\_allocated}
\newdwfnamecommands{DWATartificial}{DW\_AT\_artificial}
\newdwfnamecommands{DWATassociated}{DW\_AT\_associated}
%
\newdwfnamecommands{DWATbasetypes}{DW\_AT\_base\_types}
\newdwfnamecommands{DWATbinaryscale}{DW\_AT\_binary\_scale}
\newdwfnamecommands{DWATbitoffset}{DW\_AT\_bit\_offset}
\newdwfnamecommands{DWATbitsize}{DW\_AT\_bit\_size}
\newdwfnamecommands{DWATbitstride}{DW\_AT\_bit\_stride}
\newdwfnamecommands{DWATbyteoffset}{DW\_AT\_byte\_offset}
\newdwfnamecommands{DWATbytesize}{DW\_AT\_byte\_size}
\newdwfnamecommands{DWATbytestride}{DW\_AT\_byte\_stride}
%
\newdwfnamecommands{DWATcallcolumn}{DW\_AT\_call\_column}
\newdwfnamecommands{DWATcallfile}{DW\_AT\_call\_file}
\newdwfnamecommands{DWATcallline}{DW\_AT\_call\_line}
\newdwfnamecommands{DWATcallingconvention}{DW\_AT\_calling\_convention}
\newdwfnamecommands{DWATcommonreference}{DW\_AT\_common\_reference}
\newdwfnamecommands{DWATcompdir}{DW\_AT\_comp\_dir}
\newdwfnamecommands{DWATconstexpr}{DW\_AT\_const\_expr}
\newdwfnamecommands{DWATconstvalue}{DW\_AT\_const\_value}
\newdwfnamecommands{DWATcontainingtype}{DW\_AT\_containing\_type}
\newdwfnamecommands{DWATcount}{DW\_AT\_count}
%
\newdwfnamecommands{DWATdatabitoffset}{DW\_AT\_data\_bit\_offset}
\newdwfnamecommands{DWATdatalocation}{DW\_AT\_data\_location}
\newdwfnamecommands{DWATdatamemberlocation}{DW\_AT\_data\_member\_location}
\newdwfnamecommands{DWATdecimalscale}{DW\_AT\_decimal\_scale}
	\newcommand{\DWOPbregtwo}{\hyperlink{chap:DWOPbregn}{DW\_OP\_breg2}}		% Link, don't index...
	\newcommand{\DWOPbregthree}{\hyperlink{chap:DWOPbregn}{DW\_OP\_breg3}}		%
	\newcommand{\DWOPbregfour}{\hyperlink{chap:DWOPbregn}{DW\_OP\_breg4}}		%
	\newcommand{\DWOPbregfive}{\hyperlink{chap:DWOPbregn}{DW\_OP\_breg5}}		%
	\newcommand{\DWOPbregeleven}{\hyperlink{chap:DWOPbregn}{DW\_OP\_breg11}}	%
\newdwfnamecommands{DWOPbregx}{DW\_OP\_bregx}
\newdwfnamecommands{DWOPcalltwo}{DW\_OP\_call2}
\newdwfnamecommands{DWOPcallfour}{DW\_OP\_call4}
%
%%%%%%%%%%%%%%%
%
% .debug_*, .debug_*.dwo, et al
%
\newcommand{\dotdebugabbrev}{\addtoindex{\texttt{.debug\_abbrev}}}
\newcommand{\dotdebugaddr}{\addtoindex{\texttt{.debug\_addr}}}
\newcommand{\dotdebugaranges}{\addtoindex{\texttt{.debug\_aranges}}}
\newcommand{\dotdebugframe}{\addtoindex{\texttt{.debug\_frame}}}
%
\newcommand{\dotdata}{\addtoindex{\texttt{.data}}}
\newcommand{\dottext}{\addtoindex{\texttt{.text}}}
%
% Current section version numbers
%
\newcommand{\versiondotdebugabbrev}  {5}
\newcommand{\versiondotdebugaddr}    {5}
\newcommand{\versiondotdebugstr}     {5}
\newcommand{\versiondotdebugstroffsets}{5}
\newcommand{\versiondotdebugtypes}   {\versiondotdebuginfo}
%
% DWARF Standard Versions
%
\newcommand{\DWARFVersionI}   {\addtoindex{DWARF Version 1}}
\newcommand{\DWARFVersionII}  {\addtoindex{DWARF Version 2}}
%
\newcommand{\MDfive}{\livelink{def:MDfive}{MD5}}


\DWACCESSpublic&0x01  \\
\DWACCESSpublicTARG{}     
# Following is duplicate def
\DWACCESSpublicTARG{}     

\DWATcountLINK
% Following should generate error
\DWATcountnodwnamecommand + 3

% just shows the name. Not a target or even indexed.
% Actually an error because we never defined this above.
A \DWATnonameNAME{} attribute whose value is a

\livelinki{datarep:classreference}{reference}{reference class}
\livelinki{datarep:classstring}{string}{string class}
describe the static \livelink{chap:lexicalblock}{block} structure 
\livelink{chap:lexicalblock}{lexical block} that owns it, 
class \livelink{chap:classexprloc}{exprloc}  
using class \livelink{chap:classloclistptr}{loclistptr}
which is a \livelink{chap:classflag}{flag}.
\livelink{chap:classflag}{flag}.
attribute whose value is a \livelink{chap:classreference}{reference} to

\livetargi{chap:declarationcoordinates}{}{declaration coordinates}
oops \livetarg{chap:classreference}{}
buy \livetarg{chap:classexprloc}{}
something \livetarg{chap:classflag}{}
nice \livetarg{chap:DWATdwoidforunit}{}
today \livetarg{chap:DWATdwoidforunit}{}

\label{mylabone}
\label{mylabtwo}
\label{mylabdup}
\label{mylabdup}
\refersec{mylabone}
\refersec{danglingref}

% defines label
\begin{simplenametable}[1.9in]{Accessibility codes}{tab:goodlabel}
% No label
\begin{simplenametable}[1.9in]
% no label here, not reported
\begin{foo}[1.9in]{Accessibility codes}{badlabel}
\begin{simplenametable}[1.9in]{Accessibility codes}{chap:unusedlabel}
\refersec{tab:goodlabel}
