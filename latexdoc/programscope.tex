\chapter{Program Scope Entries}
\label{chap:programscopeentries} 
This section describes debugging information entries that
relate to different levels of program scope: compilation,
module, subprogram, and so on. Except for separate type
entries (see Section \refersec{chap:typeunitentries}), 
these entries may be thought of as
ranges of text addresses within the program.

\section{Unit Entries}
\label{chap:unitentries}
A DWARF object file is an object file that contains one or more 
DWARF compilation units, of which there are these kinds:
\addtoindexx{unit|see {compilation unit}} 
\addtoindexx{compilation unit}
\begin{itemize}
\item A \definition{full compilation unit} describes
a complete compilation, possibly in combination with
related partial compilation units and/or type units.

\item A \definition{partial compilation unit} describes
a part of a compilation (generally corresponding to an
imported module) which is imported into one or more 
related full compilation units.

\item A \definition{type unit} is a specialized unit
(similar to a compilation unit) that represents a type 
whose description may be usefully shared by multiple 
other units.
\end{itemize}

\index{conventional compilation unit|see{
       full compilation unit, partial compilation unit, type unit}}

\textit{These first three kinds of compilation unit are
sometimes called \doublequote{conventional} compilation
units--they are kinds of compilation units that were
defined prior to \DWARFVersionV. Conventional compilation units
are part of the same object file as the compiled code and
data (whether relocatable, executable, shared and so on).
The word \doublequote{conventional} is usually
omitted in these names, unless needed to distinguish them
from the similar split compilation units below.}

\needlines{4}
\begin{itemize}
\item A \definition{skeleton compilation unit} represents
the DWARF debugging information for a compilation using a
minimal description that identifies a separate split
compilation unit that provides the remainder (and most) 
of the description.
\end{itemize}

\textit{A skeleton compilation acts as a minimal conventional full
compilation (see above) that identifies and is paired with a 
corresponding split full compilation (as described below). Like
the conventional compilation units, a skeleton compilation unit
is part of the same object file as the compiled code and data.}

\begin{itemize}
\item A 
\definition{split compilation unit} describes
a complete compilation, possibly in combination with
related type compilation units. It corresponds 
to a specific skeleton compilation unit.

\item A \definition{split type unit} is a specialized
compilation unit that represents a type whose description may
be usefully shared by multiple other units.

\end{itemize}

\textit{Split compilation units and split type units may be 
contained in object files separate from those containing the 
program code and data.
These object files are not processed by a linker; thus,
split units do not depend on underlying object file relocations.}

\textit{Either a full compilation unit or a partial compilation 
unit may be logically incorporated into another compilation unit 
using an \addtoindex{imported unit entry}
(see Section \refersec{chap:importedunitentries}).}

\textit{A partial compilation unit is not defined for use
within a split object file.}

\textit{In the remainder of this document, the word 
\doublequote{compilation} in the phrase \doublequote{compilation unit} 
is generally omitted, unless it is deemed needed for clarity 
or emphasis.}

\subsection{Full and Partial Compilation Unit Entries}
\label{chap:fullandpartialcompilationunitentries}
A \addtoindex{full compilation unit}\addtoindexx{compilation unit!full} 
is represented by a debugging information entry with the tag 
\DWTAGcompileunitTARG. 
A \addtoindex{partial compilation unit}\addtoindexx{compilation unit!partial} 
is represented by a debugging information entry with the tag 
\DWTAGpartialunitTARG.

\needlines{6}
In a simple compilation, a single compilation unit with
the tag 
\DWTAGcompileunit{} represents a complete object file
and the tag 
\DWTAGpartialunit{} (as well as tag \DWTAGtypeunit) is not used. 
In a compilation
employing the DWARF space compression and duplicate elimination
techniques from 
Appendix \refersec{app:usingcompilationunits}, 
multiple compilation units using
the tags 
\DWTAGcompileunit{}, 
\DWTAGpartialunit{} and/or 
\DWTAGtypeunit{} 
are used to represent portions of an object file.

\needlines{4}
\textit{A full compilation unit typically represents the text and
data contributed to an executable by a single relocatable
object file. It may be derived from several source files,
including pre-processed header files. 
A \addtoindex{partial compilation unit} typically represents a part 
of the text and data of a relocatable object file, in a manner that 
can potentially be shared with the results of other compilations
to save space. It may be derived from an \doublequote{include file,}
template instantiation, or other implementation-dependent
portion of a compilation. A full compilation unit can also
function in a manner similar to a partial compilation unit
in some cases.
See Appendix \refersec{app:dwarfcompressionandduplicateeliminationinformative}
for discussion of related compression techniques.}

A full or partial 
compilation unit entry owns debugging information
entries that represent all or part of the declarations
made in the corresponding compilation. In the case of a
partial compilation unit, the containing scope of its owned
declarations is indicated by imported unit entries in one
or more other compilation unit entries that refer to that
partial compilation unit (see 
Section \refersec{chap:importedunitentries}).

A full or partial compilation unit entry 
may have the following attributes:
\begin{enumerate}[1. ]
\item Either a \DWATlowpc{} and 
\DWAThighpc{} pair of
\addtoindexx{high PC attribute}
attributes 
\addtoindexx{low PC attribute}
or 
\addtoindexx{ranges attribute}
a 
\DWATranges{} attribute
\addtoindexx{ranges attribute}
whose values encode 
\addtoindexx{discontiguous address ranges|see{non-contiguous address ranges}}
the contiguous or 
non-contiguous address ranges, respectively,
of the machine instructions generated for the compilation
unit (see Section \refersec{chap:codeaddressesandranges}).
  
A \DWATlowpc{} attribute 
may also be specified in combination 
\addtoindexx{ranges attribute}
with 
\DWATranges{} to specify the
\addtoindexx{ranges attribute}
default base address for use in 
\addtoindexx{location list}
location lists (see Section
\refersec{chap:locationlists}) and range lists 
\addtoindexx{range list}
(see Section \refersec{chap:noncontiguousaddressranges}).

\item \hypertarget{chap:DWATnamepathnameofcompilationsource}{}
A \DWATnameDEFN{} attribute \addtoindexx{name attribute}
whose value is a null-terminated string 
containing the full or relative path name 
(relative to the value of the \DWATcompdir{} attribute, 
see below) of the primary
source file from which the compilation unit was derived.

\item \hypertarget{chap:DWATlanguageprogramminglanguagename}{}
A 
\bb
\DWATlanguagenameDEFN{} 
\eb
attribute\addtoindexx{language name attribute}
whose constant value is an integer code 
indicating the source language of the compilation
unit. The set of language names and their meanings are given
in Table \refersec{tab:languagenames}.
\addtoindexx{ANSI-defined language names}\addtoindexx{ISO-defined language names}

\bb
\textit{The most recent list of approved language names  
and applicable versions may be found at}  
\texttt{\url{http://dwarfstd.org/languages-v6.html}}. 
\eb

\vspace{1cm}
\needlines{8}
\begin{centering}
\setlength{\extrarowheight}{0.1cm}
\begin{longtable}{l|P{4.6cm}|P{3.1cm}}
  \caption{Language names} \label{tab:languagenames} \\
  \hline \bfseries Language name & 
		\bfseries Meaning        & 
		\bfseries Version Scheme\linebreak\textit{(See Table \ref{tab:versionencodingschemes})} \\ 
  \hline
\endfirsthead
  \bfseries Language name & \bfseries Meaning & \bfseries Version Scheme \\ 
  \hline
\endhead
  \hline 
  \multicolumn{3}{l}{\emph{Continued on next page}}
\endfoot
  \hline
\endlastfoot
\bb
\DWLNAMEAdaTARG		 		& ISO \Ada 		& \LVESYYYY \\
\DWLNAMEAssemblyTARG 		& \Assembly		& \\
\DWLNAMEBLISSTARG 			& \BLISS 		& \\
\DWLNAMECTARG 				& ISO \C 		& \LVESYYYYMM \\
%\DWLANGCTARG & Non-standardized C, such as K\&R \addtoindexx{C!non-standard} \\*
\DWLNAMECplusplusTARG   	& ISO \Cplusplus& \LVESYYYYMM \\
\DWLNAMECobolTARG 			& ISO \COBOL 	& \LVESYYYY \\
\DWLNAMECPPforOpenCLTARG	& \CPPforOpenCL& \LVESVVMM \\
\DWLNAMECrystalTARG 		& \Crystal 		& \\
\DWLNAMECsharpTARG 			& \Csharp 		& \\
\DWLNAMEDTARG{}		 		& \D 			& \\
\DWLNAMEDylanTARG	 		& \Dylan 		& \\
\DWLNAMEFortranTARG 		& ISO \Fortran 	& \LVESYYYY \\
\DWLNAMEGoTARG{}	 		& \Go 			& \\
\DWLNAMEGLSLTARG{}		 	& \GLSL			& \LVESVVMMPP \\
\DWLNAMEGLSLESTARG{}	 	& \GLSLES		& \LVESVVMMPP \\
\DWLNAMEHaskellTARG{}	 	& \Haskell 		& \\
\DWLNAMEHIPTARG{}			& \HIP			& \\
\DWLNAMEHLSLTARG{}		 	& \HLSL 		& \LVESYYYY \\
\DWLNAMEHyloTARG{}			& \Hylo			& \\
\DWLNAMEJavaTARG{} 			& \Java 		& \\
\DWLNAMEJuliaTARG{}		 	& \Julia 		& \\
\DWLNAMEKotlinTARG 			& \Kotlin 		& \\
\DWLNAMEModulatwoTARG   	& ISO \Modulatwo& \\
\DWLNAMEModulathreeTARG 	& \Modulathree 	& \\
\DWLNAMEMojoTARG			& \Mojo 		& \\
\DWLNAMEMoveTARG			& \Move			& \LVESYYYYMM \\
\DWLNAMEObjCTARG         	& \ObjC 		& \LVESYYYYMM \\
\DWLNAMEObjCplusplusTARG{} 	& \ObjCplusplus & \LVESYYYYMM \\
\DWLNAMEOCamlTARG{}		  	& \OCaml 		& \\
\DWLNAMEOdinTARG{}			& \Odin			& \LVESYYYYMM \\
\DWLNAMEOpenCLCTARG\footnote{This is equivalent to DW\_LANG\_OpenCL in \DWARFVersionV}
							& \OpenCLC 		& \LVESVVMM \\
\DWLNAMEOpenCLCPPTARG		& \OpenCLCPP	& \LVESVVMM \\
\DWLNAMEPFourTARG				& \PFour		& \LVESVVMMPP \\	
\DWLNAMEPascalTARG 			& ISO \Pascal	& \LVESYYYY \\
\DWLNAMEPLITARG{}	 		& ANSI \PLI 	& \\
\DWLNAMEPythonTARG{}	 	& \Python 		& \\
\DWLNAMERenderScriptTARG	& \RenderScript{} Kernel Language & \\
\DWLNAMERubyTARG{}		 	& \Ruby 		& \LVESVVMMPP \\
\DWLNAMERustTARG{}		 	& \Rust 		& \\
\DWLNAMESwiftTARG{} 		& \Swift 		& \LVESVVMM \\*
\DWLNAMESYCLTARG			& \SYCL			& \LVESYYYYRR \\*
\DWLNAMEUPCTARG{} 			& \UPC~(Unified Parallel C) &   
								\index{Unified Parallel C|see{UPC}} \\*
\DWLNAMEZigTARG 			& \Zig 			& 
\eb
\end{longtable}
\end{centering}

\needlines{6}
\bb
\item \hypertarget{chap:DWATlanguageprogramlanguageversion}
A \DWATlanguageversion{} attribute may be specified whose constant value is
an integer value that indicates the version of the source language. This value
is encoded using one of several schemes as shown in Table
\vref{tab:versionencodingschemes}. A value of zero is equivalent to omitting 
this attribute.

\begin{table*}[ht]
\caption{Version Encoding Schemes}
\label{tab:versionencodingschemes}
\centering
\begin{threeparttable}
\begin{tabular}{l|P{10cm}}
\hline
Scheme & Encoding \\
\hline
\addtoindexx{language version encoding schemes}
\LVESYYYY   & Year in which the language definition was released. 
			\\
\LVESYYYYMM~\tnote{\dag}~
			& Year in which the language definition was released times 100
			plus the ordinal number of the month (from 1 to 12). 
			\newline	
			\textit{For example, 202206 represents June of 2022.}
			\\
\LVESYYYYRR	& Year in which the language definition was released times 100 
			plus the revision number.
			\newline
			\textit{For example, 202007 represents version 2020 revision 7 
			while 202011 represents version 2020 revision 11.} 
			\\
\LVESVVMM   & Major version number times 100 plus the minor version number.
			\newline
			\textit{For example, 306 represents version 3.6 while 312
			represents version 3.12.} 
			\\
\LVESVVMMPP	& Major version number times 10,000 plus the minor version 
			number times 100 plus the patch version number.
			\newline			
			\textit{For example, 30607 represents version 3.6.7 while 
			31215 represents version 3.12.15.} 
			\\
\hline
\end{tabular}
\begin{tablenotes}
\item[\dag]	\begin{flushleft}\footnotesize
			For the YYYYMM version scheme, to convert a version number 
			to a specific release, it is good practice to treat the 
			version numbers listed on the 
			\hbox{\texttt{\url{http://dwarfstd.org/languages-v6.html}}}
			website as the maximum version that is interpreted as 
			belonging to a specific release. This way producers can emit 
			version numbers for unreleased upcoming specifications, by  
			using, e.g., the date the compiler was built.
			\end{flushleft}
\end{tablenotes}
\end{threeparttable}
\end{table*}
\eb

\item \hypertarget{chap:DWATstmtlistlinenumberinformationforunit}{}
A \DWATstmtlistDEFN{}\addtoindexx{statement list attribute}
attribute whose value is a 
\addtoindexx{section offset!in statement list attribute}
section offset to the line number information for this compilation
unit.

This information is placed in a separate object file
section from the debugging information entries themselves. The
value of the statement list attribute is the offset in the
\dotdebugline{} section of the first byte of the line number
information for this compilation unit 
(see Section \refersec{chap:linenumberinformation}).

\item A \DWATmacrosDEFN{}\hypertarget{chap:DWATmacrosmacroinformation}{}
attribute 
\addtoindexx{macro information attribute}
whose value is a 
\addtoindexx{section offset!in macro information attribute}
section offset to the macro information for this compilation unit.

This information is placed in a separate object file section
from the debugging information entries themselves. The
value of the macro information attribute is the offset in
the \dotdebugmacro{} section of the first byte of the macro
information for this compilation unit 
(see Section \refersec{chap:macroinformation}).
\db
%\textit{The \DWATmacrosNAME{} attribute is new in \DWARFVersionV, 
%and supersedes the 
%\DWATmacroinfoDEFN{} attribute of earlier DWARF versions.
%\livetarg{chap:DWATmacroinfomacroinformation}{}
%While \DWATmacrosNAME{} and \DWATmacroinfoNAME{} attributes cannot 
%both occur in the same compilation unit, both may be found in the 
%set of units that make up an executable or shared object file. 
%The two attributes have distinct encodings to facilitate such
%coexistence.}

\needlines{6}
\item  \hypertarget{chap:DWATcompdircompilationdirectory}{}
A \DWATcompdirDEFN{} attribute\addtoindexx{compilation directory attribute} 
whose value is a
null-terminated string containing the current working directory
of the compilation command that produced this compilation
unit in whatever form makes sense for the host system.

\bb
If a relative path is used in \DWATcompdirNAME, it will be relative 
to the location of the linked image containing the \DWATcompdirNAME{}
entry.

\textit{In some cases a producer may allow the user to specify a relative
path for \DWATcompdir. There are a few cases in which this is useful,
but in general using a relative path for \DWATcompdir{} is discouraged
as it will not work well in many cases including the following:
different relative paths are used within the same build; the build
system creates multiple linked images in different directories; the
final linked image is moved before being debugged; .o files that need to
be debugged directly.}
\eb

\needlines{8}
\item  \hypertarget{chap:DWATproducercompileridentification}{}
A \DWATproducerDEFN{} attribute\addtoindexx{producer attribute}
whose value is a null-terminated string containing 
information about the compiler that produced the compilation unit. 

\textit{The actual contents of
the string will be specific to each producer, but should
begin with the name of the compiler 
\bb
producer 
\eb
or some other
identifying character sequence that will avoid confusion
with other producer values.}

\needlines{4}
\item  \hypertarget{chap:DWATidentifiercaseidentifiercaserule}{}
A \DWATidentifiercaseDEFN{} attribute 
\addtoindexx{identifier case attribute} whose integer
constant value is a code describing the treatment
of identifiers within this compilation unit. The
set of identifier case codes is given in
Table \refersec{tab:identifiercasecodes}.

\begin{simplenametable}{Identifier case codes}{tab:identifiercasecodes}
\DWIDcasesensitive{}      \\
\DWIDupcase{}             \\
\DWIDdowncase{}           \\
\DWIDcaseinsensitive{}    \\
\end{simplenametable}

\DWIDcasesensitiveTARG{} is the default for all compilation units
that do not have this attribute.  It indicates that names given
as the values of \DWATname{} attributes 
\addtoindexx{name attribute}
in debugging information
entries for the compilation unit reflect the names as they
appear in the source program. 

\textit{A debugger should be sensitive
to the case of \addtoindex{identifier names} when doing identifier 
lookups.}

\needlines{4}
\DWIDupcaseTARG{} means that the 
producer of the debugging
information for this compilation unit converted all source
names to upper case. The values of the name attributes may not
reflect the names as they appear in the source program. 

\textit{A debugger should convert all names to upper case 
when doing lookups.}

\DWIDdowncaseTARG{} means that the producer of the debugging
information for this compilation unit converted all source
names to lower case. The values of the name attributes may not
reflect the names as they appear in the source program. 

\textit{A debugger should convert all names to lower case 
when doing lookups.}

\needlines{4}
\DWIDcaseinsensitiveTARG{} means that the values of the name
attributes reflect the names as they appear in the source
program but that case is not significant.

\textit{A debugger should ignore case when doing lookups.}

\needlines{5}
\item \hypertarget{chap:DWATbasetypesprimitivedatatypesofcompilationunit}{}
A \DWATbasetypesDEFN{} attribute\addtoindexx{base types attribute} 
whose value is a \livelink{chap:classreference}{reference}. 
This attribute points to a debugging information entry
representing another compilation unit.  It may be used
to specify the compilation unit containing the base type
entries used by entries in the current compilation unit
(see Section \refersec{chap:basetypeentries}).

\needlines{6}
\textit{This attribute provides a consumer a way to find the definition
of base types for a compilation unit that does not itself
contain such definitions. This allows a consumer, for example,
to interpret a type conversion to a base type correctly.}

\item \hypertarget{chap:DWATuseUTF8compilationunitusesutf8strings}{}
A \DWATuseUTFeightDEFN{} attribute,
\addtoindexx{use UTF8 attribute}\addtoindexx{UTF-8} 
which is a \livelink{chap:classflag}{flag} whose
presence indicates that all strings (such as the names of
declared entities in the source program, or filenames in the line number table) 
are represented using the UTF-8 representation. 

\needlines{4}
\item%
\hypertarget{chap:DWATmainsubprogramunitcontainingmainorstartingsubprogram}{}
A \DWATmainsubprogramDEFN{} attribute,\addtoindexx{main subprogram attribute} 
which is a \livelink{chap:classflag}{flag},
whose presence indicates that the compilation unit contains a
subprogram that has been identified as the starting
subprogram of the program. If more than one compilation unit contains
this \nolink{flag}, any one of them may contain the starting 
function.

\textit{\addtoindex{Fortran} has a \addtoindex{PROGRAM statement}
which is used
to specify and provide a user-specified name for the main
subroutine of a program. 
\addtoindex{C} uses the name \doublequote{main} to identify
the main subprogram of a program. Some other languages provide
similar or other means to identify the main subprogram of
a program. The \DWATmainsubprogram{} attribute may also be used to
identify such subprograms (see 
Section \refersec{chap:generalsubroutineandentrypointinformation}).}

\item 
\hypertarget{chap:DWATentrypcofcompileunit}{}
\hypertarget{chap:DWATentrypcofpartialunit}{}
A \DWATentrypc{} attribute whose value is the address of the first
\addtoindexx{entry pc attribute}
executable instruction of the unit (see 
Section \refersec{chap:entryaddress}).

\needlines{8}
\item \hypertarget{chap:DWATstroffsetforindirectstringsection}{}
\bb
A \DWATstroffsetsDEFN\addtoindexx{string offset section attribute}
\eb
attribute, whose value is of class \CLASSstroffsetsptr. 
This attribute points to the 
\bb
header
\eb
of the compilation unit's contribution to the
\dotdebugstroffsets{} (or \dotdebugstroffsetsdwo{}) section. 
Indirect string references
(using \DWFORMstrxXNor) within the compilation unit are
interpreted as indices 
\bb
into the array of offsets following that header.
\eb

\needlines{6}
\item \hypertarget{chap:DWATaddrbaseforaddresstable}{}
A \DWATaddrbaseDEFN\addtoindexx{address table base attribute}
attribute, whose value is of class \CLASSaddrptr.
This attribute points to the beginning of the compilation
unit's contribution to the \dotdebugaddr{} section.
Indirect references (using \DWFORMaddrxXN, \DWOPaddrx, 
\DWOPconstx, 
\DWLLEbaseaddressx{}, \DWLLEstartxendx{}, \DWLLEstartxlength{},
\DWRLEbaseaddressx{}, \DWRLEstartxendx{} or \DWRLEstartxlength) 
within the compilation unit are interpreted as indices 
relative to this base.

\needlines{5}
\item \hypertarget{chap:DWATrnglistsbaseforrnglists}{}
A \DWATrnglistsbaseDEFN\addtoindexx{ranges table base attribute}
attribute, whose value is of class \CLASSrnglistsptr.
This attribute points to the beginning of the offsets table 
(immediately following the header) of the compilation
unit's contribution to the \dotdebugrnglists{} section.
References to range lists (using \DWFORMrnglistx)
within the compilation unit are
interpreted relative to this base.

\item \hypertarget{chap:DWATloclistsbaseinlocationlist}{}
A \DWATloclistsbaseDEFN{}\addtoindexx{location table base attribute} 
attribute, whose value is of class \CLASSloclistsptr. 
This attribute points to the beginning of the offsets table 
(immediately following the header) of the compilation unit's 
contribution to the \dotdebugloclists{} section. References to 
\bb
value lists and
\eb
location lists (using \DWFORMloclistx) within the compilation 
unit are interpreted relative to this base.
\end{enumerate}

The  base address of a compilation unit is defined as the
value of the \DWATlowpc{} attribute, if present; otherwise,
it is undefined. If the base address is undefined, then any
DWARF entry or structure defined in terms of the base address
of that compilation unit is not valid.

\needlines{8}
\subsection{Skeleton Compilation Unit Entries}
\label{chap:skeletoncompilationunitentries}
\addtoindexx{compilation unit!skeleton}
\addtoindexx{skeleton compilation unit}
When generating a \splitDWARFobjectfile{} (see 
Section \refersec{datarep:splitdwarfobjectfiles}), the
compilation unit in the \dotdebuginfo{} section is a "skeleton"
compilation unit with the tag 
\DWTAGskeletonunitTARG, which contains a
\DWATdwoname{} attribute as well as a subset of the
attributes of a full or partial compilation unit. In general,
it contains those attributes that are necessary for the consumer
to locate the object file where the split full compilation unit
can be found, and for the consumer to interpret references to
addresses in the program. 

A skeleton compilation unit has no children.

A skeleton compilation unit has 
\bb
the following attributes:
\eb

\begin{enumerate}[1. ]

\item \livetarg{chap:DWATdwonameforunit}{}
A \DWATdwonameDEFN{} attribute
\addtoindexx{split DWARF object file name attribute}
whose value is a
null-terminated string containing the full or relative
path name (relative to the value of the \DWATcompdir{} attribute, 
see below) of the object file that contains the full
compilation unit.

The value in the \HFNdwoid{} field of the unit header for 
this unit is the same as the value in the \HFNdwoid{} field 
of the unit header of the corresponding full compilation 
unit (see Section \refersec{datarep:unitheaders}).
        
\textit{The means of determining a compilation unit ID does 
not need to be similar or related to the means of 
determining a \TUsignature. However, it should 
be suitable for detecting file version skew or other 
kinds of mismatched files and for looking up a full
split unit in a DWARF package file 
(see Section \refersec{datarep:dwarfpackagefiles}).}

\end{enumerate}

A skeleton compilation unit may have additional attributes, 
which are the same as for conventional compilation unit entries 
except as noted, from among the following:
\begin{enumerate}[1. ]
\addtocounter{enumi}{1}
\item Either a \DWATlowpc{} and \DWAThighpc{} pair of attributes
or a \DWATranges{} attribute.
\item A \DWATstmtlist{} attribute.
\item A \DWATcompdir{} attribute.

\needlines{6}
\item A \DWATuseUTFeight{} attribute.

\textit{This attribute applies to strings referred to by the skeleton
compilation unit entry itself, and strings in the associated line
number information.
The representation for strings in the object file referenced 
by the \DWATdwoname{} attribute is determined by the presence 
of a \DWATuseUTFeight{} attribute in the full compilation unit 
(see Section \refersec{chap:splitfullcompilationunitentries}).}

\item A 
\bb
\DWATstroffsets{} 
\eb
attribute, for indirect strings references from the 
\bb
corresponding split full
\eb 
compilation unit.

\item A \DWATaddrbase{} attribute.

\bb
\item A \DWATrnglistsbase{} attribute, for range list entry references 
from the corresponding split full compilation unit.
\eb

\end{enumerate}

All other attributes of a compilation unit entry (described
in Section \refersec{chap:fullandpartialcompilationunitentries})
are placed in the split full compilation unit
(see \refersec{chap:splitfullcompilationunitentries}).
The attributes provided by the skeleton compilation
unit entry do not need to be repeated in the full compilation
unit entry.

\textit{The \DWATaddrbase{},
\bb 
\DWATstroffsets{}, and \DWATrnglistsbase{}
\eb
attributes provide context that may be 
necessary to interpret the contents
of the corresponding \splitDWARFobjectfile.}

\textit{The \DWATbasetypes{} attribute is not defined for a
skeleton compilation unit.}


\subsection{Split Full Compilation Unit Entries}
\label{chap:splitfullcompilationunitentries}
A \definition{split full compilation unit} is represented by a 
debugging information entry with tag \DWTAGcompileunit.
It is very similar to a conventional full compilation unit but
is logically paired with a specific skeleton compilation unit while
being physically separate.

A split full compilation unit 
may have the following attributes, 
which are the same as for conventional compilation unit entries 
except as noted:
\begin{enumerate}[1. ]
\item A \DWATname{} attribute.
\bb
\item A \DWATlanguagename{} attribute.
\item A \DWATlanguageversion{} attribute.
\eb
\item A \DWATmacros{} attribute.\db
\item A \DWATproducer{} attribute.        
\item A \DWATidentifiercase{} attribute.
\item A \DWATmainsubprogram{} attribute.
\item A \DWATentrypc{} attribute.
\item A \DWATuseUTFeight{} attribute.

\end{enumerate}

\textit{The following attributes are not part of a 
split full compilation unit entry but instead are inherited 
(if present) from the corresponding skeleton compilation unit: 
\bbeb
\DWATaddrbase{},
\DWATcompdir,
\DWAThighpc,
\DWATlowpc,
\DWATranges{}
\db
and
\DWATstmtlist.\db 
}

\textit{The \DWATbasetypes{} attribute is not defined for a
split full compilation unit.}

\bb
\textit{Use of \DWFORMsecoffset{} and other equivalent encodings 
(for example, the abbrev offset in a compilation unit header) are resolved 
relative to the beginning of the contribution of the relevant 
section within the dwo or dwp file and cannot be used for sharing 
content between multiple compilation units. \DWFORMsecoffset{} may 
not be used when a reference to content in the skeleton unit is 
required (as the value present in the dwo file could not be relocated 
during linking of the skeleton units), such as for the \CLASSaddrptr{} class.}
\eb

\needlines{6}
\subsection{Type Unit Entries}
\label{chap:typeunitentries}
\addtoindexx{type unit}
\addtoindexx{type unit|see{\textit{also} compilation unit}}
\addtoindexx{compilation unit!\textit{see also} type unit}
An object file may contain any number of separate type
unit entries, each representing a single complete type
definition. 
Each \addtoindex{type unit} must be uniquely identified by
an 8-byte signature, stored as part of the type unit, which
can be used to reference the type definition from debugging
information entries in other compilation units and type units.

Conventional and split type units are identical except for
the sections in which they are represented 
(see Section \refersec{datarep:splitdwarfobjectfiles} for details).
\addtoindexx{conventional type unit}
\addtoindexx{split type unit}
Moreover, the 
\bb
\DWATstroffsets{} 
\eb
attribute (see below) is not 
used in a split type unit.

\needlines{4}
A type unit is represented by a debugging information entry
with the tag \DWTAGtypeunitTARG. 
A \addtoindex{type unit entry} owns debugging
information entries that represent the definition of a single
type, plus additional debugging information entries that may
be necessary to include as part of the definition of the type.

\needlines{4}
A type unit entry may have the following attributes:
\begin{enumerate}[1. ]

\item A
\bb
\DWATlanguagename{} attribute, 
whose
\eb
\addtoindexx{language attribute}
constant value is an integer code indicating the source
language used to define the type. The set of language names
and their meanings are given in Table \refersec{tab:languagenames}.

\bb
\item A \DWATlanguageversion{} attribute, whose constant value is an 
integer code indicating the source language version as described in
Table \refersec{tab:versionencodingschemes}.
\eb

\needlines{4}
\item A \DWATstmtlist{} attribute\addtoindexx{statement list attribute}
whose value of class \CLASSlineptr{} points to the line number 
information for this type unit.

\textit{Because type units do not describe any code, they
do not actually need a line number table, but the line number
headers contain a list of directories and file names that
may be referenced by the \DWATdeclfile{} attribute of the
type or part of its description.} 

\textit{In an object file with a conventional compilation 
unit entry, the type unit entries may refer to (share) the 
line number table used by the compilation unit. In a type 
unit located in a split compilation unit, the 
\DWATstmtlistNAME{} attribute refers to a \doublequote{specialized}
line number table in the \dotdebuglinedwo{} section, which
contains only the list of directories and file names.}

\textit{All type unit entries in a \splitDWARFobjectfile{} may 
(but are not required to) refer to the same 
\addtoindex{specialized line number table}.}

\item A \DWATuseUTFeight{} attribute, which is a flag
whose presence indicates that all strings referred to by this type
unit entry, its children, and its associated 
\addtoindex{specialized line number table}, 
are represented using the UTF-8 representation.

\needlines{4}
\item A 
\bb
\DWATstroffsets\addtoindexx{string offsets base attribute}
\eb
attribute, whose value is of class \CLASSstroffsetsptr. 
This attribute points to the 
\bb
header
\eb 
of the type unit's contribution to
the \dotdebugstroffsets{} section. Indirect string references
(using \DWFORMstrxXNor) within the type unit are interpreted
as indices 
\bb
into the array of offsets following that header.
\eb

\end{enumerate}

\needlines{6}
A \addtoindex{type unit} entry for a given type T owns a debugging
information entry that represents a defining declaration
of type T. If the type is nested within enclosing types or
namespaces, the debugging information entry for T is nested
within debugging information entries describing its containers;
otherwise, T is a direct child of the type unit entry.

\needlines{4}
A type unit entry may also own additional debugging information
entries that represent declarations of additional types that
are referenced by type T and have not themselves been placed in
separate type units. Like T, if an additional type U is nested
within enclosing types or namespaces, the debugging information
entry for U is nested within entries describing its containers;
otherwise, U is a direct child of the type unit entry.

The containing entries for types T and U are declarations,
and the outermost containing entry for any given type T or
U is a direct child of the type unit entry. The containing
entries may be shared among the additional types and between
T and the additional types.

\textit{Examples of these kinds of relationships are found in
Section \refersec{app:signaturecomputationexample} and
Section \refersec{app:declarationscompletingnondefiningdeclarations}.}

\needlines{4}
\textit{Types are not required to be placed in type units. In general,
only large types such as structure, class, enumeration, and
union types included from header files should be considered
for separate type units. Base types and other small types
are not usually worth the overhead of placement in separate
type units. Types that are unlikely to be replicated, such
as those defined in the main source file, are also better
left in the main compilation unit.}

\section{Module, Namespace and Importing Entries}
\textit{Modules and namespaces provide a means to collect related
entities into a single entity and to manage the names of
those entities.}

\needlines{8}
\subsection{Module Entries}
\label{chap:moduleentries}
\textit{Several languages have the concept of a \doublequote{module.}
\addtoindexx{Modula-2}
A Modula\dash 2 definition module 
\addtoindexx{Modula-2!definition module}
may be represented by a module
entry containing a 
\addtoindex{declaration attribute}
(\DWATdeclaration). A
\addtoindex{Fortran 90} module 
\addtoindexx{Fortran!module (Fortran 90)}
may also be represented by a module entry
(but no declaration attribute is warranted because \addtoindex{Fortran}
has no concept of a corresponding module body).}

\needlines{4}
A module is represented by a debugging information entry
with the 
tag \DWTAGmoduleTARG.  
Module entries may own other
debugging information entries describing program entities
whose declaration scopes end at the end of the module itself.

If the module has a name, the module entry has a 
\DWATname{} attribute 
\addtoindexx{name attribute}
whose value is a null\dash terminated string containing
the module name.

\needlines{4}
The \addtoindex{module entry} may have either a 
\DWATlowpc{} and
\DWAThighpc{} 
pair 
\addtoindexx{high PC attribute}
of 
\addtoindexx{low PC attribute}
attributes or a 
\DWATranges{} attribute
\addtoindexx{ranges attribute}
whose values encode the contiguous or non-contiguous address
ranges, respectively, of the machine instructions generated for
the module initialization
code\hypertarget{chap:DWATentrypcentryaddressofmoduleinitialization}{} 
(see Section \refersec{chap:codeaddressesandranges}). 
It may also have a
\addtoindexx{entry PC attribute!for module initialization}
\DWATentrypc{} attribute whose value is the address of
the first executable instruction of that initialization code
(see Section \refersec{chap:entryaddress}).

\needlines{5}
If\hypertarget{chap:DWATprioritymodulepriority}{}
the module has been assigned a priority, it may have a
\addtoindexx{priority attribute}
\DWATpriorityDEFN{} attribute. 
The value of this attribute is a
reference to another debugging information entry describing
a variable with a constant value. The value of this variable
is the actual constant value of the module\textquoteright{s} priority,
represented as it would be on the target architecture.

\subsection{Namespace Entries}
\label{chap:namespaceentries}
\textit{\addtoindex{C++} has the notion of a namespace, which provides a way to
\addtoindexx{namespace (C++)}
implement name hiding, so that names of unrelated things
do not accidentally clash in the 
\addtoindex{global namespace} when an
application is linked together.}

A namespace is represented by a debugging information entry
with the tag \DWTAGnamespaceTARG. A namespace extension 
is\hypertarget{chap:DWATextensionpreviousnamespaceextensionororiginalnamespace}{}
represented by a \DWTAGnamespaceNAME{} entry with a 
\DWATextensionDEFN{}\addtoindexx{extension attribute}
attribute referring to the previous extension, or if there
is no previous extension, to the original 
\DWTAGnamespaceNAME{}
entry. A namespace extension entry does not need to duplicate
information in a previous extension entry of the namespace
nor need it duplicate information in the original namespace
entry. (Thus, for a namespace with a name, 
a \DWATname{} attribute 
\addtoindexx{name attribute}
need only be attached directly to the original
\DWTAGnamespaceNAME{} entry.)

\needlines{4}
Namespace and namespace extension entries may own 
\addtoindexx{namespace extension entry}
other
\addtoindexx{namespace declaration entry}
debugging information entries describing program entities
whose declarations occur in the namespace.

A namespace may have a 
\DWATexportsymbolsDEFN{}\livetarg{chap:DWATexportsymbolsofnamespace}{}
attribute\addtoindexx{export symbols attribute}
\addtoindexx{inline namespace|see{\textit{also} export symbols attribute}}
which is a \CLASSflag{} which
indicates that all member names defined within the 
namespace may be referenced as if they were defined within 
the containing namespace. 

\textit{This may be used to describe an \addtoindex{inline namespace} in \addtoindex{C++}}.

If a type, variable, or function declared in a namespace is
defined outside of the body of the namespace declaration,
that type, variable, or function definition entry has a
\DWATspecification{} attribute 
\addtoindexx{specification attribute}
whose value is a \livelink{chap:classreference}{reference} to the
debugging information entry representing the declaration of
the type, variable or function. Type, variable, or function
entries with a 
\DWATspecification{} attribute 
\addtoindexx{specification attribute}
do not need
to duplicate information provided by the declaration entry
referenced by the specification attribute.

\textit{The \addtoindex{C++} \addtoindex{global namespace}
(the 
\addtoindexx{global namespace|see{namespace (C++), global}}
namespace 
\addtoindexx{namespace (C++)!global}
referred to by
\texttt{::f}, for example) is not explicitly represented in
DWARF with a namespace entry (thus mirroring the situation
in \addtoindex{C++} source).  
Global items may be simply declared with no
reference to a namespace.}

\textit{The \addtoindex{C++} 
compilation unit specific \doublequote{unnamed namespace} may
\addtoindexx{namespace (C++)!unnamed}
\addtoindexx{unnamed namespace|see {namespace (C++), unnamed}}
be represented by a namespace entry with no name attribute in
the original namespace declaration entry (and therefore no name
attribute in any namespace extension entry of this namespace).
C++ states that declarations in the unnamed namespace are
implicitly available in the containing scope; a producer
should make this effect explicit with the \DWATexportsymbols{}
attribute, or by using a \DWTAGimportedmodule{} that is a
sibling of the namespace entry and references it.}

\textit{A compiler emitting namespace information may choose to
explicitly represent namespace extensions, or to represent the
final namespace declaration of a compilation unit; this is a
quality-of-implementation issue and no specific requirements
are given here. If only the final namespace is represented,
\addtoindexx{namespace (C++)!using declaration}
it is impossible for a debugger to interpret using declaration
references in exactly the manner defined by the 
\addtoindex{C++} language.}

\textit{For \addtoindex{C++} namespace examples, 
see Appendix \refersec{app:namespaceexamples}.}


\needlines{10}
\subsection{Imported (or Renamed) Declaration Entries} 
\label{chap:importedorrenameddeclarationentries}

\textit{Some languages support the concept of importing into or 
making accessible in a given unit certain declarations that occur
in a different module or scope. An imported declaration may 
sometimes be given another name.}

\needlines{6}
An imported declaration is represented by one or
\addtoindexx{imported declaration entry}
more debugging information entries with the 
tag \DWTAGimporteddeclarationTARG. 
When\hypertarget{chap:DWATimportimporteddeclaration}{}
an overloaded entity is imported, there is one imported 
declaration entry for each overloading. 
Each imported declaration entry has a
\DWATimportDEFN{} attribute,\addtoindexx{import attribute}
whose value is a \livelink{chap:classreference}{reference} to the
debugging information entry representing the declaration that
is being imported.

\needlines{4}
An imported declaration may also have a \DWATname{}
attribute\addtoindexx{name attribute}
whose value is a null-terminated string containing the
name by which the
imported entity is to be known in the context of the imported
declaration entry (which may be different than the name of
the entity being imported). If no name is present, then the
name by which the entity is to be known is the same as the
name of the entity being imported.

An imported declaration entry with a name attribute may be
used as a general means to rename or provide an alias for
\addtoindexx{alias declaration|see{imported declaration entry}}
an entity, regardless of the context in which the importing
declaration or the imported entity occurs.

\textit{A \addtoindex{C++}
namespace alias\hypertarget{chap:DWATimportnamespacealias}{} 
may be represented by an imported declaration entry 
\addtoindexx{namespace (C++)!alias}
with a name attribute whose value is
a null-terminated string containing the alias name
and a \DWATimportDEFN{} attribute 
whose value is a \livelink{chap:classreference}{reference} to the 
applicable original namespace or namespace extension entry.}

\textit{A \addtoindex{C++} using declaration may be represented 
by one or more
imported\hypertarget{chap:DWATimportnamespaceusingdeclaration}{} 
\addtoindexx{namespace (C++)!using declaration}
declaration entries.  When the using declaration
refers to an overloaded function, there is one imported
declaration entry corresponding to each overloading. Each
imported declaration entry has no name attribute but it does
have a \DWATimportDEFN{} attribute that refers to the entry for the
entity being imported. (\addtoindex{C++} 
provides no means to \doublequote{rename}
an imported entity, other than a namespace).}


\textit{A \addtoindex{Fortran} use statement 
\addtoindexx{Fortran!use statement}
\addtoindexx{use statement|see {Fortran, use statement}}
with an \doublequote{only list} may be
represented by a series of imported declaration entries,
one (or more) for each entity that is imported. An entity
\addtoindexx{renamed declaration|see{imported declaration entry}}
that is renamed in the importing context may be represented
by an imported declaration entry with a name attribute that
specifies the new local name.
}

\needlines{8}
\subsection{Imported Module Entries}
\label{chap:importedmoduleentries}

\textit{Some languages support the concept of importing into or making
accessible in a given unit all of the declarations contained
within a separate module or namespace.
}

\needlines{6}
An imported module declaration is represented by a debugging
information entry with 
\addtoindexx{imported module attribute}
the 
\addtoindexx{imported module entry}
tag \DWTAGimportedmoduleTARG.
An
imported module entry contains a 
\DWATimport{} attribute
\addtoindexx{import attribute}
whose value is a \livelink{chap:classreference}{reference} 
to the module or namespace entry
containing the definition and/or declaration entries for
the entities that are to be imported into the context of the
imported module entry.

\needlines{6}
An imported module declaration may own a set of imported
declaration entries, each of which refers to an entry in the
module whose corresponding entity is to be known in the context
of the imported module declaration by a name other than its
name in that module. Any entity in the module that is not
renamed in this way is known in the context of the imported
module entry by the same name as it is declared in the module.

\textit{A \addtoindex{C++} using directive
\addtoindexx{namespace (C++)!using directive}
\addtoindexx{using directive|see {namespace (C++), using directive}} 
may be represented by an imported 
module\hypertarget{chap:DWATimportnamespaceusingdirective}{}
entry, with a \DWATimportDEFN{} attribute referring to the namespace
entry of the appropriate extension of the namespace (which
might be the original namespace entry) and no owned entries.
}

\textit{A \addtoindex{Fortran} use statement 
\addtoindexx{Fortran!use statement}
with a \doublequote{rename list} may be
represented by an imported module entry with an import
attribute referring to the module and owned entries
corresponding to those entities that are renamed as part of
being imported.
}

\textit{A \addtoindex{Fortran} use statement
\addtoindexx{Fortran!use statement}
with neither a \doublequote{rename list} nor
an \doublequote{only list} may be represented by an imported module
entry with an import attribute referring to the module and
no owned child entries.
}

\textit{A use statement with an \doublequote{only list} is represented by a
series of individual imported declaration entries as described
in Section \refersec{chap:importedorrenameddeclarationentries}.
}

\needlines{12}
\textit{A \addtoindex{Fortran} use statement for an entity in a module that is
\addtoindexx{Fortran!use statement}
itself imported by a use statement without an explicit mention
may be represented by an imported declaration entry that refers
to the original debugging information entry. For example, given}
\par % Needed to end paragraph before listing so that it gets a line number
\vspace{2mm}
\begin{nlnlisting}
module A
integer X, Y, Z
end module

module B
use A
end module

module C
use B, only Q => X
end module

\end{nlnlisting}
\textit{the imported declaration entry for Q within module C refers
directly to the variable declaration entry for X in module A
because there is no explicit representation for X in module B.
}

\textit{A similar situation arises for a \addtoindex{C++} using declaration
\addtoindexx{namespace (C++)!using declaration}
\addtoindexx{using declaration|see {namespace (C++), using declaration}}
that imports an entity in terms of a namespace alias. See 
Appendix  \refersec{app:namespaceexamples}
for an example.
}

\needlines{8}
\subsection{Imported Unit Entries}
\label{chap:importedunitentries}
\hypertarget{chap:DWATimportimportedunit}{}
The place where a normal or partial compilation unit is imported is
represented by a debugging information entry with the 
\addtoindexx{imported unit entry}
tag \DWTAGimportedunitTARG. 
An imported unit entry contains a
\DWATimportDEFN{} attribute\addtoindexx{import attribute} 
whose value is a \livelink{chap:classreference}{reference} to the
normal or partial compilation unit 
\bb
entry
\eb
whose declarations logically
belong at the place of the imported unit entry.

\textit{An imported unit entry does not necessarily correspond to
any entity or construct in the source program. It is merely
\doublequote{glue} used to relate a partial unit, or a compilation
unit used as a partial unit, to a place in some other
compilation unit.}

\section{Subroutine and Entry Point Entries}
\label{chap:subroutineandentrypointentries}

The following tags exist to describe 
debugging information entries 
\addtoindexx{function entry|see{subroutine entry}}
for 
\addtoindexx{subroutine entry}
subroutines 
\addtoindexx{subprogram entry}
and entry
% FIXME: is entry point entry the right index 'entry'?
\addtoindexx{entry point entry}
points:

\nolinenumbers
\begin{tabular}{lp{9.0cm}}
\DWTAGsubprogramTARG{} & A subroutine or function \\
\DWTAGinlinedsubroutine{} & A particular inlined 
\addtoindexx{inlined subprogram entry}
instance of a subroutine or function \\
\DWTAGentrypointTARG{} & An alternate entry point \\
\end{tabular}
\par\condlinenumbers

\needlines{6}
\subsection{General Subroutine and Entry Point Information}
\label{chap:generalsubroutineandentrypointinformation}
The subroutine or entry point entry has a \DWATname{} 
attribute whose value is a null-terminated string containing the 
subroutine or entry point name.
It may also have a \DWATlinkagename{} attribute as
described in Section \refersec{chap:linkagenames}.

If the name of the subroutine described by an entry with the
\addtoindexx{subprogram entry}
tag \DWTAGsubprogram{}\hypertarget{chap:DWATexternalexternalsubroutine}{}
is visible outside of its containing
compilation unit, that entry has a 
\DWATexternalDEFN{} attribute,\addtoindexx{external attribute}
which is a \livelink{chap:classflag}{flag}.

\textit{Additional attributes for functions that are members of a
class or structure are described in 
Section \refersec{chap:memberfunctionentries}.
}

A\hypertarget{chap:DWATmainsubprogrammainorstartingsubprogram}{}
subroutine entry may contain a 
\DWATmainsubprogramDEFN{} attribute 
\addtoindexx{main subprogram attribute}
which is 
a \CLASSflag{} whose presence indicates that the
subroutine has been identified as the starting function of
the program.  If more than one subprogram contains this 
\nolink{flag},
any one of them may be the starting subroutine of the program.

\textit{See also Section \refersec{chap:unitentries}) regarding the
related use of this attribute to indicate that a compilation
unit contains the main subroutine of a program.}

\subsubsection{Calling Convention Information}
\hypertarget{chap:DWATcallingconventionforsubprograms}{}
A subroutine entry may contain a 
\DWATcallingconventionDEFN{}
\addtoindexx{calling convention attribute!for subprogram}
attribute, whose value is an 
\livelink{chap:classconstant}{integer constant}. The set of
\addtoindexi{calling convention codes for subroutines}{calling convention codes!for subroutines}
is given in Table \refersec{tab:callingconventioncodesforsubroutines}.

\begin{simplenametable}[1.4in]{Calling convention codes for subroutines}{tab:callingconventioncodesforsubroutines}
\DWCCnormal        \\
\DWCCprogram       \\
\DWCCnocall        \\
\end{simplenametable}

If this attribute is not present, or its value is the constant
\DWCCnormalTARG, then the subroutine may be safely called by
obeying the \doublequote{standard} calling conventions of the target
architecture. If the value of the calling convention attribute
is the constant \DWCCnocallTARG, the subroutine does not obey
standard calling conventions, and it may not be safe for the
debugger to call this subroutine.

\textit{Note that \DWCCnormal{} is also used as a calling convention 
code for certain types 
(see Table \refersec{tab:callingconventioncodesfortypes}).}

If the semantics of the language of the compilation unit
containing the subroutine entry distinguishes between ordinary
subroutines and subroutines that can serve as the \doublequote{main
program,} that is, subroutines that cannot be called
directly according to the ordinary calling conventions,
then the debugging information entry for such a subroutine
may have a calling convention attribute whose value is the
constant \DWCCprogramTARG.

\textit{A common debugger feature is to allow the debugger user to call
a subroutine within the subject program. In certain cases,
however, the generated code for a subroutine will not obey
the standard calling conventions for the target architecture
and will therefore not be safe to call from within a debugger.}

\textit{The \DWCCprogram{} 
value is intended to support \addtoindex{Fortran} main
\addtoindexx{Fortran!main program}
programs which in some implementations may not be callable
or which must be invoked in a special way. It is not intended
as a way of finding the entry address for the program.}


\subsubsection{Miscellaneous Subprogram Properties}
\textit{In \addtoindex{C}
there is a difference between the types of functions
declared using function prototype style declarations and
those declared using non-prototype declarations.}

A subroutine entry declared with a function prototype style
declaration may have a
\addtoindexx{prototyped attribute}
\DWATprototypedDEFN{} attribute, which is
a \CLASSflag. 
The attribute indicates whether a subroutine entry point corresponds
to a function declaration that includes parameter prototype information.

A subprogram entry may have 
a\hypertarget{chap:DWATelementalelementalpropertyofasubroutine}{}
\DWATelementalDEFN{} attribute,\addtoindexx{elemental attribute} 
which is a \livelink{chap:classflag}{flag}. 
The attribute indicates whether the subroutine
or entry point was declared with the \doublequote{elemental} keyword
or property.

A\hypertarget{chap:DWATpurepurepropertyofasubroutine}{}
subprogram entry may have a
\addtoindexx{pure attribute}
\DWATpureDEFN{} attribute, which is
a \livelink{chap:classflag}{flag}. 
The attribute indicates whether the subroutine was
declared with the \doublequote{pure} keyword or property.

A\hypertarget{chap:DWATrecursiverecursivepropertyofasubroutine}{}
subprogram entry may have a 
\addtoindexx{recursive attribute}
\DWATrecursiveDEFN{} attribute, which
is a \livelink{chap:classflag}{flag}. 
The attribute indicates whether the subroutine
or entry point was declared with the \doublequote{recursive} keyword
or property.

A subprogram entry may have a 
\DWATnoreturnDEFN{}
\livetargi{chap:DWATnoreturnofsubprogram}{attribute}{noreturn attribute}, 
which is a \CLASSflag. The attribute 
indicates whether the subprogram was declared with the \doublequote{noreturn} keyword or property 
indicating that the subprogram can be called, but will never return to its caller.

\needlines{4}
\textit{The \addtoindex{Fortran} 
language allows the keywords \texttt{elemental}, \texttt{pure}
and \texttt{recursive} to be included as part of the declaration of
a subroutine; these attributes reflect that usage. These
attributes are not relevant for languages that do not support
similar keywords or syntax. In particular, the \DWATrecursiveNAME{}
attribute is neither needed nor appropriate in languages such
as \addtoindex{C} where functions support recursion by default.}


\subsubsection{Call Site-Related Attributes}
\textit{While subprogram attributes in the previous section provide
information about the subprogram and its entry point(s) as a whole,
the following attributes provide summary information about the calls
that occur within a subprogram.}

A subroutine entry may have \DWATcallalltailcalls, 
\DWATcallallcalls{} and/or \DWATcallallsourcecalls{} 
attributes, each of which is a \CLASSflag.
\addtoindexx{call site summary information}
\addtoindexx{subroutine call site summary attributes}
These flags indicate the completeness of the call site 
information provided by call site entries (see 
Section \refersec{chap:callsiteentries}) within the subprogram.

The \DWATcallalltailcallsDEFN{}
\livetargi{chap:DWATcallalltailcallsofasubprogram}{attribute}{all tail calls summary attribute} 
indicates that every tail call 
that occurs in the code for the subprogram is described by a 
\DWTAGcallsite{} entry. 
(There may or may not be other non-tail calls to some of the same 
target subprograms.)

The \DWATcallallcallsDEFN{}
\livetargi{chap:DWATcallallcallsofasubprogram}{attribute}{all calls summary attribute} 
indicates that every non-inlined call
(either a tail call or a normal call) that occurs in the code for the subprogram
is described by a \DWTAGcallsite{} entry.

The \DWATcallallsourcecallsDEFN{}
\livetargi{chap:DWATcallallsourcecallsofasubprogram}{attribute}{all source calls summary attribute} 
indicates that every call that occurs in the
code for the subprogram, including every call inlined into it, is described by either a 
\DWTAGcallsite{} entry or a \DWTAGinlinedsubroutine{} entry; further, any call
that is optimized out is nonetheless also described using a \DWTAGcallsite{} entry 
that has neither a \DWATcallpc{} nor \DWATcallreturnpc{} attribute.

\textit{The \DWATcallallsourcecallsNAME{} attribute is intended for debugging 
information format consumers that analyze call graphs.}

\needlines{4}
If the the \DWATcallallsourcecalls{} attribute is present then the 
\DWATcallallcalls{} and \DWATcallalltailcalls{} attributes are 
also implicitly present. Similarly, if the 
\DWATcallallcalls{} attribute is present then the \DWATcallalltailcalls{} 
attribute is implicitly present.

\needlines{5}
\subsection{Subroutine and Entry Point Return Types}
\label{chap:subroutineandentrypointreturntypes}

If\hypertarget{chap:DWATtypetypeofsubroutinereturn}{}
the subroutine or entry point 
\addtoindexx{return type of subroutine}
is a function that returns a
value, then its debugging information entry has 
\addtoindexx{type attribute}
a \DWATtypeDEFN{} attribute 
to denote the type returned by that function.

\textit{Debugging information entries for 
\addtoindex{C} void functions should
not have an attribute for the return type.  }

\textit{Debugging information entries for declarations of \addtoindex{C++} 
member functions with an 
\autoreturntype{} specifier should use an unspecified type entry (see 
Section \refersec{chap:unspecifiedtypeentries}). 
The debugging information entry for the corresponding definition
should provide the deduced return type.  This practice causes the description of
the containing class to be consistent across compilation units, allowing the class
declaration to be placed into a separate type unit if desired.}


\subsection{Subroutine and Entry Point Locations}
\label{chap:subroutineandentrypointlocations}

A subroutine entry may have either a \DWATlowpc{} and
\DWAThighpc{} pair of attributes or a \DWATranges{} attribute
\addtoindexx{ranges attribute}
whose 
\addtoindexx{high PC attribute}
values 
\addtoindexx{low PC attribute}
encode the contiguous or non-contiguous address
ranges, respectively, of the machine instructions generated
for the subroutine (see 
Section \refersec{chap:codeaddressesandranges}).

A\hypertarget{chap:DWATentrypcentryaddressofsubprogram}{}
subroutine entry may also have a
\addtoindexx{entry PC attribute!for subroutine}
\DWATentrypc{} attribute
whose value is the address of the first executable instruction
of the subroutine (see 
Section \refersec{chap:entryaddress}).

An entry point has a \DWATlowpc{} attribute whose value is the
relocated address of the first machine instruction generated
for the entry point.

%\bbpareb
%\textit{While the \DWATentrypc{} attribute 
%\addtoindexx{entry pc attribute!for subroutine}
%might also seem appropriate for this purpose, historically the 
%\DWATlowpc{} attribute was used before the 
%\DWATentrypc{} was introduced (in
%\addtoindex{DWARF Version 3}). 
%There is insufficient reason to change this.}

Subroutines and entry points may also have
\bb
a \DWATaddressclassDEFN{}\addtoindexx{address class attribute}
attribute, if 
\eb
appropriate, to specify
\db
the addressing mode to be used in calling that subroutine.

A subroutine entry representing a subroutine declaration
that is not also a definition does not have code address or
range attributes.


\subsection{Declarations Owned by Subroutines and Entry Points} 
\label{chap:declarationsownedbysubroutinesandentrypoints}
\addtoindexx{subroutine formal parameters}
The declarations enclosed by a subroutine or entry point are
represented by debugging information entries that are owned
by the subroutine or entry point entry. Entries representing
\addtoindexx{formal parameter}
the formal parameters of the subroutine or entry point appear
in the same order as the corresponding declarations in the
source program.

\needlines{5}
\textit{There is no ordering requirement for entries for declarations
other than formal parameters. The formal parameter
entries may be interspersed with other entries used by formal
parameter entries, such as type entries.}

The unspecified (sometimes called \doublequote{varying}) 
parameters of a subroutine parameter list are
represented by a debugging information 
entry\addtoindexx{unspecified parameters entry}
with the tag \DWTAGunspecifiedparametersTARG.

\needlines{4}
The entry for a subroutine that includes a
\addtoindex{Fortran}
\addtoindexx{Fortran!common block}
\livelink{chap:fortrancommonblock}{common} 
\livelink{chap:commonblockentry}{block}
\addtoindexx{common block|see{Fortran common block}}
has a child entry with the 
tag \DWTAGcommoninclusionTARG. 
The\hypertarget{chap:commonreferencecommonblockusage}{}
common inclusion entry has a 
\DWATcommonreferenceDEFN{} attribute
\addtoindexx{common block reference attribute}
whose value is a \livelink{chap:classreference}{reference} 
to the debugging information entry
for the common \nolink{block} being included 
(see Section \refersec{chap:commonblockentries}).

\subsection{Low-Level Information}
\label{chap:lowlevelinformation}

\bbeb
\subsubsection{Return Address Location}
\label{chap:returnaddresslocation}
A\hypertarget{chap:DWATreturnaddrsubroutinereturnaddresssavelocation}{}
subroutine or entry point entry may have a 
\addtoindexx{return address attribute}
\DWATreturnaddrDEFN{}
attribute, whose value is a location description. The location
specified is the place where the return address for the
subroutine or entry point is stored.

\bbeb
\subsubsection{Frame Base}
\label{chap:framebase}
A\hypertarget{chap:DWATframebasesubroutineframebaseaddress}{}
subroutine or entry point entry may also have a
\addtoindexx{frame base attribute}
\DWATframebaseDEFN{} attribute, whose value is a location
description that describes the \doublequote{frame base} for the
subroutine or entry point. If the location description is
a simple register location description, the given register
contains the frame base address. If the location description is
a DWARF expression, the result of evaluating that expression
is the frame base address. Finally, for a 
\addtoindex{location list},
this interpretation applies to each location description
contained in the list of \addtoindex{location list} entries.

\textit{The use of one of the \DWOPregn{} 
operations in this context is equivalent to using 
\DWOPbregn(0) but more
compact. However, these are not equivalent in general.}

\needlines{5}
\textit{The frame base for a subprogram is typically an address
relative to the first unit of storage allocated for the
subprogram\textquoteright s stack frame. The \DWATframebase{} attribute
can be used in several ways:}
\begin{enumerate}[1. ]
\item \textit{In subprograms that need 
\addtoindexx{location list}
location lists to locate local
variables, the \DWATframebase{} can hold the needed location
list, while all variables\textquoteright\  location descriptions can be
simpler ones involving the frame base.}

\item \textit{It can be used in resolving \doublequote{up\dash level} addressing
within nested routines. 
(See also \DWATstaticlink, below)}
\end{enumerate}

\needlines{5}
\bbeb
\subsubsection{Nested subroutines and up-level references}
\label{nestedsubroutinesanduplevelreferences}
\textit{Some languages support nested subroutines. In such languages,
it is possible to reference the local variables of an
outer subroutine from within an inner subroutine. The
\DWATstaticlink{} and \DWATframebase{} attributes allow
debuggers to support this same kind of referencing.}

If\hypertarget{chap:DWATstaticlinklocationofuplevelframe}{}
a subroutine or entry point is nested, it may have a
\addtoindexx{address!uplevel|see {static link attribute}}
\addtoindexx{uplevel address|see {static link attribute}}
\DWATstaticlinkDEFN{} attribute, whose value is a location
description that computes the frame base of the relevant
instance of the subroutine that immediately encloses the
subroutine or entry point.

In the context of supporting nested subroutines, the
\DWATframebase{} attribute value obeys the following constraints:

\begin{enumerate}[1. ]
\item 
It computes a value that does not change during the
life of the subprogram, and

\item The computed value is unique among instances of
the same subroutine. 

\textit{For typical \DWATframebase{} use, this
means that a recursive subroutine\textquoteright s stack frame must have
non-zero size.}

\end{enumerate}

\textit{If a debugger is attempting to resolve an up\dash level reference
to a variable, it uses the nesting structure of DWARF to
determine which subroutine is the lexical parent and the
\DWATstaticlink{} value to identify the appropriate active
frame of the parent. It can then attempt to find the reference
within the context of the parent.}

\bbeb\subsubsection{Lanes in SIMD Vectorization}
\label{chap:lanesinsimdvectorization}
\addtoindexx{SIMD Vectorization}
\bb
\textit{SIMD instructions process multiple data elements in 
one instruction. The number of data elements that is processed 
with one instruction is typically referred to as the SIMD width. 
Each individual data element is typically referred to as SIMD lane.}

\textit{When generating code for a SIMD architecture, compilers 
may need to implicitly widen the source code to match the SIMD 
width of the instruction set they are using. Source variables 
are widened into a vector of variables, with one instance per 
SIMD lane.}

A subroutine that is implicitly vectorized may have a 
\DWATnumlanes{} attribute whose value describes the implicit 
vectorization factor and the corresponding number of lanes in 
the generated code. The value of this attribute is determined as 
described in Section \refersec{chap:staticanddynamicvaluesofattributes}.

To refer to individual lanes in such vectorized code, for example 
to describe the location of widened source variables, producers 
may use the \DWOPpushlane{} operation (see Section 
\refersec{chap:stackoperations}) to have the consumer supply 
the current focus lane for which to evaluate the expression. 
The pushed lane index must be an unsigned integer value between 
zero (inclusive) and the value of \DWATnumlanes{} (exclusive) 
at the current location.

If the attribute is omitted, its value is defined by the ABI.

\textit{If the source code had already been vectorized and is 
not further widened by the compiler, the value should be one.}

\textit{This value does not only apply to vector instructions. 
If a loop or function has been widened, the entire loop or 
function body shall be annotated with the vectorization factor.}
\eb

\needlines{8}
\subsection{Types Thrown by Exceptions}
\label{chap:typesthrownbyexceptions}

\textit{In \addtoindex{C++} a subroutine may declare a set of types which
it may validly throw.}

If a subroutine explicitly declares that it may throw
\addtoindexx{exception thrown|see{thrown type entry}}
an 
\addtoindexx{thrown exception|see{thrown type entry}}
exception of one or more types, each such type is
represented by a debugging information entry with 
\addtoindexx{thrown type entry}
the tag
\DWTAGthrowntypeTARG.  
Each such entry is a child of the entry
representing the subroutine that may throw this type. Each
thrown type entry contains 
\addtoindexx{type attribute}
a \DWATtype{} attribute, whose
value is a \livelink{chap:classreference}{reference} 
to an entry describing the type of the
exception that may be thrown.

\subsection{Function Template Instantiations}
\label{chap:functiontemplateinstantiations}

\textit{In \addtoindex{C++}, a function template is a generic definition of
a function that is instantiated differently for calls with
values of different types. DWARF does not represent the generic
template definition, but does represent each instantiation.}

\needlines{4}
A \addtoindex{function template instantiation}\addtoindexx{template instantiation!function} 
is represented by a debugging information entry with the 
\addtoindexx{subprogram entry!use for template instantiation}
tag \DWTAGsubprogram. 
With the following
exceptions, such an entry will contain the same attributes and
will have the same types of child entries as would an entry
for a subroutine defined explicitly using the instantiation
types and values. The exceptions are:

\begin{enumerate}[1. ]
\item Template parameters are described and referenced as specified in
Section \refersec{chap:templateparameters}.

\needlines{4}
\item If the compiler has generated a separate compilation unit
to hold the template instantiation and that compilation unit
has a different name from the compilation unit containing
the template definition, the name attribute for the debugging
information entry representing that compilation unit is empty
or omitted.

\item If the subprogram entry representing the template
instantiation or any of its child entries contain declaration
coordinate attributes, those attributes refer to the source
for the template definition, not to any source generated
artificially by the compiler for this instantiation.
\end{enumerate}


\needlines{8}
\subsection{Inlinable and Inlined Subroutines}
\label{chap:inlinedsubroutines}
A declaration or a definition of an inlinable subroutine
is represented by a debugging information entry with the
tag \DWTAGsubprogram.
The entry for a subroutine 
\addtoindexx{subprogram entry!use in inlined subprogram}
that is\hypertarget{chap:DWATinlineinlinedsubroutine}{}
explicitly declared to be available for inline expansion or
that was expanded inline implicitly by the compiler has a
\addtoindexx{inline attribute}
\DWATinlineDEFN{} attribute whose value is an 
\livelink{chap:classconstant}{integer constant}. The
set of values for the \DWATinline{} attribute is given in
Table \refersec{tab:inlinecodes}.

\begin{table}[ht]
\centering
\caption{Inline codes}
\label{tab:inlinecodes}
\begin{tabular}{l|P{8cm}}
\hline
Name&Meaning\\ \hline
\DWINLnotinlinedTARG{} & Not declared inline nor inlined by the
  \mbox{compiler} (equivalent to the absence of the
  containing \DWATinline{} attribute) \\
\DWINLinlinedTARG{} & Not declared inline but inlined by the \mbox{compiler} \\
\DWINLdeclarednotinlinedTARG{} & Declared inline but 
  not inlined by the \mbox{compiler} \\
\DWINLdeclaredinlinedTARG{} & Declared inline and inlined by the 
  \mbox{compiler} \\
\hline
\end{tabular}
\end{table}

\textit{In \addtoindex{C++}, a function or a constructor declared with
\addttindex{constexpr} is implicitly declared inline. The abstract
instance (see Section \refersec{chap:abstractinstances}) 
is represented by a debugging information
entry with the tag \DWTAGsubprogram. Such an entry has a
\DWATinline{} attribute whose value is \DWINLinlined.}

\needlines{4}
\subsubsection{Abstract Instances}
\label{chap:abstractinstances}
Any subroutine entry that contains a
\DWATinlineDEFN{} attribute\addtoindexx{inline attribute} 
whose value is other than 
\DWINLnotinlined{}
is known as an \definition{abstract instance root}.
\addtoindexx{abstract instance!root}
\hypertarget{chap:DWATinlineabstracttinstance}{}
Any debugging information entry that is owned (either
directly or indirectly) by an abstract instance root
is known as an 
\definition{abstract instance entry.}\addtoindexx{abstract instance!entry}
Any set of abstract instance entries that are all
children (either directly or indirectly) of some abstract
instance root, together with the root itself, is known as an
\definition{abstract instance tree.}\addtoindexx{abstract instance!tree}
However, in the case where an abstract instance tree is 
nested within another abstract instance tree, the entries in the 
\addtoindex{nested abstract instance}
tree are not considered to be entries in the outer abstract
instance tree.

\needlines{6}
Each abstract instance root is either part of a larger
\addtoindexx{abstract instance!root}
tree (which gives a context for the root) or 
\addtoindexx{specification attribute}
uses
\DWATspecification{} 
to refer to the declaration in context.

\textit{For example, in \addtoindex{C++} the context might be a namespace
declaration or a class declaration.}

\textit{Abstract instance trees are defined so that no entry is part
of more than one abstract instance tree.}

Attributes and children in an abstract instance are shared
by all concrete instances (see Section \refersec{chap:concreteinstances}).

A debugging information entry that is a member of an abstract
instance tree may not contain any attributes which describe
aspects of the subroutine which vary between distinct inlined
expansions or distinct out-of-line expansions. 

\textit{For example,
\addtoindexx{entry pc attribute!and abstract instance}
the \DWATlowpc,\addtoindexx{low PC attribute!and abstract instance}
\DWAThighpc,\addtoindexx{high PC attribute!and abstract instance} 
\DWATranges,\addtoindexx{ranges attribute!and abstract instance} 
\DWATentrypc,\addtoindexx{entry PC attribute!and abstract instance} 
\DWATlocation,\addtoindexx{location attribute!and abstract instance}
\bb
\DWATreturnaddr\addtoindexx{return address attribute!and abstract instance}
and
\DWATstartscope\addtoindexx{start scope attribute!and abstract instance}
\eb 
attributes typically should be omitted; however, this list is not 
exhaustive.}

\needlines{5}
\textit{It would not make sense normally to put these attributes into
abstract instance entries since such entries do not represent
actual (concrete) instances and thus do not actually exist at
run\dash time.  However, 
see Appendix \refersec{app:inlineouteronenormalinner} 
for a contrary example.}

\needlines{8}
The rules for the relative location of entries belonging to
abstract instance trees are exactly the same as for other
similar types of entries that are not abstract. Specifically,
the rule that requires that an entry representing a declaration
be a direct child of the entry representing the scope of the
declaration applies equally to both abstract and non-abstract
entries. Also, the ordering rules for formal parameter entries,
member entries, and so on, all apply regardless of whether
or not a given entry is abstract.

\needlines{5}
\subsubsection{Concrete Instances}
\label{chap:concreteinstances}

Each inline expansion of a subroutine is represented
by a debugging information entry with the 
tag \DWTAGinlinedsubroutineTARG. Each such entry is a direct
child of the entry that represents the scope within which
the inlining occurs.

\needlines{4}
Each inlined subroutine entry may have either a 
\DWATlowpc{}
and \DWAThighpc{} pair of attributes
\addtoindexx{high PC attribute}
\addtoindexx{low PC attribute}
or a \DWATranges{}\addtoindexx{ranges attribute}
attribute whose values encode the contiguous or non-contiguous
address ranges, respectively, of the machine instructions
generated for the inlined subroutine (see 
Section \referfol{chap:codeaddressesandranges}). 
An\hypertarget{chap:DWATentrypcentryaddressofinlinedsubprogram}{}
inlined subroutine entry may 
\addtoindexx{inlined subprogram entry!in concrete instance}
also 
\addtoindexx{inlined subprogram entry}
contain 
\addtoindexx{entry PC attribute!for inlined subprogram}
a 
\DWATentrypc{}
attribute, representing the first executable instruction of
the inline expansion (see 
Section \refersec{chap:entryaddress}).

% Positions of the 3 targets here is a bit arbitrary.
An inlined\hypertarget{chap:DWATcalllinelinenumberofinlinedsubroutinecall}{}
subroutine\hypertarget{chap:DWATcallcolumncolumnpositionofinlinedsubroutinecall}{}
entry\hypertarget{chap:DWATcallfilefilecontaininginlinedsubroutinecall}{}
may also have \DWATcallfileDEFN,
\DWATcalllineDEFN{} and \DWATcallcolumnDEFN{} attributes,
\addtoindexx{inlined call location attributes} 
each of whose
value is an \livelink{chap:classconstant}{integer constant}. 
These attributes represent the
source file, source line number, and source column number,
respectively, of the first character of the statement or
expression that caused the inline expansion. The call file,
call line, and call column attributes are interpreted in
the same way as the declaration file, declaration line, and
declaration column attributes, respectively (see 
Section \refersec{chap:declarationcoordinates}).

\textit{The call file, call line and call column coordinates do not
describe the coordinates of the subroutine declaration that
was inlined, rather they describe the coordinates of the call.
}

An inlined subroutine entry may have
a\hypertarget{chap:DWATconstexprcompiletimeconstantfunction}{}
\DWATconstexprDEFN{} attribute,\addtoindexx{constant expression attribute} 
which is a \livelink{chap:classflag}{flag} 
whose presence indicates that the
subroutine has been evaluated as a compile\dash time constant. Such
an entry may also have a \DWATconstvalue{} attribute,
whose value may be of any form that is appropriate for the
representation of the subroutine's return value. The value of
this attribute is the actual return value of the subroutine,
represented as it would be on the target architecture.

\needlines{4}
\textit{In \addtoindex{C++}, if a function or a constructor declared with 
\addttindex{constexpr}
is called with constant expressions, then the corresponding
concrete inlined instance has a 
\DWATconstexpr{} attribute,
as well as a \DWATconstvalue{} attribute whose value represents
the actual return value of the concrete inlined instance.}

\needlines{4}
Any debugging information entry that is owned (either
directly or indirectly) by a debugging information entry
with the tag \DWTAGinlinedsubroutine{} is referred to as a
\doublequote{concrete inlined instance entry.} Any entry that has
the tag 
\DWTAGinlinedsubroutine{} 
is known as a \doublequote{concrete inlined instance root.} 
Any set of concrete inlined instance
entries that are all children (either directly or indirectly)
of some concrete inlined instance root, together with the root
itself, is known as a \doublequote{concrete inlined instance tree.}
However, in the case where a concrete inlined instance tree
is nested within another concrete instance tree, the entries
in the \addtoindex{nested concrete inline instance} tree 
are not considered to
be entries in the outer concrete instance tree.

\needlines{3}
\textit{Concrete inlined instance trees are defined so that no entry
is part of more than one concrete inlined instance tree. This
simplifies later descriptions.}

Each concrete inlined instance tree is uniquely associated
with one (and only one) abstract instance tree.

\textit{Note, however, that the reverse is not true. Any given abstract
instance tree may be associated with several different concrete
inlined instance trees, or may even be associated with zero
concrete inlined instance trees.}

Concrete inlined instance entries may omit attributes that
are not specific to the concrete instance (but present in
the abstract instance) and need include only attributes that
are specific to the concrete instance (but omitted in the
abstract instance). In place of these omitted attributes, 
each\hypertarget{chap:DWATabstractorigininlineinstance}{}
concrete inlined instance entry has a
\addtoindexx{abstract origin attribute}
\DWATabstractoriginDEFN{}
attribute that may be used to obtain the missing information
(indirectly) from the associated abstract instance entry. The
value of the abstract origin attribute is a reference to the
associated abstract instance entry.

If an entry within a concrete inlined instance tree contains
attributes describing the 
\addtoindexx{declaration coordinates!in concrete instance}
\livelink{chap:declarationcoordinates}{declaration coordinates} 
of that entry, then those attributes refer to the file, line
and column of the original declaration of the subroutine,
not to the point at which it was inlined. As a consequence,
they may usually be omitted from any entry that has an abstract
origin attribute.

\needlines{4}
For each pair of entries that are associated via a
\addtoindexx{abstract origin attribute}
\DWATabstractorigin{} attribute, both members of the pair
have the same tag. So, for example, an entry with the tag
\DWTAGvariable{} can only be associated with another entry
that also has the tag \DWTAGvariable. The only exception
to this rule is that the root of a concrete instance tree
(which must always have the tag \DWTAGinlinedsubroutine)
can only be associated with the root of its associated abstract
instance tree (which must have the tag \DWTAGsubprogram).

\needlines{6}
In general, the structure and content of any given concrete
inlined instance tree will be closely analogous to the
structure and content of its associated abstract instance
tree. There are a few exceptions:

\begin{enumerate}[1. ]
\item An entry in the concrete instance tree may be omitted if
it contains only a 
\addtoindexx{abstract origin attribute}
\DWATabstractorigin{} attribute and either
has no children, or its children are omitted. Such entries
would provide no useful information. In C\dash like languages,
such entries frequently include types, including structure,
union, class, and interface types; and members of types. If any
entry within a concrete inlined instance tree needs to refer
to an entity declared within the scope of the relevant inlined
subroutine and for which no concrete instance entry exists,
the reference refers to the abstract instance entry.

\needlines{4}
\item Entries in the concrete instance tree which are associated
with entries in the abstract instance tree such that neither
has a \DWATname{} attribute,
\addtoindexx{name attribute}
and neither is referenced by
any other debugging information entry, may be omitted. This
may happen for debugging information entries in the abstract
instance trees that became unnecessary in the concrete instance
tree because of additional information available there. For
example, an anonymous variable might have been created and
described in the abstract instance tree, but because of
the actual parameters for a particular inlined expansion,
it could be described as a constant value without the need
for that separate debugging information entry.

\needlines{4}
\item A concrete instance tree may contain entries which do
not correspond to entries in the abstract instance tree
to describe new entities that are specific to a particular
inlined expansion. In that case, they will not have associated
entries in the abstract instance tree, do not contain
\addtoindexx{abstract origin attribute}
\DWATabstractorigin{} attributes, and must contain all their
own attributes directly. This allows an abstract instance tree
to omit debugging information entries for anonymous entities
that are unlikely to be needed in most inlined expansions. In
any expansion which deviates from that expectation, the
entries can be described in its concrete inlined instance tree.

\end{enumerate}

\subsubsection{Out-of-Line Instances of Inlined Subroutines}
\label{chap:outoflineinstancesofinlinedsubroutines}
Under some conditions, compilers may need to generate concrete
executable instances of inlined subroutines other than at
points where those subroutines are actually called. Such
concrete instances of inlined subroutines are referred to as
\doublequote{concrete out\dash of\dash line instances.}

\textit{In \addtoindex{C++}, for example, 
taking the address of a function declared
to be inline can necessitate the generation of a concrete
out\dash of\dash line instance of the given function.}

The DWARF representation of a concrete out-of-line instance
of an inlined subroutine is essentially the same as for a
concrete inlined instance of that subroutine (as described in
the preceding section). The representation of such a concrete
% It is critical that the hypertarget and livelink be
% separated to avoid problems with latex.
out-of-line 
\addtoindexx{abstract origin attribute}
instance 
\hypertarget{chap:DWATabstractoriginoutoflineinstance}{}
makes use of 
\DWATabstractoriginDEFN{}
attributes in exactly the same way as they are used for
a concrete inlined instance (that is, as references to
corresponding entries within the associated abstract instance
tree).

The differences between the DWARF representation of a
concrete out\dash of\dash line instance of a given subroutine and the
representation of a concrete inlined instance of that same
subroutine are as follows:
\begin{enumerate}[1. ]
\item  The root entry for a concrete out\dash of\dash line instance
of a given inlined subroutine has the same tag as does its
associated (abstract) inlined subroutine entry (that is, tag
\DWTAGsubprogram{} rather than \DWTAGinlinedsubroutine).

\item The root entry for a concrete out\dash of\dash line instance tree
is normally owned by the same parent entry that also owns
the root entry of the associated abstract instance. However,
it is not required that the abstract and out\dash of\dash line instance
trees be owned by the same parent entry.

\end{enumerate}

\subsubsection{Nested Inlined Subroutines}
\label{nestedinlinedsubroutines}
Some languages and compilers may permit the logical nesting of
a subroutine within another subroutine, and may permit either
the outer or the nested subroutine, or both, to be inlined.

For a non-inlined subroutine nested within an inlined
subroutine, the nested subroutine is described normally in
both the abstract and concrete inlined instance trees for
the outer subroutine. All rules pertaining to the abstract
and concrete instance trees for the outer subroutine apply
also to the abstract and concrete instance entries for the
nested subroutine.

\needlines{5}
For an inlined subroutine nested within another inlined
subroutine, the following rules apply to their abstract and
\addtoindexx{abstract instance!nested}
\addtoindexx{concrete instance!nested}
concrete instance trees:

\begin{enumerate}[1. ]
\item The abstract instance tree for the nested subroutine is
described within the abstract instance tree for the outer
subroutine according to the rules in 
Section \refersec{chap:abstractinstances}, and
without regard to the fact that it is within an outer abstract
instance tree.

\item Any abstract instance tree for a nested subroutine is
always omitted within the concrete instance tree for an
outer subroutine.

\item  A concrete instance tree for a nested subroutine is
always omitted within the abstract instance tree for an
outer subroutine.

\item The concrete instance tree for any inlined or 
\addtoindexx{out-of-line instance}
out-of-line
\addtoindexx{out-of-line instance|see{\textit{also} concrete out-of-line instance}}
expansion of the nested subroutine is described within a
concrete instance tree for the outer subroutine according
to the rules in 
Sections \refersec{chap:concreteinstances} or 
\referfol{chap:outoflineinstancesofinlinedsubroutines}
, respectively,
and without regard to the fact that it is within an outer
concrete instance tree.
\end{enumerate}

\textit{See Appendix \refersec{app:inliningexamples} 
for discussion and examples.}

\subsection{Trampolines}
\label{chap:trampolines}

\textit{A trampoline is a compiler\dash generated subroutine that serves 
as\hypertarget{chap:DWATtrampolinetargetsubroutine}{}
an intermediary in making a call to another subroutine. It may
adjust parameters and/or the result (if any) as appropriate
to the combined calling and called execution contexts.}

A trampoline is represented by a debugging information entry
\addtoindexx{trampoline (subprogram) entry}
with the tag \DWTAGsubprogram{} or \DWTAGinlinedsubroutine{}
that has 
\addtoindexx{trampoline attribute}
a \DWATtrampolineDEFN{} attribute. 
The value of that
attribute indicates the target subroutine of the trampoline,
that is, the subroutine to which the trampoline passes
control. (A trampoline entry may but need not also have a
\DWATartificial{} attribute.)

\needlines{5}
The value of the trampoline attribute may be represented
using any of the following forms:

\begin{itemize}
\item If the value is of class \CLASSreference{}, then the value
specifies the debugging information entry of the target
subprogram.

\item If the value is of class \CLASSaddress{}, then the value is
the relocated address of the target subprogram.

\needlines{6}
\item If the value is of class \CLASSstring{}, then the value is the
(possibly mangled) \addtoindexx{mangled names}
name of the target subprogram.

\item If the value is of class \CLASSflag, then the value true
indicates that the containing subroutine is a trampoline but
that the target subroutine is not known.
\end{itemize}


The target subprogram may itself be a trampoline. (A sequence
of trampolines necessarily ends with a non-trampoline
subprogram.)

\textit{In \addtoindex{C++}, trampolines may be used to implement 
derived virtual member functions; such trampolines typically 
adjust the implicit 
\texttt{this} parameter\index{this parameter@\texttt{this} parameter}
in the course of passing control.  
Other languages and environments may use trampolines in a manner 
sometimes known as transfer functions or transfer vectors.}

\textit{Trampolines may sometimes pass control to the target
subprogram using a branch or jump instruction instead of a
call instruction, thereby leaving no trace of their existence
in the subsequent execution context. }

\textit{This attribute helps make it feasible for a debugger to arrange
that stepping into a trampoline or setting a breakpoint in
a trampoline will result in stepping into or setting the
breakpoint in the target subroutine instead. This helps to
hide the compiler generated subprogram from the user. }

\section{Call Site Entries and Parameters}
\label{chap:callsiteentriesandparameters}
\textit{
A call site entry describes a call from one subprogram to another in the
source program. It provides information about the actual parameters of
the call so that they may be more easily accessed by a debugger. When
used together with call frame information 
(see Section \refersec{chap:callframeinformation}), 
call site entries can be useful for computing the value of an actual parameter
passed by a caller, even when the location description for the callee's
corresponding formal parameter does not provide a current location for
the formal parameter.}

\textit{The DWARF expression for computing the value of an actual parameter at
a call site may refer to registers or memory locations.  The expression
assumes these contain the values they would have at the point where the
call is executed. After the called subprogram has been entered, these
registers and memory locations might have been modified.  In order to
recover the values that existed at the point of the call (to allow
evaluation of the DWARF expression for the actual parameter), a debugger
may virtually unwind the subprogram activation 
(see Section \refersec{chap:callframeinformation}). Any
register or memory location that cannot be recovered is referred to as
"clobbered by the call."}

\needlines{5}
A source call can be compiled into different types of machine code:
\begin{itemize}
\item
A \textit{normal call} uses a call-like instruction which transfers 
control to the start of some subprogram and preserves the call site 
location for use by the callee.

\item
A \textit{tail call} uses a jump-like instruction which
transfers control to the start of some subprogram, but 
there is no call site location address to preserve
(and thus none is available using the 
virtual unwind information). 

\item
A \textit{tail recursion call} is a call
to the current subroutine which is compiled as a jump 
to the current subroutine.

\needlines{4}
\item
An \textit{inline (or inlined) call} is a call to an inlined subprogram,
where at least one instruction has the location of the inlined subprogram
or any of its blocks or inlined subprograms. 
\end{itemize}

\needlines{4}
There are also different types of \doublequote{optimized out} calls:
\begin{itemize}
\item
An \textit{optimized out (normal) call} is a call that is in unreachable code that 
has not been emitted (such as, for example, the call to \texttt{foo} in 
\texttt{if (0) foo();}).  
\item
An \textit{optimized out inline call}
is a call to an inlined subprogram which either did not expand to any instructions
or only parts of instructions belong to it and for debug information purposes those
instructions are given a location in the caller.
\end{itemize}

\DWTAGcallsite{} entries describe normal and tail calls but not tail recursion calls,
while \DWTAGinlinedsubroutine{} entries describe inlined calls 
(see Section \refersec{chap:inlinedsubroutines}).
Call site entries cannot 
\bb
fully
\eb
describe tail recursion or optimized out calls.

\bb
\textit{For optimized out calls there is no code address to use for
\DWATcallreturnpc{} or \DWATcallpc{} attributes; however, the fact
that the souce code makes a call to a certain function 
at a specific source code location and whether
some of the arguments have constant values can be useful for certain
consumers.}
\eb

\subsection{Call Site Entries}
\label{chap:callsiteentries}
A call site is represented by a debugging information entry with the tag
\DWTAGcallsiteTARG{}\addtoindexx{call site entry}. 
The entry for a call site is owned by the innermost
debugging information entry representing the scope within which the
call is present in the source program.

\needlines{4}
\textit{A scope entry (for example, a lexical block) that would not 
otherwise be present in the debugging information of a subroutine
need not be introduced solely to represent the immediately containing scope
of a call.}

\needlines{4}
The call site entry may have a
\DWATcallreturnpcDEFN{}\addtoindexx{call site return pc attribute}
\livetargi{chap:DWATcallreturnpcofcallsite}{attribute}{call return pc attribute} 
which is the return address after the call.  
The value of this attribute corresponds to the return address 
computed by call frame information in the called subprogram 
(see Section \refersec{datarep:callframeinformation}).

\textit{On many architectures the return address is the 
address immediately following the call instruction, but 
on architectures with delay slots it might
be an address after the delay slot of the call.}

The call site entry may have a 
\DWATcallpcDEFN{}\addtoindexx{call pc attribute}
\livetargi{chap:DWATcallpcofcallsite}{attribute}{call pc attribute} 
which is the address of the 
call-like instruction for a normal call or the jump-like 
instruction for a tail call.

If the call site entry corresponds to a tail call, it has the 
\DWATcalltailcallDEFN{}\addtoindexx{call tail call attribute}
\livetargi{chap:DWATcalltailcallofcallsite}{attribute}{call tail call attribute},
which is a \CLASSflag.

The call site entry may have a 
\DWATcalloriginDEFN{}\addtoindexx{call origin attribute}
\livetargi{chap:DWATcalloriginofcallsite}{attribute}{call origin attribute}
which is a \CLASSreference.  For direct calls or jumps where the called 
subprogram is known it is a reference to the called subprogram's debugging
information entry.  For indirect calls it may be a reference to a
\DWTAGvariable{}, \DWTAGformalparameter{} or \DWTAGmember{} entry representing
the subroutine pointer that is called.

\needlines{4}
The call site may have a 
\DWATcalltargetDEFN{}\addtoindexx{call target attribute}
\livetargi{chap:DWATcalltargetofcallsite}{attribute}{call target attribute} which is
a DWARF expression.  For indirect calls or jumps where it is unknown at
compile time which subprogram will be called the expression computes the
address of the subprogram that will be called.  

\textit{The DWARF expression should
not use register or memory locations that might be clobbered by the call.}

\needlines{4}
The call site entry may have a 
\DWATcalltargetclobberedDEFN{}\addtoindexx{call target clobbered attribute}
\livetargi{chap:DWATcalltargetclobberedofcallsite}{attribute}{call target clobbered attribute}
which is a DWARF expression.  For indirect calls or jumps where the
address is not computable without use of registers or memory locations that
might be clobbered by the call the \DWATcalltargetclobberedNAME{}
attribute is used instead of the \DWATcalltarget{} attribute.

\textit{The expression of a call target clobbered attribute may only be 
valid at the time the call or call-like transfer of control is executed.}

The call site entry may have a \DWATtypeDEFN{}\addtoindexx{call type attribute}
\livetargi{chap:DWATtypeofcallsite}{attribute}{type attribute!of call site entry}
referencing a debugging information entry for the type of the called function.  

\textit{When \DWATcallorigin{} is present, \DWATtypeNAME{} is usually omitted.}

The call site entry may have 
\DWATcallfileDEFN{}\addtoindexx{call file attribute}, 
\DWATcalllineDEFN{}\addtoindexx{call line attribute} and 
\DWATcallcolumnDEFN{}\addtoindexx{call column attribute} 
\livetargi{chap:DWATcallfileofcallsite}{attributes,}{call file attribute!of call site entry}
\livetargi{chap:DWATcalllineofcallsite}{}{call line attribute!of call site entry}
\livetargi{chap:DWATcallcolumnofcallsite}{}{call column attribute!of call site entry}
each of whose value is an integer constant.
These attributes represent the source file, source line number, and source
column number, respectively, of the first character of the call statement or
expression.  The call file, call line, and call column attributes are
interpreted in the same way as the declaration file, declaration
line, and declaration column attributes, respectively 
(see Section \refersec{chap:declarationcoordinates}).

\textit{The call file, call line and call column coordinates do 
not describe the coordinates of the subroutine declaration that 
was called, rather they describe the coordinates of the call.}

\needlines{5}
\subsection{Call Site Parameters}
\label{chap:callsiteparameters}
The call site entry may own 
\DWTAGcallsiteparameterTARG{}\index{call site parameter entry} 
debugging information entries representing the parameters passed 
to the call.
Call site parameter entries occur in the same order as the 
corresponding parameters in the source.
Each such entry has a \DWATlocation{} attribute which is a location 
description. This location description 
describes where the parameter is passed
(usually either some register, or a memory location expressible as 
the contents of the stack register plus some offset).

\needlines{4}
Each \DWTAGcallsiteparameter{} entry may have a 
\DWATcallvalueDEFN{}\addtoindexx{call value attribute}
\livetargi{chap:DWATcallvalueofcallparameter}{attribute}{call value attribute}
which is a DWARF expression 
which when evaluated yields the value of the parameter at the time of the call.

\textit{If it is not
possible to avoid registers or memory locations that might be clobbered by
the call in the expression, then the \DWATcallvalueNAME{} attribute should
not be provided. The reason for the restriction is that the value of the parameter may be
needed in the midst of the callee, where the call clobbered registers or
memory might be already clobbered, and if the consumer is not assured by
the producer it can safely use those values, the consumer can not safely
use the values at all.}

For parameters passed by reference, where the code passes a pointer to
a location which contains the parameter, or for reference type parameters,
the \DWTAGcallsiteparameter{} entry may also have a
\DWATcalldatalocationDEFN{}\addtoindexx{call data location attribute}
\livetargi{chap:DWATcalldatalocationofcallparameter}{attribute}{call data location attribute}
whose value is a location description and a
\DWATcalldatavalueDEFN{}\addtoindexx{call data value attribute}
\livetargi{chap:DWATcalldatavalueofcallparameter}{attribute}{call data value attribute}
whose value is a DWARF expression.  The \DWATcalldatalocationNAME{} attribute
\addtoindexx{call data location attribute} 
describes where the referenced value lives during the call.  If it is just 
\DWOPpushobjectaddress{}, it may be left out.  The 
\DWATcalldatavalueNAME{} attribute describes the value in that location. 
The expression should not use registers or memory
locations that might be clobbered by the call, as it might be evaluated after 
virtually unwinding from the called function back to the caller.

\needlines{4}
Each call site parameter entry may also have a 
\DWATcallparameterDEFN{}\addtoindexx{call parameter attribute}
\livetargi{chap:DWATcallparameterofcallparameter}{attribute}{call parameter attribute}
which contains a reference to a \DWTAGformalparameter{} entry,
\DWATtype{} attribute referencing the type of the parameter or 
\DWATname{} attribute describing the parameter's name.

\textit{Examples using call site entries and related attributes are 
found in Appendix \refersec{app:callsiteexamples}.}

\needlines{8}
\section{Lexical Block Entries}
\label{chap:lexicalblockentries}

\textit{A 
lexical \livetargi{chap:lexicalblock}{block}{lexical block} 
is 
\addtoindexx{lexical block}
a bracketed sequence of source statements
that may contain any number of declarations. In some languages
(including \addtoindex{C} and \addtoindex{C++}),
\nolink{blocks} can be nested within other
\nolink{blocks} to any depth.}

% We do not need to link to the preceding paragraph.
A lexical \nolink{block} is represented by a debugging information
entry with the 
tag \DWTAGlexicalblockTARG.

The lexical \livetargi{chap:lexicalblockentry}{block}{lexical block entry} 
entry may have 
either a \DWATlowpc{} and
\DWAThighpc{} pair of 
attributes 
\addtoindexx{high PC attribute}
or 
\addtoindexx{low PC attribute}
a 
\DWATranges{} attribute
\addtoindexx{ranges attribute}
whose values encode the contiguous or non-contiguous address
ranges, respectively, of the machine instructions generated
for the lexical \nolink{block} 
(see Section \refersec{chap:codeaddressesandranges}).

A\hypertarget{chap:DWATentrypcoflexicalblock}{}
lexical block entry may also have a
\addtoindexx{entry PC attribute!for lexical block}
\DWATentrypc{} attribute
whose value is the address of the first executable instruction
of the lexical block (see 
Section \refersec{chap:entryaddress}).

If a name has been given to the lexical \nolink{block} 
in the source program, then the corresponding 
lexical \nolink{block} entry has a
\DWATname{} attribute whose 
\addtoindexx{name attribute}
value is a null-terminated string
containing the name of the lexical \nolink{block}.

\textit{This is not the same as a \addtoindex{C} or 
\addtoindex{C++} label (see Section \refersec{chap:labelentries}).}

The lexical \nolink{block} entry owns debugging 
information entries that describe the declarations 
within that lexical \nolink{block}. There is
one such debugging information entry for each local declaration
of an identifier or inner lexical \nolink{block}.

\needlines{10}
\section{Label Entries}
\label{chap:labelentries}
\textit{A label is a way of identifying a source location.
A labeled statement is usually the target of one or more 
\doublequote{go to} statements.}

\needlines{4}
A label is represented by a debugging information entry with
\addtoindexx{label entry} the tag \DWTAGlabelTARG. 
The entry for a label is owned by
the debugging information entry representing the scope within
which the name of the label could be legally referenced within
the source program.

The label entry has a \DWATlowpc{} attribute whose value
is the address of the first executable instruction for the 
location identified by the label in
the source program.  The label entry also has a 
\DWATname{} attribute 
\addtoindexx{name attribute}
whose value is a null-terminated string containing
the name of the label.   


\section{With Statement Entries}
\label{chap:withstatemententries}

\textit{Both \addtoindex{Pascal} and 
\addtoindexx{Modula-2}
Modula-2 support the concept of a \doublequote{with}
statement. The with statement specifies a sequence of
executable statements within which the fields of a record
variable may be referenced, unqualified by the name of the
record variable.}

A with statement is represented by a
\addtoindexi{debugging information entry}{with statement entry}
with the tag \DWTAGwithstmtTARG.

A with statement entry may have either a 
\DWATlowpc{} and
\DWAThighpc{} pair of attributes 
\addtoindexx{low PC attribute}
\addtoindexx{high PC attribute}
or a
\DWATranges{} attribute
\addtoindexx{ranges attribute}
whose values encode the contiguous or non-contiguous address
ranges, respectively, of the machine instructions generated
for the with statement 
(see Section \refersec{chap:codeaddressesandranges}).

A\hypertarget{chap:DWATentrypcofwithstmt}{}
with statement entry may also have a
\addtoindexx{entry PC attribute!for with statement}
\DWATentrypc{} attribute
whose value is the address of the first executable instruction
of the with statement (see 
Section \refersec{chap:entryaddress}).

\needlines{5}
The with statement entry has a
\addtoindexx{type attribute}
\DWATtype{} attribute, denoting
the type of record whose fields may be referenced without full
qualification within the body of the statement. It also has
\addtoindexx{location attribute}
a \DWATlocation{} attribute, describing how to find the base
address of the record object referenced within the body of
the with statement.

\needlines{6}
\section{Try and Catch Block Entries}
\label{chap:tryandcatchblockentries}
\livetarg{chap:tryandcatchblockentries}{}
\textit{In \addtoindex{C++}, a \livelink{chap:lexicalblock}{lexical block} may be 
designated as a \doublequote{catch \nolink{block}.} 
A catch \nolink{block} is an exception handler that 
handles exceptions thrown by an immediately preceding 
\doublequote{try \nolink{block}.}
A catch \nolink{block} 
designates the type of the exception that it can handle.}

A \livetarg{chap:tryblock}{try block} is represented 
by a debugging information entry
\addtoindexx{try block entry}
with the tag \DWTAGtryblockTARG.  
A \livetarg{chap:catchblock}{catch block} is represented by
a debugging information entry
\addtoindexx{catch block entry}
with the tag \DWTAGcatchblockTARG.

Both try and catch \nolink{block} entries may have either a
\DWATlowpc{} and 
\DWAThighpc{} pair of attributes 
\addtoindexx{low PC attribute}
\addtoindexx{high PC attribute}
or a
\DWATranges{} attribute 
\addtoindexx{ranges attribute}
whose values encode the contiguous
or non-contiguous address ranges, respectively, of the
machine instructions generated for the \nolink{block}
(see Section \refersec{chap:codeaddressesandranges}).

A\hypertarget{chap:DWATentrypcoftryblock}{}
try or catch\hypertarget{chap:DWATentrypcofcatchblock}{}
block entry may also have a
\addtoindexx{entry PC attribute!for try block}
\addtoindexx{entry PC attribute!for catch block}
\DWATentrypc{} attribute
whose value is the address of the first executable instruction
of the try or catch block 
(see Section \refersec{chap:entryaddress}).

\needlines{4}
Catch \nolink{block} entries have at least one child entry, 
an entry representing the type of exception accepted by
that catch \nolink{block}. 
This child entry has one of the tags
\DWTAGformalparameter{}\addtoindexx{formal parameter entry!in catch block}
or
\DWTAGunspecifiedparameters{},
\addtoindexx{unspecified parameters entry!in catch block}
and will have the same form as other parameter entries.

The siblings immediately following a try \nolink{block} 
entry are its corresponding catch \nolink{block} entries.

\needlines{8}
\section{Declarations with Reduced Scope}
\label{declarationswithreducedscope}
\hypertarget{chap:DWATstartscopeofdeclaration}{}
Any debugging information entry for a declaration 
(including objects, subprograms, types and modules) whose scope 
has an address range that is a subset of the address range for 
the lexical scope most closely enclosing the declared entity 
may have a 
\DWATstartscopeDEFN{}\addtoindexx{start scope attribute}
attribute to specify that reduced range of addresses. 

\needlines{12}
There are two cases:
\begin{enumerate}[1. ]
\item If the address range for the scope of the entry 
includes all of addresses for the containing scope except 
for a contiguous sequence of bytes at the beginning of the 
address range for the containing scope, then the address is 
specified using a value of class \CLASSconstant. 

\begin{enumerate}[a) ]
\item If the address
range of the containing scope is contiguous, the value of 
this attribute is the offset in bytes of the beginning of 
the address range for the scope of the object from the low 
PC value of the debugging information entry that defines
that containing scope. 
\item If the address range of the containing 
scope is non-contiguous 
(see \refersec{chap:noncontiguousaddressranges})
the value of this attribute is the offset in bytes of the 
beginning of the address range for the scope of the entity 
from the beginning of the first \addtoindex{range list} entry
for the containing scope that is not a base address entry
\db
or an end-of-list entry.
\end{enumerate}

\needlines{4}
\item Otherwise, the set of addresses for the scope of the 
entity is specified using a value of class \CLASSrnglistsptr{}. 
This value indicates the beginning of a \addtoindex{range list}
(see Section \refersec{chap:noncontiguousaddressranges}).
\end{enumerate}

\textit{For example, the scope of a variable may begin somewhere 
in the midst of a lexical \livelink{chap:lexicalblock}{block} in a 
language that allows executable code in a
\nolink{block} before a variable declaration, or where one declaration
containing initialization code may change the scope of a
subsequent declaration.}  

\needlines{4}
\textit{Consider the following example \addtoindex{C} code:}
\par % Needed to end paragraph before listing so that it gets a line number
\begin{nlnlisting}
float x = 99.99;
int myfunc()
{
    float f = x;
    float x = 88.99;
    return 0;
}
\end{nlnlisting}

\textit{\addtoindex{C} scoping rules require that the value of the 
variable \texttt{x} assigned to the variable \texttt{f} in the 
initialization sequence is the value of the global variable \texttt{x}, 
rather than the local \texttt{x}, because the scope of the local variable 
\texttt{x} only starts after the full declarator for the local \texttt{x}.}

\textit{Due to optimization, the scope of an object may be
non-contiguous and require use of a \addtoindex{range list} even when
the containing scope is contiguous. Conversely, the scope of
an object may not require its own \addtoindex{range list} even when the
containing scope is non-contiguous.}
