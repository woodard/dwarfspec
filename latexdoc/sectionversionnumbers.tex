
\chapter[Section Version Numbers (Informative)]{DWARF Section Version Numbers (Informative)}
\label{app:dwarfsectionversionnumbersinformative}
\addtoindexx{version number!summary by section}

% The table format looks rather different that V4 and earlier
% as latex took up too much space and left no room for V5 
% or later.  The new format uses space more efficiently.

Most DWARF sections have a version number in the section
header. This version number is not tied to the DWARF standard
revision numbers, but instead is incremented when incompatible
changes to that section are made. The DWARF standard that
a producer is following is not explicitly encoded in the
file. Version numbers in the section headers are represented
as two byte unsigned integers. 

Table \refersec{tab:sectionversionnumbers}
shows what version
numbers are in use for each section. In that table:
\begin{itemize}
\setlength{\itemsep}{0em}
\item  \doublequote{V2} means \addtoindex{DWARF Version 2}, published July 27, 1993.
\item  \doublequote{V3} means \addtoindex{DWARF Version 3}, published December 20, 2005.
\item  \doublequote{V4} means \addtoindex{DWARF Version 4}, published June 10, 2010.
\item  \doublequote{V5} means \addtoindex{DWARF Version 5}\footnote{Higher numbers are reserved for future use.}, published 
			\ifthenelse{\boolean{isdraft}}{\textit{<to be determined>}}{\docdate}.
\end{itemize}

There are sections with no version number encoded in them;
they are only accessed via the 
\dotdebuginfo{} 
sections and so an incompatible change in those sections'
format would be represented by a change in the 
\dotdebuginfo{} section version number.

\needlines{10}
\begin{centering}
\setlength{\extrarowheight}{0.1cm}
\begin{longtable}{lcccc}
  \caption{Section version numbers} \label{tab:sectionversionnumbers} \\
  \hline 
  \bfseries Section Name &\bfseries V2 &\bfseries V3 &\bfseries V4 
                         &\bfseries V5 \\ 
  \hline
\endfirsthead
   \bfseries Section Name &\bfseries V2 &\bfseries V3 &\bfseries V4 &\bfseries V5 \\ \hline
\endhead
  \hline \emph{Continued on next page}
\endfoot
  \hline
\endlastfoot
\dotdebugabbrev{}   & * & * & * & * \\
\dotdebugaddr{}	    & - & - & - & 5 \\
\dotdebugaranges{}  & 2 & 2 & 2 & 2 \\
\dotdebugframe{}\footnote{\textit{For the \dotdebugframe{} section, version 2 is unused.}}
                    & 1 & 3 & 4 & 4 \\
\dotdebuginfo{}     & 2 & 3 & 4 & 5 \\
\dotdebugline{}     & 2 & 3 & 4 & 5 \\\
\dotdebuglinestr{}  & - & - & - & * \\
\dotdebugloc{}      & * & * & * & - \\
\dotdebugloclists{} & - & - & - & 5 \\
\dotdebugmacinfo{}  & * & * & * & - \\*
\dotdebugmacro{}    & - & - & - & 5 \\
\dotdebugnames{}    & - & - & - & 5 \\
\dotdebugpubnames{} & 2 & 2 & 2 & - \\
\dotdebugpubtypes{} & - & 2 & 2 & - \\
\dotdebugranges{}   & - & * & * & - \\
\dotdebugrnglists{} & - & - & - & 5 \\
\dotdebugstr{}      & * & * & * & * \\
\dotdebugstroffsets & - & - & - & 5 \\
\dotdebugsup        & - & - & - & 5 \\
\dotdebugtypes{}    & - & - & 4 & - \\
\\
\hspace{3.5cm}\textit{(split object sections)}
\\
\dotdebugabbrevdwo  & - & - & - & * \\
\dotdebuginfodwo    & - & - & - & 5 \\
\dotdebuglinedwo    & - & - & - & 5 \\
\dotdebugloclistsdwo& - & - & - & 5 \\
\dotdebugmacrodwo   & - & - & - & 5 \\
\dotdebugrnglistsdwo& - & - & - & 5 \\

\dotdebugstrdwo     & - & - & - & * \\
\dotdebugstroffsetsdwo 
                    & - & - & - & 5 \\

\hspace{3.5cm}\textit{(package file sections)}
\\
\dotdebugcuindex{}  & - & - & - & 5 \\
\dotdebugtuindex{}  & - & - & - & 5 \\
\end{longtable}
\end{centering}

\needlines{8}
Notes:
\begin{itemize}
\item  \doublequote{*} means that a version number is not applicable
(the section does not include a header or the section's header does not include a version).
\item  \doublequote{-} means that the section was not defined in that
version of the DWARF standard.
\item  The version numbers for corresponding .debug\_<kind> and .debug\_<kind>.dwo 
sections are the same.
\end{itemize}


