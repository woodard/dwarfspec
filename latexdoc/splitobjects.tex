\chapter[Split DWARF Object Files (Informative)]{Split DWARF Object Files (Informative)}
\label{app:splitdwarfobjectsinformative}
\addtoindexx{DWARF compression}
\addtoindexx{DWARF duplicate elimination|see{\textit{also} DWARF compression}}
\addtoindexx{DWARF duplicate elimination|see{\textit{also} split DWARF object file}}
With the traditional DWARF format, debug information is designed
with the expectation that it will be processed by the linker to
produce an output binary with complete debug information, and
with fully-resolved references to locations within the
application. For very large applications, however, this approach
can result in excessively large link times and excessively large
output files. 

Several vendors have independently developed
proprietary approaches that allow the debug information to remain
in the relocatable object files, so that the linker does not have
to process the debug information or copy it to the output file.
These approaches have all required that additional information be
made available to the debug information consumer, and that the
consumer perform some minimal amount of relocation in order to
interpret the debug info correctly. The additional information
required, in the form of load maps or symbol tables, and the
details of the relocation are not covered by the DWARF
specification, and vary with each vendor's implementation.

Section \refersec{datarep:splitdwarfobjectfiles} describes a
platform-independent mechanism that allows a producer to
split the debugging information into relocatable and
non-relocatable partitions. This Appendix describes the use
of \splitDWARFobjectfile{s} and provides some illustrative
examples.

\section{Overview}
\label{app:splitoverview}
\DWARFVersionV{} introduces an optional set of debugging sections
that allow the compiler to partition the debugging information
into a set of (small) sections that require link-time relocation
and a set of (large) sections that do not. The sections that
require relocation are written to the relocatable object file as
usual, and are linked into the final executable. The sections
that do not require relocation, however, can be written to the
relocatable object (.o) file but ignored by the linker, or they
can be written to a separate DWARF object (.dwo{}) 
\addtoindexx{\texttt{.dwo} file extension} file
that need not be accessed by the linker.

\needlines{4}
The optional set of debugging sections includes the following:
\begin{itemize}
\item
\dotdebugabbrevdwo{} - Contains the abbreviations table(s) used by
the \dotdebuginfodwo{} section.
\item
\dotdebuginfodwo{} - Contains the \DWTAGcompileunit{} and
\DWTAGtypeunit{} DIEs and
their descendants. This is the bulk of the debugging
information for the compilation unit that is normally found
in the \dotdebuginfo{} section.
\item
\dotdebugloclistsdwo{} - Contains the location lists referenced by
the debugging information entries in the \dotdebuginfodwo{}
section. This contains the location lists normally found in 
the \dotdebugloclists{} section.
\item
\dotdebugstrdwo{} - Contains the string table for all indirect
strings referenced by the debugging information in the
\dotdebuginfodwo{} sections.
\item
\dotdebugstroffsetsdwo{} - Contains the string offsets table
for the strings in the \dotdebugstrdwo{}{} section.
\item
\dotdebugmacrodwo{} - Contains macro definition information,
normally found in the \dotdebugmacro{} section.
\item
\dotdebuglinedwo{} - Contains \addtoindex{specialized line number table}s 
for the type units in the \dotdebuginfodwo{} section. These tables
contain only the directory and filename lists needed to
interpret \DWATdeclfile{} attributes in the debugging
information entries. Actual line number tables remain in the
\dotdebugline{} section, and remain in the relocatable object
(.o) files.

\end{itemize}

In a \texttt{.dwo} file, there is no benefit to having a separate string
section for directories and file names because the primary
string table will never be stripped. Accordingly, no
\texttt{.debug\_line\_str.dwo} section is defined. Content descriptions 
corresponding to \DWFORMlinestrp{} in an executable file (for example, 
in the skeleton compilation unit) instead use one of the forms
\DWFORMstrxXNor. This allows
directory and file name strings to be merged with general
strings and across compilations in package files 
(where they are not subject to potential stripping).
\bb
This merge is facilitated by the requirement that all references to
the \dotdebugstrdwo{} string table are made indirectly through the
\dotdebugstroffsetsdwo{} section so that only that section needs to
be modified during string merging (see Section 
\refersec{datarep:secondpartition}).
\eb
\db

In order for the consumer to locate and process the debug
information, the compiler must produce a small amount of debug
information that passes through the linker into the output
binary. A skeleton \dotdebuginfo{} section for each compilation unit
contains a reference to the corresponding \texttt{.o} or \texttt{.dwo}
file, and the \dotdebugline{} section (which is typically small
compared to the \dotdebuginfo{} sections) is
linked into the output binary, as is the \dotdebugaddr{} section.

\needlines{6}
The debug sections that continue to be linked into the
output binary include the following:
\begin{itemize}
\item
\dotdebugabbrev{} - Contains the abbreviation codes used by the
skeleton \dotdebuginfo{} section.
\item
\dotdebugaddr{} - Contains references to loadable sections,
indexed by attributes of one of the forms \DWFORMaddrxXN{}, 
or location expression \DWOPaddrx{} opcodes.
\item
\dotdebugaranges{} - Contains the accelerated range lookup table
for the compilation unit.
\item
\dotdebugframe{} - Contains the frame tables.
\item
\dotdebuginfo{} - Contains a skeleton 
skeleton compilation unit DIE, which has no children.
\item
\dotdebugline{} - Contains the line number tables.
(These could be moved to the .dwo file, but in
order to do so, each \DWLNEsetaddress{} opcode would need to
be replaced by a new opcode that referenced an entry in the
\dotdebugaddr{} section. Furthermore, leaving this section in the
.o file allows many debug info consumers to remain unaware of
.dwo files.)
\item
\dotdebuglinestr{} - Contains strings for file names used in
combination with the \dotdebugline{} section.
\item
\dotdebugnames{} - Contains the names for use in
building an index section. 
The section header refers to a
compilation unit offset, which is the offset of the
skeleton compilation unit in the \dotdebuginfo{} section.

\item
\dotdebugstr{} - Contains any strings referenced by the skeleton
\dotdebuginfo{} sections (via \DWFORMstrp{}, 
\bb 
\DWFORMstrpeight{},
\eb
\DWFORMstrxXNor{}).

\item
\dotdebugstroffsets{} - Contains the string offsets table for
the strings in the \dotdebugstr{} section (if one of the forms
\DWFORMstrxXNor{} is used).
\end{itemize}

\needlines{6}
The skeleton compilation unit DIE may have the following attributes:
\par
\begin{nolinenumbersenv}
\autocols[0pt]{c}{3}{l}{
\DWATaddrbase{},
\DWATcompdir{},
\DWATdwoname{},
%\DWATdwoid{},
\DWAThighpc{},
\DWATlowpc{},
\DWATranges{},
\DWATstmtlist{},
\bb
\DWATstroffsets{}
\eb
}
\end{nolinenumbersenv}

All other attributes of the compilation unit DIE are moved to
the full DIE in the \dotdebuginfodwo{} section.

The \HFNdwoid{} field is present in headers of the skeleton DIE 
and the header of the full DIE, so that a consumer
can verify a match.

\needlines{4}
Relocations are neither necessary nor useful in 
\texttt{.dwo} files, because the \texttt{.dwo}  
files contain only debugging information that does not need to be
processed by a linker. Relocations are rendered unnecessary by 
these strategies:

\begin{enumerate}[1. ]
\item Some values needing relocation are kept in the \texttt{.o} file
(for example, references to the line number program from the skeleton
compilation unit).

\item Some values do not need a relocation because they refer from
one \dotdwo{} section to another \dotdwo{} section
in the same compilation unit. 

\item Some values that need a relocation to refer to a relocatable 
program address use one of the \DWFORMaddrxXNor{} forms, 
referencing a relocatable 
value in the \dotdebugaddr{} section (which remains in the .o file).

\end{enumerate}


Table \refersec{tab:unitattributesbyunitkind} summarizes which
attributes are defined for use in the various 
kinds of compilation units (see Section \refersec{chap:unitentries}). 
It compares and contrasts both conventional and split object-related
kinds.

\begin{table}[ht]
\caption{Unit attributes by unit kind}
\label{tab:unitattributesbyunitkind}
\begin{tabular}{P{5.5cm}|cc|ccc}
\hline
                        & \multicolumn{5}{c}{\bfseries Unit Kind} \\
                        & \multicolumn{2}{c}{\bfseries Conventional} 
                                              & \multicolumn{3}{c}{\bfseries Skeleton and Split} \\
\bfseries Attribute     & Full \&    &  Type  &  Skeleton & Split Full & Split Type \\
                        & Partial    &        &           &            &            \\
\hline
\DWATaddrbase           & \chkmk  &        &  \chkmk   &        &         \\
\hline
\DWATbasetypes          & \chkmk  &        &           &        &         \\
\hline
\DWATcompdir            & \chkmk  &        &  \chkmk   &        &         \\
\hline
%\DWATdwoid             &         &        &  \chkmk   & \chkmk &         \\
%\hline
\DWATdwoname            &         &        &  \chkmk   &        &         \\
\hline
\DWATentrypc            & \chkmk  &        &           & \chkmk &         \\
\hline
\DWAThighpc             & \chkmk  &        &  \chkmk   &        &         \\
\hline
\DWATidentifiercase     & \chkmk  &        &           & \chkmk &         \\
\hline
\bbeb
\DWATlanguagename       & \chkmk  & \chkmk &           & \chkmk & \chkmk  \\
\hline
\bbeb
\DWATlanguageversion    & \chkmk  & \chkmk &           & \chkmk & \chkmk  \\
\hline
\DWATloclistsbase       & \chkmk  &        &           &        &         
\\
\hline
\DWATlowpc              & \chkmk  &        &  \chkmk   &        &         \\
\hline
\DWATmacros             & \chkmk  &        &           & \chkmk &         \\
\hline
\DWATmainsubprogram     & \chkmk  &        &           & \chkmk &         \\
\hline
\DWATname               & \chkmk  &        &           & \chkmk &         \\
\hline
\DWATproducer           & \chkmk  &        &           & \chkmk &         \\
\hline
\DWATranges             & \chkmk  &        &           & \chkmk &         \\
\hline
\DWATrnglistsbase       & \chkmk  &        &\bbeb\chkmk&        &         \\
\hline
\DWATstmtlist           & \chkmk  & \chkmk &  \chkmk   &        & \chkmk  \\
\hline
\bbeb \DWATstroffsets   & \chkmk  & \chkmk &  \chkmk   &        &         \\
\hline
\DWATuseUTFeight        & \chkmk  & \chkmk &  \chkmk   & \chkmk & \chkmk  \\
\hline
\end{tabular}
\end{table}

\needlines{8}
The split dwarf object file design depends on having an index of 
debugging information available to the consumer. For name lookups, 
the consumer can use the \dotdebugnames{} index section (see 
Section \refersec{chap:acceleratedaccess}) to 
locate a skeleton compilation unit. The
\DWATcompdir{} and \DWATdwoname{} attributes in the skeleton
compilation unit can then be used to locate the corresponding
DWARF object file for the compilation unit. Similarly, for an
address lookup, the consumer can use the \dotdebugaranges{} table,
which will also lead to a skeleton compilation unit. For a file
and line number lookup, the skeleton compilation units can be
used to locate the line number tables.

\clearpage

\section{Split DWARF Object File Example}
\label{app:splitdwarfobjectfileexample}
\addtoindexx{split DWARF object file!example}
Consider the example source code in 
Figure \refersec{fig:splitobjectexamplesourcefragment1}, 
Figure \refersec{fig:splitobjectexamplesourcefragment2} and
Figure \refersec{fig:splitobjectexamplesourcefragment3}.
When compiled with split DWARF, we will have two DWARF object files,
\texttt{demo1.o} and \texttt{demo2.o}, and two \splitDWARFobjectfile{s}, 
\texttt{demo1.dwo} and \texttt{demo2.dwo}.

In this section, we will use this example to show how the
connections between the relocatable object file and the split
DWARF object file are maintained through the linking process. In
the next section, we will use this same example to show how two
or more split DWARF object files are combined into a DWARF
package file.

\vspace{2cm}
\begin{figure}[ht]
\textit{File demo1.cc}
\begin{nlnlisting}
#include "demo.h"

bool Box::contains(const Point& p) const
{
    return (p.x() >= ll_.x() && p.x() <= ur_.x() &&
            p.y() >= ll_.y() && p.y() <= ur_.y());
}
\end{nlnlisting}
\caption{Split object example: source fragment \#1}
\label{fig:splitobjectexamplesourcefragment1}
\end{figure}

\begin{figure}[ht]
\textit{File demo2.cc}
\begin{nlnlisting}
#include "demo.h"

bool Line::clip(const Box& b)
{
  float slope = (end_.y() - start_.y()) / (end_.x() - start_.x());
  while (1) {
    // Trivial acceptance.
    if (b.contains(start_) && b.contains(end_)) return true;

    // Trivial rejection.
    if (start_.x() < b.l() && end_.x() < b.l()) return false;
    if (start_.x() > b.r() && end_.x() > b.r()) return false;
    if (start_.y() < b.b() && end_.y() < b.b()) return false;
    if (start_.y() > b.t() && end_.y() > b.t()) return false;

    if (b.contains(start_)) {
      // Swap points so that start_ is outside the clipping 
      // rectangle.
      Point temp = start_;
      start_ = end_;
      end_ = temp;
    }

    if (start_.x() < b.l())
      start_ = Point(b.l(), 
                     start_.y() + (b.l() - start_.x()) * slope);
    else if (start_.x() > b.r())
      start_ = Point(b.r(), 
                     start_.y() + (b.r() - start_.x()) * slope);
    else if (start_.y() < b.b())
      start_ = Point(start_.x() + (b.b() - start_.y()) / slope, 
                     b.b());
    else if (start_.y() > b.t())
      start_ = Point(start_.x() + (b.t() - start_.y()) / slope, 
                     b.t());
  }
}
\end{nlnlisting}
\caption{Split object example: source fragment \#2}
\label{fig:splitobjectexamplesourcefragment2}
\end{figure}

\begin{figure}[ht]
\textit{File demo.h}
\begin{nlnlisting}
class A {
  public:
    Point(float x, float y) : x_(x), y_(y){}
    float x() const { return x_; }
    float y() const { return y_; }
  private:
    float x_;
    float y_;
};

class Line {
  public:
    Line(Point start, Point end) : start_(start), end_(end){}
    bool clip(const Box& b);
    Point start() const { return start_; }
    Point end() const { return end_; }
  private:
    Point start_;
    Point end_;
};

class Box {
  public:
    Box(float l, float r, float b, float t) : ll_(l, b), ur_(r, t){}
    Box(Point ll, Point ur) : ll_(ll), ur_(ur){}
    bool contains(const Point& p) const;
    float l() const { return ll_.x(); }
    float r() const { return ur_.x(); }
    float b() const { return ll_.y(); }
    float t() const { return ur_.y(); }
  private:
    Point ll_;
    Point ur_;
};

\end{nlnlisting}
\caption{Split object example: source fragment \#3}
\label{fig:splitobjectexamplesourcefragment3}
\end{figure}

\clearpage
\subsection{Contents of the Object Files}
The object files each contain the following sections of debug
information:
\begin{alltt}
  \dotdebugabbrev
  \dotdebuginfo
  \dotdebugline
  \dotdebugstr
  \dotdebugaddr
  \dotdebugnames
  \dotdebugaranges
\end{alltt}

The \dotdebugabbrev{} section contains just a single entry describing
the skeleton compilation unit DIE.

The DWARF description in the \dotdebuginfo{} section 
contains just a single DIE, the skeleton compilation unit, 
which may look like 
Figure \referfol{fig:splitdwafexampleskeletondwarfdescription}.

\begin{figure}[ht]
\begin{dwflisting}
\begin{alltt}

    \DWTAGskeletonunit
        \DWATcompdir: (reference to directory name in .debug_str)
        \DWATdwoname: (reference to "demo1.dwo" in .debug_str)
        \DWATaddrbase: (reference to .debug_addr section)
        \DWATstmtlist: (reference to .debug_line section)
      
\end{alltt}
\end{dwflisting}
\caption{Split object example: skeleton DWARF description}
\label{fig:splitdwafexampleskeletondwarfdescription}
\end{figure}

The \DWATcompdir{} and \DWATdwoname{} attributes provide the
location of the corresponding \splitDWARFobjectfile{} that
contains the full debug information; that file is normally
expected to be in the same directory as the object file itself.

The \HFNdwoid{} field in the header of the skeleton unit provides 
an ID or key for the debug information contained in the 
DWARF object file. This ID serves
two purposes: it can be used to verify that the debug information
in the \splitDWARFobjectfile{} matches the information in the object
file, and it can be used to find the debug information in a DWARF
package file.

The \DWATaddrbase{} attribute contains the relocatable offset of
this object file's contribution to the \dotdebugaddr{} section.

The \DWATstmtlist{} attribute contains the relocatable offset of
this file's contribution to the \dotdebugline{} table.

\needlines{6}
The \dotdebugline{} section contains the full line number table for
the compiled code in the object file. As shown in
Figure \refersec{fig:splitobjectexamplesourcefragment1}, the line
number program header lists the two file names, \texttt{demo.h} and
\texttt{demo1.cc}, and contains line number programs for
\texttt{Box::contains}, \texttt{Point::x}, and \texttt{Point::y}.

The \dotdebugstr{} section contains the strings referenced indirectly
by the compilation unit DIE and by the line number program.

The \dotdebugaddr{} section contains relocatable addresses of
locations in the loadable text and data that are referenced by
debugging information entries in the split DWARF object. In the
example in \refersec{fig:splitobjectexamplesourcefragment3}, 
\texttt{demo1.o} may have three entries:
\par
\begin{nolinenumbersenv}
\begin{center}
\begin{tabular}{cl}
Slot & Location referenced \\
\hline
   0   &  low PC value for \texttt{Box::contains}  \\
   1   &  low PC value for \texttt{Point::x}       \\
   2   &  low PC value for \texttt{Point::y}       \\
\end{tabular}
\end{center}
\end{nolinenumbersenv}

\needlines{4}
The \dotdebugnames{}
section contains the names defined by the debugging
information in the \splitDWARFobjectfile{} 
(see Section \refersec{chap:contentsofthenameindex}), 
and references the skeleton compilation unit. 
When linked together into a final executable,
they can be used by a DWARF consumer to lookup a name to find one
or more skeleton compilation units that provide information about
that name. From the skeleton compilation unit, the consumer can
find the \splitDWARFobjectfile{} that it can then read to get the full
DWARF information.

The \dotdebugaranges{} section contains the PC ranges defined in this
compilation unit, and allow a DWARF consumer to map a PC value to
a skeleton compilation unit, and then to a \splitDWARFobjectfile.


\subsection{Contents of the Linked Executable File}
When \texttt{demo1.o} and \texttt{demo2.o} are linked together (along with 
a main program and other necessary library routines that we will ignore here
for simplicity), the resulting executable file will contain at least
the two skeleton compilation units in the \dotdebuginfo{} section, as shown in 
Figure \referfol{fig:splitobjectexampleexecutablefiledwarfexcerpts}.

\begin{figure}[ht]
\begin{dwflisting}
\begin{alltt}

    \DWTAGskeletonunit
        \DWATcompdir: (reference to directory name in \dotdebugstr)
        \DWATdwoname: (reference to "demo1.dwo" in \dotdebugstr)
        \DWATaddrbase: 48 (offset in \dotdebugaddr)
        \DWATstmtlist: 120 (offset in \dotdebugline)
    \DWTAGskeletonunit
        \DWATcompdir: (reference to directory name in \dotdebugstr)
        \DWATdwoname: (reference to "demo2.dwo" in \dotdebugstr)
        \DWATaddrbase: 80 (offset in \dotdebugaddr)
        \DWATstmtlist: 200 (offset in \dotdebugline)

\end{alltt}
\end{dwflisting}
\caption{Split object example: executable file DWARF excerpts}
\label{fig:splitobjectexampleexecutablefiledwarfexcerpts}
\end{figure}

Each skeleton compilation unit has a \DWATstmtlist{} attribute,
which provides the relocated offset to that compilation unit's
contribution in the executable's \dotdebugline{} section. In this
example, the line number information for \texttt{demo1.dwo} begins at
offset 120, and for \texttt{demo2.dwo}, it begins at offset 200.

\needlines{4}
Each skeleton compilation unit also has a \DWATaddrbase{}
attribute, which provides the relocated offset to that
compilation unit's contribution in the executable's \dotdebugaddr{}
section. Unlike the \DWATstmtlist{} attribute, the offset refers
to the first address table slot, not to the section header. In
this example, we see that the first address (slot 0) from \texttt{demo1.o}
begins at offset 48. Because the \dotdebugaddr{} section contains an
8-byte header, the object file's contribution to the section
actually begins at offset 40 (for a 64-bit DWARF object, the
header would be 16 bytes long, and the value for the
\DWATaddrbase{} attribute would then be 56). All attributes in
\texttt{demo1.dwo} that use \DWFORMaddrxXNor{} 
would then refer to address
table slots relative to that offset. Likewise, the \dotdebugaddr{}
contribution from \texttt{demo2.dwo} begins at offset 72, and its first
address slot is at offset 80. Because these contributions have
been processed by the linker, they contain relocated values for
the addresses in the program that are referred to by the debug
information.

The linked executable will also contain \dotdebugabbrev,
\dotdebugstr{}, \dotdebugnames{} and \dotdebugaranges{} sections, each the
result of combining and relocating the contributions from the
relocatable object files.

\needlines{10}
\subsection{Contents of the Split DWARF Object Files}
The \splitDWARFobjectfile{s} each contain the following sections:
\begin{alltt}
  \dotdebugabbrevdwo
  \dotdebuginfodwo{} (for the compilation unit)
  \dotdebuginfodwo{} (one COMDAT section for each type unit)
  \dotdebugloclistsdwo
  \dotdebuglinedwo
  \dotdebugmacrodwo
  \dotdebugrnglistsdwo
  \dotdebugstroffsetsdwo
  \dotdebugstrdwo
\end{alltt}
The \dotdebugabbrevdwo{} section contains the abbreviation
declarations for the debugging information entries in the
\dotdebuginfodwo{} section. 

The \dotdebuginfodwo{} section containing the compilation unit
contains the full debugging information for the compile unit, and
looks much like a normal \dotdebuginfo{} section in a non-split
object file, with the following exceptions:
\begin{itemize}
\item The \DWTAGcompileunit{} DIE does not need to repeat the
\DWATranges, \DWATlowpc, \DWAThighpc, and
\DWATstmtlist{} attributes that are provided in the skeleton
compilation unit.

\item References to strings in the string table use the 
form code \DWFORMstrxXNor, referring to slots in the
\dotdebugstroffsetsdwo{} section.

\needlines{4}
\item References to relocatable addresses in the object file use
one of the form codes \DWFORMaddrxXNor, referring to slots in the
\dotdebugaddr{} table, relative to the base offset given by
\DWATaddrbase{} in the skeleton compilation unit.
\end{itemize}

\vspace*{1mm}
Figure \referfol{fig:splitobjectexampledemoonedwodwarfexcerpts} presents
excerpts from the \dotdebuginfodwo{} section for \texttt{demo1.dwo}.

\begin{figure}[ht]
\vspace{1mm}
\figurepart{1}{2}
\begin{dwflisting}
\begin{alltt}

    \DWTAGcompileunit
        \DWATproducer [\DWFORMstrx]: (slot 15) (producer string)
        \DWATlanguagename: \DWLNAMECplusplus
        \DWATname [\DWFORMstrx]: (slot 7) "demo1.cc"
        \DWATcompdir [\DWFORMstrx]: (slot 4) (directory name)
1$:     \DWTAGclasstype
            \DWATname [\DWFORMstrx]: (slot 12) "Point"
            \DWATsignature [\DWFORMrefsigeight]: 0x2f33248f03ff18ab
            \DWATdeclaration: true
2$:         \DWTAGsubprogram
                \DWATexternal: true
                \DWATname [\DWFORMstrx]: (slot 12) "Point"
                \DWATdeclfile: 1
                \DWATdeclline: 5
                \DWATlinkagename [\DWFORMstrx]: (slot 16) "_ZN5PointC4Eff"
                \DWATaccessibility: \DWACCESSpublic
                \DWATdeclaration: true
            ...
3$:     \DWTAGclasstype
            \DWATname [\DWFORMstring]: "Box"
            \DWATsignature [\DWFORMrefsigeight{}]: 0xe97a3917c5a6529b
            \DWATdeclaration: true
          ...
4$:         \DWTAGsubprogram
                \DWATexternal: true
                \DWATname [\DWFORMstrx]: (slot 0) "contains"
                \DWATdeclfile: 1
                \DWATdeclline: 28
                \DWATlinkagename [\DWFORMstrx: (slot 8) 
                                                  "_ZNK3Box8containsERK5Point"
                \DWATtype: (reference to 7$)
                \DWATaccessibility: \DWACCESSpublic
                \DWATdeclaration: true
          ...

\end{alltt}
\end{dwflisting}
\caption{Split object example: \texttt{demo1.dwo} excerpts}
\label{fig:splitobjectexampledemoonedwodwarfexcerpts}
\end{figure}
        
\begin{figure}
\figurepart{2}{2}
\begin{dwflisting}
\begin{alltt}

5$:   \DWTAGsubprogram
          \DWATspecification: (reference to 4$)
          \DWATdeclfile: 2
          \DWATdeclline: 3
          \DWATlowpc [\DWFORMaddrx]: (slot 0)
          \DWAThighpc [\DWFORMdataeight]: 0xbb
          \DWATframebase: \DWOPcallframecfa
          \DWATobjectpointer: (reference to 6$)
6$:       \DWTAGformalparameter
              \DWATname [\DWFORMstrx]: (slot 13): "this"
              \DWATtype: (reference to 8$)
              \DWATartificial: true
              \DWATlocation: \DWOPfbreg(-24)
          \DWTAGformalparameter
              \DWATname [\DWFORMstring]: "p"
              \DWATdeclfile: 2
              \DWATdeclline: 3
              \DWATtype: (reference to 11$)
              \DWATlocation: \DWOPfbreg(-32)
      ...
7$:   \DWTAGbasetype
          \DWATbytesize: 1
          \DWATencoding: \DWATEboolean
          \DWATname [\DWFORMstrx]: (slot 5) "bool"
      ...
8$:   \DWTAGconsttype
          \DWATtype: (reference to 9$)
9$:   \DWTAGpointertype
          \DWATbytesize: 8
          \DWATtype: (reference to 10$)
10$:  \DWTAGconsttype
          \DWATtype: (reference to 3$)
      ...
11$:  \DWTAGconsttype
          \DWATtype: (reference to 12$)
12$:  \DWTAGreferencetype
          \DWATbytesize: 8
          \DWATtype: (reference to 13$)
13$:  \DWTAGconsttype
          \DWATtype: (reference to 1$)
      ...
\end{alltt}
\end{dwflisting}
\begin{center}
\vspace{3mm}
Figure~\ref{fig:splitobjectexampledemoonedwodwarfexcerpts}: Split object example: \texttt{demo1.dwo} DWARF excerpts \textit{(concluded)}
\end{center}
\end{figure}

\needlines{4}
In the defining declaration for \texttt{Box::contains} at 5\$, the
\DWATlowpc{} attribute is represented using \DWFORMaddrx, which refers 
to slot 0 in the \dotdebugaddr{} table from \texttt{demo1.o}.
That slot contains the relocated address of the beginning of the
function.

\needlines{6}
Each type unit is contained in its own COMDAT \dotdebuginfodwo{}
section, and looks like a normal type unit in a non-split object,
except that the \DWTAGtypeunit{} DIE contains a \DWATstmtlist{}
attribute that refers to a specialized \dotdebuglinedwo{}
\addtoindexx{type unit!specialized \texttt{.debug\_line.dwo} section in}
\addtoindexx{specialized \texttt{.debug\_line.dwo} section}
section. This
section contains a normal line number
program header with a list of include directories and filenames,
but no line number program. This section is used only as a
reference for filenames needed for \DWATdeclfile{} attributes
within the type unit.

The \dotdebugstroffsetsdwo{} section contains an entry for each
unique string in the string table. 
Each entry in the table is the offset of the string, which is
contained in the \dotdebugstrdwo{} section. 

In a split DWARF object file, all references to
strings go through this table (there are no
other offsets to \dotdebugstrdwo{} in a split
DWARF object file). That is, there
is no use of \DWFORMstrp{} in a split DWARF object file.

The offsets in these slots have no associated relocations, 
because they are not part of a relocatable object file.
When combined into a DWARF package file, however, each 
slot must be adjusted to refer to the appropriate offset 
within the merged string table (\dotdebugstrdwo{}).
The tool that builds the DWARF package file must understand 
the structure of the \dotdebugstroffsetsdwo{} section in 
order to apply the necessary adjustments. 
Section \refersec{app:dwarfpackagefileexample} presents
an example of a DWARF package file.

\needlines{4}
The \dotdebugrnglistsdwo{} section contains range lists referenced by any
\DWATranges{} attributes in the split DWARF object. In our example,
\texttt{demo1.o} would have just a single range list for the compilation unit,
with range list entries for the function \texttt{Box::contains} and for
out-of-line copies of the inline functions \texttt{Point::x} and \texttt{Point::y}.

The \dotdebugloclistsdwo{} section contains the location lists referenced
by \DWATlocation{} attributes in the \dotdebuginfodwo{} section. This
section has a similar format to the \dotdebugloclists{} section in a
non-split object, but 
\bb
the section
\eb
has some small differences as explained
in Section \refersec{datarep:locationlists}. 

\begin{figure}[b]
\figurepart{1}{2}
\begin{dwflisting}
\begin{alltt}

1$: \DWTAGclasstype
        \DWATname [\DWFORMstrx]: (slot 20) "Line"
        \DWATsignature [\DWFORMrefsigeight]: 0x79c7ef0eae7375d1
        \DWATdeclaration: true
        ...
2$:     \DWTAGsubprogram
            \DWATexternal: true
            \DWATname [\DWFORMstrx]: (slot 19) "clip"
            \DWATdeclfile: 2
            \DWATdeclline: 16
            \DWATlinkagename [\DWFORMstrx]: (slot 2) "_ZN4Line4clipERK3Box"
            \DWATtype: (reference to DIE for bool)
            \DWATaccessibility: \DWACCESSpublic
            \DWATdeclaration: true
        ...

\end{alltt}
\end{dwflisting}
\caption{Split object example: \texttt{demo2.dwo} DWARF \dotdebuginfodwo{} \mbox{excerpts}}
\label{fig:splitobjectexampledemotwodwodwarfdebuginfodwoexcerpts}
\end{figure}

\begin{figure}[t]
\figurepart{2}{2}
\begin{dwflisting}
\begin{alltt}

3$:   \DWTAGsubprogram
          \DWATspecification: (reference to 2$)
          \DWATdeclfile: 1
          \DWATdeclline: 3
          \DWATlowpc [\DWFORMaddrx]: (slot 32)
          \DWAThighpc [\DWFORMdataeight]: 0x1ec
          \DWATframebase: \DWOPcallframecfa
          \DWATobjectpointer: (reference to 4$)
4$:       \DWTAGformalparameter
              \DWATname: (indexed string: 0x11): this
              \DWATtype: (reference to DIE for type const Point* const)
              \DWATartificial: 1
              \DWATlocation: 0x0 (location list)
5$:       \DWTAGformalparameter
              \DWATname: b
              \DWATdeclfile: 1
              \DWATdeclline: 3
              \DWATtype: (reference to DIE for type const Box& const)
              \DWATlocation [\DWFORMsecoffset]: 0x2a
6$:       \DWTAGlexicalblock
              \DWATlowpc [\DWFORMaddrx]: (slot 17)
              \DWAThighpc: 0x1d5
7$:           \DWTAGvariable
                  \DWATname [\DWFORMstrx]: (slot 28): "slope"
                  \DWATdeclfile: 1
                  \DWATdeclline: 5
                  \DWATtype: (reference to DIE for type float)
                  \DWATlocation [\DWFORMsecoffset]: 0x49

\end{alltt}
\end{dwflisting}
\begin{center}
\vspace{3mm}
Figure~\ref{fig:splitobjectexampledemotwodwodwarfdebuginfodwoexcerpts}: 
			Split object example: \texttt{demo2.dwo} DWARF \dotdebuginfodwo{} excerpts \textit{(concluded)}
\end{center}
\end{figure}


In \texttt{demo2.dwo} as shown in 
Figure \refersec{fig:splitobjectexampledemotwodwodwarfdebuginfodwoexcerpts}, 
the debugging information for \texttt{Line::clip} 
starting at \texttt{2\$} describes a local 
variable \texttt{slope} at \texttt{7\$}
whose location varies based on the PC.
Figure \refersec{fig:splitobjectexampledemotwodwodwarfdebuglocdwoexcerpts} 
presents some excerpts from the \dotdebuginfodwo{} section for 
\texttt{demo2.dwo}.

\clearpage

In Figure \refersec{fig:splitobjectexampledemotwodwodwarfdebuginfodwoexcerpts},
the \DWTAGformalparameter{} entries at \texttt{4\$} and \texttt{5\$} refer to the
location lists at offset \texttt{0x0} and \texttt{0x2a}, respectively, and the
\DWTAGvariable{} entry for \texttt{slope} 
refers to the location list at offset \texttt{0x49}. 
Figure \refersec{fig:splitobjectexampledemotwodwodwarfdebuglocdwoexcerpts}
shows a representation of the
location lists at those offsets in the \dotdebugloclistsdwo{} section.

% Special commands for use in the folloing table
\newcommand{\XXLLEsl}{\hyperlink{chap:DWLLEstartlength}{start\_length}
                      \index{DW\_LLE\_start\_length}}
\newcommand{\XXLLEeol}{\hyperlink{chap:DWLLEendoflist}{end\_of\_list}
                       \index{DW\_LLE\_end\_of\_list}}

\begin{figure}[ht]
\begin{dwflisting}
\begin{center}
\begin{tabular}{rl|rr|rl}
   \multicolumn{2}{c}{\textbf{Entry type}}
                & \multicolumn{2}{c}{\textbf{Range}} 
                                & \multicolumn{2}{l}{\hspace{6mm}\textbf{Counted Location Description}} \\
offset& (DW\_LLE\_*)   
                & start& length & length & expression \\
\hline
&&&&& \\

0x00 & \XXLLEsl &  [9] & 0x002f & 0x01 & \DWOPregfive~(rdi) \\
0x09 & \XXLLEsl & [11] & 0x01b9 & 0x01 & \DWOPregthree~(rbx) \\
0x12 & \XXLLEsl & [29] & 0x0003 & 0x03 & \DWOPbregtwelve~(r12): -8;\\
     &          &      &        &      & \DWOPstackvalue \\
0x1d & \XXLLEsl & [31] & 0x0001 & 0x03 & \DWOPentryvalue: \\
     &          &      &        &      & (\DWOPregfive~(rdi)); \\
     &          &      &        &      & \DWOPstackvalue \\
0x29 & \XXLLEeol &&&& \\
------ &&&&& \\

0x2a & \XXLLEsl &  [9] & 0x002f & 0x01 & \DWOPregfour~(rsi)) \\
0x33 & \XXLLEsl & [11] & 0x01ba & 0x03 & \DWOPregsix~(rbp)) \\
0x3c & \XXLLEsl & [30] & 0x0003 & 0x03 & \DWOPentryvalue: \\
     &          &      &        &      & (\DWOPregfour~(rsi)); \\
     &          &      &        &      & \DWOPstackvalue \\
0x48 & \XXLLEeol &&&& \\
------ &&&&& \\

0x49 & \XXLLEsl & [10] & 0x0004 & 0x01 & \DWOPregeighteen~(xmm1) \\
0x52 & \XXLLEsl & [11] & 0x01bd & 0x02 & \DWOPfbreg: -36 \\
0x5c & \XXLLEeol &&&& \\
&&&& \\
\end{tabular}
\end{center}
\end{dwflisting}
\caption{Split object example: \texttt{demo2.dwo} DWARF \dotdebugloclistsdwo{} \mbox{excerpts}}
\label{fig:splitobjectexampledemotwodwodwarfdebuglocdwoexcerpts}
\end{figure}

In each \DWLLEstartlength{} entry, 
the start field is the index
of a slot in the \dotdebugaddr{} section, relative to the base
offset defined by the compilations unit's \DWATaddrbase{}
attribute. The \dotdebugaddr{} slots referenced by these entries give
the relocated address of a label within the function where the
address range begins. 
The following length field gives the length of the
address range. The location, consisting of its own length and
a DWARF expression, is last.

\clearpage
\section{DWARF Package File Example}
\label{app:dwarfpackagefileexample}
\addtoindexx{DWARF duplicate elimination!examples}

A \addtoindex{DWARF package file} 
(see Section \refersec{datarep:dwarfpackagefiles}) 
is a collection of split DWARF object files.
In general, it will be much smaller than the sum of the split
DWARF object files, because the packaging process removes duplicate
type units and merges the string tables. Aside from those two
optimizations, however, each compilation unit and each type unit
from a split DWARF object file is copied verbatim into the package
file.

The package file contains the same set of sections as a split
DWARF object file, plus two additional sections described below.

The packaging utility, like a linker, combines sections of the
same name by concatenation. While a split DWARF object may
contain multiple \dotdebuginfodwo{} sections, one for the
compilation unit, and one for each type unit, a package file
contains a single \dotdebuginfodwo{} section. The combined
\dotdebuginfodwo{} section contains each compilation unit and one
copy of each type unit (discarding any duplicate type
signatures).

As part of merging the string tables, the packaging utility
treats the \dotdebugstrdwo{} and \dotdebugstroffsetsdwo{}
sections specially. Rather than
combining them by simple concatenation, it instead builds a new
string table consisting of the unique strings from each input
string table. Because all references to these strings use
form \DWFORMstrx{},
the packaging utility only needs to adjust the
string offsets in each \dotdebugstroffsetsdwo{} contribution after
building the new \dotdebugstrdwo{} section.

Each compilation unit or type unit consists of a set of
inter-related contributions to each section in the package file.
For example, a compilation unit may have contributions in
\dotdebuginfodwo{}, \dotdebugabbrevdwo{}, \dotdebuglinedwo{},
\dotdebugstroffsetsdwo{}, and so on. In order to maintain the ability 
for a consumer to follow references between these sections, the
package file contains two additional sections: a compilation unit
(CU) index, and a type unit (TU) index. These indexes allow a
consumer to look up a compilation unit (by its \CUsignature) or 
a type unit (by its \TUsignature), and locate each contribution 
that belongs to that unit.

For example, consider a package file, \texttt{demo.dwp}, formed by
combining \texttt{demo1.dwo} and \texttt{demo2.dwo} from the previous example
(see Appendix \refersec{app:splitdwarfobjectfileexample}). 
For an executable file named "\texttt{demo}" (or "\texttt{demo.exe}"), a debugger would
typically expect to find \texttt{demo.dwp} in the same directory as the
executable file.
The resulting package file would contain the sections shown in Figure
\refersec{fig:sectionsandcontributionsinapackagefile}, 
with contributions from each input file as shown.

\begin{figure}[ht]
\begin{center}
\begin{tabular}{P{4.7cm}|P{8cm}}
\hline
\bfseries Section & \bfseries Source of section contributions \\
\hline
  \dotdebugabbrevdwo{}
&    \dotdebugabbrevdwo{} from \texttt{demo1.dwo} \newline
     \dotdebugabbrevdwo{} from \texttt{demo2.dwo} \\
\hline \newline
  \dotdebuginfodwo{} \newline (for the compilation units and type units)
&    compilation unit from \texttt{demo1.dwo} \newline
     compilation unit from \texttt{demo2.dwo} \newline
     type unit for class \texttt{Box} from \texttt{demo1.dwo}   \newline
     type unit for class \texttt{Point} from \texttt{demo1.dwo} \newline
     type unit for class \texttt{Line} from \texttt{demo2.dwo}  \\
\hline
  \dotdebugrnglistsdwo{} 
&    \dotdebugrnglistsdwo{} from \texttt{demo1.dwo} \newline
     \dotdebugrnglistsdwo{} from \texttt{demo2.dwo} \\
\hline
  \dotdebugloclistsdwo{}
&    \dotdebugloclistsdwo{} from \texttt{demo1.dwo} \newline
     \dotdebugloclistsdwo{} from \texttt{demo2.dwo} \\
\hline
  \dotdebuglinedwo{}
&    \dotdebuglinedwo{} from \texttt{demo1.dwo} \newline
     \dotdebuglinedwo{} from \texttt{demo2.dwo} \\
\hline
  \dotdebugstroffsetsdwo{}
&    \dotdebugstroffsetsdwo{} from \texttt{demo1.dwo}, \hspace*{6mm}adjusted \newline
     \dotdebugstroffsetsdwo{} from \texttt{demo2.dwo}, \hspace*{6mm}adjusted \\
\hline
  \dotdebugstrdwo{}
&    merged string table generated by package utility \\
\hline
  \dotdebugcuindex
&    CU index generated by package utility \\
\hline
  \dotdebugtuindex
&    TU index generated by package utility \\
\hline
\end{tabular}
\end{center}
\caption{Sections and contributions in example package file \texttt{demo.dwp}}
\label{fig:sectionsandcontributionsinapackagefile}
\end{figure}

\needlines{4}
The \dotdebugabbrevdwo{}, \dotdebugrnglistsdwo{}, 
\dotdebugloclistsdwo{} and \dotdebuglinedwo{}
sections are copied over from the two \texttt{.dwo} files as
individual contributions to the corresponding sections in the
\texttt{.dwp} file. 
The offset of each contribution within 
the combined section and the size of each contribution is recorded
as part of the CU and TU index sections.

The \dotdebuginfodwo{} sections corresponding to each compilation 
unit are copied as individual contributions to the combined
\dotdebuginfodwo{} section, and one copy of each type unit 
is also copied. The type units for class \texttt{Box} and class 
\texttt{Point}, for example, are contained in both \texttt{demo1.dwo} 
and \texttt{demo2.dwo}, but only one instance of each is copied into 
the package file.

\needlines{4}
The \dotdebugstrdwo{} sections from each file are merged to
form a new string table with no duplicates, requiring the
adjustment of all references to those strings. The
\dotdebugstroffsetsdwo{} sections from the \texttt{.dwo} files 
are copied as individual contributions, but the string table offset
in each slot of those contributions is adjusted to point to
the correct offset in the merged string table.

The \dotdebugcuindex{} and \dotdebugtuindex{} sections provide a
directory to these contributions. 
Figure \referfol{fig:examplecuindexsection} shows an example CU
index section containing the two compilation units from 
\texttt{demo1.dwo} and \texttt{demo2.dwo}. The CU index shows that 
for the compilation unit from \texttt{demo1.dwo}, with \CUsignature{} 
\texttt{0x044e413b8a2d1b8f}, its contribution to the \dotdebuginfodwo{} 
section begins at offset 0, and is 325 bytes long. For the compilation 
unit from \texttt{demo2.dwo}, with \CUsignature{} 
\texttt{0xb5f0ecf455e7e97e}, its contribution to the \dotdebuginfodwo{}
section begins at offset 325, and is 673 bytes long.

Likewise, we can find the contributions to the related sections.
In Figure \refersec{fig:splitobjectexampledemotwodwodwarfdebuglocdwoexcerpts}, 
we see that the \DWTAGvariable{} DIE at \texttt{7\$} has a
reference to a location list at offset 0x49 (decimal 73). Because
this is part of the compilation unit for \texttt{demo2.dwo}, with 
unit signature \texttt{0xb5f0ecf455e7e97e}, we see that its contribution 
to \dotdebugloclistsdwo{} begins at offset 84, so the location list from
Figure \refersec{fig:splitobjectexampledemotwodwodwarfdebuglocdwoexcerpts} 
can be found in \texttt{demo.dwp} at offset 157 (84 + 73) in
the combined \dotdebugloclistsdwo{} section.

\begin{figure}[ht]
\begin{center}
\begin{tabular}{lrrrrrrr}
\\
  \multicolumn{7}{c}{Section header} \\
\hline \\
  \multicolumn{2}{l}{Version:}&                 5  &&&&&\\
  \multicolumn{2}{l}{Number of columns:}&       6  &&&&&\\
  \multicolumn{2}{l}{Number of used entries:}&  2  &&&&&\\
  \multicolumn{2}{l}{Number of slots:}&         16 &&&&&\\
\\
  \multicolumn{7}{c}{Offset table} \\
  \hline
  slot&  signature&                       info&   abbrev&      loc&     line& str\_off&    rng \\ \\
    14& \texttt{0xb5f0ecf455e7e97e} &      325&      452&       84&       52&       72&    350 \\
    15& \texttt{0x044e413b8a2d1b8f} &        0&        0&        0&        0&        0&      0 \\
\\
  \multicolumn{7}{c}{Size table} \\
  \hline
  slot&                    &     info&   abbrev&      loc&     line& str\_off&    rng \\ \\
    14&                    &      673&      593&       93&       52&      120&     34 \\
    15&                    &      325&      452&       84&       52&       72&     15 \\
\\ \hline 
\end{tabular}
\end{center}
\caption{Example CU index section}
\label{fig:examplecuindexsection}
\end{figure}

\needlines{4}
Figure \referfol{fig:exampletuindexsection} 
shows an example TU index section containing the
three type units for classes \texttt{Box}, \texttt{Point}, and 
\texttt{Line}. Each type unit
contains contributions from \dotdebuginfodwo{}, \dotdebugabbrevdwo{},
\dotdebuglinedwo{} and \dotdebugstroffsetsdwo{}. In this example, the
type units for classes \texttt{Box} and \texttt{Point} come from 
\texttt{demo1.dwo}, and
share the abbreviations table, line number table, and string
offsets table with the compilation unit from \texttt{demo1.dwo}. 
Likewise, the type unit for class \texttt{Line} shares tables 
from \texttt{demo2.dwo}. 

The sharing of these tables between compilation units and type units
is typical for some implementations, but is not required by the
DWARF standard.

\begin{figure}[ht]
\begin{center}
\begin{tabular}{lrrrrr}
\\
  \multicolumn{6}{c}{Section header} \\
\hline \\
  \multicolumn{2}{l}{Version:}&                 5 \\
  \multicolumn{2}{l}{Number of columns:}&       4 \\
  \multicolumn{2}{l}{Number of used entries:}&  3 \\
  \multicolumn{2}{l}{Number of slots:}&         32 \\
\\
  \multicolumn{6}{c}{Offset table} \\
  \hline
  slot&  signature&                    info&   abbrev&     line& str\_off \\ \\
  11& \texttt{0x2f33248f03ff18ab}&     1321&        0&        0&        0 \\
  17& \texttt{0x79c7ef0eae7375d1}&     1488&      452&       52&       72 \\
  27& \texttt{0xe97a3917c5a6529b}&      998&        0&        0&        0 \\
\\
  \multicolumn{6}{c}{Size table} \\
  \hline
  slot&                          &     info&   abbrev&     line& str\_off \\ \\
  11&                            &      167&      452&       52&       72 \\
  17&                            &      217&      593&       52&      120 \\
  27&                            &      323&      452&       52&       72 \\
\\
\hline
\end{tabular}
\end{center}
\caption{Example TU index section}
\label{fig:exampletuindexsection}
\end{figure}

