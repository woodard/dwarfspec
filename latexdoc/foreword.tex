\renewcommand{\abstractname}{\Large Foreword}
\begin{abstract}
\setlength{\parindent}{0pt}
\nonzeroparskip
\ \break
The \dwf\ Committee was originally organized in 1988 as the
Programming Languages Special Interest Group (PLSIG) of Unix
International, Inc., a trade group organized to promote Unix
System V Release 4 (SVR4).

PLSIG drafted a standard for \DWARFVersionI, compatible with
the DWARF debugging format used at the time by SVR4 compilers
and debuggers from AT\&T.  This was published as Revision 1.1.0
on October 6, 1992. PLSIG also designed the \addtoindex{DWARF Version 2}
format, which followed the same general philosophy as Version
1, but with significant new functionality and a more compact,
though incompatible, encoding.  An industry review draft
of \DWARFVersionII{} was published as Revision 2.0.0 on July
27, 1993.

Unix International dissolved shortly after the draft of
Version 2 was released; no industry comments were received or
addressed, and no final standard was released. The committee
mailing list was hosted by OpenGroup (formerly XOpen).

The Committee reorganized in October, 1999, and met for the
next several years to address issues that had been noted with
\DWARFVersionII{} as well as to add a number of new features.
In mid-2003, the Committee became a workgroup under the Free
Standards Group (FSG), an industry consortium chartered to
promote open standards. \DWARFVersionIII{} was published on
December 20, 2005, following industry review and comment.

The DWARF Committee withdrew from the Free Standards Group
in February, 2007, when FSG merged with the Open Source
Development Labs to form The Linux Foundation, more narrowly
focused on promoting Linux. The DWARF Committee has been
independent since that time.

It is the intention of the DWARF Committee that migrating from
an earlier version of the DWARF standard to the current version
should be straightforward and easily accomplished.
Almost all constructs from \DWARFVersionII{}
onward have been retained unchanged in
\bb
\DWARFVersionVI,
\eb
although a few
have been compatibly superseded by improved constructs which are
more compact and/or more expressive.

This document was created using the \LaTeX{} document preparation
system.

\clearpage

{\bfseries The \dwf{} Committee}

The \dwf{} Committee is open to compiler and debugger
developers who have experience with source language debugging
and debugging formats, and have an interest in promoting or
extending the DWARF debugging format.

DWARF Committee members contributing to Version
\bb
6
\eb
are:
\begin{center}
\begin{tabular}{ll}
\bbeb
Todd Allen              & Concurrent Real-Time \\
\bbeb
Pedro Alves				& Pedro Alves Services \\
David Anderson, Associate Editor \\
\bbeb
David Blaikie			& Google \\
\db
Ron Brender, Editor \\
Andrew Cagney \\
Eric Christopher        & Google \\
\bbeb
Cary Coutant, Chair (from March 2023) \\
\bbeb
John DelSignore         & Perforce \\
\bbeb
Jonas Devlieghere		& Apple \\
\bbeb
Michael Eager, past Chair (to February 2023)   & Eager Consulting \\
\bbeb
Jini Susan George       & AMD \\
\bbeb
\db
Tommy Hoffner           & Untether AI \\
Jakub Jel\'{i}nek       & Red Hat \\
\bbeb
Simon Marchi			& EfficiOS \\
\db
Jason Merrill			& Red Hat \\
\bbeb
\db
Markus Metzger			& Intel \\
\bbeb
Jeremy Morse			& Sony \\
Adrian Prantl           & Apple \\
Hafiz Abid Qadeer       & Mentor Graphics \\
Paul Robinson			& Sony \\
\bbeb
Tom Russell				& Sony \\
\bbeb
F\={a}ng-rui S\`{o}ng	& Google \\
\bbeb
Caroline Tice			& Google \\
\bbeb
Tom Tromey				& Adacore \\
\bbeb
Tony Tye				& AMD \\
\bbeb
Keith Walker            & Arm \\
\bbeb
\db
Mark Wielaard			& Red Hat \\
Brock Wyma              & Intel \\
Jian Xu                 & IBM \\
\bbeb
Zoran Zaric				& AMD \\
\end{tabular}
\end{center}

For further information about
DWARF or the DWARF Committee,
see:

% For unclear reasons, simply adding the url
% to the sentence here makes the url extend past the
% end of the line and generates a warning and
% ugly output.
\begin{myindentpara}{8mm}
\url{http://www.dwarfstd.org}
\end{myindentpara}

\clearpage
{\bfseries How to Use This Document}

This document is intended to be usable in online as well as
traditional paper forms. Both online and paper forms include
page numbers, a Table of Contents, a List of Figures,
a List of Tables and an Index.

Text in normal font describes
required aspects of the DWARF format.  Text in \textit{italics} is
explanatory or supplementary material, and not part of the
format definition itself.

\textit{Online Form}

In the online form, \textcolor{blue}{blue text}
is used to indicate hyperlinks.
Most hyperlinks link to the definition of a term or
construct, or to a cited Section or Figure.
However, attributes
in particular are often used in more than one way or context so
that there is no single definition; for attributes, hyperlinks
link to the introductory table of all attributes which in turn
contains hyperlinks for the multiple usages.

The occurrence of
a DWARF name in its definition (or one of its definitions in the
case of some attributes) is shown in \definition{red text}.
Other occurrences of the same name in the same or possibly following
paragraphs are generally in normal text color.)

The Table of Contents, List of Figures, List of Tables and Index
provide hyperlinks to the respective items and places.

\textit{Paper Form}

In the traditional paper form, the appearance of the hyperlinks
and definitions on a page of paper does not distract the eye
because the blue hyperlinks and the color used for definitions
are typically imaged by black and white printers in
a manner nearly indistinguishable from other text.
(Hyperlinks are not underlined for this same reason.)


\end{abstract}
